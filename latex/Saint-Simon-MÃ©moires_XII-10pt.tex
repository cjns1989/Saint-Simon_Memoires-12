\PassOptionsToPackage{unicode=true}{hyperref} % options for packages loaded elsewhere
\PassOptionsToPackage{hyphens}{url}
%
\documentclass[oneside,10pt,french,]{extbook} % cjns1989 - 27112019 - added the oneside option: so that the text jumps left & right when reading on a tablet/ereader
\usepackage{lmodern}
\usepackage{amssymb,amsmath}
\usepackage{ifxetex,ifluatex}
\usepackage{fixltx2e} % provides \textsubscript
\ifnum 0\ifxetex 1\fi\ifluatex 1\fi=0 % if pdftex
  \usepackage[T1]{fontenc}
  \usepackage[utf8]{inputenc}
  \usepackage{textcomp} % provides euro and other symbols
\else % if luatex or xelatex
  \usepackage{unicode-math}
  \defaultfontfeatures{Ligatures=TeX,Scale=MatchLowercase}
%   \setmainfont[]{EBGaramond-Regular}
    \setmainfont[Numbers={OldStyle,Proportional}]{EBGaramond-Regular}      % cjns1989 - 20191129 - old style numbers 
\fi
% use upquote if available, for straight quotes in verbatim environments
\IfFileExists{upquote.sty}{\usepackage{upquote}}{}
% use microtype if available
\IfFileExists{microtype.sty}{%
\usepackage[]{microtype}
\UseMicrotypeSet[protrusion]{basicmath} % disable protrusion for tt fonts
}{}
\usepackage{hyperref}
\hypersetup{
            pdftitle={SAINT-SIMON},
            pdfauthor={Mémoires XII},
            pdfborder={0 0 0},
            breaklinks=true}
\urlstyle{same}  % don't use monospace font for urls
\usepackage[papersize={4.80 in, 6.40  in},left=.5 in,right=.5 in]{geometry}
\setlength{\emergencystretch}{3em}  % prevent overfull lines
\providecommand{\tightlist}{%
  \setlength{\itemsep}{0pt}\setlength{\parskip}{0pt}}
\setcounter{secnumdepth}{0}

% set default figure placement to htbp
\makeatletter
\def\fps@figure{htbp}
\makeatother

\usepackage{ragged2e}
\usepackage{epigraph}
\renewcommand{\textflush}{flushepinormal}

\usepackage{indentfirst}
\usepackage{relsize}

\usepackage{fancyhdr}
\pagestyle{fancy}
\fancyhf{}
\fancyhead[R]{\thepage}
\renewcommand{\headrulewidth}{0pt}
\usepackage{quoting}
\usepackage{ragged2e}

\newlength\mylen
\settowidth\mylen{...................}

\usepackage{stackengine}
\usepackage{graphicx}
\def\asterism{\par\vspace{1em}{\centering\scalebox{.9}{%
  \stackon[-0.6pt]{\bfseries*~*}{\bfseries*}}\par}\vspace{.8em}\par}

\usepackage{titlesec}
\titleformat{\chapter}[display]
  {\normalfont\bfseries\filcenter}{}{0pt}{\Large}
\titleformat{\section}[display]
  {\normalfont\bfseries\filcenter}{}{0pt}{\Large}
\titleformat{\subsection}[display]
  {\normalfont\bfseries\filcenter}{}{0pt}{\Large}

\setcounter{secnumdepth}{1}
\ifnum 0\ifxetex 1\fi\ifluatex 1\fi=0 % if pdftex
  \usepackage[shorthands=off,main=french]{babel}
\else
  % load polyglossia as late as possible as it *could* call bidi if RTL lang (e.g. Hebrew or Arabic)
%   \usepackage{polyglossia}
%   \setmainlanguage[]{french}
%   \usepackage[french]{babel} % cjns1989 - 1.43 version of polyglossia on this system does not allow disabling the autospacing feature
\fi

\title{SAINT-SIMON}
\author{Mémoires XII}
\date{}

\begin{document}
\maketitle

\hypertarget{chapitre-premier.}{%
\chapter{CHAPITRE PREMIER.}\label{chapitre-premier.}}

1715

~

{\textsc{Chute de la princesse des Ursins.}} {\textsc{- Réflexions.}}
{\textsc{- Comtesse douairière d'Altamire camarera-mayor, et le prince
de Cellamare grand écuyer de la reine.}} {\textsc{- Cardinal del Giudice
rappelé.}} {\textsc{- Macañas et Orry chassés d'Espagne.}} {\textsc{-
Pompadour remercié, et le duc de Saint-Aignan ambassadeur en Espagne.}}
{\textsc{- Tolède donné à un simple curé.}} {\textsc{- Mort de la
duchesse d'Aveiro et du marquis de Mancera.}} {\textsc{- Succès de la
reine près du roi d'Espagne.}} {\textsc{- Sa préférence pour les
Italiens.}} {\textsc{- Mort de la comtesse de Roye à Londres\,; sa
famille.}} {\textsc{- Mariage du comte de Poitiers avec
M\textsuperscript{lle} de Malause.}} {\textsc{- Mariage d'Ancezune avec
une fille de Torcy.}} {\textsc{- Les Caderousse.}} {\textsc{- Mariage du
fils d'O avec une fille de Lassai, et d'Arpajon avec la fille de
Montargis.}} {\textsc{- Statue avortée du maréchal de Montrevel.}}
{\textsc{- Ambassadeur de Perse, plus que douteux, à Paris.}} {\textsc{-
Son entrée\,; sa première audience\,; sa conduite.}} {\textsc{-
Magnificences étalées devant lui.}} {\textsc{- Citation à Malte sans
effet comme sans cause effective.}} {\textsc{- Le grand prieur y va sans
avoir pu voir le roi.}} {\textsc{- Cent mille livres à Bonrepos.}}
{\textsc{- La Chapelle, un des premiers commis de la marine, tout à
Pontchartrain, et sa femme chassés par la jalousie et les artifices de
Pontchartrain.}} {\textsc{- Électeur de Bavière visite à Blois la reine
de Pologne, sa belle-mère\,; fait à Compiègne la noce de sa maîtresse
avec le comte d'Albert\,; prend congé du roi à Versailles en
particulier, et s'en va dans ses États.}}

~

On a vu que la princesse des Ursins s'était enfin perdue avec le roi et
M\textsuperscript{me} de Maintenon. Le roi ne lui avait pu pardonner
l'audace de sa souveraineté, l'obstacle que son opiniâtreté, voilée de
celle qu'elle inspirait au roi d'Espagne, avait mis si longtemps à sa
paix, malgré tout ce que le roi avait pu faire, et qui ne vint à bout de
faire abandonner cette folie, qu'aucun des alliés n'avait voulu écouter,
qu'en lui déclarant enfin qu'il l'abandonnerait à ses propres forces. Le
roi avait vivement senti la frayeur que le roi d'Espagne ne l'épousât,
et ensuite l'autorité sans voile et sans bornes qu'elle avait prise sur
le roi d'Espagne, dans la solitaire captivité où elle le retenait au
palais de Medina-Celi. Enfin le roi se sentit piqué jusqu'au fond de
l'âme du mariage de Parme, négocié et conclu sans lui en avoir donné la
moindre participation. Roi partout, et dans sa famille plus que partout
ailleurs, s'il était possible, il n'était pas accoutumé à voir marier
ses enfants en étranger. Le choix en soi ne lui pouvait plaire, et la
manière y ajouta tout. M\textsuperscript{me} de Maintenon qui, comme on
l'a vu, n'avait jamais soutenu et porté M\textsuperscript{me} des Ursins
au point d'autorité et de puissance où elle était parvenue que pour
régner par elle en Espagne, ce qu'elle ne pouvait espérer par les
ministres, sentit vivement l'affranchissement de son joug, par
l'indépendance entière dont elle gouverna depuis la mort de la reine, et
l'abus qu'elle faisait avec si peu de ménagement de toute la confiance
du roi d'Espagne. Elle fut encore plus sensible que le roi à la frayeur
de la voir reine d'Espagne, elle qui avait manqué par deux fois sa
déclaration de reine de France, si positivement promise. Enfin la
souveraineté, qui la laissait si loin derrière M\textsuperscript{me} des
Ursins, l'avait rendue son ennemie\,; et le mariage de Parme, fait à
l'entier insu du roi et d'elle, ne lui laissait plus d'espérance
d'influer sur l'Espagne par la princesse des Ursins. La perte de
celle-ci fut donc conclue entre le roi et M\textsuperscript{me} de
Maintenon, mais d'une manière si secrète, devant et depuis, que je n'ai
connu personne qui ait pénétré de qui ils se servirent, ni ce qu'ils
firent pour l'exécuter. Il est de la bonne foi d'avouer ses ténèbres, et
de ne donner pas des fictions et des inventions à la place de ce qu'on
ignore. Il faut raconter l'événement avec exactitude, et ne donner après
ses courtes réflexions que pour ce qu'elles peuvent valoir.

La reine d'Espagne s'avançait vers Madrid avec ce qui avait été la
recevoir aux frontières d'équipage, de maison et de gardes du roi
d'Espagne. Albéroni était à sa suite depuis Parme, et le duc de
Saint-Aignan depuis le lieu où il l'avait jointe en France. La princesse
des Ursins avait pris auprès d'elle la charge de camarera-mayor, comme
elle l'avait auprès de la feue reine, et avait nommé toute sa maison,
qu'elle avait remplie de ses créatures, hommes et femmes. Elle n'avait
eu garde de quitter le roi de loin\,; ainsi elle le suivit à
Guadalaxara, petite ville appartenant au duc de l'Infantado, qui y a
fait un panthéon aux cordeliers beaucoup plus petit que celui de
l'Escurial, sur le même modèle, et qui, pour la richesse et l'art, ne
lui cède guère en beauté. J'aurai lieu d'en parler ailleurs. Guadalaxara
est sur le chemin de Madrid à Burgos, par conséquent de France, à peu
près de distance de Madrid quelque chose de plus que de Paris à
Fontainebleau. Le palais qu'y ont les ducs de l'Infantado est vaste,
beau, bien meublé, et en est habité quelquefois. Ce fut jusque-là que le
roi d'Espagne voulut s'avancer, et dans la chapelle de ce palais qu'il
résolut de célébrer son mariage, quoiqu'il l'eût été, comme on l'a vu, à
Parme par procureur. Le voyage fut ajusté des deux côtés de façon que le
roi n'arrivât à Guadalaxara que la surveille de la reine.

Il fit ce petit voyage accompagné de ceux que la princesse des Ursins
avait mis auprès de lui, pour lui tenir toujours compagnie et n'en
laisser approcher qui que ce soit. Elle suivait dans son carrosse pour
arriver en même temps\,; et dès en arrivant, le roi s'enfermait seul
avec elle et ne voyait plus personne jusqu'à son coucher. Les
retardements des chemins et de la saison avaient conduit à Noël. Ce fut
le 22 décembre que le roi d'Espagne arriva à Guadalaxara. Le lendemain
23, surveille de Noël, la princesse des Ursins partit avec une très
légère suite pour aller à sept lieues plus loin à une petite villette
nommée Quadraqué, où la reine devait coucher ce même soir.
M\textsuperscript{me} des Ursins comptait aller jouir de toute la
reconnaissance de la grandeur inespérable qu'elle lui procurait, passer
la soirée avec elle, et l'accompagner le lendemain dans son carrosse à
Guadalaxara. Elle trouva à Quadraqué la reine arrivée\,; elle mit pied à
terre en un logis qu'on lui avait préparé vis-à-vis et tout près de
celui de la reine. Elle était venue en grand habit de cour et parée.
Elle ne fit que se rajuster un peu, et s'en alla chez la reine. La
froideur et la sécheresse de sa réception la surprit d'abord
extrêmement\,; elle l'attribua d'abord à l'embarras de la reine, et
tâcha de réchauffer cette glace. Le monde cependant s'écoula par respect
pour les laisser seules.

Alors la conversation commença. La reine ne la laissa pas continuer, se
mit incontinent sur les reproches qu'elle lui manquait de respect par
l'habillement avec lequel elle paraissait devant elle, et par ses
manières. M\textsuperscript{me} des Ursins, dont l'habit était régulier,
et qui, par ses manières respectueuses et ses discours propres à ramener
la reine, se croyait bien éloignée de mériter cette sortie de sa part,
fut étrangement surprise et voulut s'excuser\,; mais voilà tout aussitôt
la reine aux paroles offensantes, à s'écrier, à appeler, à demander des
officiers des gardes, et à commander avec injure à M\textsuperscript{me}
des Ursins de sortir de sa présence. Elle voulut parler et se défendre
des reproches qu'elle recevait, la reine, redoublant de furie et de
menaces, se mit à crier qu'on fît sortir cette folle de sa présence et
de son logis, et l'en fit mettre dehors par les épaules. À l'instant
elle appelle Amenzaga, lieutenant des gardes du corps, qui commandait le
détachement qui était auprès d'elle, et en même temps, l'écuyer qui
commandait ses équipages\,; ordonne au premier d'arrêter
M\textsuperscript{me} des Ursins et de ne la point quitter qu'il ne
l'eût mise dans un carrosse avec deux officiers des gardes sûrs et une
quinzaine de gardes autour du carrosse\,; au deuxième, de faire
sur-le-champ venir un carrosse à six chevaux et deux ou trois valets de
pied, de faire partir sur l'heure la princesse des Ursins vers Burgos et
Bayonne, et de ne se point arrêter. Amenzaga voulut représenter à la
reine qu'il n'y avait que le roi d'Espagne qui eût le pouvoir qu'elle
voulait prendre\,; elle lui demanda fièrement s'il n'avait pas un ordre
du roi d'Espagne de lui obéir en tout, sans réserve et sans
représentation. Il était vrai qu'il l'avait, et qui que ce fût n'en
savait rien.

M\textsuperscript{me} des Ursins fut donc arrêtée à l'instant et mise en
carrosse avec une de ses femmes de chambre, sans avoir eu le temps de
changer d'habit ni de coiffure, de prendre aucune précaution contre le
froid, d'emporter ni argent ni aucune autre chose, ni elle ni sa femme
de chambre, et sans aucune sorte de nourriture dans son carrosse, ni
chemise, ni quoi que ce soit pour changer ou se coucher. Elle fut donc
embarquée ainsi avec les deux officiers des gardes, qui se trouvèrent
prêts dans le moment ainsi que le carrosse, elle en grand habit et parée
comme elle était sortie de chez la reine. Dans ce très court tumulte
elle voulut envoyer à la reine, qui s'emporta de nouveau de ce qu'elle
n'avait pas encore obéi, et la fit partir à l'instant. Il était lors
près de sept heures du soir, la surveille de Noël, la terre toute
couverte de glace et de neige, et le froid extrême et fort vif et
piquant, comme il est toujours en Espagne. Dès que la reine sut la
princesse des Ursins hors de Quadraqué, elle écrivit au roi d'Espagne
par un officier des gardes qu'elle dépêcha à Guadalaxara. La nuit était
si obscure qu'on ne voyait qu'à la faveur de la neige.

Il n'est pas aisé de se représenter l'état de M\textsuperscript{me} des
Ursins dans ce carrosse. L'excès de l'étonnement et de l'étourdissement
prévalut d'abord, et suspendit tout autre sentiment\,: mais bientôt la
douleur, le dépit, la rage et le désespoir se firent place. Succédèrent
à leur tour les tristes et profondes réflexions sur une démarche aussi
violente et aussi inouïe, d'ailleurs si peu fondée en cause, en raisons,
en prétextes même les plus légers, enfin en autorité, et sur
l'impression qu'elle allait faire à Guadalaxara\,; et de là les
espérances en la surprise du roi d'Espagne, en sa colère, en son amitié
et sa confiance pour elle, en ce groupe de serviteurs si attachés à elle
dont elle l'avait environné, qui se trouveraient si intéressés à exciter
le roi en sa faveur. La longue nuit d'hiver se passa ainsi tout entière,
avec un froid terrible, rien pour s'en garantir, et tel que le cocher en
perdit une main. La matinée s'avança\,; nécessité fut de s'arrêter pour
faire repaître les chevaux\,; mais pour les hommes il n'y a quoi que ce
soit dans les hôtelleries d'Espagne, où on vous indique seulement où se
vend chaque chose dont on a besoin. La viande est ordinairement
vivante\,; le vin épais, plat et violent\,; le pain se colle à la
muraille\,; l'eau souvent ne vaut rien\,; de lits, il n'y en a que pour
les muletiers, en sorte qu'il faut tout porter avec soi\,; et
M\textsuperscript{me} des Ursins ni ce qui était avec elle n'avaient
chose quelconque. Les œufs, où elle en put trouver, fut leur unique
ressource, et encore à la coque, frais ou non, pendant toute la route.

Jusqu'à cette repue des chevaux, le silence avait été profond et non
interrompu. Là il se rompit. Pendant toute cette longue nuit, la
princesse des Ursins avait eu le loisir de penser aux propos qu'elle
tiendrait, et à composer son visage. Elle parla de son extrême surprise,
et de ce peu qui s'était passé entre la reine et elle. Réciproquement
les deux officiers des gardes, accoutumés comme toute l'Espagne à la
craindre et à la respecter plus que leur roi, lui répondirent ce qu'ils
purent du fond de cet abîme d'étonnement dont ils n'étaient pas encore
revenus. Bientôt il fallut atteler et partir. Bientôt aussi la princesse
des Ursins trouva que le secours qu'elle espérait du roi d'Espagne
tardait bien à lui arriver. Ni repos, ni vivres, ni de quoi se
déshabiller jusqu'à Saint-Jean de Luz. À mesure qu'elle s'éloignait, que
le temps coulait, qu'il ne lui venait point de nouvelle, elle comprit
qu'elle n'avait plus d'espérances à former. On peut juger quelle rage
succéda dans une femme aussi ambitieuse, aussi accoutumée à régner
publiquement, aussi rapidement et indignement précipitée du faîte de la
toute puissance par la main qu'elle avait elle-même choisie pour être le
plus solide appui de la continuation et de la durée de toute sa
grandeur. La reine n'avait point répondu aux deux dernières lettres que
M\textsuperscript{me} des Ursins lui avait écrites\,; cette négligence
affectée lui avait dû être de mauvais augure, mais qui aurait pu
imaginer un traitement aussi étrange et aussi inouï\,?

Ses neveux, Lanti et Chalais, qui eurent permission de l'aller joindre,
achevèrent de l'accabler. Elle fut fidèle à elle-même. Il ne lui échappa
ni larmes, ni regrets, ni reproches, ni la plus légère faiblesse\,; pas
une plainte, même du froid excessif, du dénûment entier de toutes sortes
de besoins, des fatigues extrêmes d'un pareil voyage. Les deux officiers
qui la gardaient à vue n'en sortaient point d'admiration. Enfin elle
trouva la fin de ses maux corporels et de sa garde à vue à Saint-Jean de
Luz, où elle arriva le 14 janvier, et où elle trouva enfin un lit, et
d'emprunt de quoi se déshabiller, et se coucher, et manger. Là elle
recouvra sa liberté. Les gardes, leurs officiers et le carrosse qui
l'avait amenée s'en retournèrent\,; elle demeura avec sa femme de
chambre et ses neveux. Elle eut loisir de penser à ce qu'elle pouvait
attendre de Versailles. Malgré la folie de sa souveraineté si longuement
poussée, et sa hardiesse d'avoir fait le mariage du roi d'Espagne sans
la participation du roi, elle se flatta de trouver encore des ressources
dans une cour qu'elle avait si longuement domptée. Ce fut de Saint-Jean
de Luz qu'elle dépêcha un courrier chargé de lettres pour le roi, pour
M\textsuperscript{me} de Maintenon, pour ses amis. Elle y rendit
brièvement compte du coup de foudre qu'elle venait d'essuyer, et
demandait la permission de venir à la cour pour y rendre compte plus en
détail. Elle attendit le retour de son courrier en ce premier lieu de
liberté et de repos, qui par lui-même est fort agréable. Mais ce premier
courrier parti, elle le fit suivre par Lanti chargé de lettres écrites
moins à la hâte et d'instructions, qui vit le roi dans son cabinet à
Versailles le dernier janvier, avec lequel il ne demeura que quelques
moments. On sut par lui que, dès que M\textsuperscript{me} des Ursins
eut dépêché son premier courrier, elle avait envoyé à Bayonne faire des
compliments à la reine douairière d'Espagne, qui ne voulut pas les
recevoir. Que de cruelles mortifications à la chute du trône\,! Revenons
maintenant à Guadalaxara.

L'officier des gardes que la reine y dépêcha avec une lettre pour le roi
d'Espagne, dès que la princesse des Ursins fut hors de Quadraqué, trouva
le roi qui s'allait bientôt coucher. Il parut ému, fit une courte
réponse à la reine, et ne donna aucun ordre. L'officier repartit
sur-le-champ. Le singulier est que le secret fut si bien gardé qu'il ne
transpira que le lendemain sur les dix heures du matin. On peut penser
quelle émotion saisit toute la cour, et les divers mouvements de tout ce
qui se trouva à Guadalaxara. Personne toutefois n'osa parler au roi, et
on était en grande attente de ce que contenait sa réponse à la reine. La
matinée achevant de s'écouler sans qu'on ouït parler de rien, on
commença à se persuader que c'en était fait de M\textsuperscript{me} des
Ursins pour l'Espagne. Chalais et Lanti se hasardèrent de demander au
roi la permission de l'aller trouver, et de l'accompagner dans l'abandon
où elle était\,; non seulement il le leur permit, mais il les chargea
d'une lettre de simple honnêteté par laquelle il lui manda qu'il était
bien fâché de ce qui s'était passé, qu'il n'avait pu opposer son
autorité a la volonté de la reine, qu'il lui conservait ses pensions et
qu'il aurait soin de les lui faire payer. Il tint parole, et tant
qu'elle a vécu depuis elle les a très exactement touchées.

La reine arriva l'après-midi de la veille de Noël, à l'heure marquée, à
Guadalaxara, comme s'il ne se fût rien passé. Le roi de même la reçut à
l'escalier, lui donna la main, et tout de suite la mena à la chapelle,
où le mariage fut aussitôt célébré de nouveau, car en Espagne la coutume
est de marier l'après-dînée\,; de là dans sa chambre, où sur-le-champ
ils se mirent au lit, avant six heures du soir pour se lever pour la
messe de minuit. Ce qui se passa entre eux sur l'événement de la veille
fut entièrement ignoré. Il n'y en eut pas plus d'éclaircissements dans
la suite. Le lendemain, jour de Noël, le roi déclara qu'il n'y aurait
aucun changement dans la maison de la reine, toute composée par
M\textsuperscript{me} des Ursins, ce qui remit un peu le calme dans les
esprits. Le lendemain de Noël, le roi et la reine seuls ensemble dans un
carrosse, et suivis de toute la cour, prirent le chemin de Madrid, où il
ne fut pas plus question de la princesse des Ursins que si jamais le roi
d'Espagne ne l'eût connue. Le roi son grand-père ne marqua pas la plus
légère surprise à la nouvelle que lui en apporta un courrier que le duc
de Saint-Aignan lui dépêcha de Quadraqué même, dont toute la cour fut
remplie d'émotion et d'effroi, après l'y avoir vue si triomphante.

Rassemblons maintenant quelques traits qui aideront à percer ces
ténèbres\,: ce mot échappé du roi à Torcy, qu'il ne put entendre, qu'il
rendit à Castries, son ami, et chevalier d'honneur de
M\textsuperscript{me} la duchesse d'Orléans, par qui nous le sûmes, et
que dans son mystère je jugeai qu'il s'agissait de la princesse des
Ursins et d'une disgrâce\,; une querelle d'Allemand, sans raison
apparente, sans cause, sans prétexte, faite au premier instant du
tête-à-tête par la reine à la princesse des Ursins, et subitement
poussée au delà des dernières extrémités. Peut-on penser qu'une fille de
Parme, élevée dans un grenier par une mère impérieuse, eût osé prendre
d'elle-même une hardiesse de cette nature, inouïe à l'égard d'une
personne de cette considération à tous égards, dans la confiance entière
du roi d'Espagne et régnant à découvert, à six lieues du roi d'Espagne,
qu'elle n'avait pas encore vu\,? La chose s'éclaircit par l'ordre si
fort inusité et si secret qu'Amenzaga avait du roi d'Espagne d'obéir en
tout à la reine sans réserve et sans réplique, et qu'on ne sut qu'à
l'instant de l'ordre qu'elle lui donna de l'arrêter et de la faire
partir.

{[}Remarquons enfin{]} la tranquillité avec laquelle le roi et le roi
d'Espagne, chacun de son côté, reçurent le premier avis de cet
événement, et l'inaction du roi d'Espagne, la froideur de sa lettre à
M\textsuperscript{me} des Ursins, et sa parfaite incurie de ce qu'une
personne, si chérie encore la veille, pouvait devenir jour et nuit par
des chemins pleins de glace et de neige, dénuée de tout sans exception.
Il faut se souvenir que l'autre fois que le roi fit chasser la princesse
des Ursins, pour l'ouverture de la lettre de l'abbé d'Estrées au roi, et
{[}pour{]} la note qu'elle avait remise dessus, on n'osa hasarder
l'exécution en présence du roi d'Espagne. Le roi voulut exprès qu'il
partît pour la frontière du Portugal, et que de là il signât l'ordre qui
fut porté à la princesse des Ursins de partir et de se retirer en
Italie. Ce second tome ressemble fort en cela au premier. Ajoutons, ce
que j'ai su du maréchal de Brancas, que, longtemps après cette dernière
disgrâce, Albéroni, alors petit compagnon, et qui suivit la reine de
Parme à Madrid, avait conté qu'étant pendant ce voyage seul un soir avec
elle, elle lui parut agitée, se promenant à grands pas dans la chambre,
prononçant de fois à autre des mots entrecoupés, puis s'échauffant, il
entendit le nom de M\textsuperscript{me} des Ursins lui échapper, et
tout de suite\,: «\,Je la chasserai d'abord.\,» Il s'écria à la reine et
voulut lui représenter le danger, la folie, l'inutilité de l'entreprise,
dont il était tout hors de lui. «\,Taisez-vous sur toutes choses,\,» lui
dit la reine, «\,et que ce que vous avez entendu ne vous échappe jamais.
Ne me parlez point, je sais bien ce que je fais.\,» Tout cela ensemble
jette une grande lumière sur une catastrophe également étonnante en la
chose et en la manière, et fait bien voir le roi auteur, le roi
d'Espagne consentant et contribuant par l'ordre si extraordinaire donné
à Amenzaga, et la reine actrice et chargée de l'exécution, en quelque
sorte que ce fût, par les deux rois. La suite en France confirmera cette
opinion.

La chute de la princesse des Ursins fit de grands changements en
Espagne. La comtesse d'Altamire fut nommée en sa place camarera-mayor.
C'était une des plus grandes dames d'Espagne. Elle était d'elle duchesse
héritière de Cardone. Son mari était mort il y avait quelques années,
ayant passé par les plus grands emplois et par l'ambassade de Rome.
J'aurai lieu de parler d'elle ailleurs, de ses enfants, de leurs
alliances. Cellamare, neveu du cardinal del Giudice, fut nommé son grand
écuyer\,; et le cardinal del Giudice ne tarda pas à retourner à Madrid,
et en considération. Par une suite naturelle, Macañas fut disgracié\,;
lui et Orry eurent ordre de sortir d'Espagne, ce dernier sans voir le
roi, avec la malédiction publique. Il fut très mal reçu ici\,; mais ses
provisions étaient bien faites. Macañas emporta les regrets de tout le
monde, ceux du roi même, qui lui continua ses pensions et sa confiance,
et s'en servit au dehors en plusieurs choses et affaires secrètes.
Pompadour, qui n'avait été nommé ambassadeur en Espagne que pour amuser
M\textsuperscript{me} des Ursins, fut remercié\,; et le duc de
Saint-Aignan revêtu de ce caractère, comme il pensait à s'en revenir
après avoir conduit la reine à Madrid.

Cette princesse n'oublia rien pour plaire au roi son mari, et y réussit
au delà de ses espérances. Elle aimait fort les Italiens, et les avança
toujours tant qu'elle put, quels qu'ils fussent, au préjudice de tous
autres, dont les Espagnols et les Flamands furent fort jaloux. Ce crayon
léger suffira pour le présent. Le roi d'Espagne fit en ce temps-ci une
action qui fut extrêmement applaudie. Un simple curé s'était tellement
accrédité par sa vie et sa conduite, qu'il se trouva en état de rendre
des services très considérables dans les temps les plus calamiteux. Il
fit fournir la nourriture à la cavalerie et aux troupes par le pays, et
beaucoup de soldats. Il procura aussi des dons en argent, et sans s'être
jamais montré ni approché de la cour, ni {[}avoir{]} changé rien en la
simplicité de sa vie. Tolède vaquait depuis assez longtemps\,; c'était
l'objet des plus ardents désirs du cardinal del Giudice, et des manèges
du duc de Giovenazzo, son frère, qui était conseiller d'État. Le curé
fut choisi\,; et quand sa nomination fut partie pour Rome, le cardinal
del Giudice eut permission de revenir à la cour. La duchesse d'Aveiro
mourut en même temps à Madrid\,; elle était mère du duc d'Arcos et du
duc de Baños\,; elle avait figuré toute sa vie. On en a suffisamment
parlé ailleurs, ainsi que du marquis de Mancera, qui, à cent sept ans,
mourut aussi en même temps, et l'un et l'autre à Madrid. On a si souvent
parlé de cet illustre vieillard qu'on n'y ajoutera rien davantage.

La comtesse de Roye mourut fort âgée en Angleterre. Elle y avait perdu
son mari depuis quelques années, et elle y laissa deux filles\,: l'une
veuve sans enfants du comte de Strafford\,;-l'autre, fille et un fils
non marié. Elle était soeur de MM. les maréchaux-ducs de Duras et de
Lorges. On a vu ailleurs comment la révocation de l'édit de Nantes fit
retirer le comte et la comtesse de Roye en Danemark, les grands
établissements qu'ils y eurent, la ridicule aventure qui les leur fit
quitter pour passer en Angleterre, où ils n'en trouvèrent aucun. Elle
était très opiniâtre huguenote, et avait empêché la conversion de son
mari. M\textsuperscript{me} de Pontchartrain, le comte de Roucy-Blansac,
le chevalier de Roye et le marquis de Roye étaient aussi ses enfants,
demeurés en France.

Une autre sœur de ces deux maréchaux et de la comtesse de Roye avait
épousé M. de Malause, des bâtards de Bourbon. Le calvinisme et le peu de
dot avaient fait ce mariage. Il en avait eu un fils qui laissa plusieurs
enfants, entre autres une fille élevée à Paris à la Ville-l'Évêque. Nous
avions tous grande envie de la marier\,; M. et M\textsuperscript{me} de
Lauzun en prirent assez de soin. Sa mère était morte\,; et la veuve de
son père était fort extraordinaire, et ne sortait point de ses terres de
Languedoc. Nous sûmes que le comte de Poitiers était arrivé à Paris pour
faire ses exercices. Il était de la branche de Saint-Vallier, de cette
grande et illustre maison, et il était le seul mâle de cet ancien nom.
Son père et sa mère étaient morts\,; il avait dix-huit ou dix-neuf ans,
et de grandes terres en Franche-Comté. Il désirait une alliance, un
appui, et les moyens d'avoir des emplois de guerre et de cheminer\,; il
trouva ce qu'il désirait dans la plus proche parenté de
M\textsuperscript{lle} de Malause\,; et nous un grand seigneur dont le
nom était pour aller à tout, les biens pour le soutenir grandement, et
le personnel à souhait. Il n'y eut donc pas grande difficulté en ce
mariage, qui se fit à l'hôtel de Lauzun.

Torcy maria une de ses filles à d'Ancezune, fils de Caderousse et de
M\textsuperscript{lle} d'Oraison, et petit-fils du vieux Caderousse\,;
leur nom est Cadart, leur bien au comtat d'Avignon. Le vieux Caderousse
s'était ruiné à ne rien faire, son fils et sa belle-fille avaient achevé
à jouer. La paresse du fils l'avait enterré de bonne heure. Son père
avait fait l'esprit et l'important, puis le dévot. Il avait primé où il
avait pu, fort à l'hôtel de Bouillon, et avait fort été autrefois dans
les bonnes compagnies. Il y avait encore à glaner en mettant quelque
ordre à leurs biens. Ils voulaient pousser d'Ancezune, et se trouvaient
sans crédit\,; Torcy voulait donner peu à sa fille, et le mariage se
fit. Par l'événement, d'Ancezune se trouva aussi obscur et aussi
paresseux que son père, impuissant de plus, et quitta bientôt le service
sans avoir presque servi ni paru à la cour. Il se jeta à Sceaux, où il
fut un des inutiles tenants de M\textsuperscript{me} du Maine aussi bien
que son père. Ils avaient pourtant tous de l'esprit et fort orné, mais
la paresse les écrasa. Le fils avait fait une campagne aide de camp du
maréchal de Boufflers. Excédé de cette vie, on le vint éveiller un matin
à cinq heures, et lui dire que le maréchal était déjà à cheval\,:
«\,Comment, dit-il, à cheval, et je n'y suis pas\,! tire mon rideau, je
ne suis pas digne de voir le jour\,;» et se rendormit de plus belle. Le
père était duc du pape, ce qui est moins que rien, nul rang ni
distinction à Rome, ni nulle autre part qu'à Avignon, où ils ont
quelques distinctions chez le vice-légat, ce à quoi elles se bornent
toutes. M\textsuperscript{me} de Torcy ne voulut jamais faire casser le
mariage pour impuissance, car cela lui fut proposé.
M\textsuperscript{me} d'Ancezune, fort laide et avec beaucoup d'esprit,
de grâces, d'intrigue, de manège, d'agaceries, eut un moment le don de
plaire. Elle crut après devoir se jeter dans la plus haute dévotion\,;
l'ennui l'en tira bientôt, et le goût de l'intrigue la fit frapper à
bien des portes. Son père enfin l'arrêta, et sa santé après eut de quoi
l'occuper, sans changer son goût ni ses grâces.

Lassai avait une fille de la bâtarde de M. le Prince qu'il avait
épousée, et dont la tête était fort égarée. Il la maria au fils d'O\,;
c'était la faim et la soif. M\textsuperscript{me} la Princesse fit leur
noce chez elle.

Le marquis d'Arpajon, lieutenant général et chevalier de la Toison d'or,
épousa en même temps une fille de Montargis, garde du trésor royal,
extrêmement riche, dont la mère était fille de Mansart.

Le maréchal de Montrevel, bas et misérable courtisan, avait imaginé
d'imiter le feu maréchal-duc de La Feuillade, et de donner à Bordeaux le
vieux réchauffé de sa statue et de sa place des Victoires. Il vivait
d'industrie, toujours aux dépens d'autrui, comme il avait fait toute sa
vie. Il voulut donc engager la ville de Bordeaux à toute la dépense de
la fonte de la statue, de son érection et de la place qu'il destinait
pour elle. La ville n'osa refuser tout à tait, mais s'y prêta mal
volontiers. Montrevel, qui en avait déjà fait sa cour au roi, se flatta
de l'appui de son autorité, mais il trouva Desmarets en son chemin, à
qui les négociants et le commerce de Bordeaux fuient plus chers que
cette folie violente. Elle avorta ainsi, et Montrevel retourna à
Bordeaux plein de dépit et chargé de confusion.

Un ambassadeur de Perse était arrivé à Charenton, défrayé depuis son
débarquement\,: le roi s'en fit une grande fête, et Pontchartrain lui en
fit fort sa cour. Il fut accusé d'avoir créé cette ambassade, en
laquelle en effet il ne parut rien de réel, et que toutes les manières
de l'ambassadeur démentirent, ainsi que sa misérable suite et la
pauvreté des présents qu'il apporta. Nulle instruction ni pouvoir du roi
de Perse, ni d'aucun de ses ministres. C'était une espèce d'intendant de
province, que le gouverneur chargea de quelques affaires particulières
de négoce, que Pontchartrain travestit en ambassadeur, et dont le roi
presque seul demeura la dupe. Il fit son entrée le jeudi 7 février à
Paris, à cheval, entre le maréchal de Matignon et le baron de Breteuil,
introducteur des ambassadeurs, avec lequel il eut souvent des
grossièretés de bas marchand\,; et tant de folles disputes sur le
cérémonial avec le maréchal de Matignon, que, dès qu'il l'eut remis à
l'hôtel des ambassadeurs extraordinaires, il le laissa là sans
l'accompagner dans sa chambre, comme c'est la règle, et s'en alla faire
ses plaintes au roi, qui l'approuva en tout, et trouva l'ambassadeur
très malappris. Sa suite fut pitoyable. Torcy le fut voir aussitôt. Il
s'excusa à lui sur la lune d'alors, qu'il prétendait lui être contraire,
de toutes les impertinences qu'il avait faites\,; et obtint par la même
raison de différer sa première audience, contre la règle qui la fixe au
surlendemain de l'entrée.

Dans ce même temps, Dippy mourut, qui était interprète du roi pour les
langues orientales. Il fallut faire venir un curé d'auprès d'Amboise,
qui avait passé plusieurs années en Perse, pour remplacer cet
interprète. Il s'en acquitta très bien, et en fut mal récompensé. Le
hasard me le fit fort connaître et entretenir. C'était un homme de bien,
sage, sensé, qui connaissoit fort les moeurs et le gouvernement de
Perse, ainsi que la langue, et qui, par tout ce qu'il vit et connut de
cet ambassadeur, auprès duquel il demeura toujours tant qu'il fut à
Paris, jugea toujours que l'ambassade était supposée, et l'ambassadeur
un marchand de fort peu de chose, fort embarrassé à soutenir son
personnage, où tout lui manquait. Le roi, à qui on la donna toujours
pour véritable, et qui fut presque le seul de sa cour qui le crut de
bonne foi, se trouva extrêmement flatté d'une ambassade de Perse sans se
l'être attirée par aucun envoi. Il en parla souvent avec complaisance,
et voulut que toute la cour fût de la dernière magnificence le jour de
l'audience, qui fut le mardi 19 février\,; lui-même en donna l'exemple,
qui fut suivi avec la plus grande profusion.

On plaça un magnifique trône, élevé de plusieurs marches, dans le bout
de la galerie, adossé au salon qui joint l'appartement de la reine, et
des gradins à divers étages de bancs des deux côtés de la galerie,
superbement ornée ainsi que tout le grand appartement. Les gradins les
plus proches du trône étaient pour les dames de la cour, les autres pour
les hommes et pour les bayeuses\footnote{Vieux mot indiquant des
  personnes qui regardent avec un air étonné. Il vient du verbe
  \emph{bayer} ( tenir la bouche béante en regardant quelque chose ).}\,;
mais on n'y laissait entrer hommes ni femmes que fort parés. Le roi
prêta une garniture de perles et de diamants au duc du Maine, et une de
pierres de couleur au comte de Toulouse. M. le duc d'Orléans avait un
habit de velours bleu, brodé en mosaïque, tout chamarré de perles et de
diamants, qui remporta le prix de la parure et du bon goût. La maison
royale, les princes et princesses du sang et les bâtards s'assemblèrent
dans le cabinet du roi.

Les cours, les toits, l'avenue, fourmillaient de monde, à quoi le roi
s'amusa fort par ses fenêtres, et y prit grand plaisir en attendant
l'ambassadeur, qui arriva sur les onze heures dans les carrosses du roi,
avec le maréchal de Matignon et le baron de Breteuil, introducteur des
ambassadeurs. Ils montèrent à cheval dans l'avenue, et précédés de la
suite de l'ambassadeur, ils vinrent mettre pied à terre dans la grande
cour, à l'appartement du colonel des gardes, par le cabinet. Cette suite
parut fort misérable en tout, et le prétendu ambassadeur fort embarrassé
et fort mal vêtu, les présents au-dessous de rien. Alors le roi,
accompagné de ce qui remplissait son cabinet, entra dans la galerie, se
fit voir aux dames des gradins\,; les derniers étaient pour les
princesses du sang. Il avait un habit d'étoffe or et noir, avec l'ordre
par-dessus, ainsi que le très peu de chevaliers qui le portaient
ordinairement dessous\,; son habit était garni des plus beaux diamants
de la couronne, il y en avait pour douze millions cinq cent mille
livres\,; il ployait sous le poids, et parut fort cassé, maigri et très
méchant visage. Il se plaça sur le trône, les princes du sang et bâtards
debout à ses côtés, qui ne se couvrirent point. On avait ménagé un petit
degré et un espace derrière le trône pour Madame et pour
M\textsuperscript{me} la duchesse de Berry qui était dans sa première
année de deuil, et pour leurs principales dames. Elles étaient là
incognito et fort peu vues, mais voyant et entendant tout. Elles
entrèrent et sortirent par l'appartement de la reine, qui n'avait pas
été ouvert depuis la mort de M\textsuperscript{me} la Dauphine. La
duchesse de Ventadour était debout à la droite du roi, tenant le roi
d'aujourd'hui par la lisière. L'électeur de Bavière était sur le second
gradin avec les dames qu'il avait amenées\,; et le comte de Lusace,
c'est-à-dire le prince électeur de Saxe, sur celui de la princesse de
Conti, fille de M. le Prince. Coypel, peintre, et Boze, secrétaire de
l'Académie des inscriptions, étaient au bas du trône, l'un pour en faire
le tableau, l'autre la relation. Pontchartrain n'avait rien oublié pour
flatter le roi, lui faire accroire que cette ambassade ramenait l'apogée
de son ancienne gloire, en un mot le jouer impudemment pour lui plaire.

Personne déjà n'en était plus la dupe que ce monarque. L'ambassadeur
arriva par le grand escalier des ambassadeurs, traversa le grand
appartement, et entra dans la galerie par le salon opposé à celui contre
lequel le trône était adossé. La splendeur du spectacle acheva de le
déconcerter. Il se fâcha une fois ou deux pendant l'audience contre son
interprète, et fit soupçonner qu'il entendait un peu le français. Au
sortir de l'audience, il fut traité à dîner par les officiers du roi,
comme on a accoutumé. Il fut ensuite saluer le roi d'aujourd'hui dans
l'appartement de la reine qu'on avait superbement orné, de là voir
Pontchartrain et Torcy, où il monta en carrosse pour retourner à Paris.
Les présents, aussi peu dignes du roi de Perse que du roi, consistèrent
en tout en cent-quatre perles fort médiocres, deux cents turquoises fort
vilaines et deux boîtes d'or pleines d'un baume qui est rare, sort d'un
rocher renfermé dans un antre, et se congèle un peu par la suite du
temps. On le dit merveilleux pour les blessures. Le roi ordonna qu'on ne
défit rien dans la galerie ni dans le grand appartement. Il avait résolu
de donner l'audience de congé dans le même lieu et avec la même
magnificence qu'il avait donné cette première audience à ce prétendu
ambassadeur. Il eut pour commissaires Torcy, Pontchartrain et Desmarets,
dont Pontchartrain se trouva fort embarrassé.

Le grand maître de Malte, persuadé que les Turcs allaient attaquer son
île, fit faire aux chevaliers les citations pour s'y rendre. Il envoya
des vaisseaux à Marseille, tant pour les passer que pour lui en apporter
force munitions de guerre et de bouche. Le grand prieur, qui faisait
toujours son séjour à Lyon, fit demander au roi la permission de venir
prendre congé de lui pour y aller. Il fut refusé de voir le roi et de
s'approcher de Paris, et eut liberté de se rendre à Malte. Le roi y
destina quatre bataillons des troupes de terre, et deux de celles de la
marine, cent canonniers, beaucoup de mineurs, le tout payé par la
Religion. L'électeur de Trèves, comme grand prieur de Castille,
{[}arma{]} deux bataillons à ses dépens\,; mais ces troupes eurent
bientôt un contre-ordre, ainsi que Renault, lieutenant général des
armées navales, que le grand maître avait obtenu du roi. Le grand prieur
qui était allé à Malte, y fut salué, en arrivant, de vingt-trois coups
de canon, et reçu par tous les grand-croix et les carrosses du grand
maître, ce que le grand prieur fit publier. Les chevaliers les plus
pressés en furent pour leur voyage, les autres furent contremandés, les
Turcs n'avaient aucun dessein sur Malte.

Le roi donna cent mille francs à Bonrepos, qu'il lui avait promis il y
avait longtemps, en considération des dépenses qu'il avait faites
pendant ses ambassades en Danemark et en Hollande.

La Chapelle, un des premiers commis de la marine, fut subitement chassé,
et sa femme\,; son emploi donné\,; lui et sa femme eurent ordre en même
temps de se retirer à Paris. C'étaient deux personnes que leurs qualités
et leurs talents avaient fort distinguées de leur état, et qui l'un et
l'autre s'étaient acquis beaucoup d'amis considérables. La Chapelle et
sa femme avaient toujours été dans la confiance du chancelier, de la
chancelière, de M. et de M\textsuperscript{me} de Pontchartrain. La
Chapelle faisait plusieurs lettres de la main de Pontchartrain qu'il
contrefaisait fort bien, et lui avait donné ainsi la réputation de bien
écrire. Pontchartrain, délivré de famille, entra en jalousie du mérite
et des amis de La Chapelle et de sa femme. Il résolut de s'en défaire\,;
et, pour y parvenir à coup sûr, de s'en faire encore un mérite. Le
jansénisme et le P. Tellier firent son affaire. Il eut le dépit que tout
ce qu'il y eut de considérable à Versailles, en hommes et en femmes,
accourut chez ces exilés, au moment que la chose fut sue, et que
personne ne se méprit a l'auteur, qui encourut de plus en plus la haine
et la malédiction publique.

L'électeur de Bavière alla, de sa petite maison de Saint-Cloud, voir la
reine de Pologne, sa belle-mère, qu'il n'avait jamais vue. Il ne coucha
point à Blois, où elle était, et s'en revint aussitôt. Il était pressé
de retourner à Compiègne faire le mariage du comte d'Albert avec
M\textsuperscript{me} de Montigny, sa maîtresse publique depuis bien des
années. Elle était des bâtards de Brabant, sœur du feu prince de
Berghes, grand d'Espagne, et chevalier de la Toison d'or, gendre du duc
de Rohan-Chabot. Le comte d'Albert n'avait rien, l'électeur le faisait
subsister. Il trouvait de grands biens dans ce mariage, dont l'infamie
avait toujours été rejetée par le duc de Chevreuse avec toute
l'indignation qu'elle méritait. Sa mort leva le principal obstacle\,; il
passa sur tous les autres. Outre les solides avantages que lui fit
l'électeur, il y ajouta toute l'aisance de la vie, en le faisant son
grand écuyer, avec la permission du roi. La noce s'en fit à Compiègne,
sans aucun parent du comte d'Albert, d'où, incessamment après, tout ce
bagage, et la cour, et les équipages de l'électeur, prirent le chemin de
la Bavière. Ce prince vit le roi dans son cabinet par les derrières au
sortir du sermon, l'après-dînée du vendredi 22 mars à Versailles. Le roi
l'embrassa à diverses reprises\,; et l'électeur prit congé, et s'en
retourna à Paris, chez d'Antin, où il soupa avec M\textsuperscript{me}
la Duchesse et beaucoup de dames. Il y joua et y coucha, et partit le
lendemain matin pour retourner dans ses États.

\hypertarget{chapitre-ii.}{%
\chapter{CHAPITRE II.}\label{chapitre-ii.}}

1715

~

{\textsc{Mort à Rome du cardinal de Bouillon.}} {\textsc{- Précis de sa
vie.}} {\textsc{- Cause et genre de sa mort.}} {\textsc{- Son
caractère.}} {\textsc{- Cardinal de Bouillon méprisé et délaissé à
Rome.}} {\textsc{- Imagine pour les cardinaux la distinction de
conserver leur calotte sur leur tête, parlant au pape, lesquels lui en
donnent le démenti.}} {\textsc{- La rage l'en saisit, et il en crève.}}
{\textsc{- Personnel du cardinal de Bouillon.}} {\textsc{- Belle et
singulière retraite du cardinal Marescotti.}} {\textsc{- Quel il fut\,;
sa mort.}} {\textsc{- Voyage du duc et de la duchesse de Savoie en
Sicile.}} {\textsc{- Conduite de ce nouveau roi dans sa famille et avec
son fils aîné.}} {\textsc{- Rare mérite de ce prince, et sa mort causée
par la jalousie et les duretés de son père.}} {\textsc{- Voysin, comme
chancelier, va prendre sa place au parlement.}} {\textsc{- Tallard,
démis à son fils, ne peut être pair.}} {\textsc{- Son fils l'est fait au
lieu de lui.}} {\textsc{- Affaires de Suisse en deux mots.}} {\textsc{-
Renouvellement très-mal à propos de l'alliance des seuls cantons
catholiques avec la France.}} {\textsc{- Changements en Espagne.}}
{\textsc{- Orry, chassé d'Espagne et de la cour en France.}} {\textsc{-
Veragua et Frigilliane chefs des conseils de marine et du commerce, et
de celui des Indes.}} {\textsc{- Cellamare ambassadeur en France.}}
{\textsc{- Chalais et Lanti ont défense de retourner en Espagne.}}
{\textsc{- Giudice chef des affaires étrangères et de justice, et
gouverneur du prince des Asturies.}} {\textsc{- P. Robinet chassé\,; P.
Daubenton confesseur du roi d'Espagne en sa place.}} {\textsc{- Leur
caractère.}} {\textsc{- Flotte et Renaut en liberté.}} {\textsc{-
Réconciliation de M. le duc d'Orléans avec le roi d'Espagne.}}
{\textsc{- Alonzo Manriquez fait duc del Arco, grand d'Espagne et grand
écuyer.}} {\textsc{- Son caractère et sa fortune.}} {\textsc{- Valouse
premier écuyer.}} {\textsc{- Montalègre sommelier du corps\,; sa
fortune\,; son caractère.}} {\textsc{- Valero vice-roi du Mexique\,; sa
fortune\,; son caractère.}} {\textsc{- Princesse des Ursins à Paris.}}
{\textsc{- Dégoûts qu'elle essuie.}} {\textsc{- Je passe huit heures de
suite tête à tête avec elle.}} {\textsc{- Court et triste voyage de la
princesse des Ursins à Versailles.}} {\textsc{- Elle obtient quarante
mille livres de rente sur la ville, au lieu de sa pension de vingt mille
livres.}}

~

Le cardinal de Bouillon mourut à Rome le 7 mars de cette année, a
soixante et onze ans et six mois\,; il y fit une fin digne de sa vie.
Quoiqu'on ait souvent parlé de lui en ces Mémoires, la singularité de ce
personnage si étrange mérite au moins un court abrégé par dates. Il
était né à Turenne, le 24 août 1643 \footnote{Voy. notes à la fin du
  volume.}, dans l'apogée de sa plus proche famille. On a vu par quel
art le roi se crut quitte à bon marché de lui donner sa nomination, qui
le fit cardinal, 5 août 1669. Il n'avait pas vingt-six ans faits. En
décembre 1671, à vingt-huit ans et quelques mois, il fut grand aumônier
de France par la mort du cardinal Antoine Barberin, et eut rapidement
les abbayes de Cluny, Saint-Ouen de Rouen, Saint-Waast d'Arras,
Saint-Martin de Pontoise, Saint-Pierre de Beaujeu, Tournus et Vigogne.
Il se trouva aux conclaves où furent élus Clément X, Innocent XI,
Alexandre VIII, Innocent XII et Clément XI qu'il sacra évêque avant son
couronnement. Il ouvrit la porte sainte à Rome pour le grand jubilé de
1700, par l'indisposition du pape et celle du doyen du sacré collège
dont il était sous-doyen, et dont sa vanité fit faire des tableaux. Il
devint doyen et évêque d'Ostie et de Velletri par la mort du cardinal
Cibo, de la manière qui a été rapportée. Il fut aussi grand doyen de
Liège et prévôt de Strasbourg, et songea toujours à se revêtir ou ses
neveux, de ces deux évêchés. Le premier lui coûta un exil avec la
déclaration formelle du roi contre lui, l'autre le précipita dans
l'abîme d'où il ne put sortir.

L'éclat de M. de Turenne, son oncle, le mit fort avant dans la faveur du
roi. La brouillerie ouverte de ce fameux capitaine avec le puissant
Louvois lui ouvrit la confiance du roi, parce que M. de Turenne obtint
que tout ce qu'il écrirait au roi de l'armée, et ce que le roi lui
écrirait aussi ne passerait point par Louvois, mais uniquement par le
cardinal de Bouillon. Louvois ne voyait pas moins les lettres de M. de
Turenne, et n'était guère moins maître des ordres et des réponses du roi
à M. de Turenne\,; mais, comme il était censé ignorer les unes et les
autres, c'était au roi que ce général écrivait au lieu du secrétaire
d'État, et le roi, au lieu du secrétaire d'État, qui lui faisait
réponse, ou qui directement lui envoyait ses ordres. Cela faisait donc
un commerce continuel entre le roi et le cardinal de Bouillon, à qui,
pour abréger des écritures, le roi disait mille choses et mille détails
de bouche pour les mander de sa part à son oncle, cela qui initiait
d'autant plus le cardinal de Bouillon dans les affaires que M. de
Turenne se mêlait aussi assez souvent de projets, de négociations et de
commerces secrets, du su du roi, qui, pendant qu'il était sur la
frontière ou à l'armée, passaient tous par le cardinal de Bouillon. La
présence de M. de Turenne à la cour l'y rehaussait encore, et sa mort
même fut une occasion d'entrer de plus en plus avec le roi, d'en être
mieux traité, par la commune douleur, et {[}d'obtenir{]} un surcroît de
grandeur par la majesté de ses obsèques, où néanmoins le roi défendit
tout titre ou toute qualité de prince. Le duc de Bouillon et le comte
d'Auvergne, ses frères, étaient\,: l'un, grand chambellan et gouverneur
d'Auvergne\,; l'autre avait succédé à M. de Turenne au gouvernement de
Limousin, et à la charge de colonel général de la cavalerie. Ses deux
sœurs avaient épousé\,: l'une, le duc d'Elbœuf\,; l'autre, un frère de
l'électeur de Bavière, oncle de M\textsuperscript{me} la Dauphine.
M\textsuperscript{me} de Bouillon, avec des sœurs et des cousines
germaines si prodigieusement établies, vivait en reine de Paris\,; et la
comtesse d'Auvergne avait presque des États en Hollande.

Le cardinal de Bouillon vivait dans la plus brillante et la plus
magnifique splendeur. La considération, les distinctions, la faveur la
plus marquée éclataient en tout\,; il se permettait toutes choses, et le
roi souffrait tout d'un cardinal. Nul homme si heureux pour ce monde,
s'il avait bien voulu se contenter d'un bonheur aussi accompli\,; mais
il l'était trop pour pouvoir monter plus haut, et le cardinal de
Bouillon, accoutumé par le rang accordé à sa maison aux usurpations et
aux chimères, croyait reculer quand il n'avançait pas. Ses diverses
tentatives déplurent. Il prétendit, au mariage de M\textsuperscript{me}
la Duchesse, manger avec le roi à la noce\,; il y échoua avec
l'indignation du roi qui le chassa, et qui bientôt après l'empêcha
publiquement d'être élu évêque de Liège. Il se raccrocha, se remit mieux
que jamais, et fut souvent chargé des affaires du roi à Rome, et de son
secret aux conclaves. On a vu les liaisons qui le firent retourner à
Rome en 1697, et obtenir en même temps la coadjutorerie de Cluny pour
son neveu l'abbé d'Auvergne. On a vu la hardiesse et la duplicité avec
laquelle il trompa le pape et le roi, pour faire ce même neveu cardinal,
et combien sa plus que fourberie fut reconnue à Versailles et au
Vatican. On a vu le personnage qu'il fit dans l'affaire et dans la
condamnation du livre de Fénelon, archevêque de Cambrai, ce qui commença
sa disgrâce\,; et la fureur avec laquelle il se conduisit sur la
coadjutorerie de Strasbourg en 1700, qui la combla. La désobéissance
formelle à ses rappels réitérés en France lui coûta sa charge, dont il
fut privé, et la saisie de tous ses revenus. Il voulut être doyen du
sacré collège. Il subit, après y être parvenu, son exil à Cluny, à la
fin de 1700. Pendant dix ans., il n'est souplesse ni bassesse qu'il ne
tentât, comme on l'a vu, ni misère d'orgueil qu'il ne montrât sans
cesse. Il s'occupa à lutter contre les moines de Cluny. Il y essuya les
plus grands dégoûts et quelquefois les affronts. Le désespoir qu'il
conçut d'une situation si différente de celle qui avait achevé de le
gâter et de le perdre lui fit prendre le parti de l'évasion, et enfanta
cette lettre également folle, ingrate, insolente et criminelle, qu'il
écrivit au roi. La mort de son neveu, déserteur en Hollande, le dégoût
de ses hauteurs, l'orgueilleux dérangement de ses manières, tournèrent
bientôt en mépris le grand accueil qu'il avait reçu aux Pays-Bas. Son
procédé avec la duchesse d'Aremberg, et l'indigne mariage de sa fille,
veuve de son neveu, qu'il fit pour devenir maître des biens des enfants
qu'il avait laissés, la conviction juridique et publique de cette
infamie, celle du procès qu'il perdit là-dessus contre la duchesse
d'Aremberg, achevèrent de le déshonorer, et de lui rendre le séjour des
Pays-Bas insupportable. Il n'avait plus que Rome où pouvoir aller. Il
sentit, par l'expérience qu'il en avait déjà faite, tout le poids de ses
différentes situations sur ce grand théâtre. Il y alla donc le plus
lentement qu'il put, et y arriva vers Pâques de 1712.

Le mépris et l'embarras de l'y voir l'y avaient devancé. Il espéra en
vain des égards, que le pape ne lui put refuser pour la part qu'il avait
eue à son exaltation, et pour avoir été sacré de sa main. Il attendit
des retours de son crédit et de sa magnificence passée\,; il se flatta
de retrouver dès amis de son ancienne splendeur, et des généreux touchés
de sa fortune présente, enfin il compta sur la grandeur de la place de
doyen du sacré collège, qu'il se promettait de bien faire valoir. Saisi
dans tous ses revenus, il ne jouissait que d'Ostie. Il avait eu soin de
beaucoup épargner et amasser pendant son exil, mais ces sommes, quelque
considérables qu'elles fussent, il n'y toucha qu'à regret et le moins
qu'il put. Il se mit donc au noviciat des jésuites, ses inaltérables
amis de tous les temps, et il y vécut en cardinal pauvre. Tout ce qui
n'était pas brouillé sans mesure avec le roi n'osa le voir, ni avoir
secrètement avec lui aucun commerce. L'échange de Sedan non consommé
jusqu'à cette heure, et le rang de sa maison, l'un et l'autre si aisé à
détruire, lui furent une cruelle bride qui le retint de se livrer
publiquement aux ennemis de la France, qui même le méprisèrent trop pour
le rechercher. Il fut donc sans crédit à Rome, n'y eut que la
considération d'écorce qui ne se put refuser au doyen des cardinaux,
avec les accès au pape, que cette place, et ce qu'on a vu de personnel
entre eux, lui avaient acquis, mais sans aucune estime. On peut juger ce
qu'un homme si prodigieusement et en même temps si bassement superbe,
aussi touché du petit comme du grand, dut souffrir d'un contraste si
accablant sur ce premier théâtre de l'univers, où il se trouvait si
honteusement en spectacle. Parmi ces tourments, et dans la première
place à Rome après le pape, cet orgueilleux imagina d'introduire une
distinction nouvelle.

C'est la coutume en Italie parmi les ecclésiastiques d'ôter sa calotte
en parlant à un beaucoup plus grand que soi, et les cardinaux ont
toujours les leurs à la main lorsqu'ils parlent au pape. Le cardinal de
Bouillon trouva qu'il serait d'une grande distinction pour les cardinaux
de conserver seuls leur calotte sur leur tête en parlant au pape. Il lui
en parla\,; le pape sourit et ne voulut pas le refuser\,; mais il y mit
que cela ne se ferait que de concert et avec le consentement de tous les
cardinaux. Bouillon en parla aux plus considérables, mais en petit
nombre, jugeant des autres par lui-même, persuadé qu'ils seraient tous
ravis de cette distinction, de l'invention de laquelle ils lui sauraient
le meilleur gré du monde. Ceux à qui il en parla lui répondirent
ambigument\,; ils ne voulurent ni s'engager à cette fantaisie ni prêter
le collet au cardinal de Bouillon, qui plein de son idée crut les avoir
persuadés, et qu'ils persuaderaient les autres. Incontinent après, il y
eut un consistoire indiqué. Le pape y est au haut bout seul, assis dans
un fauteuil, les cardinaux sur des bancs des deux côtés\,; et, après que
ce qui se doit passer en consistoire est achevé, le doyen des cardinaux
se lève et va parler au pape, et après lui tous les cardinaux qui
veulent lui dire quelque chose. Les matières finies, le cardinal de
Bouillon alla le premier parler au pape, ayant sa calotte sur la tête.
Dès qu'on s'en aperçut voilà un murmure général qui s'éleva jusqu'à
l'interrompre. Il retourna assez embarrassé à sa place, mais il le fut
bien davantage lorsqu'il vit aller les autres cardinaux au pape, et tous
la calotte à la main. Il ne put, malgré son trouble, s'empêcher de faire
signe à ceux à qui il avait parlé de mettre leur calotte sur leur
tête\,; ce fut sans succès auprès de chacun. Il frémissait de sa place
et le montrait\,; il n'y gagna que la honte, et il sortit du consistoire
plein de dépit et de confusion. Ce fut bien pis lorsqu'il apprit que le
sacré collège se voulait plaindre au pape d'une innovation, qu'un
particulier, quoique doyen, n'était pas en droit de faire, et d'en
demander justice et réparation. Le pape, à la vérité, détourna cet orage
par son autorité en faveur de Bouillon, mais il le blâma fort d'avoir
hasardé la chose sans en être convenu avec tous les cardinaux, comme il
le lui avait prescrit. Le bruit n'en continua que plus fort parmi le
sacré collège qui élit le pape, qui est si intéressé en sa grandeur, qui
tient de lui toute la sienne, et qui n'en connait point à ses dépens.
Bien loin de se trouver flatté de cette imagination de Bouillon, dont
l'orgueil et les chimères lui étaient toujours suspects, et qui avait
perdu toute considération personnelle et toute estime parmi les
cardinaux, la prélature et partout à Rome qui se moqua continuellement
de lui, qui, dans les premiers jours, avait aigri son affaire pensant la
renouer en parlant à d'autres cardinaux\,; les propos furent si
unanimes, si vifs, si peu ménagés qu'il en fut encore plus touché que de
l'affront public d'avoir échoué. Alors il ne put plus se cacher à
lui-même le mépris et l'aversion dans lesquels il était généralement
tombé, lui qui jusqu'alors s'était toujours efforcé de se persuader le
contraire. Il en tomba malade aussitôt de rage\,; et de rage il en
mourut en cinq ou six jours, chose étrange pour un homme si familiarisé
avec la rage, et qui en vivait depuis plusieurs années. Personne à Rome
ne le regretta, ni en France, si ce n'est peut-être les Bouillon. Le roi
le méprisa au point de ne pas même nommer son nom.

Le cardinal de Bouillon était un homme fort maigre, brun, de grandeur
ordinaire, de taille aisée et bien prise. Son visage n'aurait eu rien de
marqué s'il avait eu les yeux comme un autre\,; mais, outre qu'ils
étaient fort près du nez, ils le regardaient tous deux à la fois jusqu'à
faire croire qu'ils s'y voulaient joindre. Cette loucherie qui était
continuelle faisait peur, et lui donnait une physionomie hideuse. Il
portait des habits gris, doublés de rouge, avec des boutons d'or
d'orfèvrerie à pointe, d'assez beaux diamants\,; jamais vêtu comme un
autre, et toujours d'invention, pour se donner une distinction. Il avait
de l'esprit mais confus, savait peu, fort l'air et les manières du grand
monde, ouvert, accueillant, poli d'ordinaire, mais tout cela était mêlé
de tant d'air de supériorité qu'on était blessé même de ses politesses.
On n'était pas moins importuné de son infatigable attention au rang
qu'il prétendait jusqu'à la minutie, à primer dans la conversation, à la
ramener toujours à soi ou aux siens avec la plus dégoûtante vanité. Les
besoins le rendaient souple jusqu'au plus bas valetage. Il n'avait
d'amis que pour les dominer et se les sacrifier. Vendu corps et âme aux
jésuites, et eux réciproquement à lui, il trouva en eux mille
importantes ressources dans les divers états de sa vie, jusqu'à des
instruments de ses félonies. Sa vie en aucun temps n'eut
d'ecclésiastique et de chrétien que ce qui servait à sa vanité.

Son luxe fut continuel et prodigieux en tout\,; son faste le plus
recherché, et le plus industrieux pour établir et jouir de toute la
grandeur qu'il imaginait. Ses mœurs étaient infâmes, il ne s'en cachait
pas\,; et le roi, qui abhorra toujours ce vice jusque dans son propre
frère, le souffrit dans M. de Vendôme et dans le cardinal de Bouillon,
non seulement sans peine, mais il en fit longtemps ses favoris. Peu
d'hommes distingués se sont déshonorés aussi complètement que celui-là,
et sur autant de chapitres les plus importants. Ses débauches, son
ingratitude, ses félonies\,; la fabrication du cartulaire de Brioude
pour se faire descendre des ducs d'Aquitaine, juridiquement prouvée,
condamnée, lacérée, le faussaire condamné sur son propre aveu, les
Bouillon forcés d'avouer tout au roi et aux juges, et le cardinal de
Bouillon prouvé et avoué l'inventeur et celui qui avait mis de Bar en
besogne de cette fabrication, de concert avec son frère et ses neveux\,;
le trait de double tromperie, lui chargé des affaires du roi à Rome,
pour duper le roi et le pape l'un par l'autre pour faire l'abbé
d'Auvergne cardinal\,; le spectacle de désobéissance donné à Rome\,; sa
prétention de n'en devoir point au roi\,; la folie de sa lettre en
s'évadant\,; l'infamie et la cause plus infâme encore du mariage qu'il
fit de sa nièce avec Mesy, plaidée et prouvée juridiquement aux
Pays-Bas\,; toutes les misères qui précédèrent sa fuite\,; l'audace de
se faire élire abbé de Saint-Amand par avarice, contre les bulles du
pape, sur la nomination du roi\,; on ne finirait pas si on voulait
reprendre toutes les manières dont il s'est déshonoré, et les excès de
son ingratitude et de ses félonies, lui qui devait au roi les biens, les
charges, les dignités, le rang, les établissements de sa maison, après
ce qu'elle avait commis contre Henri IV qui le premier l'avait élevée,
Louis XIII et Louis XIV dans sa minorité, et qui lui-même ne fut doyen
des cardinaux, en désobéissant avec tant d'éclat, que par avoir été
cardinal à vingt-six ans de la nomination du roi. Il eut en mourant la
vanité de nommer six cardinaux pour ses exécuteurs testamentaires, lui
qui ne pouvait disposer de rien en France, et qui n'avait que ce qu'il
avait porté d'argent, de pierreries et d'argenterie à Rome. On peut dire
de lui qu'il ne put être surpassé en orgueil que par Lucifer, auquel il
sacrifia tout comme à sa seule divinité.

Je ne puis mieux placer la conduite d'un autre cardinal si édifiante, si
sage et si sainte, qu'en contraste avec celle du cardinal de Bouillon,
et qui par sa singularité même mérite la curiosité, parce qu'elle n'a
point eu d'exemple auparavant ni d'imitateurs après, et je ne
l'avancerai que de deux mois. Galeas Marescotti, né 1er octobre 1627,
était d'une famille de Rome, noble, ancienne, alliée à la maison des
Ursins et à d'autres fort considérables. Il fut d'abord archevêque de
Corinthe \emph{in partibus}, nonce en Pologne, après en Espagne pendant
la minorité de Charles II. Clément X le fit cardinal, 27 mai 1675, à
moins de quarante-huit ans. Il s'était acquis beaucoup de réputation de
piété et de savoir dans sa prélature, et de capacité dans ses
nonciatures\,; et il passa depuis pour un des plus hommes d'honneur et
de bien, et des plus habiles du sacré collège. Aussi y passa-t-il par
toutes les plus grandes charges qui se donnent au mérite. Il fut légat
de Ferrare, et ensuite secrétaire d'État, deux emplois dont le premier
n'a qu'un temps limité, l'autre finit avec le pape qui l'a donné. Il eut
depuis plusieurs emplois importants, entre autres celui de préfet du
saint-office, et qui l'est tellement que les papes se le sont presque
toujours réservé depuis. Il eut d'autres préfectures, la protection des
dominicains, et d'autres grands ordres, et devint en 1708 chef de
l'ordre des cardinaux-prêtres. Il avait alors plus de quatre-vingts ans,
et ne voulut point passer à sou tour d'option dans l'ordre des
cardinaux-évêques. Peu de temps après il cessa tout commerce ordinaire,
et se renferma aux fonctions indispensables.

Lorsqu'il se fut accoutumé peu à peu à cette sorte de séparation, qui
était grande pour lui, parce qu'il était extrêmement honoré, visité et
consulté, il pria le pape de disposer de ses emplois, et de le dispenser
de toute fonction de cardinal, résolu de ne plus entrer même au
conclave. Sa santé était vigoureuse, et sa tête comme à cinquante ans.
Le pape résista longtemps et céda enfin à ses instances. Il déclara en
même temps qu'il ne recevrait plus les visites des nouveaux cardinaux,
ni celles des ambassadeurs, qu'il n'en rendrait aucunes, et, pour y
couper court, il alla prendre congé du pape et le supplier de le
dispenser de plus aller à son palais. Il se renferma dans le sien, d'où
il ne sortit plus que pour aller rarement dire la messe dehors certains
jours fort solennels. Il partagea tout son temps entre la prière et les
lectures spirituelles, dans une continuelle préparation à la mort. Comme
il était fort aimé et fort honoré, et qu'il était savant, il choisit ce
qu'il y avait de plus pieux et de plus doctes religieux de tous les
ordres, et à qui leurs emplois le pouvaient permettre, pour venir tous
les jours chez lui à une heure marquée, toujours la même\,; de manière
que, depuis le moment qu'il se levait jusqu'à celui qu'il se couchait,
il n'était pas un moment seul et changeait de compagnie presque toutes
les heures. Il priait avec les uns, les autres lui faisaient des
lectures sur lesquelles ils faisaient des réflexions, enfin il y en
avait qui après son repas servaient une heure a sa récréation Parmi ces
exercices rien de faible ni de triste, mais toujours une grande présence
de Dieu et du compte qu'il se préparait à lui rendre, sans jamais rien
de vain ni de mondain. Il avait été fort aumônier toute sa vie, il le
devint encore davantage. Au mois de mai de cette année, il remit au pape
tout ce qu'il avait de bénéfices et de pensions sur des bénéfices, et ne
conserva que le revenu de son patrimoine. Il fut visité par Clément XI
plusieurs fois, et par les papes, ses successeurs, sans qu'il soit
jamais retourné en leur palais. Les cardinaux non seulement l'invitèrent
d'entrer aux conclaves qu'il y eut depuis, et l'en pressèrent
inutilement, mais quoiqu'il fût demeuré chez lui, et que les cardinaux
n'aient point de voix quand ils ne sont pas dans le conclave, il ne
laissa pas d'en être consulté plusieurs fois, et d'influer sur les
élections qui s'y firent. Benoît XIII l'alla voir aussitôt après son
exaltation, et les autres papes lui firent lé même honneur. Il ne
démentit pas d'un seul point la vie qu'il avait embrassée jusqu'à sa
mort, arrivée le 3 juillet 1726, ayant joui d'une bonne santé jusqu'à
cette dernière maladie, et de toute sa tête jusqu'à la mort qui eut
toutes les marques de celles des prédestinés. Il avait près de
quatre-vingt-dix-neuf ans, et fut regretté comme s'il n'en avait eu que
cinquante, des pauvres surtout dont il était le père, sans toutefois
avoir fait tort à sa famille. Le pape assista lui-même à ses obsèques
avec le sacré collège\,; il avait plus de cinquante ans de cardinalat.
Disons encore un mot d'Italie. Le duc de Savoie, nouveau roi de Sicile,
était allé, comme on l'a dit, en prendre possession, s'y faire
couronner, connaître le pays et les gens, et en tirer tout ce qui lui
fut possible. Il avait mené la reine sa femme, qui y fut aussi
couronnée, et laissé à Turin un conseil bien choisi, de peu de
personnes, pour gouverner en son absence. Il avait offert la régence à
la duchesse sa mère, qui le pria de l'en dispenser. Jamais il ne lui
avait pardonné de l'avoir voulu faire roi de Portugal, en épousant
l'infante, sa cousine germaine, et y allant demeurer. Il lui pardonnait
aussi peu d'être toute française, et adorée dans tous ses États et dans
sa cour. Sa jalousie avait été fort poussée, ainsi que les dégoûts qu'il
lui avait donnés. Il n'y avait entre eux qu'une sèche bienséance. Ces
raisons firent que la régence fut froidement offerte et sagement
refusée. L'épouse, aussi française que la mère, n'était pas plus
heureuse. La belle-mère et la belle-fille vécurent toute leur vie dans
la plus intime amitié et dans la confiance la plus parfaite. C'est ce
qui obligea le roi de Sicile à la mener avec lui, pour qu'elle ne fût
pas régente, et Madame Royale par elle. Il déclara régent le prince de
Piémont, son fils aîné, qui était grand et bien fait pour son âge, et
qui d'ailleurs promettait toutes choses. Il chargea le conseil qu'il
laissa de l'instruire et de lui rendre compte de tout pour le former aux
affaires, et d'essayer quelquefois avec opiniâtreté à le laisser faire
en certaines choses pour voir comment il s'y prendrait.

Le jeune prince s'appliqua et devint capable jusqu'à étonner le
conseil\,; et par la facilité de son accès, la sagesse et la justesse de
ses réponses, sa modestie, sa politesse, son désir de plaire et
d'obliger, le déplaisir qu'il montrait quand il était obligé de refuser,
et l'adoucissement qu'il y savait mettre, lui acquirent tous les coeurs.
C'en était trop pour un père jaloux, qui eût été au désespoir d'avoir un
fils sans talents pour gouverner, mais qui {[}était{]} jaloux de son
ombre, et qui avait trop de pénétration pour ne pas sentir qu'il était
redouté, mais nullement aimé dans sa cour ni dans son pays, trouvait un
fils aîné, de seize ans, trop avancé dans l'estime et dans l'affection
générale, et qui l'avait trop bien su mériter. Son accueil à son retour
et ses louanges à son fils furent fort sèches. Après le premier compte
rendu, il ne l'admit plus en aucunes affaires, et les ministres eurent
défense de lui rien communiquer. Le jeune prince sentit amèrement un
procédé si peu mérité, et le souffrit sans se plaindre ni paraître même
mécontent. Son père l'était infiniment de voir sa cour également
empressée autour du prince, et après son retour en user par amour et par
attachement pour son fils commesidéjà il eût régné. Il lui refusa donc
jusqu'aux plus petites choses pour le décréditer, et pour diminuer cette
foule et cette complaisance que tous prenaient en lui par la crainte de
déplaire et de reculer la fortune. Le prince y fut extrêmement sensible,
sans se déranger en rien de sa modestie, de ses respects et de ses
devoirs. Cependant le carnaval arriva\,; les dames qui, pendant la
régence du prince, lui avaient fait leur cour chez Madame Royale, et en
étaient fort connues, lui demandèrent un bal. Il ne crut pas déplaire en
s'engageant d'en demander la permission au roi son père. Les affaires
n'avaient aucun trait avec un bal, et ce plaisir était de son âge, de la
saison, et convenait dans une cour. Il en fit donc la demande. Le roi de
Sicile, qui le voulait décréditer et le mortifier en toutes façons, le
refusa avec la plus grande dureté\,; ce fut la dernière, après tant
d'autres, et la dernière goutte qui fit verser le verre.

Le prince ne put soutenir un traitement si barbare si peu mérité,
souffert avec tant de respect et de douceur, et auquel il n'apercevait
ni bornes ni mesures. La fièvre le prit la nuit\,; il en confia la cause
a la princesse de Carignan, sa soeur naturelle, qui me l'a conté, et à
qui il avait accoutumé de s'ouvrir uniquement sur les traitements qu'il
recevait. Il l'assura qu'il avait le coeur flétri et qu'il n'en
reviendrait pas, et avec peu de regret à la vie sous un tel père. Il ne
parla pas si librement aux médecins, mais il les assura toujours qu'il
n'en reviendrait pas\,; et avec la même douceur il se disposa à la mort,
et ne pensa plus qu'à l'autre vie. Sa maladie ne dura que cinq ou six
jours\,; les deux princesses, mère et grand'mère, la cour, la ville
étaient dans le dernier désespoir. Le malheureux père y tomba
lui-même\,; il sentit en ces derniers jours tout ce que valait son fils,
tout ce qu'il allait perdre, et ne put se dissimuler qu'il en était le
bourreau. Mais l'impression était faite\,; ses caresses tardives ne
purent rappeler le prince à la vie. Si ce père, barbarement politique,
avait pu lire dans l'avenir et voir de si loin quel traitement le fils
qui lui restait lui préparait, son désespoir eût été au comble. Il eut
la douleur de perdre un fils accompli, généralement reconnu et goûté
comme tel, d'en voir sa cour, sa ville, ses États dans la plus vive
douleur, et dans la conviction entière que sa jalousie l'avait fait
mourir. Retournons maintenant en France.

Voysin regorgeait des plus grands dons de la fortune, chancelier et
garde des sceaux, ministre et secrétaire d'État au département de la
guerre, avec plus d'autorité que Louvois, conseil intime de
M\textsuperscript{me} de Maintenon et de M. du Maine, instrument du
testament du roi et de tout ce que sa vieille et son bâtard se
proposaient encore d'en arracher, ministre unique de l'affaire de la
constitution, et dans la plus intime confiance et dépendance des chefs
de ce redoutable parti, et l'âme aussi cautérisée qu'eux, il nageait
dans la plus solide et la plus entière confiance du roi, et dans la
puissance la plus étendue. Il voulut jouir de sa gloire, et aller
triompher au parlement en qualité de chancelier de France, où son propre
grand-père paternel avait été longtemps greffier criminel, et sans être
monté plus haut crut avoir fait une fortune. Le chancelier de
Pontchartrain et bien d'autres chanceliers n'y avaient jamais été, et il
se trouvera peu ou point d'exemple qu'aucun y ait été sans occasion
nécessaire, et seulement comme celui-ci pour le plaisir et la vanité d'y
aller. Il s'y fit suivre par plus de cent officiers, et accompagner de
tout ce qui lui fut permis de conseillers d'État et de maîtres des
requêtes. Il n'oublia rien de la pompe de sa marche, de sa réception et
de sa reconduite. Son discours montra plus de fortune que de talents.
Aucun pair ni prince du sang ne s'y trouva\,; ils ne marchent point pour
la robe.

Tallard séchait sur pied de n'avoir encore rien recueilli d'avoir livré
le cardinal de Rohan au P. Tellier jusqu'à en avoir fait son esclave. La
jalousie le perçait de voir que cela même eut fait les princes de Rohan
et d'Espinoy ducs et pairs, tandis qu'on le laissait, et qu'il était
d'autant plus pressé qu'il voyait le roi diminuer tous les jours. Il ne
voulut pas en être la dupe, et fit tant de bruit aux Rohan et au P.
Tellier, qu'ils n'osèrent le pousser à bout. Vouloir et pouvoir était
même chose auprès du roi et de M\textsuperscript{me} de Maintenon pour
les maîtres de la constitution. Elle leur était trop chère et sacrée
pour se dispenser d'en payer les dettes, et elle n'en avait contracté
aucune si utile que celle que Tallard s'était acquise sur elle en lui
livrant le cardinal de Rohan. Tallard fut donc déclaré pair de France\,;
mais quand il fallut en venir à la mécanique des expéditions, la chose
fut trouvée impossible, parce qu'il n'avait qu'un duché vérifié qu'il
avait cédé à son fils en le mariant. On tourna, on chercha, mais à la
fin il fallut que le père se contentât, en enrageant, que la pairie fût
érigée pour son fils, et de demeurer lui comme il était.

Le comte du Luc, ambassadeur en Suisse, fit en ce temps-ci une faute
dont la France et la Suisse se ressentent encore. Les cantons
catholiques et protestants étaient depuis longtemps animés les uns
contre les autres\,; la longue affaire de l'abbé de Saint-Gall les avait
mis aux prises, et quelquefois aux armes. L'intérêt de la maison
d'Autriche entretint sous main ce feu pour abaisser les cantons les uns
par les autres et en profiter. Passionné, jeune, emporté, violent et
sans expérience, y était nonce du pape, et il aigrit les choses de plus
en plus. Du Luc était occupé du renouvellement de l'alliance de la
France avec tout le corps helvétique, et les ministres de la maison
d'Autriche à l'empêcher, à quoi rien n'était plus propre que
d'entretenir la division dans la république. Du Luc espéra forcer les
protestants par les catholiques, plus nombreux à la vérité, mais
incomparablement plus faibles\,; il conclut le renouvellement d'alliance
avec ces derniers. Les cantons protestants, animés par les émissaires de
Vienne, de Londres, de Hollande, imputèrent ce traité à affront et n'ont
jamais voulu ouïr parler depuis de renouveler leur alliance avec la
France, et les armes à la main s'en sont souvent vengés sur les cantons
catholiques, et leur ont durement fait sentir leur supériorité.

Pendant que la princesse des Ursins s'acheminait lentement vers Paris,
sa catastrophe produisit de grands changements en Espagne. Orry l'avait
devancée, et trouva en arrivant à Paris défense d'approcher de la
cour\,; il courut même fortune de la prison et de pis. Le cardinal del
Giudice {[}fut{]} non seulement rappelé, comme on l'a vu, mais mis à la
tête des affaires politiques, de justice et religion\,; le duc de
Veragua eut celles de la marine et du commerce\,; le vieux marquis
Frigilliane fut fait chef du conseil des Indes\,; le marquis de Bedmar
le demeura du conseil de guerre\,; et le prince de Cellamare, fils du
duc de Giovenazzo, conseiller d'État, frère du cardinal del Giudice, qui
venait, comme on l'a vu, d'être fait grand écuyer de la reine, fut nommé
ambassadeur en France. Chalais et Lanti, neveux de M\textsuperscript{me}
des Ursins, qui avaient eu, comme on l'a vu, permission de la joindre en
chemin, et qu'elle avait envoyés l'un après l'autre devant elle à Paris,
y reçurent défense de retourner en Espagne, ce qui embarrassa fort Lanti
qui était Italien et qui n'avait rien ici, et Chalais encore plus, à qui
le roi refusait la jouissance du rang et des honneurs de grand
d'Espagne, qu'il ne lui avait permis qu'à cette condition-là d'accepter.

Peu de jours après, le cardinal del Giudice fut fait gouverneur du
prince des Asturies, emploi fort étrange pour un prêtre. Dans ce rayon
de fortune, qui avait déjà, comme on l'a vu, expatrié Macañas, il
n'oublia point la générosité avec laquelle le P. Robinet avait résisté à
sa faveur, jointe alors à l'autorité de M\textsuperscript{me} des
Ursins, pour l'archevêché de Tolède que le cardinal et la princesse des
Ursins demandaient vivement, et que Giudice fut au moment d'obtenir,
lorsqu'avec l'applaudissement général de la cour de la ville, de toute
l'Espagne, le P. Robinet l'emporta, et le fit donner à cet illustre curé
de village dont j'ai parlé ailleurs. Un prêtre et un Italien n'oublient
guère. Giudice profita de sa faveur pour faire chasser Robinet, qui se
retira en France, où il vécut très content simple jésuite à Strasbourg,
sans se mêler à rien. Le P. Daubenton, lors assistant du général des
jésuites à Rome, celui-là même qui, seul avec le cardinal Fabroni, avait
concerté et fabriqué la constitution \emph{Unigenitus}, fut rappelé au
confessionnal du roi d'Espagne. Ce changement de confesseur fut un grand
et long malheur pour les deux couronnes. Robinet n'avait nul intérêt,
aucune ambition, n'était point entaché d'ultramontanisme, et n'était
jésuite qu'autant que l'honneur et sa conscience le lui permettaient. Il
était solidement homme de bien\,; aussi voulait-il le bien pour le bien,
et y était également hardi et sage. Toute la cour et toute l'Espagne
l'aimaient et l'honoraient, s'y confiaient\,; il ne s'en élevait et ne
s'en estimait pas davantage, et il était droit, vrai et ennemi de toute
intrigue. On verra ailleurs le parfait contraste de son successeur avec
lui.

Un mois après, Flotte et Renaut furent mis en liberté. La chute de
M\textsuperscript{me} des Ursins fit voir clair au roi d'Espagne sur
bien des choses. C'était elle qui avait fait arrêter ces deux
domestiques de M. le duc d'Orléans, et qui, soutenue de
M\textsuperscript{me} de Maintenon par leur haine commune, et de
Monseigneur poussé par la cabale qui le gouvernait, ne visait pas à
moins qu'à la tête de M. le duc d'Orléans, comme je l'ai raconté en son
lieu. La reine d'Espagne, qui devenait fort maîtresse, ne cherchait qu'à
détruire ce que M\textsuperscript{me} des Ursins avait édifié\,;
peut-être l'âge et la santé du roi la persuadèrent-ils tacitement de
raccommoder le roi d'Espagne avec un prince à qui on ne pouvait, le cas
arrivant, ôter la régence. Ainsi, sans que M. le duc d'Orléans y
songeât, ni personne pour lui, le roi d'Espagne écrivit au roi qu'ayant
enfin reconnu l'innocence de Flotte et de Renaut, et la fausseté des
accusations faites contre eux, il avait ordonné qu'on les mît en
liberté. Le roi d'Espagne ajouta dans la même lettre que, dans le désir
qu'il avait de se réconcilier avec M. le duc d'Orléans, il laissait au
roi d'en ordonner la manière. La surprise fut grande à la réception de
cette lettre, et la rage de M\textsuperscript{me} de Maintenon. Un
pareil désaveu, sur une affaire qu'on avait poussée si étrangement loin
auprès du roi, lui pouvait faire ouvrir les yeux sur des calomnies plus
atroces et plus domestiques. M. du Maine en trembla, et glissa sur ce
fâcheux pas avec adresse et silence. M. le duc d'Orléans écrivit au roi
d'Espagne, de concert avec le roi, et en reçut une réponse la plus
honnête. Flotte et Renaut reçurent ordre de M. le duc d'Orléans d'aller
à Madrid remercier le roi et la reine, dont ils furent bien reçus, et de
revenir aussitôt en France où ils voudraient, excepté Paris et ses
environs, pour prévenir sagement les questions et les propos qu'on se
plairait à leur faire tenir. Ils touchèrent promptement en Espagne de
quoi payer toutes les dettes qu'ils y avaient faites, et la dépense de
leur retour, par ordre de M. le duc d'Orléans, qui leur donna à leur
arrivée une gratification et une pension honnête.

Il faut achever les changements d'Espagne, d'autant que je ne les
préviens que de six semaines. Alonzo Manriquez était un homme de
qualité, et le seul pour qui le roi d'Espagne eut invariablement une
amitié constante. Il aimait aussi le roi avec attachement\,; il était
grand, de taille aisée, fort bien fait, avec un air noble et un visage
agréable, et, chose rare pour un Espagnol, il était blond et avait de
belles dents. Son esprit était médiocre, mais sage et mesuré au dernier
point\,; éloigné de se mêler d'affaires et de cabales, et tout aussi
éloigné de faire sa cour à aucun ministre, même à la princesse des
Ursins\,; d'ailleurs l'homme le plus affable, le plus poli, le plus
gracieux, de l'accès le plus facile. Son affection pour le roi d'Espagne
lui en avait donné pour les François. Il n'était pas riche, mais autant
qu'il le pouvait généreux et libéral. Dès qu'il fut grand seigneur, il
devint magnifique et conserva les mêmes moeurs. Il était fort réservé à
rendre de bons offices et à parler au roi pour quelqu'un, non que
l'inclination ne l'y portât, mais il en sentait le danger avec un prince
aussi dépendant d'autrui. C'était un des plus grands toréadors de toute
l'Espagne, et qui se consolait le moins qu'on eût banni ces combats, où
il avait fait de grandes folies avec une grande valeur. C'est lui qui
fut obligé de se retirer dans un couvent au plus vite, en attendant que
sa grâce lui fût expédiée, et qui la fut promptement, pour avoir sauté à
bas de son cheval et tiré le pied de la feue reine de son étrier, tombée
et traînée par le sien, à qui il sauva ainsi la vie. Sa femme, qui avait
beaucoup de mérite, et qui était Enriquez, et avec qui il a toujours
vécu dans la plus grande union, avait souvent des musiques chez elle, et
ils en eurent une fort bonne à eux quand ils se virent en état de
figurer. Ils voyaient beaucoup plus de monde que tous les autres
seigneurs espagnols, et bien plus librement. Alonzo Manriquez fut
majordome du roi, puis premier écuyer, qui ne ressemble en rien au
nôtre, comme on le verra ailleurs. Il quitta en ce temps-ci cette
charge, parce qu'il fut fait grand d'Espagne sous le titre de duc del
Arco, et qu'un grand d'Espagne ne peut être premier écuyer.

Valouse, gentilhomme de Provence, nourri page du roi, puis écuyer
particulier de M. le duc d'Anjou, qui l'avait suivi en Espagne, où, avec
peu d'esprit, il se gouverna toujours fort sagement, et se maintint dans
les bonnes grâces de son maître et des divers gouvernements, fut fait
premier écuyer. Le roi d'Espagne fit en même temps persuader au duc de
La Mirandole, qui était grand écuyer, de se démettre de cette charge, en
lui en conservant les honneurs et les appointements\,; il y consentit,
et le duc del Arco fut fait grand écuyer. Il était aussi gentilhomme de
la chambre et seul en exercice avec le marquis de Santa-Cruz,
majordome-major de la reine. J'aurai ailleurs occasion de parler de ces
deux seigneurs. Le duc del Arco ne ploya jamais sous Albéroni, qui ne
l'aimait pas, mais qui n'osa jamais se hasarder de l'entamer. C'était un
des plus honnêtes et des plus accomplis hommes d'Espagne, doux, modeste,
mais digne et haut aussi dans les occasions. Il montra beaucoup de
valeur dans les campagnes d'Italie et d'Espagne, qu'il fit à la suite de
son maître. Il était aussi parfaitement désintéressé avant et depuis sa
fortune. Il ne demanda jamais rien au roi pour soi\,; il avait une des
moindres commanderies de Saint-Jacques et n'en voulut point d'autres. Il
portait cet ordre à la boutonnière, comme ils font tous, et avait le
portrait du roi d'Espagne au revers de la médaille.

La charge de sommelier du corps ou de grand chambellan vaquait depuis la
mort du duc d'Albe, arrivée à Paris pendant son ambassade, en sorte
qu'il ne l'avait jamais faite. L'ancien des gentils hommes de la chambre
l'exerce dans le cas d'absence ou de vacance\,; et c'était le marquis de
Montalègre, grand d'Espagne, qui l'était, et qui avait toujours suppléé.
Il était Guzman, et avait épousé une sœur du marquis de Los Balbazès,
qui était Spinola. Il avait été une espèce de favori de Charles II, qui
lui avait donné la compagnie des hallebardiers qu'il avait encore, qui
était lors la seule garde des rois d'Espagne, avec certaine canaille de
lanciers en petit nombre et qui ne suivaient qu'à cheval, qui
demandaient l'aumône à la porte du palais. Philippe V les abolit en
arrivant en Espagne, et mit les hallebardiers sur le pied et avec
l'habillement des Cent-Suisses de la garde du roi. Ce marquis de
Montalègre était un fort honnête homme, assez borné, qui ne se mêlait de
rien\,; mais poli, honnête, généreux, et qui vivait fort retiré à
l'espagnole.

Le duc de Liñarez, vice-roi du Mexique, avait obtenu son rappel. Il
était vice-roi de l'avènement de Philippe V à la couronne, et lui avait
envoyé de grands secours d'argent. Le marquis de Valero fut envoyé à sa
place. Il était frère du duc de Bejar et oncle de Zuniga, qu'on a vu
servir dans nos armées. Le roi d'Espagne avait toujours aimé ce marquis
de Valero, il l'avait en arrivant trouvé majordome, et avait toujours
cherché à l'élever. C'était un vrai Espagnol, plein d'honneur, de
courage et de fidélité, mais austère et inflexible, et qui n'était pas
sans capacité. À son retour il fut grand d'Espagne et sommelier du corps
avec beaucoup de crédit, dont il n'abusa jamais, et s'en servit
utilement pour le roi et la monarchie. Ce fut dommage qu'il ne vécût pas
assez. Il n'eut point d'enfants, et sa grandesse retourna à des neveux.

Enfin la princesse des Ursins arriva à Paris, et vint descendre et loger
chez le duc de Noirmoutiers, son frère, dans une petite maison des
Jacobins, qu'il occupait dans la rue Saint-Dominique, porte à porte de
la mienne. Ce voyage dut lui paraître bien différent du dernier qu'elle
avait fait en France, où elle avait paru la reine de la cour. Peu de
gens, outre ses anciens amis et ceux de son ancienne cabale, la vinrent
voir, et néanmoins quelques envieux s'y mêlèrent\,; ce qui fit assez de
concours les premiers jours, après quoi les visites s'éclaircirent, et
la solitude domina dès qu'on eut vu le succès de son voyage à
Versailles, qu'on lui laissa attendre plusieurs jours. M. le duc
d'Orléans, raccommodé avec le roi d'Espagne, sentit qu'il était
solidement de son intérêt, encore plus que d'une faible vengeance, de
montrer par quelque éclat que ce n'était qu'à la haine et à l'artifice
de la princesse des Ursins qu'il devait celui de son affaire d'Espagne,
qui avait été si près de lui {[}faire{]} porter la tête sur l'échafaud.
M\textsuperscript{me} de Maintenon avec M. du Maine, et tous leurs
puissants ressorts, soutenus de l'intérêt de la cabale de Meudon,
étaient ceux qui avaient poussé à l'extrémité cette affaire, que
M\textsuperscript{me} des Ursins leur avait présentée. Mais les temps
étaient changés, Monseigneur était mort, et la cabale de Meudon
anéantie. M\textsuperscript{me} de Maintenon avait tourné le dos à
M\textsuperscript{me} des Ursins\,; ainsi M. le duc d'Orléans, libre à
l'égard de cette dernière ennemie, ne crut pas la devoir ménager. Il y
fut poussé par M\textsuperscript{me} la duchesse d'Orléans, et plus
encore par Madame, tellement qu'il pria le roi de défendre à la
princesse des Ursins de se trouver en pas un lieu, même dans Versailles,
où M\textsuperscript{me} la duchesse de Berry, Madame, M. {[}le duc{]}
et M\textsuperscript{me} la duchesse d'Orléans se pourraient rencontrer,
lesquels firent en même temps une défense étroite à toutes leurs maisons
de la voir, et demandèrent la même chose aux personnes qui leur étaient
particulièrement attachées. Cet éclat fit un grand bruit, montra à
découvert l'abandon de M\textsuperscript{me} de Maintenon,
l'inconsidération du roi, et devint un grand embarras pour la princesse
des Ursins.

Je n'avais pu trouver que M. le duc d'Orléans eût tort dans cette
conduite, qui faisait retomber à plomb sur les artifices tout ce qu'on
avait voulu lui imputer, et qui se trouvait très heureusement placée au
moment de la liberté rendue à Flotte et à Renaut, et de sa
réconciliation avec le roi d'Espagne. Mais je lui représentai qu'ayant
toujours été ami particulier de M\textsuperscript{me} des Ursins,
laissant à part sa conduite envers lui, et ne mettant point de
proportion dans mon attachement pour lui avec mon amitié pour elle, je
ne pouvais oublier les marques qu'elle m'en avait toujours données,
particulièrement en ce dernier voyage si triomphant, comme je l'ai
expliqué en son temps, et qu'il me serait dur de ne la point voir. Nous
capitulâmes donc, et M. {[}le duc{]} et M\textsuperscript{me} la
duchesse d'Orléans me permirent de la voir deux fois\,: une alors,
l'autre quand elle partirait, avec parole que je n'irais pas une
troisième, et que M\textsuperscript{me} de Saint-Simon ne la verrait
point, à cause d'eux et de M\textsuperscript{me} la duchesse de Berry,
ce que nous digérâmes mal volontiers, mais il en fallut passer par là.
Comme je voulus au moins profiter de ma bisque, je fis dire à
M\textsuperscript{me} des Ursins les entraves où je me trouvais, et que,
voulant au moins la voir à mon aise le très peu que je le pouvais, je
lui laisserais passer les premiers jours et son premier voyage à la cour
avant de lui demander audience. Mon message fut très bien reçu, elle
savait depuis longues années où j'en étais avec M. le duc d'Orléans,
elle ne fut point surprise de ces entraves, et me sut au contraire bon
gré de ce que j'avais obtenu. Quelques jours donc après qu'elle eut été
à Versailles, j'allai chez elle à deux heures après midi. Aussitôt elle
ferma sa porte sans exception, et je fus tête à tête avec elle
jusqu'après dix heures du soir.

On peut juger combien de choses passent en revue dans un aussi long
entretien. Je lui trouvai la même amitié et la même ouverture, beaucoup
de sagesse sur M. le duc d'Orléans et les siens, et de franchise sur
tout le reste. Elle me conta sa catastrophe sans jamais y mêler le roi,
ni le roi d'Espagne, duquel elle se loua toujours\,; mais sans se lâcher
sur la reine, elle me prédit ce qu'on a vu depuis. Elle ne me dissimula
rien de sa surprise, des mauvais traitements, jusqu'aux grosses injures
de propos délibéré, de son départ, de son voyage, de son état, de tout
ce qu'elle avait essuyé. Elle me parla fort naturellement aussi de son
voyage de Versailles, de sa désagréable situation à Paris, de la feue
reine, du roi d'Espagne, de diverses personnes qui de son temps y
avaient figuré dans le gouvernement et dehors, enfin des vues
incertaines et diverses d'une honnête retraite, dont le lieu était
combattu dans son esprit. Ces huit heures de conversation avec une
personne qui y fournissait tant de choses curieuses me parurent huit
moments. L'heure du souper, même tardive, nous sépara, avec mille
protestations vraies et réciproques, et un pareil regret entre elle et
M\textsuperscript{me} de Saint-Simon de ne pouvoir se voir. Elle me
promit de m'avertir de son départ à temps de passer encore une journée
ensemble.

Son voyage à Versailles se passa peu agréablement. Elle alla le matin du
mercredi 27 mars, dîner à Versailles chez la duchesse du Lude qui y
demeurait toujours. Elle y resta jusqu'à une demi-heure près de celle
que le roi devait passer chez M\textsuperscript{me} de Maintenon, où
elle alla l'attendre seule avec elle\,; elle n'y demeura guère plus en
tiers avec eux, et se retira après à la ville, chez
M\textsuperscript{me} Adam, femme d'un premier commis des affaires
étrangères, qui lui donna à souper et à coucher, et où elle fut très peu
visitée. Le lendemain elle dîna chez la duchesse de Ventadour, et s'en
retourna à Paris. Elle obtint peu après de remettre sa pension du roi,
moyennant une augmentation en rentes sur l'hôtel de ville, dont elle eut
quarante mille livres de rente. Cela était, outre l'augmentation du
double, plus solide qu'une pension, qu'elle ne doutait pas de perdre dès
que M. le duc d'Orléans en deviendrait le maître. Elle songeait à se
retirer en Hollande\,; mais les États généraux ne voulurent point d'elle
à la Haye ni à Amsterdam. Elle avait compté sur la Haye. Elle pensa
alors à Utrecht, mais elle s'en dégoûta bientôt, et tourna ses projets
sur l'Italie. Elle ne retourna plus à la cour que pour en prendre congé.
M. du Maine, en reconnaissance des grandeurs qu'elle avait procurées à
M. de Vendôme en Espagne, lui valut cette grâce pécuniaire du roi.

\hypertarget{chapitre-iii.}{%
\chapter{CHAPITRE III.}\label{chapitre-iii.}}

1715

~

{\textsc{Le comte de Lusace et les princes d'Anhalt et de Darmstadt à la
chasse avec le roi.}} {\textsc{- Bolingbroke à Paris\,; sa
catastrophe.}} {\textsc{- Stairs ambassadeur d'Angleterre à Paris\,; son
caractère.}} {\textsc{- Mariage du fils unique du comte de Matignon,
fait duc, avec la fille aînée du prince de Monaco, et ses étranges
concessions et conditions.}} {\textsc{- Cinq cent mille livres, etc.,
sur le non-complet des troupes, données au chancelier Voysin.}}
{\textsc{- Le Camus, premier président de la cour des aides, prévôt et
grand maître des cérémonies de l'ordre.}} {\textsc{- Mort de la comtesse
d'Acigné\,; du duc de Richelieu\,; de la princesse d'Harcourt\,; de
Sézanne, dont la Toison est donnée à un de ses neveux.}} {\textsc{- Mort
du docteur Burnet, évêque de Salisbury, et de l'abbé d'Estrades.}}
{\textsc{- Mariage de Castelmoron avec la fille de Fontanieu\,;
d'Heudicourt avec la fille de Surville\,; du troisième fils du duc de
Rohan avec la comtesse de Jarnac\,; de Cayeux avec la fille de
Pomponne\,; de Saint-Sulpice avec la fille du comte d'Estaing. Éclipse
de soleil.}} {\textsc{- Bout de l'an de M. le duc de Berry.}} {\textsc{-
Le roi fait quitter le grand deuil avant le temps à
M\textsuperscript{me} la duchesse de Berry, et la mène jouer dans le
salon à Marly.}} {\textsc{- Elle en obtient quatre dames pour la
suivre\,: M\textsuperscript{me}s de Coettenfao, de Brancas, de Clermont,
de Pons.}} {\textsc{- M\textsuperscript{me}s d'Armentières et de Beauvau
succèdent peu après aux deux premières.}} {\textsc{- Mort de
M\textsuperscript{me} de Coettenfao, qui me donne presque tout son bien,
que je rends sans y toucher à M\,; de Coettenfao.}} {\textsc{-
Précaution nouvelle et extraordinaire du parlement de Paris contre les
fidéicommis.}} {\textsc{- Coettenfao m'envoie furtivement pour soixante
mille livres de belle vaisselle, qu'il me force après d'accepter.}}
{\textsc{- Dernier voyage du roi à Marly.}} {\textsc{- La reine
d'Angleterre à Plombières.}} {\textsc{- Chamlay, en apoplexie, va à
Bourbon.}} {\textsc{- Effiat à Marly.}} {\textsc{- Crayon de ce
personnage.}} {\textsc{- Étrange trait de lui avec moi.}} {\textsc{-
M\textsuperscript{me} de Nassau à la Bastille.}} {\textsc{- Maladie de
M\textsuperscript{me} la duchesse d'Orléans, dont on tâche de
profiter.}} {\textsc{- Paris ouverts en Angleterre sur la mort prochaine
du roi, qui par hasard les voit dans une gazette de Hollande.}}
{\textsc{- Prince de Dombes visité par les ambassadeurs comme les
princes du sang.}} {\textsc{- Adresse là-dessus du duc du Maine.}}
{\textsc{- Il obtient la qualité et le titre de prince du sang pour lui
et sa postérité, et pour son frère, par une nouvelle et très précise
déclaration du roi, incontinent enregistrée au parlement.}} {\textsc{-
Sainte-Maure conserve les livrées et les voitures de M. le duc de
Berry.}} {\textsc{- Prince électoral de Saxe prend congé du roi dans son
cabinet à Marly.}} {\textsc{- M\textsuperscript{me} de Maintenon lui
fait les honneurs de Saint-Cyr.}} {\textsc{- Mort de Ducasse\,; sa
fortune, son caractère.}} {\textsc{- Mort de Nesmond, évêque de
Bayeux.}}

~

Le roi alla de Versailles courre le cerf dans la forêt de Marly, et y
fit donner des chevaux au comte de Lusace, c'est-à-dire le prince
électeur de Saxe, au palatin son gouverneur, et aux princes d'Anhalt et
de Darmstadt\,; et le lendemain il convia dans la galerie le comte de
Lusace à la volerie, où Sa Majesté allait.

Un autre étranger arriva en même temps qui éprouva le sort ici de la
princesse des Ursins. Je parle du lord Saint-Jean, plus connu sous le
nom de vicomte de Bolingbroke, par les mains duquel avait passé le
traité de Londres qui força les alliés à conclure la paix d'Utrecht, et
lequel, dans la fin de la négociation de Londres, fut envoyé ici passer
huit ou dix jours par la reine Anne, où il fut reçu avec tant de
distinction, comme je l'ai marqué en son lieu. Son sort en Angleterre
avait changé comme celui de la princesse des Ursins en Espagne, avec
cette différence que notre cour fut bien fâchée de la disgrâce de ce
ministre et de n'oser le voir. Le nouveau roi avait changé tout le
ministère, et remis les whigs en place, d'où il avait chassé les torys.
Ces premiers profitèrent de ce retour pour exercer leurs haines
particulières. Ils attaquèrent les ministres de la reine Anne, et leur
firent un crime d'avoir fait la paix. Prior, qui s'en était fort mêlé
sous ces ministres de la reine Anne, vendit leur secret et ce qu'il put
avoir de papiers à leurs persécuteurs qui étaient aussi les siens, pour
se tirer d'oppression par cette infamie. Bolingbroke, le plus noté de
tous pour avoir eu la principale part à la paix, se trouva aussi dans le
plus grand danger, et en même temps le moins établi. Il lutta un temps,
et lorsqu'il vit qu'il n'y avait point de ressources, il fit un discours
très nerveux en plein parlement, et en même temps très libre et très
fort contre la harangue du roi d'Angleterre, et tout de suite passa en
France. Il vint demeurer à Paris, mais sans aller à la cour, ni voir
publiquement nos ministres et nos personnages. J'aurai ailleurs lieu de
parler de lui.

Il y avait déjà quelque temps que le lord Stairs était ici de la part du
roi d'Angleterre, avec la patente d'ambassadeur, dont il fut fort
longtemps sans prendre le caractère. C'était un Écossais grand et bien
fait, qui avait l'ordre du Chardon ou de Saint-André d'Écosse. Il
portait le nez au vent avec un air insolent qu'il soutenait des plus
audacieux propos sur les ouvrages de Mardick, les démolitions de
Dunkerque, le commerce, et toutes sortes de querelles et de chicanes, en
sorte qu'on le jugeait moins chargé d'entretenir la paix, et de faire
les affaires de son pays, que de causer une rupture. Il poussa si loin
la patience et la douceur naturelle de Torcy, que ce ministre ne voulut
plus traiter avec lui. Stairs même était si peu mesuré dans les
audiences qu'il demandait fréquemment, et avec la plus grande hauteur,
que le roi prit le parti de ne le plus entendre. Il tâchait à se mêler
avec ce qu'il pouvait de meilleure compagnie, qui se lassa bientôt de
ses discours, dont il répandait l'impudence aux promenades publiques,
aux spectacles et chez lui, où il cherchait à s'attirer du monde par sa
bonne chère. J'aurai lieu plus d'une fois de parler de ce personnage qui
ne sut que trop bien jouer le sien et faire peur, tandis qu'il en
mourait intérieurement lui-même, et avec grande raison. C'était un homme
d'esprit, de toute espèce d'entreprises, qui était dans les troupes où
il avait servi sous le duc de Marlborough, et qui haïssait
merveilleusement la France. Il parlait aisément, éloquemment, et
démesurément sur tous chapitres, avec la dernière liberté.

Le roi fit à M. le Grand les grâces les plus singulières et les plus
sans exemple, pour M. de Monaco, son gendre, qui s'était raccommodé avec
lui depuis la rupture, qui a été racontée, du mariage du fils du comte
de Roucy avec sa fille, auquel M\textsuperscript{me} de Monaco et M. le
Grand son père ne voulurent jamais consentir, et qui n'avait pas en
effet de quoi remplir par ses biens les vues que M. de Monaco s'était
proposées. Il n'avait que des filles, et il était hors d'espérance
d'avoir d'autres enfants. Il était mal dans ses affaires, il cherchait
franchement à trafiquer sa dignité avec sa fille aînée. Il n'avait point
de crédit, la paresse italienne l'avait retenir à Monaco depuis la mort
de son père, il n'en sortit même plus, mais il espéra tout du crédit de
M. le Grand, et il ne s'y trompa point. Les grandes barrières de la
succession à la couronne étaient franchies\,; après celles-là nulles
antres ne pouvaient sembler considérables et les grâces en ce genre
accordées à M. de La Rochefoucauld ne pouvaient pas être refusées à son
rival perpétuel en faveur. Il fallait à M. de Monaco un homme de qualité
qui voulût bien quitter à jamais, pour soi et pour sa postérité, son
nom, ses armes, ses livrées, pour prendre en seul le nom, les armes et
les livrées de Grimaldi. Il était nécessaire aussi qu'il fût assez riche
pour donner quelque argent à M. de Monaco, se charger de la dot de ses
deux filles cadettes, et payer outre cela un grand nombre de gros
créanciers qui tourmentaient M. de Monaco. Ce n'était pas tout encore\,;
il fallait quelque fonds et un ample viager à l'abbé de Monaco son
frère, lequel y tenait ferme pour céder ses droits. Il fallait de plus
que tout cela fût si net et si assuré que M. de Monaco fût libéré
parfaitement, et à son aise et en repos pour tout le reste de sa vie.

Le défaut de moyens avait rompu l'affaire du fils du comte de Roucy.
Matignon, grâce aux trésors qu'il avait tirés du ministère de Chamillart
et à sa propre économie, avait de quoi satisfaire à tant de grands
besoins de M. de Monaco. Il n'avait pu réussir à se faire duc
d'Estouteville\,; il n'était point en situation d'espérer que le roi le
fît duc et pair de pure grâce\,; il se livra donc à une occasion unique
d'acheter cette dignité, pour en parler franchement. Son marché fait
avec M. de Monaco, il fut question de la seule chose qui le lui avait
fait faire, en laquelle toute impossibilité se trouvait, si on n'eût pas
été dans un temps où le roi ne voulait plus rien trouver d'impossible.
Valentinois avait été érigé en duché-pairie pour mâles uniquement, et
les femelles exclues, en 1642, en faveur du grand-père de M. de Monaco,
lorsqu'il chassa de Monaco la garnison espagnole, qu'il y en reçut une
française, et qu'il se mit sous la protection de la France\,: première
difficulté pour faire passer la dignité à une femelle. Elle subsistait
en la personne de M. de Monaco, elle n'était donc pas éteinte,
conséquemment point susceptible d'érection nouvelle. Il est vrai que
Henri Gondi, duc de Retz, petit-fils du maréchal-duc de Retz, et par sa
mère du duc de Longueville, n'ayant que deux filles, obtint en 1634,
c'est-à-dire vingt-cinq ans avant sa mort, une érection nouvelle de Retz
en faveur de Pierre Gondi avec rang nouveau, en épousant la fille aînée
de Henri Gondi duc de Retz, sa cousine issue de germaine, énormité dont
jusqu'alors on n'avait point vu d'exemple, et qui même n'avait pas été
imaginée. Ce Pierre Gondi, nouveau duc de Retz, en même temps que son
beau-père démis, était frère du fameux coadjuteur de Paris, si connu
sous le nom de dernier cardinal de Retz, et père de la duchesse de
Lesdiguières, dernière Gondi en France, mère du duc de Lesdiguières,
gendre du maréchal de Duras. Tout cela fut accordé à M. de Monaco\,;
mais comme les énormités n'ont plus de bornes quand les justes barrières
sont une fois franchies, en voici d'autres qu'il obtint.

Au cas que M. de Monaco pût avoir un fils, tout lui retournait, et la
dignité même de duc et pair de l'ancienneté de 1642\,; le fils de
Matignon demeurait duc sa vie durant comme un duc et pair démis, et son
fils ne pouvait jamais prétendre d'y revenir ni les siens, mais il
reprenait, mais sans aucun rang ni honneurs, son nom, ses armes, ses
livrées ainsi que toute la postérité du fils de Matignon et de la fille
de Monaco. Ainsi M. de Monaco vendit sa dignité et sa fille très
chèrement, et se réserva de la retenir s'il avait un fils. Rien de plus
monstrueux ne se pouvait imaginer après l'habilité à la couronne, et les
grandeurs des bâtards du roi et de M\textsuperscript{me} de Montespan.
Ce prodige de concession n'eut pas lieu parce que M. de Monaco n'eut
point de fils. Il y eut encore d'autres choses passées entre M. de
Monaco et M. de Matignon, touchant la réversion des biens en cas de
naissance d'un fils. Comme le mariage ne se pouvait faire sans aplanir
auparavant des difficultés intrinsèques et qu'il était pourtant très
nécessaire d'en bien assurer le fondement, toutes ces monstrueuses
concessions furent énoncées par un brevet du 24 juillet 1715. Le 20
octobre suivant, six semaines après la mort du roi, le fils de Matignon
épousa à Monaco la fille aînée de M. de Monaco. Au mois de décembre
suivant, les lettres d'érection furent expédiées conformément en tout au
brevet du 24 juillet précédent\,; en quoi M. le duc d'Orléans, régent,
ni le conseil de régence, ne trouvèrent point de difficulté, parce que
la concession du feu roi avait été publique, qu'ils en avaient tous
connaissance, et que ce brevet, expédié du vivant du roi, en faisait
foi. Par les mêmes raisons le parlement enregistra sans difficulté les
lettres d'érection, le 2 septembre 1716, dès qu'elles y furent
présentées, et le nouveau duc de Valentinois y fut reçu comme pair de
France le 14 décembre suivant.

Le roi fit présent à Voysin, chancelier et secrétaire d'État ayant le
département de la guerre, du revenant-bon du non-complet des troupes,
qu'il dit aller à cinq cent mille livres. Cette libéralité était bien
due aux services de cette âme damnée de la constitution, de
M\textsuperscript{me} de Maintenon et de M. du Maine, et à l'unique
dépositaire des manèges et du testament du roi\,; mais il fit
étrangement crier le public dont ce front d'airain eut toute honte bue.

Sa Majesté accorda à Le Camus, encore fort jeune, la place et l'exercice
de premier président de la cour des aides qu'avait son grand-père, et
l'agrément de la charge de prévôt et grand maître des cérémonies de
l'ordre que lui vendit Pontchartrain en retenant les marques de l'ordre.

La comtesse d'Acigné, dernière, par elle et par son défunt mari, de
cette bonne et ancienne maison de Bretagne, mourut fort âgée à Paris. Le
duc de Richelieu, son gendre, et qui n'avait de fils que de sa fille, la
suivit de fort près, à quatre-vingt-six ans. J'en ai suffisamment parlé
en plusieurs endroits pour le faire connaître, ainsi que de la princesse
d'Harcourt, soeur de la duchesse de Brancas, qui mourut assez
brusquement chez elle à Clermont, et qui ne laissa de regrets à
personne.

Sézanne, frère de père du duc d'Harcourt, et de mère de la duchesse
d'Harcourt, était mort depuis quelque temps d'une longue maladie, dont
il avait rapporté d'Italie les premiers commencements, et à laquelle les
médecins ne connurent rien. Le duc de Mantoue avait un sérail de
maîtresses dont il était fort jaloux. Sézanne ne s'en contraignit pas,
et on crut qu'il en avait été payé à l'italienne. Il ne laissa point
d'enfants. C'était un jeune homme bien fait, que la fortune de son frère
avait gâté, qui sans cela eût valu quelque chose, et qui ne se fit point
regretter. Son frère lui avait fait donner la Toison qui lui était
destinée\,; il envoya un de ses fils cadets en reporter le collier en
Espagne, dans l'espérance qu'il lui serait donné, en quoi son espérance
ne fut pas trompée.

Le fameux docteur Burnet, évêque de Salisbury, si connu par ses
ouvrages, et par le secret qu'il eut de l'entreprise du prince d'Orange
sur l'Angleterre, avec lequel il y passa lors de la révolution whig, le
plus déclaré pour ce parti malgré son épiscopat, mourut en ce même
temps.

L'abbé d'Estrades mourut aussi à Chaillot, où sa pauvreté lui avait fait
louer une maison depuis bien des années pour y vivre à meilleur marché
et en retraite. Il était fils du maréchal d'Estrades, et avait très bien
réussi à Venise et à Turin, où il avait été ambassadeur, mais il s'y
était fort endetté. Il vécut fort exemplairement et fort solitairement à
Chaillot. Ses dettes étaient presque toutes payées. Il avait l'abbaye de
Moissac et dix mille livres de pension sur les abbayes de l'abbé de
Lyonne. On aurait pu se servir fort utilement de lui, mais on ne voulait
que des gens qui pussent et voulussent bien se ruiner, et non pas de
ceux qui s'étaient déjà ruinés en ambassades.

M. de Lauzun maria Castelmoron, son neveu, qui n'était pas riche, à la
fille de Fontanieu, qui de laquais de Crosat était devenu son commis,
puis son caissier, et qui y avait acquis de grands biens avec lesquels
s'était poussé, et était devenu pour son argent garde-meuble de la
couronne, qui est l'inspection en détail de tous les meubles faits et à
faire pour le roi, et de l'ameublement et du démeublement de toutes les
maisons royales. Heudicourt épousa, pour se recrépir, une fille de
Surville\,; et Cayeux, fils de Gamaches, épousa la fille de M. de
Pomponne, fils du ministre d'État. Le troisième fils du duc de Rohan
épousa aussi sa cousine de même nom, comtesse de Jarnac, veuve d'un
cadet de Montendre-lez-La-Rochefoucauld, dont elle n'avait point eu
d'enfants. Ce fut une fortune pour ce troisième cadet du duc de Rohan
qu'elle préféra au second\,; mais elle stipula qu'il quitterait le
service et Paris, et qu'il irait avec elle vivre à Jarnac, qui est un
fort beau lieu en Poitou, dont elle ne voulait point sortir. Elle
parlait en héritière très riche à cadet qui n'avait rien, et qui se
trouva heureux de l'épouser et de se conformer à toutes ses volontés.

Le marquis de Saint-Sulpice-Crussol épousa en même temps la fille du
comte d'Estaing, qui fut longtemps depuis chevalier de l'ordre.

Le roi, étant à Marly, s'arrêta dans ses jardins avant la messe, pour
s'y amuser à voir une éclipse de soleil, sur les neuf heures du matin.
Toutes les dames y étaient longtemps auparavant. Cassini, fameux
astronome, y était venu de l'Observatoire avec des lunettes pour la
faire bien remarquer, le vendredi 3 mai. Le lendemain on fit, à
Saint-Denis, le bout de l'an de M. le duc de Berry, où l'évêque de Séez,
Turgot, officia, qui avait été son premier aumônier\,; M. le duc
d'Orléans et quelques princes du sang s'y trouvèrent.

Dès le lendemain le roi fit quitter le grand deuil à
M\textsuperscript{me} la duchesse de Berry, qui devait durer encore six
semaines, et la mena lui-même dans le salon, où il la fit jouer. On a vu
souvent ici combien le roi était peiné du grand deuil, et le peu de
mesure qu'il y garda dans sa plus proche famille. M\textsuperscript{me}
la duchesse de Berry souhaitait fort d'avoir des dames, depuis la mort
de M\textsuperscript{me} la Dauphine, à l'instar des dames du palais. Il
y avait longtemps que M\textsuperscript{me} de Saint-Simon avait obtenu
du roi que M\textsuperscript{me} de Coettenfao, femme de son chevalier
d'honneur, pût la suivre quand M\textsuperscript{me} de Saint-Simon et
M\textsuperscript{me} de La Vieuville ne le pourraient pas. Cette
dernière était à Paris, hors d'espérance que sa santé se rétablit.
M\textsuperscript{me} la duchesse de Berry obtint donc quatre dames,
mais sans titre de dames du palais. Elle proposa M\textsuperscript{me}
de Coettenfao\,; la marquise de Brancas, dont il a été parlé plus d'une
fois\,; M\textsuperscript{me} de Clermont, dont le mari avait été
capitaine des gardes de M. le duc de Berry, et qui était fille de
M\textsuperscript{me} d'O\,; et M\textsuperscript{me} de Pons, dont le
mari avait été maître de la garde-robe de M. le duc de Berry. Elles
furent toutes quatre acceptées par le roi pour accompagner
M\textsuperscript{me} la duchesse de Berry, et deux à deux à Marly, avec
quatre mille livres d'appointements. La marquise de Brancas n'en fit
jamais de fonctions et s'en alla en Provence, d'où elle ne revint
plus\,; et M\textsuperscript{me} de Coettenfao mourut fort peu de temps
après cette nomination. Quelque temps après M\textsuperscript{me}s
d'Armentières et de Beauvau eurent leurs places.

La mort de M\textsuperscript{me} de Coettenfao me donna des affaires
auxquelles je ne m'attendais pas. Elle était peu de chose, fille d'un
conseiller au parlement et d'une fille de cette M\textsuperscript{me} de
Motteville, dont nous avons de si bons Mémoires de la régence de la
reine Anne d'Autriche. M\textsuperscript{me} de Coettenfao n'avait point
d'enfants ni d'héritiers proches. Son mari, qui était depuis bien des
années extrêmement de mes amis, et que j'avais fait chevalier d'honneur
de M\textsuperscript{me} la duchesse de Berry, m'avait prié, les trois
dernières campagnes, de lui garder une cassette, et en cas de mort de la
remettre à sa femme. Elle tomba fort malade, et m'envoya prier, à Marly
où j'étais, de lui aller parler à Paris. J'y fus aussitôt\,; elle se
hâta de me remettre la même cassette, sans me rien dire au delà, ni de
ce qu'elle contenait, ni de ce qu'elle voulait que j'en fisse, et acheva
de me parler derrière un paravent, car elle était encore debout, fort
troublée de ce que sa mère, avec qui elle logeait, entra dans la
chambre. J'emportai la cassette chez moi, et retournai à Marly. À huit
ou dix jours de là elle mourut. Il fallut articuler cette cassette, et
l'envoyer ouvrir chez le lieutenant civil.

On y trouva un testament, par lequel elle me donnait tout ce dont elle
pouvait disposer, qui allait à plus de cinq cent mille francs.
J'entendis aisément, sans que personne m'en ouvrît la bouche, ce que
c'était que ce grand présent. Je le dis à Coettenfao et à son frère,
évêque d'Avranches, et je pris toutes mes mesures pour recueillir cette
succession et la remettre sur-le-champ à Coettenfao. Les héritiers et la
mère se préparèrent à me la disputer, moi à me défendre. Je me croyais
bien fort parce que, qui que ce soit ne m'ayant parlé de ce legs, encore
moins de l'objet de son usage, j'étais en état de jurer là-dessus en
plein parlement\,; mais il venait d'y intervenir tout nouvellement un
arrêt fort étrange en haine de ces sortes de fidéicommis

M\textsuperscript{me} d'Isenghien-Rhodes, morte sans enfants, avait
donné tout son bien à l'abbé de Thou, homme de la plus grande probité et
fort de ses amis et de M. d'Isenghien. Il n'avait pas su le moindre mot
de ce legs que par l'ouverture du testament, encore moins lui avait-on
insinué l'usage\,; il était donc en mêmes termes où je me trouvais, et
en toute liberté de jurer là-dessus en plein parlement. Mais le
parlement alla plus loin qu'il n'avait encore fait\,; et, par une
nouveauté qu'il introduisit dont il n'y avait point encore eu d'exemple,
non seulement il exigea de l'abbé de Thou le serment accoutumé «\, qu'il
n'avait eu aucune connaissance du legs à lui fait, ni que ce legs fût en
effet un fidéicommis pour le rendre à un autre\,;» mais il exigea son
serment de garder le legs à son profit, et de {[}ne{]} le donner à
personne, à faute de quoi le testament serait cassé et déclaré nul. Je
ne sais comment l'abbé de Thou l'entendit\,; mais, voyant le testament
cassé à faute de serment de garder le legs et de {[}ne{]} le donner à
personne, il sauta le bâton, et prêta le serment, au moyen duquel le
legs lui fut payé.

Pour moi, qui ne voulais du mien que pour le remettre à M. de
Coettenfao, parce que je voyais bien qu'il ne pouvait m'avoir été fait
que pour cet usage, je ne voulus pas hasarder le serment que l'abbé de
Thou avait prêté\,; et pour l'éviter, j'évoquai l'affaire au parlement
de Rouen sur les parentés de ceux qui me disputaient, parce que le
parlement de Rouen, où il m'était resté des amis depuis le procès que
j'y avais gagné contre M. de Brissac, la duchesse d'Aumont, etc., ne
s'était pas encore avisé du serment que le parlement de Paris avait fait
prêter à l'abbé de Thou, et que j'espérais bien qu'il ne me l'imposerait
pas. Pour achever cette affaire tout de suite, elle s'instruisit à
Rouen. Mes parties s'y rendirent, et y publièrent que je ne soutenais ce
procès que par bienséance, que je ne me souciais point du succès, parce
qu'on jugeait bien que ce n'était pas pour moi que je plaidais, et que
je le prouvais par mon absence. Coettenfao et l'évêque d'Avranches, qui
étaient à Rouen, m'en avertirent. Je partis deux jours après pour m'y
rendre, malgré les affaires dont j'étais alors occupé. Je vis toits les
juges et mes anciens amis\,; je ne négligeai rien de tout ce qui pouvait
servir au gain du procès\,; et je demeurai huit ou dix jours à montrer
que c'était très sérieusement et pour moi que je le soutenais, et que je
n'oubliais rien pour l'emporter. Ce voyage changea la face de
l'affaire\,; la mère et les héritiers eurent peur et me firent proposer
un accommodement. Je le refusai, et en avertis Coettenfao et son frère.
Je leur dis que, comme ils savaient bien, par ce que je leur en avais
déclaré d'abord, que je n'en mettrais pas un sou dans ma poche,
m'accommoder ou non, m'accommoder d'une façon ou d'une autre m'étaient
choses entièrement indifférentes\,; que c'était à eux à voir ce qui leur
convenait le mieux, et à me faire agir en conséquence. Malgré mon refus,
les parties me firent faire encore des propositions\,; et tant fut
procédé que Coettenfao et son frère réglèrent l'accommodement de manière
que la plus grande partie me fut cédée. Alors Coettenfao et son frère
aimèrent mieux cela que l'incertitude d'un arrêt et les longueurs de la
chicane. Ils me prièrent d'y passer, et je signai l'accommodement avec
les parties, et le moment d'après je fis les signatures et tout ce qui
était nécessaire pour que tout ce qui me revenait fût mis, sans entrer
en mes mains, entre celles de M. de Coettenfao, qui toucha tout
aussitôt.

À quatre ou cinq mois de là, lui et son frère firent faire une belle et
bonne vaisselle à mes armes, avec un secret profond et fort bien observé
jusqu'à deux jours après qu'elle fut apportée chez moi, et laissée par
des crocheteurs, sans dire ce que c'était que ces ballots, ni de quelle
part. Ils s'enfuirent dès qu'ils les eurent déchargés.
M\textsuperscript{lle} d'Avaise, demoiselle de bon lieu et de grande
vertu, mais pauvre, qui était à M\textsuperscript{me} la duchesse
d'Orléans avec distinction, et que j'avais fait faire première femme de
chambre de M\textsuperscript{me} la duchesse de Berry, en avait
découvert quelque chose, et nous en avertit. Il y avait pour plus de
vingt mille écus de vaisselle. Nous en parlâmes à Coettenfao, qui nia
tant qu'il put, mais qui le put jusqu'au bout, et qui ne la voulut
jamais reprendre, quelque chose que M\textsuperscript{me} de Saint-Simon
et moi pussions faire. Nous n'en avions que de faïence depuis que tout
le monde avait envoyé la sienne à la Monnaie. Ainsi l'affaire de cette
succession finit de la sorte galamment des deux parts. Je sus après que
cette cassette, que je gardai trois campagnes de suite à Coettenfao,
contenait cette disposition de sa femme. Il était riche de lui\,; cette
augmentation ne lui nuisait pas, car il vivait à l'armée et partout fort
honorablement. Il était lieutenant général, distingué par ses actions et
par son désintéressement, et adoré et très estimé dans la maison du roi,
où il était premier sous-lieutenant des chevau-légers de la garde. Je
lui fis donner devant moi parole par M. le duc d'Orléans, régent alors,
de le faire chevalier de l'ordre à la première promotion qu'il y
aurait\,; mais ce prince en avait tant donné de pareilles qu'il trouva
plus court de ne point faire de promotion, et de manquer à toutes plutôt
qu'à plusieurs, parce qu'il ne pouvait excéder le nombre de cent porté
par les statuts.

Le roi partit le mercredi 12 juin pour Marly\,: ce fut son dernier
voyage\,; et la reine d'Angleterre partit le lendemain en litière pour
aller prendre les eaux de Plombières, plus encore pour y voir le roi son
fils. Chamlay, dont j'ai parlé souvent, et qui était de tous les voyages
de Marly, tomba en apoplexie, et partit aussitôt pour Bourbon. Son
logement fut donné au marquis d'Effiat. La santé du roi diminuait à vue
d'œil, et M. du Maine, à qui le marquis d'Effiat était vendu de longue
main, sans que M. le duc d'Orléans le voulût croire ni rien diminuer de
sa confiance en lui, était nécessaire à M. du Maine dans un aussi long
Marly, où le roi pouvait mourir, et où il était si important d'être bien
informé des mesures de M. le duc d'Orléans, et de lui en faire inspirer
de fausses. C'était un homme de sac et de corde, d'autant plus dangereux
qu'il avait beaucoup d'esprit et de sens, fort avare, fort particulier,
fort débauché, mais avec sobriété pour conserver sa santé. Il était
grand chasseur, et jusqu'à ces derniers temps chez lui, fort seul avec
les chiens de M. le duc d'Orléans. Il avait, comme on l'a vu, empoisonné
la première femme de Monsieur, avec le poison que le chevalier de
Lorraine lui avait envoyé de Rome, duquel il fut toute sa vie intime, et
du maréchal de Villeroy. Je ne lui avais jamais parlé lorsqu'il vint à
Marly. Je n'ignorais pas ses menées avec M. du Maine, même avec
M\textsuperscript{me} de Maintenon, et tout me déplaisait en lui.

Lorsqu'il fat à Marly, et ce fut au bout de quatre jours de l'arrivée,
M\textsuperscript{me} la duchesse d'Orléans me fit de grandes plaintes
du délabrement et de la mauvaise administration des biens et revenus de
M. le duc d'Orléans, me vanta la capacité et le mérite du marquis
d'Effiat, son attachement pour M. le duc d'Orléans, son déplaisir de
voir aller ses affaires en décadence, la facilité avec laquelle il les
remettrait en bon état et les revenus plus qu'au courant, si on lui en
voulait donner le soin et l'autorité, qu'il ne voulait pas demander mais
qu'il accepterait volontiers par amitié s'ils lui étaient offerts\,;
qu'elle en avait raisonné avec lui sur ce pied-là. Elle ajouta qu'elle
voudrait fort que je connusse le marquis d'Effiat, avec force louanges
pour lui et pour moi\,; et conclut par me prier de parler à M. le duc
d'Orléans du dérangement de ses affaires, du mauvais effet que cela
faisait, pour un prince destiné à l'administration publique dans une
minorité, et de lui proposer d'en remettre le soin et l'autorité au
marquis d'Effiat. Je ne goûtai rien de tout cela. Je me défendis des
nouvelles connaissances\,; et on verra en son lieu que
M\textsuperscript{me} la duchesse d'Orléans était bien moins femme que
soeur. Je lui dis que j'avais toute ma vie observé de ne parler\,;
jamais à M. le duc d'Orléans de ses affaires, ni du Palais-Royal\,; que
je me trouvais si bien de cette coutume que je ne pouvais la changer. Ma
fermeté n'ébranla point la sienne. Elle me pressa, elle me tourmenta, me
força enfin de représenter à M. le duc d'Orléans le discrédit, et les
suites de la mauvaise administration de ses affaires, de prendre mon
temps que le marquis d'Effiat serait avec lui, qu'il m'appuierait dans
cette conversation, que je viendrais à proposer tout de suite à M. le
duc d'Orléans de prier Effiat de s'en mêler avec toute autorité, qu'il
ne le refuserait pas en face, ni d'Effiat d'y entrer pour les rectifier.

Deux jours après, sans avoir vu le marquis d'Effiat, je le trouvai chez
M. le duc d'Orléans, où je ne serais pas entré en tiers sans la promesse
que M\textsuperscript{me} la duchesse d'Orléans m'avait arrachée. Nous
causâmes quelque temps de choses indifférentes, enfin je fis ma
représentation, et tout de suite ma conclusion. Ils me laissèrent tous
deux dire jusqu'au bout\,; et quand j'eus fini, M. le duc d'Orléans me
dit qu'il ne savait pas où je prenais le dérangement de ses affaires, et
le mauvais effet qu'il faisait dans le public\,; de là il se mit à en
vanter le bon ordre. Je répondis que je croyais pourtant en être bien
informé, et par gens qui n'y prenaient d'autre intérêt que le sien\,;
puis regardant le marquis d'Effiat, qui avait gardé là-dessus le plus
profond silence, je dis à M. le duc d'Orléans de demander à d'Effiat ce
qu'il en savait et pensait, qui en pouvait être mieux informé peut-être
que les personnes qui m'avaient parlé. Là-dessus d'Effiat me dit
qu'elles étaient sûrement très mal informées, qu'il n'avait jamais suivi
de près les choses qui ne le regardaient point, mais qu'il en savait
pourtant assez pour pouvoir m'assurer que les affaires de M. le duc
d'Orléans étaient dans le meilleur ordre du monde, les mieux
administrées, et renchérit longuement sur ce que M. le duc d'Orléans
m'avait répondu. Ils se renvoyèrent même la balle l'un à l'autre avec
complaisance, tandis que j'étais plongé dans un silence d'admiration et
d'indignation. J'en sortis enfin par témoigner que j'étais ravi qu'on se
fût mépris là-dessus en me parlant\,; et peu à peu la conversation se
remit sur choses indifférentes\,; c'était ce que je souhaitais pour
lever le siège avec bienséance. Je n'en perdis pas le moment\,; et je
passai tout de suite chez M\textsuperscript{me} la duchesse d'Orléans, à
qui je dis d'arrivée de ne me parler de sa vie de son marquis d'Effiat,
et lui contai ce qui venait de se passer. Elle m'en parut fort étonnée,
mais point déprise du marquis d'Effiat, qui tenait à elle par des
endroits plus chers\,; mais j'y gagnai qu'elle n'osa jamais plus me
nommer son nom. J'évitai depuis fort aisément de rencontrer Effiat chez
M. le duc d'Orléans, et de l'approcher dans le salon où lui aussi ne me
cherchait pas\,; mais force politesses de sa part dans ces lieux publics
quand l'occasion s'en offrait, sans se rebuter de la froideur des
miennes. Il n'est pas temps encore de parler de tout cet intérieur de M.
{[}le duc{]} et de M\textsuperscript{me} la duchesse d'Orléans, et de ce
peu de gens qui encore alors approchaient de ce prince.

À propos d'honnêtes gens, le marquis de Nesle avait une sœur fort laide,
qui avait épousé un Nassau, de branche très cadette, qui servait
l'Espagne d'officier général, et qui avait eu la Toison. C'était la faim
et la soif ensemble. Le mari était un fort honnête homme et brave,
d'ailleurs un fort pauvre homme, qui avait laissé brelander sa femme à
son gré, qui vivait de ce métier et de l'argent des cartes. Toute laide
qu'elle était, elle avait eu des aventures vilaines qui avaient fait du
bruit. Le mari se fâcha, elle prit le parti de le plaider\,; de part et
d'autre il se dit d'étranges choses. Le mari à la fin présenta un placet
au roi, par lequel il lui demandait, sans toutefois en avoir besoin, la
permission d'accuser sa femme d'adultère, et d'attaquer en justice ceux
qui l'avaient commis avec elle. Il y avait encore pis\,: il prétendait
avoir preuve en main qu'elle avait voulu l'empoisonner et qu'il l'avait
échappé belle. Les Mailly s'effrayèrent de l'échafaud et obtinrent
qu'elle serait conduite à la Bastille\,; elle en est sortie depuis et a
bien fait encore parler d'elle. Elle n'a point eu d'enfants, et son mari
est mort longtemps après cette aventure. On la crut mariée depuis à un
avocat obscur.

Les mêmes personnes, qui n'avaient rien oublié, par leurs manèges et par
leurs émissaires, pour persuader le roi, Paris, toute la France et les
pays étrangers de mettre les malheurs domestiques de la maison royale
sur le compte de M. le duc d'Orléans, et qui de temps en temps savaient
renouveler et entretenir ces bruits avec art, ne laissèrent pas tomber
une maladie de M\textsuperscript{me} la duchesse d'Orléans, qui fut
bizarre, longue, et où les médecins dirent qu'ils n'entendaient rien.
Elle était pourtant facile à comprendre\,; et, sans être médecin, je la
lui avais prédite. Ces princesses ont toutes des fantaisies que rien ne
peut détourner.

Celle-ci, non contente d'un magnifique appartement et très complet à
Versailles, s'avisa de se faire un cabinet d'un bouge cul-de-sac à la
ruelle de son lit, qui lui servait d'une garde-robe, où on ne voyait
clair que par le haut d'un vitrage qui donnait sur la galerie. Elle y
fit une cheminée et des ornements tant qu'elle put. Le lieu était si
petit qu'il contenait à peine cinq ou six personnes, encore à la faveur
d'un grand enfoncement qu'elle fit faire en grattant et cavant un gros
mur vis-à-vis la cheminée, ou elle pratiqua une niche à se coucher tout
de son long. Il la fallut enduire de plâtre pour unir ce qui était rompu
et raboteux partout\,; la boiser aurait trop étréci. Elle la meubla donc
par-dessus ce plâtre qu'on ne faisait que mettre, et tout aussitôt elle
y passa ses journées. Je l'avertis que rien n'était si pernicieux que ce
plâtre neuf dans lequel elle était couchée\,; je lui en citai force
exemples. Je lui rappelai la mort de cette forte et robuste maréchale
d'Estrées, qui mourut pour avoir eu les prémices d'une chambre neuve à
Marly. Rien ne prit\,; elle en fut châtiée. Des douleurs partout et une
fièvre irrégulière, tantôt forte, tantôt faible, une soif continuelle et
point d'appétit\,; c'était moins une maladie en forme qu'une langueur
insupportable. Elle se lassa enfin des remèdes et des médecins,
s'affranchit des uns et des autres, et avec le temps elle guérit
parfaitement sans secours, au grand regret, je pense, de qui en avait
préparé l'affreux paquet à M. le duc d'Orléans, quelques fortes raisons
d'ailleurs de toute espèce qu'il pût y avoir de désirer sa conservation.

Quoiqu'il ne soit pas encore temps de parler de l'état de la santé du
roi, on la voyait décliner sensiblement, et son appétit, qui était fort
grand et toujours égal, très considérablement diminué. Si l'attention y
était grande au milieu de sa cour, où il n'avait pas néanmoins changé la
moindre chose en la manière accoutumée de sa vie ni en l'arrangement
divers de ses journées, toujours les mêmes dans leur diversité, les pays
étrangers n'y étaient pas moins attentifs et guère moins bien informés.
Les paris s'ouvrirent donc en Angleterre que sa vie passerait ou ne
passerait pas le 1er septembre, c'est-à-dire environ trois mois, et,
quoique le roi voulût tout savoir, on peut juger que personne ne fut
pressé de lui apprendre ces nouvelles de Londres. Il se faisait
ordinairement lire les gazettes de Hollande en particulier par Torcy,
souvent après le conseil d'État. Un jour qu'à cette heure-là Torcy lui
faisait cette lecture qu'il n'avait point parcourue auparavant, il
rencontra ces paris à l'article de Londres\,; il s'arrêta, balbutia et
les sauta. Le roi, qui s'en aperçut aisément, lui demanda la cause de
son embarras, ce qu'il passait et pourquoi\,; Torcy rougit jusqu'au
blanc des yeux, dit ce qu'il put, enfin que c'était quelque impertinence
indigne de lui être lue. Le roi insista\,; Torcy aussi, dans le dernier
embarras\,; enfin il ne put résister aux commandements réitérés\,; il
lui lut les paris tout du long. Le roi ne fit pas semblant d'en être
touche, mais il le fut profondément, et au point que s'étant mis à table
incontinent après, il ne put se tenir d'en parler en regardant la
compagnie, mais sans faire mention de la gazette.

C'était à Marly, où quelquefois j'allais faire ma cour au commencement
du petit couvert, et le hasard fit que j'y étais ce jour-là. Le roi me
regarda comme les autres, mais comme exigeant quelque réponse. Je me
gardai bien d'ouvrir la bouche, et je baissai les yeux. Cheverny, homme
pourtant fort sage, ne fut pas si discret, et fit une assez longue et
mauvaise rapsodie de pareils bruits, venus de Vienne à Copenhague,
pendant qu'il y était ambassadeur, il y avait dix-sept ou dix-huit ans.
Le roi le laissa bavarder, et n'y prit point. Il parut touché en homme
qui ne le voulait pas paraître. On vit qu'il fit ce qu'il put pour
manger et pour montrer qu'il mangeait avec appétit. Mais on remarquait
en même temps que les morceaux lui croissaient à la bouche\,: cette
bagatelle ne laissa pas d'augmenter la circonspection de la cour,
surtout de ceux qui, par leur position, avaient lieu d'y être plus
attentifs que les autres. Il se répandit qu'un aide de camp de Stairs,
retourné depuis peu en Angleterre, avait donné occasion à ces paris, par
ce qu'il avait publié de la santé du roi. Stairs, à qui cela revint,
s'en montra fort peiné, et dit que c'était un fripon qu'il avait chassé.

Il parut que cette aventure fut un coup d'éperon pour combler de plus en
plus la grandeur des bâtards. M. du Maine sentait qu'il n'avait point de
temps à perdre, et secondé de M\textsuperscript{me} de Maintenon et des
manèges du chancelier, il sut profiter de tous les moments. Rien n'avait
été si long ni plus difficile que de ployer les ambassadeurs à traiter
les bâtards du roi comme les princes du sang. À la fin ils les
visitèrent comme ces princes, et n'y mirent plus de différence. M. du
Maine voulut que ses enfants eussent le même honneur que lui à cet
égard, puisque comme lui ils étaient déclarés et leur postérité habiles
à succéder à la couronne. Il se servit habilement de l'occasion du
dernier de tous les ambassadeurs et du frère de sa créature la plus
abandonnée. Le bailli de Mesmes avait été nommé à l'ambassade de Malte
en France, à la sollicitation du roi, séduit par M. du Maine, lequel
avait décoré son entrée de tous ses gens et de tous ses chevaux. L'ordre
de Malte est trop sous la main du roi de France pour oser lui déplaire
et contester un cérémonial si désiré. Le frère du premier président
n'était pas non plus pour faire le difficile, tellement que ce fut lui
qui, le premier de tous les ambassadeurs, visita en pleine cérémonie le
prince de Dombes, comme il avait visité tous les princes du sang et les
deux bâtards. Cette démarche fit grand bruit et déplut également aux
ambassadeurs, pour qui la planche était faite, et aux princes du sang\,;
ceux-ci cherchèrent à s'en venger et ne firent qu'approfondir la plaie.

À huit jours de là, M. du Maine présenta une requête au parlement dans
le cours du procès de la succession de M. le Prince, dans laquelle il
prenait la qualité de prince du sang. Il s'y croyait fondé par l'édit
bien enregistré, qui le rendait habile et les siens à succéder à la
couronne, qui est la qualité distinctive et qui fait l'essence des
princes du sang. M. le Duc s'y opposa, et, avec M. le prince de Conti,
quoique uni d'intérêt en ce procès avec M. du Maine, demanda
juridiquement la radiation de la qualité de prince du sang, mal à propos
prise par le duc du Maine\,: cela fit grand bruit, mais il fut court\,;
car autres huit jours après, il parut une nouvelle déclaration qui
enjoignit au parlement d'admettre en tous actes judiciaires et jugements
le titre et la qualité de prince du sang pour le duc du Maine, sa
postérité et le comte de Toulouse, et de n'en faire en quoi que ce soit
la moindre différence d'avec les princes du sang, toutefois après le
dernier de tous. La déclaration témoigne surprise, et quelque chose de
plus, de ce que cette qualité et titre avait pu être contestée et
souffrir la moindre difficulté, après la manière dont les précédents
édits enregistrés étaient énoncés. Celui-ci fut aussi enregistré tout
aussitôt qu'il fut porté au parlement.

Sainte-Maure, qui avait été premier écuyer de M. le duc de Berry,
s'avisa en quittant son deuil de demander permission au roi de
conserver, sa vie durant, et à ses dépens, les livrées de ce prince et
ses armes à ses voitures. Les dernières étaient pour entrer à ce moyen,
comme ceux qui ont les honneurs du Louvre, les autres pour user
lentement de toutes les livrées qui lui pouvaient durer toute sa vie, et
en épargner les habits. Il se trouva que Hautefort, qui avait été
premier écuyer de la reine, oncle paternel de tous les Hautefort, et que
sa charge avait fait chevalier de l'ordre, avait eu la même concession.
Sur cet exemple, le roi l'accorda à Sainte-Maure.

Le comte de Lusace, c'est-à-dire le prince électoral de Saxe, maintenant
électeur et roi de Pologne, après son père, vint prendre congé du roi
dans son cabinet à Marly, qui lui fit beaucoup d'honnêtetés, et au
palatin de Livonie, qui était le surintendant de sa conduite et de son
voyage, et qui s'était acquis par la sienne, ici et partout, beaucoup de
réputation. Le roi envoya au comte de Lusace une épée de diamants de
quarante mille écus, au palatin de Livonie son portrait enrichi de fort
beaux diamants, et le même présent, mais moindre en valeur, au baron
Haageri, gouverneur du prince. Il avait témoigné souhaiter fort de voir
Saint-Cyr, et cela s'était toujours différé. M\textsuperscript{me} de
Maintenon lui avait donné jour au dimanche 2 juin. Elle l'attendait, et,
après lui avoir fait voir toute la maison, elle lui avait préparé la
comédie d'\emph{Esther}, jouée par les demoiselles\,; mais la fièvre
prit au prince, qui envoya faire ses excuses, et supplier
M\textsuperscript{me} de Maintenon que la bonté qu'elle avait ne fût que
différée, et cela fut remis au mardi 11 juin, qu'il se trouva en état
d'y aller. Il partit peu de jours après pour la Saxe. Il se conduisit
avec beaucoup de sagesse, de politesse, et pourtant de dignité, et vit
fort la meilleure compagnie.

Ducasse mourut fort âgé, et plus cassé encore de fatigues et de
blessures. Il était fils d'un vendeur de jambons de Bayonne, et de ce
pays-là où ils sont assez volontiers gens de mer. Il aima mieux
s'embarquer que suivre le métier de son père, et se fit flibustier. Il
se fit bientôt remarquer parmi eux par sa valeur, son jugement, son
humanité. En peu de temps ses actions l'élevèrent à la qualité d'un de
leurs chefs. Ses expéditions furent heureuses, et il y gagna beaucoup.
Sa réputation le tira de ce métier pour entrer dans la marine du roi, où
il fut capitaine de vaisseau. Il se signala si bien dans ce nouvel état,
qu'il devint promptement chef d'escadre, puis lieutenant général, grades
dans lesquels il fit glorieusement parler de lui, et où il eut encore le
bonheur de gagner gros sans soupçon de bassesse. Il servit si utilement
le roi d'Espagne, même de sa bourse, qu'il eut la Toison, qui n'était
pas accoutumée à tomber sur de pareilles épaules. La considération
générale qu'il s'était acquise même du roi et de ses ministres, ni
l'autorité où sa capacité et ses succès l'avaient établi dans la marine
ne purent le gâter. C'était un grand homme maigre, commandeur de
Saint-Louis, qui avec l'air d'un corsaire, et beaucoup de feu et de
vivacité, était doux, poli, respectueux, affable, et qui ne se méconnut
jamais. Il était fort obligeant, et avait beaucoup d'esprit avec une
sorte d'éloquence naturelle, et même hors des choses de son métier, il y
avait plaisir et profit à l'entendre raisonner. Il aimait l'État et le
bien pour le bien, qui est chose devenue bien rare.

Nesmond, évêque de Bayeux, mourut aussi doyen de l'épiscopat en France,
à quatre-vingt-six ans. C'était de ces vrais saints qui attirent, malgré
eux, une vénération qu'on ne peut leur refuser, et dont la simplicité
donne à tous les moments à rire. Aussi, disait-on de lui, qu'il disait
la messe tous les matins, et qu'il ne savait plus après ce qu'il disait
du reste de la journée. L'innocence parfaite de ses mœurs, jointe à un
esprit très borné, lui laissait échapper des ordures à tout propos, dont
il n'avait pas le moindre soupçon, et qui rendaient sa compagnie
embarrassante aux femmes, jusque-là que la présidente Lamoignon, sa
nièce, renvoyait sa fille, qui épousa depuis le président Nicolaï, dès
qu'il entrait chez elle. La même cause le rendait dangereux sur le
prochain, dont il parlait très librement. On le lui faisait remarquer
après. Il disait que c'étaient choses publiques qui n'apprenaient rien à
personne. S'il trouvait qu'il eût blessé les gens, il ne balançait pas à
leur aller demander pardon. Il reprit un jour un de ses curés d'avoir
été à une noce. Le curé se défendit sur l'exemple de Notre-Seigneur aux
noces de Cana. «\,Voyez-vous, monsieur le curé, répliqua-t-il, ce n'est
pas là ce qu'il a fait de mieux.\,» Quel blasphème dans une autre
bouche\,! Ce bon homme croyait fort bien répliquer et d'une manière
édifiante, et il est vrai aussi que de lui on le prenait de même.\,»

C'était un vrai pasteur, toujours résidant, tout occupé du soin de son
diocèse, de ses visites, de ses fonctions jusque tout à la fin de sa
vie, et avec plus d'esprit et de sens que Dieu ne lui en avait donné
pour tout le reste. Il était riche de patrimoine\,; son évêché l'était
aussi\,: il eut l'industrie de le doubler sans grever personne. Il
vivait fort honorablement, mais sans délicatesse, fort épiscopalement
avec modestie et avec économie. Au bout de l'année, il ne lui restait
pas un écu, et tout allait aux pauvres et en bonnes oeuvres. Tant que le
roi Jacques vécut en France, il lui donnait tous les ans dix mille écus,
et jamais on ne l'a su qu'après la mort de l'évêque non plus que
quantité d'autres oeuvres nobles et grandes qui faisaient marier et
subsister la pauvre noblesse de son diocèse. Ses gens le tenaient de
court tant qu'ils pouvaient sur les aumônes de sa poche, et lui les
trompait tant qu'il pouvait aussi pour donner. Allant à Paris, quelqu'un
lui dit qu'il prierait quelqu'un de ses gens de se charger de cent louis
d'or qu'il avait à payer à un tel à Paris. L'évêque répondit qu'il s'en
voulait charger lui-même, et n'eut point de patience qu'il ne les eût.
Par les chemins il donnait à tous les pauvres, aux hôpitaux, aux pauvres
couvents des lieux par où il passait. Ses gens n'imaginaient pas d'où il
avait pris de quoi faire des aumônes si abondantes. Elles furent au
point qu'il donna la dernière pistole avant d'arriver à Paris. Le
lendemain qu'il fut arrivé, il dit à celui qui avait soin de ses
affaires et qu'il savait avoir de l'argent à lui, d'aller porter cent
louis à un tel et ce fut par là que ses gens surent d'où étaient venues
les aumônes du voyage. Le roi, qui connaissoit sa vertu, le traitait
avec bonté, et une sorte de considération même dans le peu qu'il
paraissait devant lui, et le bon évêque était libre avec le roi, comme
s'il l'eût vu tous les jours. C'était le meilleur et le plus doux des
hommes, avec un air quelquefois grondeur, et le plus éloigné de toute
voie de fait et d'autorité. Nul bruit jamais dans son diocèse qu'il
laissa dans la plus profonde paix, et ses affaires en grand ordre. Sa
mort fut le désespoir des pauvres et l'affliction amère de tout son
diocèse. Il ne laissait pourtant pas d'être dangereux en vesperies, mais
ce n'était qu'avec des gens qu'il ne savait plus par où prendre, et ce
trait, entre beaucoup d'autres, montrera le zèle qui l'animait. Il avait
un procès considérable au parlement de Rouen, qui l'obligea d'y aller.
Un des premiers présidents à mortier, et qui par sa capacité et son
autorité menait le plus la grand'chambre et le reste de la compagnie,
avait chez lui une femme mariée qu'il entretenait publiquement, et il
avait forcé la sienne par ses mauvais traitements à se mettre dans un
couvent. Le bon évêque alla donc chez ce président, qui était un de ses
juges, pour l'entretenir de son affaire. Le portier dit qu'il n'y était
pas. Le prélat insista, le portier l'assura que le président était
sorti, mais que s'il voulait entrer et voir madame en l'attendant,
qu'elle y était. «\,Comment, madame, s'écria l'évêque, eh\,! de bon
cœur, ajoutat-il, je suis ravi de joie\,; et depuis quand est-elle
revenue chez M. le président\,? --- Mais ce n'est pas
M\textsuperscript{me} sa femme, répondit le portier, dont je parle,
c'est de madame\ldots. --- Fi, fi, fi, répliqua l'évêque avec feu, je ne
veux point entrer, c'est une vilaine, une vilaine, je vous le dis, une
vilaine que je ne veux pas voir, dites-le bien à M. le président de ma
part, et que cela est honteux à un magistrat comme lui de maltraiter,
comme il fait, M\textsuperscript{me} sa femme, une honnête femme et
vertueuse comme elle est, et donner ce scandale, et vivre avec cette
gueuse, et encore à son âge. Fi, fi, fi, cela est infâme, dites-le-lui
bien de ma part, encore une fois, et que je ne reviendrai pas ici.\,»
Voilà la belle sollicitation que fit ce bon homme. Le rare est qu'il
gagna son procès, et que ce président l'y servit à merveille. Il ne se
raccommoda pourtant pas avec lui. Ce conte fit rire toute la ville de
Rouen, et vint jusqu'à Paris. J'ai connu si peu d'évêques qui
ressemblassent à celui-ci que je n'ai pu me refuser tout cet article.

\hypertarget{chapitre-iv.}{%
\chapter{CHAPITRE IV.}\label{chapitre-iv.}}

1715

~

{\textsc{Mort du cardinal Sala.}} {\textsc{- Son extraction, sa fortune,
son caractère.}} {\textsc{- Bissy cardinal.}} {\textsc{- Extraction des
Bissy.}} {\textsc{- Trois autres cardinaux italiens.}} {\textsc{-
Extraction, caractère et fortune de Massei.}} {\textsc{- Moeurs et
caractère du nonce Bentivoglio.}} {\textsc{- Jésuites obtiennent un
arrêt qui rend leurs religieux renvoyés par leurs supérieurs capables de
revenir à partage dans leur famille jusqu'à l'âge de trente-trois ans.}}
{\textsc{- Majorque, etc., soumise au roi d'Espagne par le chevalier
d'Asfeld, qui en a la Toison.}} {\textsc{- Prostitution inouïe des
Toisons.}} {\textsc{- Rubi chef de la révolte de Catalogne\,; quel.}}
{\textsc{- Premier président marie sa seconde fille au fils d'Ambres.}}
{\textsc{- Succès de ce mariage.}} {\textsc{- Quelles étaient les deux
filles du premier président.}} {\textsc{- Mariage du duc de La
Rocheguyon avec M\textsuperscript{lle} de Toiras.}} {\textsc{-
Cellamare, ambassadeur d'Espagne, arrive à Paris, puis à Marly, où il
s'établit.}} {\textsc{- Petitesse du roi.}} {\textsc{-
Boulainvilliers\,; quel il était.}} {\textsc{- Son caractère\,; ses
prédictions vraies et fausses.}} {\textsc{- Voysin obtient six cent
mille livres de gratification sur le non-complet des troupes.}}
{\textsc{- Le roi veut aller faire enregistrer la constitution en lit de
justice sans modification.}} {\textsc{- Curieux entretien là-dessus par
ses suites entre M. le duc d'Orléans et moi, mais sans effet, parce que
le roi ne put aller au parlement.}} {\textsc{- Mort et caractère de
Chauvelin, avocat général\,; sa dépouille.}} {\textsc{- Sédition des
troupes sur le pain.}} {\textsc{- Belle fin et mort du maréchal Rosen.}}
{\textsc{- Duc d'Ormond se sauve d'Angleterre en France.}} {\textsc{-
Princesse des Ursins prend congé du roi à Marly, où je la vois pour la
dernière fois.}} {\textsc{- Incertitude de la princesse des Ursins où
fixer sa demeure.}} {\textsc{- Elle se hâte de gagner Lyon, puis
Chambéry\,; s'établit à Gênes, enfin à Rome.}} {\textsc{- Sa vie à Rome
jusqu'à sa mort.}}

~

Le cardinal Sala, prélat d'une autre trempe, mourut peu de jours après,
allant à Rome prendre son chapeau. C'était un Catalan de la lie du
peuple, qui se trouva de l'esprit et de l'ambition, et qui, pour se
tirer de sa bassesse et tenter la fortune, se fit bénédictin dans le
pays. Le hasard fit que l'archiduc étant venu à Barcelone, ses écuyers
prirent le père de Sala pour son cocher. Le fils chercha à mettre ce
hasard à profit, et à se faire connaître à l'archiduc, et compter par
ses ministres. Son esprit était tout à fait tourné à l'intrigue et à la
sédition. Il la jeta dans tous les monastères de la ville et de la
province, et y parut partout comme le chef, le conducteur et le plus
séditieux. Il rendit en effet de grands services à l'archiduc par sa
hardiesse et par l'adresse de ses manèges, tellement qu'il parut
nécessaire à ce prince d'élever Sala pour le mettre en état de servir
plus en grand. Cette considération le fit évêque de Girone. Ses progrès
séditieux furent tels dans cette dignité que l'archiduc le fit passer à
l'évêché de Barcelone, où il se rendit si considérable même à
l'archiduc, qu'il en obtint sa nomination au cardinalat, et de forcer le
pape, malgré sa juste répugnance pour un tel sujet, de le déclarer
cardinal, lorsque la prospérité des armes des alliés eut obligé le pape
de reconnaître enfin l'archiduc comme roi d'Espagne, et de n'oser
déplaire en rien à l'empereur.

Le roi d'Espagne se tint fort offensé de cette promotion, et proscrivit
Sala sans y avoir égard. Lorsque la Catalogne se trouva hors de moyens
de soutenir sa révolte, et que Barcelone se vit menacée d'un siège et
des châtiments de sa rébellion, les chefs, pour la plupart, gagnèrent
les montagnes, ou sortirent du pays. Sala s'embarqua et gagna Avignon
comme il put. Il y fut châtié par des infirmités qui l'y retinrent
presque toujours au lit, mais sans amortir l'esprit de sédition qui lui
était passé en nature. Il n'oublia rien pour retourner à Barcelone,
malgré le roi d'Espagne. L'empereur en pressa le pape de tout son
pouvoir, et le pape, qui redoutait sa puissance en Italie, et qui
n'ignorait pas l'affection de l'archiduc, lors empereur, pour Sala,
chercha à ébranler le roi d'Espagne par toutes sortes de voies, et ne
cessait de lui représenter la violence de tenir un évêque éloigné de son
troupeau, et banni de son diocèse. La fermeté du roi d'Espagne fit
trouver au pape un tempérament pour gagner du temps, sans offenser les
deux monarques. Ce fut d'ordonner à Sala de venir avant toutes choses
recevoir son chapeau. Il partit donc là-dessus d'Avignon, enragé de
n'avoir pu réussir à retourner à Barcelone, malgré le roi d'Espagne, et
se mit en chemin pour aller à Rome. Il mourut étant fort près d'y
arriver, et finit ainsi l'embarras du pape, de l'empereur et roi
d'Espagne, à son occasion. Le roi d'Espagne, maître de la Catalogne et
de Barcelone, y nomma sans difficulté un autre évêque, à qui le pape
envoya des bulles aussitôt après.

Cet honnête cardinal fut tout en même temps dignement remplacé dans le
sacré collège par un prélat de moins basse étoffe, d'autant de feu et
d'ambition, et à qui les moyens ne coûtèrent pas davantage pour arriver
à ce but de la dernière fortune ecclésiastique, auquel il travaillait
depuis si longtemps par toute espèce de moyens, qui ne furent peut-être
pas si ouvertement odieux, puisque les mêmes occasions n'existaient pas
pour lui, mais qui en autres genres n'en durent guère à ceux-là en
valeur intrinsèque, comme on en a vu divers traits répandus ici en
divers temps, et comme on en remarquera d'autres tous parfaitement
conformes à la prophétie qu'on a vue ici, {[}que{]} la parfaite
connaissance qu'avait son père de ce fils lui en avait fait faite. On
juge bien à ces derniers mots que je parle de Bissy, évêque de Meaux et
abbé de Saint-Germain des Prés, qui, par l'autorité du roi et les
intrigues intéressées des jésuites auxquels de toute sa vie il était
vendu corps et âme, parvint à faire consentir aux couronnes que sa
promotion fût avancée. Elle la fut donc de près de quatre ans, puisqu'il
fut fait lui quatrième avec trois Italiens, qui étaient un Caraccioli,
évêque d'Averse, illustre encore plus par la sainteté de sa vie que par
sa naissance, Scotti, gouverneur de Rome, et Marini, maître de chambre
du pape.

Massei, camérier, confident du pape, vint apporter la barrette an
nouveau cardinal. Massei était fils du trompette de la ville de
Florence\,; il était entré domestique du prélat Albano, dès sa jeunesse.
C'était un homme d'esprit et de sens, qui était de bonnes moeurs, sage
et mesuré. Ces qualités plurent à son maître qui, peu à peu, l'éleva
dans sa médiocre maison, et lui donna une confiance qui fut toujours
constante. Lé prélat Albano, devenu cardinal, le fit son maître de
chambre, puis camérier, lorsqu'il fut parvenu au souverain pontificat.
Je m'étends sur Massei, parce qu'il succéda enfin à Bentivoglio à la
nonciature de France, où il se fit aimer, estimer et considérer par ses
bonnes et droites intentions, et la sagesse et la mesure de sa conduite
{[}autant{]} que l'autre s'y était fait abhorrer comme le plus dangereux
fou, le plus séditieux, et le plus débauché prêtre, et le plus chien
enragé qui soit venu d'Italie, peut-être même pendant la Ligue.
Longtemps après Massei fut cardinal et fort regretté en France qu'il ne
quitta qu'avec larmes\,; et\,: où il aurait voulu passer sa vie, s'il
avait pu y avoir de quoi vivre avec dignité, et que le cardinalat eût pu
compatir avec la nonciature. Il n'en sorti pas avec moins d'estime
{[}pour retourner{]} à Rome, où tôt après il eut une des trois grandes
légations qu'il exerça avec la même capacité. Il vit encore avec la même
capacité à quatre-vingts ans, évêque d'Ancône. Il vint de l'abbaye
Saint-Germain des Près, dans une carrosse du roi, à Marly, le jeudi 18
juillet\,; il y présenta au roi à la fin de sa messe la barrette dans un
bassin de vermeil, qui la mit sur la tête de Bissy, lequel alla aussi
prendre l'habit rouge dans la sacristie, vint faire son remercîment au
roi à la porte de son cabinet, et s'en retourna avec Massei à Paris,
qu'il logea, voitura et défraya tant qu'il fut à Paris, suivant la
coutume.

Ces Bissy s'appellent Thiard, sont de Bourgogne, ont été petits juges,
puis conseillers aux présidiaux du Mâconnais et du Charolais, devinrent
lieutenants généraux de ces petites juridictions, acquirent Bissy qui
n'était rien, dont peu à peu ils firent une petite terre, et l'accrurent
après que leur petite fortune les eut portés dans les parlements de
Dijon et de Dôle\footnote{Le parlement de Franche-Comté fut établi dans
  les premiers temps à Dôle. Louis XIV le transféra à Besançon en 1676.},
où ils furent conseillers, puis présidents, et ont eu enfin un premier
président en celui de Dôle. Leur belle date est leur Pontus Thiard, né à
Bissy, en 1521\,; qui se rendit célèbre par les lettres, et dont le père
était lieutenant général de ces justices subalternes aux bailliages du
Mâconnais et Charolais. C'était un temps où les savants ranimés par
François Ier brillaient\,; celui-ci était le premier poète latin de son
temps, et en commerce avec tous les illustres. Cela lui valut l'évêché
de Châlon-sur-Saône, qu'il fit passer à son neveu. Ce premier président
du parlement de Dôle, dont les enfants quittèrent la robe, était le
grand-père du père du vieux Bissy, père du cardinal.

Les jésuites, transportés de voir désormais Bissy en état de figurer à
leur gré, eurent en même temps un autre sujet de grande joie. Il le faut
expliquer. Ils ont les trois vœux ordinaires à tous les religieux,
pauvreté, chasteté, obéissance, dont le dernier est rigoureusement
observé chez eux. La plupart en demeurent là, et ne vont pas jusqu'au
quatrième, où ils n'admettent qu'après un long examen de dévouement et
de talents\,; c'est un secret impénétrable. Eux-mêmes ne savent pas qui
d'entre eux est du quatrième vœu, et jusqu'à ceux qui y ont été admis ne
connaissent pas tous ceux qui l'ont fait. Jusqu'à ce quatrième vœu
exclusivement, les jésuites ne sont point liés à leurs religieux\,: ils
les peuvent renvoyer, et comme le réciproque n'y est pas, cela est d'un
grand avantage pour leur compagnie. Ceux-là seuls qui ont fait le
quatrième vœu sont réputés profès\,; les autres s'appellent parmi eux
coadjuteurs spirituels. Ces derniers ne sont exclus d'aucuns des emplois
qui ne sont pas importants au gouvernement secret, en sorte qu'il y en a
de ce degré qui sont même provinciaux\footnote{On appelait provinciaux,
  dans plusieurs ordres religieux, les supérieurs dont l'autorité
  s'étendait sur toutes les maisons de cet ordre comprises dans une
  certaine circonscription territoriale.}. Aucuns de ceux-là ne peuvent
quitter la compagnie, parce qu'ils ont fait les trois vœux solennels\,;
mais comme à son égard ils ne sont pas profès, parce qu'ils n'ont pas
fait le quatrième, la compagnie peut les renvoyer sans aucune forme, et
simplement par un ordre de se retirer et de quitter l'habit. Ainsi un
coadjuteur spirituel vieux, et ayant passé par les emplois, peut
toujours être renvoyé, et même sans savoir pourquoi.

L'inconvénient était de mettre à la mendicité des gens crus engagés par
leurs familles et qui avaient fait leurs partages sur ce pied-là,
autorisés par les lois qui réputent morts civilement ceux qui ont fait
les trois vœux solennels, où que ce puisse être, et qui n'ont point
réclamé contre dans les trois ans suivants juridiquement décidés
valables. Les jésuites avaient tenté d'y remédier à l'occasion d'un P.
d'Aubercourt qu'ils avaient renvoyé. Cela forma un grand procès où le
public était fort intéressé dans l'exception que les jésuites tentaient
d'usurper, parce qu'un jésuite, renvoyé de la sorte au bout de dix, de
vingt, de trente ans quelquefois, aurait ruiné sa famille par le rapport
de son partage et de tout ce qui pouvait être échu depuis de successions
et d'augmentations de biens dont il aurait eu sa part, et les intérêts,
comme s'il n'avait jamais fait de voeux. Les jésuites, qui n'espéraient
obtenir ce renversement dans aucun tribunal, eurent le crédit de faire
porter l'affaire devant le roi, qui, de son autorité et malgré tout ce
que purent dire presque tous les juges et le chancelier de
Pontchartrain, leur adjugea la plupart de ce qu'ils demandaient. J'en ai
parlé dans le temps. Le P. Tellier, voyant le roi menacer une ruine
prochaine, tenta d'obtenir le reste de ce qu'ils n'avaient pu obtenir
lors du procès d'Aubercourt. La demande fut comme l'autre fois portée
devant le roi, qui, comme l'autre fois, admit quelques conseillers
d'État pour être juges avec ses ministres en sa présence. Il y eut en
tout douze juges qui n'imitèrent pas tous les premiers. Grisenoire,
maître des requêtes, fort jeune, qui longtemps depuis a été garde des
sceaux, désigné chancelier et premier ministre, dont il fit les
fonctions sous le cardinal Fleury, qui, à la fin de sa vie, le dépouilla
et le chassa, fut rapporteur. Son âge, son ambition, sa qualité de fils
de Chauvelin, conseiller d'État, et plus encore de frère de Chauvelin,
avocat général au parlement, dévoué avec abandon aux jésuites, leur en
fit tout espérer. Il fit le plus beau rapport du monde, mais le plus
fort contre eux et le plus nerveux, qui lui fit d'autant plus d'honneur,
qu'on était plus éloigné de s'y attendre. Six furent de son avis, six
contre. Le roi fut pour ces derniers, et l'arrêt passa presque comme le
P. Tellier le voulait, sans nul égard au public ni au renversement des
familles\,: L'unique modération qui fut mise est la fixation de l'âge à
trente-trois ans, jusques auquel les jésuites renvoyés peuvent désormais
hériter, comme si jamais ils n'avaient été engagés\,; mais au delà de
cet âge, ils n'héritent plus. Il est vrai que cette fixation diminua la
joie des bons pères qui ne voulaient aucunes bornes à la faculté
d'hériter.

Le chevalier d'Asfeld, lieutenant général, qui longtemps depuis a été
maréchal de France, fut chargé de la réduction de l'île de Majorque, qui
n'a de ville que Majorque, appelée aussi Palma, qui est la capitale, et
Alcudia. Il débarqua à Portopedo avec douze bataillons espagnols, autant
de français, et huit cents chevaux, sans y trouver aucune résistance,
tandis qu'on préparait à Barcelone un pareil embarquement pour l'aller
joindre. Il alla assiéger Alcudia, où, dès que la tranchée fut ouverte,
les bourgeois obligèrent la garnison, qui n'était que de quatre cents
hommes, à se rendre. Palma n'attendit point d'être attaquée. Le marquis
de Rubi, principal chef de la révolte de Catalogne, y commandait et dans
toute l'île avec commission de l'empereur. Il livra une des portes,
obtint tous les honneurs de la guerre, et d'être transporté avec ses
troupes en Sardaigne, au lieu de Naples qu'il avait demandé. Il refusa
en se soumettant et acceptant l'amnistie du roi d'Espagne, de se retirer
chez lui avec la restitution de ses biens en Catalogne, qui n'était pas
grand'chose. C'était un fort petit gentilhomme, qui n'avait jamais servi
avant cette révolte, et qui fit mieux de demeurer attaché à l'empereur,
qui dans la suite lui donna des commandements considérables. Il avait
dans Palma un régiment des troupes de l'empereur, de douze cents hommes.
Il ne tint pas aux instances les plus pressantes d'un capitaine de
vaisseau anglais qui s'y trouva et à ses promesses du prompt et puissant
secours, d'engager les troupes et les bourgeois à se bien défendre. Les
îles Caprara, Dragonera et Iviça qui avait une place à cinq bastions, se
soumirent en même temps. Elles sont fort voisines de celle de Majorque,
et se trouvaient sous le même commandement.

Le roi d'Espagne, pour une expédition si facile, envoya la Toison au
chevalier d'Asfeld, que le roi lui permit d'accepter. Il était fils d'un
marchand de drap, dont la boutique et l'enseigne sont encore dans la
rue\footnote{Le nom est laissé en blanc dans le manuscrit.}\ldots. On a
vu l'extraction de Ducasse\,; Bay, fils d'un cabaretier de Besançon,
l'eut aussi. Ces nobles choix furent dans la suite comblés par celui
d'un homme de robe et de plume, ce qui n'a jamais été vu dans aucun
grand ordre. Morville, en qui ce rare exemple fut fait, en témoigna sa
reconnaissance par le renvoi de l'infante, qui se fit très indignement
un mois après, dont il fut le promoteur, jusqu'à soutenir en plein
conseil que le roi d'Espagne ne pouvait faire ni bien ni mal en Europe,
et que, sans nulle sorte de façons ni de précautions, il fallait lui
renvoyer sa fille, même par le coche, pour que cela fût plus tôt fait.
Il voulait plaire à M. le Duc, lors premier ministre.

On a vu la folie qui prit de l'un à l'autre de se promener les nuits au
Cours, et d'y donner quelquefois des soupers et des musiques. La même
fantaisie continua {[}cette année{]}\,; mais les indécences qui s'y
commirent, et quelque chose de pis, malgré les flambeaux que la plupart
des carrosses y portaient, firent défendre ces promenades nocturnes, et
qui cessèrent pour toujours au commencement de juillet.

Le premier président, qui était veuf, n'avait que deux filles. Elles
étaient riches\,; et, pour contenter les fantasques, l'une était noire,
huileuse, laide à effrayer, sotte et bégueule à l'avenant, dévote à
merveilles\,; l'autre rousse comme une vache, le teint blanc, de
l'esprit et du monde, et le désir de liberté et de primer. Quoique la
cadette, elle fut mariée la première à Lautrec, fils d'Ambres, qui avait
la bonté d'en être amoureux. Il fut mal payé de ses feux\,; jamais il ne
put adoucir sa belle, qui sentit à qui elle avait affaire, et qui sut
s'en avantager. Le pauvre mari en quitta le service et Paris, la vérité
est que ce ne fut pas une perte, et se confina en province. Ils n'eurent
point d'enfants. C'est le frère aîné de Lautrec, aujourd'hui lieutenant
général et chevalier de l'ordre, qui a épousé une sœur du duc de Rohan.

Le duc de La Rochefoucauld maria en même temps le duc de La Rocheguyon,
son fils, aujourd'hui duc de La Rochefoucauld, à M\textsuperscript{lle}
de Toiras, riche héritière, née et élevée en Languedoc, auprès de sa
mère, d'où elle n'était jamais sortie. Bâville, intendant ou plutôt roi
de cette province, fit ce mariage. Il était ami intime de la mère, et on
a vu la raison de l'intimité qui s'est entretenue entre MM. de La
Rochefoucauld et les Lamoignon, depuis l'adroite et hardie vérification
des lettres d'érection de La Rochefoucauld. Cette héritière était la
dernière de cette maison, et ne descendait point du maréchal de Toiras,
qui ne fut point marié. Sa grand'mère était Élisabeth d'Amboise,
comtesse d'Aubijoux, qui, par le hasard de son frère, qui fut tué en
duel par Boisdavid, hérita d'une partie de ses biens.

Le prince de Cellamare, ambassadeur d'Espagne, arriva à Paris. Quatre
jours après, il vint à Marly au lever du roi, qui lui donna aussitôt
après audience dans son cabinet. Il alla de là chez M. le duc d'Orléans,
à qui il présenta une lettre du roi d'Espagne fort obligeante, en
réponse de celle qu'on a vu que ce prince lui avait écrite. Fort peu
après\,; cet ambassadeur revint faire sa cour à Marly. Le roi lui promit
le premier logement qui y vaquerait. Ici et en Espagne, l'ambassadeur
est de droit de tous les voyages, comme ambassadeur de la maison.
M\textsuperscript{me} de Saint-Simon, qui avait besoin des eaux de
Forges, demanda la permission d'y aller peu de temps après. Nous étions
logés au premier pavillon en bas du côté de la chapelle. Le jour qu'elle
allait à Paris, nous fûmes surpris de voir arriver Bloin, comme nous
allions nous mettre à table, suivi de quelques garçons du garde-meuble.
Il me dit que le roi l'avait chargé de me prier de céder ce bas de
pavillon au prince de Cellamare, et d'aller dans un logement vis-à-vis
de la chapelle, en haut, sans expliquer comment il était vide. Il
m'assura que le roi voulait que je fusse bien et que j'y serais très
commodément. Il ajouta que le roi désirait que je déménageasse aussitôt
pour m'y établir, et qu'il en avait tant d'impatience, qu'il lui avait
ordonné d'amener des garçons du garde-meuble pour aider à mes gens à
tout transporter promptement. Nous dînâmes, M\textsuperscript{me} de
Saint-Simon partit, et je déménageai aussitôt. Mes gens me dirent que
quantité de garçons du garde-meuble étaient venus, et Bloin encore une
fois, et que tout avait été fait en un moment. Je ne savais à quoi
attribuer une telle précipitation\,: je le sus enfin en m'allant
coucher.

Mes gens me contèrent que j'étais dans le logement de Courtenvaux, qui
par sa charge de capitaine des Cent-Suisses en avait un fixe auprès de
ceux des autres charges de la chambre, garde-robe et chapelle\,; que sur
les dix heures une chaise de poste était arrivée. C'était Courtenvaux
qui, surpris de voir de la lumière dans sa chambre à travers les vitres,
avait envoyé savoir ce que c'était. Son laquais monta tout botté, qui
fut encore plus {[}surpris{]} de trouver là mes gens établis, et qui
l'alla dire à son maître. Il renvoya dire que c'était son logement, et
qu'il fallait bien qu'il y couchât. Mes gens contèrent à son valet la
façon dont j'avais déménagé, et répondirent qu'ils ne sortiraient point
de là, et que son maître n'avait qu'à aller trouver Bloin, et voir avec
lui ce qu'il deviendrait. Courtenvaux n'eut pas d'autre parti à prendre.
Bloin lui dit, de la part du roi, qu'il y avait dix-huit jours qu'il
était absent sans congé\,; que cela lui arrivait tous les voyages\,; que
le roi était las de cette liberté, et qu'il avait exprès rempli son
logement avec hâte pour qu'il n'y pût pas rentrer, lui apprendre à
vivre, et lui donner le dégoût d'être exclu de Marly pour le reste du
voyage. Voilà de ces petitesses dont la couronne n'affranchit point
l'humanité.

Le duc de Noailles était fort en liaison avec Boulainvilliers, et
m'avait fait faire connaissance avec lui. C'était un homme de qualité
qui se prétendait de la maison de Croï, qui n'était pas fort accommodé,
qui avait peu servi, et qui avait de l'esprit et beaucoup de lettres. Il
possédait extrêmement les histoires, celle de France surtout, à laquelle
il s'était fort appliqué, particulièrement à l'ancien génie et à
l'ancien gouvernement français, et aux divers degrés de sa déclinaison à
la forme présente. Il avait aussi creusé les généalogies du royaume, et
personne ne lui disputait sa capacité\,; et fort peu de gens sa
supériorité en ces deux gentes qu'une mémoire parfaite\,; exacte et
nette soutenait beaucoup. C'était un homme simple, doux, humble même par
nature, quoiqu'il se sentît fort, très éloigné de se targuer de rien,
qui expliquait volontiers ce qu'il savait sans chercher à rien montrer,
et dont la modestie était rare en tout genre. Mais il était curieux au
dernier point, et avait aussi l'esprit tellement libre, que rien n'était
capable de retenir sa curiosité. Il s'était donc adonné à l'astrologie,
et il avait la réputation d'y avoir très bien réussi. Il était fort
retenu sur cet article\,; il n'y avait que ses amis particuliers qui
pussent lui en parler et à qui il voulût bien répondre. Le duc de
Noailles était avide de cette sorte de curiosité, et y donnait, tant
qu'il pouvait trouver des gens qui passassent pour avoir de quoi la
satisfaire.

Boulainvilliers, dont la famille et les affaires étaient fort dérangées,
se tenait fort souvent en sa terre de Saint-Cère, vers la mer, au pays
de Caux, qui n'est pas fort éloigné de Forges. Il y vint voir des gens
de sa connaissance, et, je crois, écumer les nouvelles dont ses calculs
le rendaient curieux. Il y fut voir M\textsuperscript{me} de Saint-Simon
et la tourna tant pour apprendre des nouvelles du roi, qu'elle n'eut pas
peine à comprendre qu'il croyait en avoir trouvé de plus sûres que
celles qui s'en disaient. Elle lui fit connaître sa pensée\,; il se
défendit quelque temps, et à la fin il se rendit. Elle lui demanda donc
ce qu'il croyait de la santé du roi qui diminuait à vue d'œil, mais dont
la fin ne paraissait pas encore prochaine, et qui n'avait rien changé
dans le cours de ses journées, ni dans quoi que ce fût de sa manière
accoutumée de vivre. On était lors au 15 ou 16 août. Boulainvilliers ne
lui dissimula point qu'il ne croyait pas que le roi en eût encore pour
longtemps, et après s'être encore laissé presser, il lui dit qu'il
croyait qu'il mourrait le jour de Saint-Louis, mais qu'il n'avait pas
encore pu vérifier ses calculs avec assez d'exactitude pour en
répondre\,; que néanmoins il était assuré que le roi serait à
l'extrémité ce jour-là, et que s'il le passait, il mourrait certainement
le 3 septembre suivant. Deux jours après, voyant le roi s'affaiblir, je
mandai à M\textsuperscript{me} de Saint-Simon de revenir. Elle partit
aussitôt, et en arrivant me raconta ce que je viens de rapporter. Il
avait prédit, longtemps avant la mort du roi d'Espagne, que Monseigneur
ni aucun de ses trois fils ne régneraient en France. Il prévit de
plusieurs années la mort de son fils unique et la sienne à lui, que
l'événement vérifia, mais il se trompa lourdement sur beaucoup d'autres,
tels que le roi d'aujourd'hui, qu'il crut devoir mourir bientôt, et à
diverses reprises, le cardinal et la maréchale de Noailles, M. le duc de
Grammont et M. Le Blanc qui devaient être tués dans une sédition à
Paris, M. le duc d'Orléans mourir après deux ans de prison et sans en
être sorti. Je n'en citerai pas davantage de faux et de vrai\,; c'en est
assez pour montrer la fausseté, la vanité, le néant de cette prétendue
science qui séduit tant de gens d'esprit, et dont Boulainvilliers
lui-même, tout épris qu'il en fût, avait la bonne foi d'avouer qu'elle
n'était fondée sur aucun principe.

M. du Maine ne fut pas le seul à tirer tout le possible des derniers
temps de la vie du roi. Voysin l'avait assez bien servi pour en être
encore payé, outre les charges dans lesquelles il régnait, mais qui
étaient nécessaires au règne et à l'apothéose du duc du Maine et des
siens. Voysin voulait du bien, n'ayant plus de places ni d'honneurs à
prétendre. Il obtint deux cent mille écus sur le revenant-bon du
non-complet des troupes, qui excitèrent contre lui un cri universel qui
fut la moindre de ses inquiétudes\footnote{Passage omis dans les
  précédentes éditions depuis \emph{M. du Maine}, jusqu'à \emph{de ses
  inquiétudes}.}.

Le P. Tellier, qui n'avait pu venir à bout de son concile national, où
lui et Bissy se faisaient fort de faire recevoir la constitution, voyait
avec désespoir le risque qu'elle courait si le roi mourait avant qu'elle
fût reçue. Il y fit donc un dernier effort. Le roi manda plusieurs fois
là-dessus le premier président et le parquet à Marly. D'Aguesseau,
procureur général, était celui qui tenait le plus ferme. Mesmes, premier
président, nageait entre la cour et sa compagnie. Fleury, premier avocat
général, mettait tout son esprit et toute sa finesse, et personne
n'avait plus de l'un et de l'autre, à gagner du temps sans trop
s'opposer de front. Chauvelin, autre avocat général plein d'esprit, de
savoir, de lumières, n'avait de dieu ni de loi que sa fortune. Il était
vendu aux jésuites, et à tout ce qui la lui pouvait procurer et avancer.
Tellier, sûr de lui, l'avait mis dans la confiance secrète du roi qui le
mandait souvent depuis près d'un an, le faisait entrer par les
derrières, et travaillait secrètement tête à tête avec lui. Blancménil,
fils de Lamoignon, valet à tout faire, et comme tous les siens esclave
des jésuites, n'était pas pour payer d'autre chose que de courbettes. On
se doutait de quelque résolution violente sur quelques mots échappés au
roi, exprès sans doute pour intimider. La femme du procureur général,
sœur de d'Ormesson, exhorta son mari à être d'autant plus ferme qu'il se
trouvait mai accompagné, et comme il allait partir pour Marly, elle le
conjura, en l'embrassant, d'oublier qu'il eût femme et enfants, de
compter sa charge et sa fortune pour rien, et pour tout son honneur et
sa conscience. De si vertueuses paroles eurent leur effet. Il soutint le
choc presque seul. Il parla toujours avec tant de respect, de lumière et
de force que les autres n'osèrent l'abandonner, de manière que le roi,
outré d'une telle résistance, s'en prit tellement à lui, qu'il fut au
moment de perdre sa charge. Fleury, qui l'avait le mieux secondé, eut
toute la peur pour la sienne, mais cette violence, qui n'eût fait
qu'aigrir les esprits, ne faisait pas l'affaire du P. Tellier. Quoique
très sensible au charme de la vengeance, il ne voulut pas se détourner,
et fit tant auprès du roi qu'il força toutes ses presque invincibles
répugnances, et jusqu'à sa santé, de manière que le roi déclara qu'au
retour de Marly il irait à Paris tenir un lit de justice, et voir enfin
lui-même s'il aurait le crédit de faire enregistrer la constitution sans
modification. Il le manda au parlement, où la terreur se répandit, mais
non si générale que la chose ne pût être bien balancée, mais surtout à
la cour et dans le grand monde, où on ne s'entretenait plus d'autre
chose.

M. le duc d'Orléans, qui n'ignorait pas ce que je pensais sur la
constitution, et qui m'avait souvent dit ce qu'il en pensait lui-même,
me demanda ce que je ferais en cette occasion. Je lui répondis que le
devoir et le serment des pairs est précis sur l'obligation d'assister le
roi dans ses hautes et importantes affaires\,; qu'on était parvenu à
rendre telle une friponnerie d'école\,; que les pairs seraient invités à
ce lit de justice, comme ils le sont toujours de la part du roi par le
grand maître des cérémonies\,; que je ne balancerais donc pas à m'y
trouver. Qu'auparavant je ne laisserais en état d'être trouvé que ce que
je voudrais bien qui le fût\,; que je tiendrais quelque argent prêt et
ma chaise de poste\,; qu'après cela j'irais au lit de justice, et
qu'ayant ma conscience, mon honneur, les lois du royaume, justice et
vérité à garder et à en répondre, je me garderais bien d'opiner du
bonnet, mais que je parlerais de toute ma force contre la constitution,
son enregistrement, sa réception, avec tout le respect pour le roi et
pour son autorité et toutes les mesures que j'y pourrais mettre, bien
persuadé en même temps que je ne retournerais pas de la séance chez moi,
et que je m'en tiendrais quitte à bon marché pour l'exil, si je n'allais
à la Bastille. À cette prompte réponse, M. le duc d'Orléans, qui me
connaissoit trop bien pour douter de la vérité et de la fermeté de ma
résolution, me regarda un moment, puis m'embrassa, et me dit qu'il était
ravi de me savoir ce parti pris\,; que non seulement il l'approuvait,
mais qu'il en userait tout de même, avec cette différence, dont tout le
poids ne l'ébranlerait pas, qu'il parlerait d'une place qui n'avait rien
entre le roi et lui, qui ne perdrait pas un mot de son discours, le
regarderait depuis les pieds jusqu'à la tête, et frémirait tellement de
colère de se voir ainsi résister en face par lui, qu'il ne savait tout
ce qu'il lui en pourrait arriver. Nous nous en reparlâmes plusieurs
fois, nous affermissant réciproquement l'un l'autre jusqu'à ce que, de
retour à Versailles, et le roi sur le point d'aller au parlement, sa
santé ne le lui permit pas, et le lit de justice tomba, et
l'enregistrement qu'il avait si à cœur. Je ne me serais pas étendu sur
une résolution qui ne put avoir lieu, si je n'avais cru également
important et curieux de montrer M. le duc d'Orléans tel qu'il se montra
lui-même à moi, pour le voir après tel qu'il fut sur cette même matière,
toute la même et sans changement, sinon plus développée, plus évidente
et, s'il était possible, encore plus odieuse à tous égards.

Chauvelin, avocat général, mourut incontinent après de la petite vérole,
ainsi que son ami Rothelin, et un troisième qui avaient soupé ensemble
la veille que ce mal leur prit, qui les tua au troisième jour. Ce
magistrat, qui visait à la plus haute fortune, décoré, chose sans
exemple au parquet, d'une charge de l'ordre du Saint-Esprit, initié dans
la plus grande confiance du roi d'affaires secrètes, et qui, pour ne
s'en pas éloigner et se ménager, avait refusé la commission de Rome qui
fut donnée à Amelot, avait une figure agréable, beaucoup d'esprit,
d'adresse, d'intrigue, de capacité et de ressources, une éloquence
aisée, une grande facilité à concevoir, à travailler, à s'énoncer, à
parler sur-le-champ. Trop ambitieux pour s'arrêter aux moyens de la
satisfaire, trop touché des plaisirs pour y trouver une barrière dans sa
santé et son travail. Il était encore dans la première jeunesse, et se
tua ainsi avant le temps. Il ne laissa qu'une fille, mariée depuis au
président Talon, et un fils devenu président à mortier. Son père eut la
permission de vendre sa charge de l'ordre, et obtint la charge d'avocat
général pour son second fils Grisenoire, qu'on vient de voir rapporteur
de l'affaire des jésuites, qui ne le lui avaient pas pardonné. C'est le
même qui a eu les sceaux sous le cardinal de Fleury, et dont l'élévation
et la chute ont été proportionnées. Le père, conseiller d'État, était un
fort bon homme\,: je ne sais où ses deux fils avaient pris tant
d'ambition.

Voysin, dont la dureté et l'incapacité étaient égales, et qui pouvait
avoir ses raisons personnelles de favoriser les munitionnaires, força
les troupes, malgré toutes sortes de représentations, de prendre le pain
de munition, et à plus haut prix qu'aux marchés. Peu à peu il se fit une
traînée d'intelligence de sédition dans les garnisons, depuis Strasbourg
jusqu'aux places maritimes de Flandre, qui éclata tout à coup, et où
quelques officiers furent tués en voulant imposer à leurs soldats.
Heureusement du Bourg, qui commandait à Strasbourg et en Alsace, et qui
fut bien secondé par les officiers de tous rangs, l'étouffa dans sa
naissance, en faisant distribuer de l'argent aux troupes, mais en les
obligeant aussi à prendre le pain, pour n'en avoir pas le démenti. Cet
exemple porta coup sur toute la traînée\,; tout fut apaisé, mais avec de
l'argent partout, et peu à peu on ne les força plus à prendre le pain.

Le maréchal Rosen mourut à quatre-vingt-huit ans, sain de corps et
d'esprit jusqu'à cet âge. On l'a fait connaître lors de sa promotion au
bâton. Il ne commanda jamais d'armée, et il n'en était pas capable, mais
souvent des ailes, de gros détachements, et la cavalerie dont il fut
longtemps mestre de camp général, et tout cela avec capacité. Il était
ordinairement chargé d'assembler l'armée à l'ouverture des campagnes.
Fâcheux souvent à cheval, emporté pour rien, et pour cela évité des
officiers principaux\,; à pied et à table qu'il tenait grande et
délicate le meilleur homme du monde, doux, poli, prévenant, généreux,
serviable, et fort libre de sa bourse à qui en avait besoin\,; toujours
singulièrement bien monté. C'était un grand homme, fort maigre, qui
avait extrêmement l'air d'un homme de guerre, et qui parlait un jargon
partie français et allemand. Il avait de l'esprit et de la finesse\,: il
avait connu le faible du roi et de ses ministres pour les étrangers\,;
il reprochait à son fils de parler trop bien français, qui d'ailleurs
était un pauvre homme, mais brave, et qui est mort lieutenant général.
Il l'avait marié à une Grammont, de Franche-Comté, qui se trouva une
très habile femme pour le dedans et pour le dehors, qui s'attacha fort à
lui, et qu'il aima beaucoup\,; avec cela sage et vertueuse. Après la
paix de Ryswick, il se retira dans une terre qu'il avait en haute
Alsace, dont il avait fort bien accommodé le château et les jardins. Sa
belle-fille tenait la maison, et y avait toujours bonne compagnie\,: le
maréchal n'en sortit plus qu'une fois l'année pour venir voir le roi qui
le recevait toujours avec distinction, et passer huit ou dix jours au
plus à Paris ou à la cour. Il se bâtit ensuite une petite maison au bout
de ses jardins, où il se retira vers quatre-vingts ans, pour ne plus
songer qu'à son salut. Il voyait quelquefois la compagnie au château, et
se retirait promptement chez lui, passant sa journée en exercices de
piété, en bonnes œuvres, et à prendre l'air à pied ou à cheval. On ne
peut faire une fin plus digne, plus sage ni plus chrétienne\,; c'était
aussi un fort honnête homme.

En ce même temps la persécution était extrême en Angleterre contre les
principaux torys, surtout contre les ministres de la reine Anne, et tous
ceux qui avaient eu part à sa paix. On en a déjà parlé ailleurs. Le
comte d'Oxford, qui avait été grand trésorier, et dont la cour voulait
avoir la tête, se défendit si puissamment lui-même à la barre du
parlement, et en même temps si noblement, que, contre toute espérance,
il se tira d'affaire. Le duc d'Ormond, non moins menacé, se trouva
investi dans sa maison de Richemont près de Londres. Il se sauva, passa
en France, et arriva en ce temps-ci à Paris.

L'état du roi, dont la santé baissait à vue d'œil, fit peur à la
princesse des Ursins de se trouver peut-être tout à coup sous la main de
M. le duc d'Orléans. Elle songea donc tout de bon à s'y dérober, sans
savoir néanmoins encore où elle fixerait sa demeure, et fit demander au
roi la permission de venir prendre congé de lui à Marly. Elle y vint de
Paris le mardi 6 août, mesurée pour arriver à l'heure de la sortie du
dîner du roi, c'est-à-dire sur les deux heures. Elle fut aussitôt admise
dans le cabinet du roi, avec qui elle demeura plus d'une bonne
demi-heure tête à tête. Elle passa tout de suite dans celui de
M\textsuperscript{me} de Maintenon, avec qui elle fut une heure, et de
là s'en alla monter en carrosse pour s'en retourner à Paris. Je ne sus
qu'elle prenait congé que par son arrivée à Marly, où j'étais en peine
de la pouvoir rencontrer. Le hasard fit que je m'avisai d'aller chercher
son carrosse pour m'informer à ses gens de ce qu'elle devenait dans
Marly, et un autre hasard l'y fit arriver en chaise comme je leur
parlais. Elle me parut fort aise de me rencontrer, et me fit monter avec
elle dans son carrosse, où nous ne demeurâmes guère moins d'une heure à
nous entretenir fort librement. Elle ne me dissimula point ses craintes,
la froideur qu'elle avait sentie pour elle dans ses deux audiences, à
travers toute la politesse que le roi et M\textsuperscript{me} de
Maintenon lui avaient témoignée, le vide qu'elle trouvait à la cour, et
même à Paris, enfin l'incertitude où elle était encore sur le choix de
sa demeure, tout cela avec détail et néanmoins sans plaintes, sans
regrets, sans faiblesse, toujours mesurée, toujours comme s'il se fût
agi d'une autre, et supérieure aux événements. Elle toucha légèrement
l'Espagne, le crédit et l'ascendant même que la reine y prenait sur le
roi, me faisant entendre que cela ne pouvait être autrement, coulant
légèrement et modestement sur la reine, se louant toujours des bontés du
roi d'Espagne. La crainte du spectacle des passants lui fit mettre fin à
notre conversation. Elle me fit mille amitiés et son regret de
l'abréger, me promit de m'avertir avant son départ, pour me donner
encore une journée, me dit mille choses pour M\textsuperscript{me} de
Saint-Simon, et me témoigna être sensible à la marque d'amitié que je
lui donnais là, malgré l'engagement où j'étais avec M. le duc d'Orléans.
Dès que je l'eus vue partir, j'allai chez M. le duc d'Orléans à qui je
dis ce que je venais de faire\,; que ce n'était point une visite, mais
une rencontre\,; qu'il était vrai que je n'avais pu m'empêcher de la
chercher, sans préjudice de la visite du départ qu'il m'avait permise.
Lui et M\textsuperscript{me} la duchesse d'Orléans ne le trouvèrent
point mauvais\,; ils avaient en plein triomphé d'elle, et ils étaient
sur le point de la voir sortir de France pour toujours, et sans espoir
en Espagne.

Jusqu'alors M\textsuperscript{me} des Ursins, amusée par un reste d'amis
ou de connaissances grossi par ceux de M. de Noirmoutiers chez qui elle
logeait, et qui en avait beaucoup, s'était lentement occupée à
l'arrangement de ses affaires dans un si grand changement, et à retirer
ses effets d'Espagne. La frayeur de se pouvoir trouver fort promptement
sous la main d'un prince qu'elle avait si cruellement offensé, et qui
lui montrait depuis son arrivée en France qu'il le sentait, précipita
toutes ses mesures. Sa terreur s'augmenta par le changement prodigieux
qu'elle trouva dans le roi en cette dernière audience, depuis celle
qu'elle en avait eue à son arrivée. Elle ne douta plus que sa fin ne fût
très prochaine, et toute son attention ne se tourna plus qu'à la
prévenir et à être bien avertie sur une santé qu'elle croyait faire
uniquement sa sûreté en France. Effrayée de nouveau par les avis qu'elle
en reçut, elle ne se donna plus le temps de rien\,; et partit
précipitamment le 14 d'août, accompagnée de ses deux neveux jusqu'à
Essonne. Elle n'eut pas le loisir de penser à m'avertir, de sorte que
depuis notre conversation à Marly dans son carrosse, je ne l'ai plus
revue. Elle ne respira que lorsqu'elle fut arrivée à Lyon.

Elle avait abandonné le projet de se retirer en Hollande\,; où les États
généraux ne la voulaient point. Elle en fut elle-même dégoûtée par
l'égalité et l'unisson d'une république qui contrebalança en elle le
plaisir de la liberté dont on y jouit. Mais elle ne pouvait se résoudre
à retourner à Rome, théâtre où elle avait régné autrefois, et de s'y
remontrer proscrite, vieille, comme dans un asile. Elle craignait encore
d'y être mal reçue, après la nonciature fermée en Espagne et les démêlés
qu'il y avait eu entre les deux cours. Elle y avait perdu beaucoup
d'amis et de connaissances\,; tout y était renouvelé depuis quinze ans
d'absence, et elle sentait tout l'embarras qu'elle y trouverait à
l'égard des ministres de l'empereur et des deux couronnes, et de leurs
principaux partisans. Turin n'était pas une cour digne d'elle\,; le roi
de Sardaigne n'en avait pas toujours été content, et ils en savaient
trop tous deux l'un pour l'autre. À Venise, elle n'eût su que faire ni
que devenir. Agitée de la sorte sans avoir pu se déterminer, elle apprit
l'extrémité du roi, toujours grossie par les nouvelles. La peur la
saisit de se trouver à sa mort dans le royaume. Elle partit à l'instant,
sans savoir où aller, et uniquement pour en sortir elle alla à Chambéry,
comme au lieu de sûreté le plus proche, et y arriva hors d'haleine. Ce
lieu fut sa première station. Elle s'y donna le loisir de choisir où se
fixer et de s'arranger pour s'y établir. Tout bien examiné, elle préféra
Gênes\,; la liberté lui en plut\,; le commerce d'une riche et nombreuse
noblesse, la beauté du lieu et du climat, une manière de centre et de
milieu entre Madrid, Paris et Rome, où elle entretenait toujours du
commerce, et était affamée de tout ce qui s'y passait. Le renversement
de tant de si grandes réalités et de desseins plus hauts encore, n'avait
pu venir à bout de ses espérances, bien moins de ses désirs. Déterminée
enfin pour Gênes, elle y passa. Elle y fut bien reçue, elle espéra y
fixer ses tabernacles, elle y passa quelques années\,; mais à la fin
l'ennui la gagna, peut-être le dépit de n'y être pas assez comptée. Elle
ne pouvait vivre sans se mêler, et de quoi se mêler à Gênes quand on est
femme et surannée\,? Elle tourna donc toutes ses pensées vers Rome\,;
elle en sonda la cour, elle se rapprocha avec effort de son frère le
cardinal de La Trémoille, réchauffa ce qui lui était d'ancien commerce,
renoua avec qui elle put décemment, tâta le pavé partout, mais sur
toutes choses fut attentive à s'assurer du traitement qu'elle recevrait
de tout ce qui tenait à la France et à l'Espagne. Elle quitta donc Gênes
et retourna dans son nid.

Elle n'y fut pas longtemps sans s'attacher au roi et à la reine
d'Angleterre, et ne s'y attacha pas longtemps sans les gouverner et
bientôt à découvert. Quelle triste ressource\,! Mais enfin c'était une
idée de cour et un petit fumet d'affaires pour qui ne s'en pouvait plus
passer. Elle acheva ainsi sa vie dans une grande santé de corps et
d'esprit et dans une prodigieuse opulence, qui n'était pas inutile aussi
à cette déplorable cour. Du reste, médiocrement considérée à Rome,
nullement comptée, désertée de ce qui sentait l'Espagne, médiocrement
visitée de ce qui était français, mais sans rien essayer de la part du
régent, bien payée de la France et de l'Espagne, toujours occupée du
monde, de ce qu'elle avait été, de ce qu'elle n'était plus, mais sans
bassesse, avec courage et grandeur. La perte qu'elle fit, en janvier
1720, du cardinal de La Trémoille, ne laissa pas, sans amitié de part ni
d'autre, de lui faire un vide. Elle le survécut de trois ans, conserva
toute sa santé, sa force, son esprit jusqu'à la mort, et fut emportée, à
plus de quatre-vingts ans, par une fort courte maladie, à Rome, le 5
décembre 1722\,; Elle eut le plaisir de voir M\textsuperscript{me} de
Maintenon oubliée et anéantie dans Saint-Cyr, et celui de lui survivre,
et la joie de voir arriver, l'un après l'autre, à Rome, ses deux ennemis
aussi profondément disgraciés qu'elle, dont l'un tombait d'aussi haut,
les cardinaux del Giudice et Albéroni, et de jouir de la parfaite
inconsidération, pour ne pas dire mépris, où ils tombèrent tous deux.
Cette mort qui, quelques années plus tôt, eût retenti par toute
l'Europe, ne fit pas la plus légère sensation. La petite cour
d'Angleterre la regretta, quelques amis particuliers dont je fus du
nombre et ne m'en cachai point, quoique, à cause de M. le duc d'Orléans,
demeuré sans commerce avec elle\,; du reste, personne ne sembla s'être
aperçu qu'elle fût disparue. Ce fut néanmoins une personne si
extraordinaire dans tout le cours de sa longue vie, et qui a partout si
grandement et si singulièrement figuré, quoique en diverses manières\,;
dont l'esprit, le courage, l'industrie et les ressources ont été si
rares\,; enfin le règne si absolu en Espagne et si à découvert, et le
caractère si soutenu et si unique, que sa vie mériterait d'être écrite,
et tiendrait place entre les plus curieux morceaux de l'histoire des
temps où elle a vécu.

\hypertarget{chapitre-v.}{%
\chapter{CHAPITRE V.}\label{chapitre-v.}}

~

{\textsc{Nécessité d'interrompre un peu le reste si court de la vie du
roi.}} {\textsc{- Première partie du caractère de M. le duc d'Orléans.}}
{\textsc{- Débonnaireté et son histoire.}} {\textsc{- Malheur de
l'éducation et de la jeunesse de M. le duc d'Orléans.}} {\textsc{- Folie
de l'abbé Dubois, qui le perd auprès du roi pour toujours.}} {\textsc{-
Caractère de l'abbé depuis cardinal Dubois.}} {\textsc{- Deuxième partie
du caractère de M. le duc d'Orléans.}} {\textsc{- M. le duc d'Orléans
excellemment peint par Madame.}} {\textsc{- Aventure du faux marquis de
Ruffec.}} {\textsc{- Quel était M. le duc d'Orléans sur la religion.}}
{\textsc{- Caractère de M\textsuperscript{me} la duchesse d'Orléans.}}
{\textsc{- Saint-Pierre et sa femme\,; leur caractère.}} {\textsc{-
Duchesse Sforce.}} {\textsc{- Courte digression sur les Sforce.}}
{\textsc{- Caractère de la duchesse Sforce.}}

~

Le règne de Louis XIV, conduit jusqu'à sa dernière extrémité, ne laisse
plus à rapporter maintenant que ce qui s'est passé dans le dernier mois
de sa vie, encore au plus. Ces derniers événements, si curieux et si
importants à exposer dans la plus exacte vérité et netteté, et dans leur
ordre le plus exact, sont tellement liés avec ceux qui suivent
immédiatement la mort de ce monarque, qu'il n'est pas possible de les
séparer. Il n'est pas moins curieux et nécessaire aussi d'exposer les
projets, les pensées, les difficultés, les différents partis qui
roulèrent dans la tête du prince qui allait nécessairement être à la
tête du royaume pendant la minorité, quelques mesures que
M\textsuperscript{me} de Maintenon et le duc du Maine eussent pu prendre
pour ne lui laisser que le nom de régent, et ce qu'ils n'avaient pu lui
ôter, et quelle sorte d'administration il voulut établir. C'est donc ici
le lieu d'expliquer tant de choses, après quoi on reprendra la narration
du dernier mois de la vie du feu roi, et des choses qui l'ont suivie.
Mais avant d'entrer dans cette épineuse carrière, il est à propos de
faire bien connaître, si l'on peut, celui qui en est le premier
personnage, ses entraves intérieures et extérieures, et tout ce qui lui
appartient personnellement. Je dis si l'on peut, parce que je n'ai de ma
vie rien connu de si éminemment contradictoire et si parfaitement en
tout que M. le duc d'Orléans. On s'apercevra aisément qu'encore que je
le visse à nu depuis tant d'années, qu'il ne se cachât pas à moi, que
j'aie été dans ces dernières années-ci le seul homme qui le voulût voir,
et l'unique avec lequel il pût s'ouvrir et s'ouvrit en effet à cœur
ouvert et par confiance et par nécessité, on sentira, dis-je, que je ne
le connaissois pas encore, et que lui-même aussi ne se connaissoit pas
parfaitement. Pour le tableau de la cour, des personnages, des desseins,
des brigues, des partis, il se trouve tout fait par tout ce qui a été
raconté et expliqué jusqu'ici. En se le rappelant on verra d'un coup
d'œil quelle était la cour de Louis XIV en ces derniers temps de sa vie,
et le détail mis au jour de toutes les différentes parties de tout le
groupe de ce spectacle.

M. le duc d'Orléans était de taillé médiocre au plus, fort plein, sans
être gros, l'air et le port aisé et fort noble, le visage large,
agréable, fort haut en couleur, le poil noir et la perruque de même.
Quoiqu'il eût fort mal dansé, et médiocrement réussi à l'académie, il
avait dans le visage, dans le geste, dans toutes ses manières une grâce
infinie, et si naturelle qu'elle ornait jusqu'à ses moindres actions, et
les plus communes. Avec beaucoup d'aisance quand rien ne le
contraignait, il était doux, accueillant, ouvert, d'un accès facile et
charmant, le son de la voix agréable, et un don de la parole qui lui
était tout particulier en quelque genre que ce pût être, avec une
facilité et une netteté que rien ne surprenait, et qui surprenait
toujours. Son éloquence était naturelle jusque dans les discours les
plus communs et les plus journaliers, dont la justesse était égale sur
les sciences les plus abstraites qu'il rendait claires, sur les affaires
du gouvernement, de politique, de finance, de justice, de guerre, de
cour, de conversation ordinaire, et de toutes sortes d'arts et de
mécanique. Il ne se servait pas moins utilement des histoires et des
Mémoires, et connaissoit fort les maisons. Les personnages de tous les
temps et leurs vies lui étaient présents, et les intrigues des anciennes
cours comme celles de son temps. À l'entendre, on lui aurait cru une
vaste lecture. Rien moins. Il parcourait légèrement, mais sa mémoire
était si singulière qu'il n'oubliait ni choses, ni noms, ni dates, qu'il
rendait avec précision\,; et son appréhension était si forte qu'en
parcourant ainsi, c'était en lui comme s'il eût tout lu fort exactement.
Il excellait à parler sur-le-champ, et en justesse et en vivacité, soit
de bons mots, soit de reparties. Il m'a souvent reproché, et d'autres
plus que lui, que je ne le gâtais pas, mais je lui ai souvent aussi
donné une louange qui est méritée par bien peu de gens, et qui
n'appartenait à personne si justement qu'à lui\,: c'est qu'outre qu'il
avait infiniment d'esprit et de plusieurs sortes, la perspicacité
singulière du sien se trouvait jointe à une si grande justesse, qu'il ne
se serait jamais trompé en aucune affaire s'il avait suivi la première
appréhension de son esprit sur chacune. Il prenait quelquefois cette
louange de moi pour un reproche, et il n'avait pas toujours tort, mais
elle n'en était pas moins vraie. Avec cela nulle présomption, nulle
trace de supériorité d'esprit ni de connaissance, raisonnant comme
d'égal à égal avec tous, et donnant toujours de la surprise aux plus
habiles. Rien de contraignant ni d'imposant dans la société, et
quoiqu'il sentît bien ce qu'il était, et de façon même de ne le pouvoir
oublier en sa présence, il mettait tout le monde à l'aise, et lui-même
comme au niveau des autres.

Il gardait fort son rang en tout genre avec les princes du sang, et
personne n'avait l'air, le discours, ni les manières plus respectueuses
que lui ni plus nobles avec le roi et avec les fils de France. Monsieur
avait hérité en plein de la valeur des rois ses père et grand-père, et
l'avait transmise tout entière à son fils. Quoiqu'il n'eût aucun
penchant à la médisance, beaucoup moins à ce qu'on appelle être méchant,
il était dangereux sur la valeur des autres. Il ne cherchait jamais à en
parler, modeste et silencieux même à cet égard sur ce qui lui était
personnel, et racontait toujours les choses de cette nature où il avait
eu le plus de part, donnant avec équité toute louange aux autres et ne
parlant jamais de soi, mais il se passait difficilement de pincer ceux
qu'il ne trouvait pas ce qu'il appelait francs du collier, et on lui
sentait un mépris et une répugnance naturelle à l'égard de ceux qu'il
avait lieu de croire tels. Aussi avait-il le faible de croire ressembler
en tout à Henri IV, de l'affecter dans ses façons, dans ses reparties,
de se le persuader jusque dans sa taille et la forme de son visage, et
de n'être touché d'aucune autre louange ni flatterie comme de celle-là
qui lui allait au cœur. C'est une complaisance à laquelle je n'ai jamais
pu me ployer. Je sentais trop qu'il ne recherchait pas moins cette
ressemblance dans les vices de ce grand prince que dans ses vertus, et
que les uns ne faisaient pas moins son admiration que les autres. Comme
Henri IV, il était naturellement bon, humain, compatissant, et cet homme
si cruellement accusé du crime le plus noir et le plus inhumain, je n'en
ai point connu de plus naturellement opposé au crime de la destruction
des autres, ni plus singulièrement éloigné de faire peine même à
personne, jusque-là qu'il se peut dire que sa douceur, son humanité, sa
facilité avaient tourné en défaut, et je ne craindrai pas de dire qu'il
tourna en vice la suprême vertu du pardon des ennemis, dont la
prodigalité sans cause ni choix tenait trop près de l'insensible, et lui
a causé bien des inconvénients fâcheux et des maux dont la suite
fournira des exemples et des preuves.

Je me souviens qu'un an peut-être avant la mort du roi, étant monté de
bonne heure après dîner chez M\textsuperscript{me} la duchesse d'Orléans
à Marly, je la trouvai au lit pour quelque migraine, et M. le duc
d'Orléans seul dans la chambre, assis dans le fauteuil du chevet du lit.
À peine fus-je assis que M\textsuperscript{me} la duchesse d'Orléans se
mit à me raconter un fait du prince et du cardinal de Rohan, arrivé
depuis peu de jours, et prouvé avec la plus claire évidence. Il roulait
sur des mesures contre M. le duc d'Orléans pour le présent et l'avenir,
et sur le fondement de ces exécrables imputations si à la mode par le
crédit et le cours que M\textsuperscript{me} de Maintenon et M. du Maine
s'appliquaient sans cesse à leur donner. Je me récriai d'autant plus que
M. le duc d'Orléans avait toujours distingué et recherché, je ne sais
pourquoi, ces deux frères, et qu'il croyait pouvoir compter sur eux\,:
«\,Et que dites-vous de M. le duc d'Orléans, ajouta-t-elle ensuite, qui,
depuis qu'il le sait, qu'il n'en doute pas, et qu'il n'en peut -douter,
leur fait tout aussi bien qu'à l'ordinaire\,?» À l'instant je regardai
M. le duc d'Orléans qui n'avait dit que quelques mots pour confirmer le
récit de la chose à mesure qu'il se faisait, et qui était couché
négligemment dans sa chaise, et je lui dis avec feu\,: «\,Pour cela,
monsieur, il faut dire la vérité, c'est que depuis Louis le Débonnaire
il n'y en eut jamais un si débonnaire que vous.\,» À ces mots, il se
releva dans sa chaise, rouge de colère jusqu'au blanc des yeux,
balbutiant de dépit contre moi qui lui disais, prétendait-il, des choses
fâcheuses, et contre M\textsuperscript{me} la duchesse d'Orléans qui les
lui avait procurées, et qui riait. «\,Courage, monsieur, ajoutai-je,
traitez bien vos ennemis, et fâchez-vous contre vos serviteurs. Je suis
ravi de vous voir en colère, c'est signe que j'ai mis le doigt sur
l'apostume\,; quand on la presse, le malade crie. Je voudrais en faire
sortir tout le pus, et après cela vous seriez tout un autre homme et
tout autrement compté.\,» Il grommela encore un peu et puis s'apaisa.
C'est là une des deux occasions seules où il se soit jamais mis en vraie
colère contre moi. Je rapporterai l'autre en son temps.

Deux ou trois ans après la mort du roi, je causais à un coin de la
longue et grande pièce de l'appartement des Tuileries, comme le conseil
de régence allait commencer dans cette même pièce où il se tenait
toujours tandis que M. le duc d'Orléans était tout à l'autre bout,
parlant à quelqu'un, dans une fenêtre. Je m'entendis appeler comme de
main en main\,; on me dit que M. le duc d'Orléans me voulait parler.
Cela arrivait souvent en se mettant au conseil. J'allai donc à cette
fenêtre où il était demeuré. Je trouvai un maintien sérieux, un air
concentré, un visage fâché qui me surprit beaucoup. «\,Monsieur me
dit-il d'abordée, j'ai fort à me plaindre de vous que j'ai toute ma vie
compté pour le meilleur de mes amis. --- Moi, monsieur\,! plus étonné
encore, qu'y a-t-il donc, lui dis-je, s'il vous plaît\,? --- Ce qu'il y
a, répondit-il avec une mine encore plus colère, chose que vous ne
sauriez nier, des vers que vous avez faits contre moi. --- Moi, des
vers\,! répliquai-je\,; eh\,! qui diable vous conte de ces
sottises-là\,? et depuis près de quarante ans que vous me connaissez,
est-ce que vous ne savez pas que de ma vie je n'ai pu faire, non pas
deux vers, mais un seul\,? --- Non, par\ldots, reprit-il, vous ne pouvez
nier ceux-là, et tout de suite me chante un \emph{pont-neu} à sa louange
dont le refrain était\,: \emph{Notre régent est débonnaire, la, la, il
est débonnaire}, avec un grand éclat de rire. --- Comment, lui dis-je,
vous vous en souvenez encore\,! et en riant aussi, pour la vengeance que
vous en prenez, souvenez-vous-en du moins à bon escient.\,» Il demeura à
rire longtemps, à ne s'en pouvoir empêcher avant de se mettre au
conseil. Je n'ai pas craint d'écrire cette bagatelle, parce qu'il me
semble qu'elle peint.

Il aimait fort la liberté, et autant pour les autres que pour lui-même.
Il me vantait un jour l'Angleterre sur ce point, où il n'y a point
d'exils ni de lettres de cachet, et le roi ne peut défendre que l'entrée
de son palais ni tenir personne en prison, et sur cela me conta en se
délectant, car tous nos princes vivaient lors, qu'outre la duchesse de
Portsmouth, Charles II avait bien eu de petites maîtresses\,; que le
grand prieur, jeune et aimable en ce temps-là, qui s'était fait chasser
pour quelque sottise, était allé passer son exil en Angleterre, où il
avait été fort bien reçu du roi. Pour le remercîment, il lui débaucha
une de ces petites maîtresses dont le roi était si passionné alors,
qu'il lui fit demander grâce, lui offrit de l'argent, et s'engagea de le
raccommoder en France. Le grand prieur tint bon. Charles lui fit
défendre le palais. Il s'en moqua et allait tous les jours à la comédie
avec sa conquête, et s'y plaçait vis-à-vis du roi. Enfin le roi
d'Angleterre, ne sachant plus que faire pour s'en délivrer, pria
tellement le roi de le rappeler en France, qu'il le fut. Mais le grand
prieur tint bon, dit qu'il se trouvait bien en Angleterre, et continua
son manège. Charles outré en vint jusqu'à faire confidence au roi de
l'état où le mettait le grand prieur, et obtint un commandement si
absolu et si prompt qu'il le fit repasser incontinent en France. M. le
duc d'Orléans admirait cela, et je ne sais s'il n'aurait pas voulu être
le grand prieur. Je lui répondis que j'admirais moi-même que le
petit-fils d'un roi de France se pût complaire dans un si insolent
procédé que moi sujet, et qui, comme lui, n'avais aucun trait au trône,
je trouvais plus que scandaleux et extrêmement punissable. Il n'en
relâcha rien, et faisait toujours cette histoire avec volupté. Aussi
d'ambition de régner ni de gouverner n'en avait-il aucune. S'il fit une
pointe tout à fait insensée pour l'Espagne, c'est qu'on la lui avait
mise dans la tête. Il ne songea même, comme on le verra, tout de bon à
gouverner que lorsque force fut d'être perdu et déshonoré, ou d'exercer
les droits de sa naissance\,; et, quant à régner, je ne craindrai pas de
répondre que jamais il ne le désira, et que, le cas forcé arrivé, il
s'en serait trouvé également importuné et embarrassé. Que voulait-il
donc\,? me demandera-t-on\,; commander les armées tant que la guerre
aurait duré, et se divertir le reste du temps sans contrainte ni à lui
ni à autrui.

C'était en effet à quoi il était extrêmement propre. Une valeur
naturelle, tranquille, qui lui laissait tout voir, tout prévoir, et
porter les remèdes, une grande étendue d'esprit pour les échecs d'une
campagne, pour les projets, pour se munir de tout ce qui convenait à
l'exécution, pour s'en aider à point nommé, pour s'établir d'avance des
ressources et savoir en profiter bout à bout, et user aussi avec une
sage diligence et vigueur de tous les avantages que lui pouvait
présenter le sort des armes. On peut dire qu'il était capitaine,
ingénieur, intendant d'armée, qu'il connaissoit la force des troupes, le
nom et la capacité des officiers, et les plus distingués de chaque
corps, {[}savait{]} s'en faire adorer, les tenir néanmoins en
discipline, exécuter, en manquant de tout, les choses les plus
difficiles. C'est ce qui a été admiré en Espagne, et pleuré en Italie,
quand il y prévit tout, et que Marsin lui arrêta les bras sur tout. Ses
combinaisons étaient justes et solides tant sur les matières de guerre
que sur celles d'État\,; il est étonnant jusqu'à quel détail il en
embrassait toutes les parties sans confusion, les avantages et les
désavantages des partis qui se présentaient à prendre, la netteté avec
laquelle il les comprenait et savait les exposer, enfin la variété
infinie et la justesse de toutes ses connaissances sans en montrer
jamais, ni avoir en effet meilleure opinion de soi.

Quel homme aussi au-dessus des autres, et en tout genre connu\,! et quel
homme plus expressément formé pour faire le bonheur de la France,
lorsqu'il eut à la gouverner\,! Ajoutons-y une qualité essentielle,
c'est qu'il avait plus de trente-six ans à la mort des Dauphins et près
de trente-huit à celle de M. le duc de Berry, qu'il avait passés
particulier, éloigné entièrement de toute idée de pouvoir arriver au
timon\,; courtisan battu des orages et des tempêtes, et qui avait vécu
de façon à connaître tous les personnages, et la plupart de ce qui ne
l'était pas\,; en un mot l'avantage d'avoir mené une vie privée avec les
hommes, et acquis toutes les connaissances, qui, sans cela, ne se
suppléent point d'ailleurs. Voilà le beau, le très beau sans doute et le
très rare. Malheureusement il y a une contre-partie qu'il faut
maintenant exposer, et ne craindre pas quelque légère répétition, pour
le mieux faire, de ce qu'on a pu voir ailleurs.

Ce prince si heureusement né pour être l'honneur et le chef-d'oeuvre
d'une éducation, n'y fut pas heureux. Saint-Laurent, homme de peu, qui
n'était même chez Monsieur que sous-introducteur des ambassadeurs, fut
le premier à qui il fut confié. C'était un homme à choisir par
préférence dans toute l'Europe pour l'éducation des rois. Il mourut
avant que son élève fût hors de sous la férule, et par le plus grand des
malheurs, sa mort fut telle et si prompte qu'il n'eut pas le temps de
penser en quelles mains il le laissait, ni d'imaginer qui s'y ancrerait
en titre. On a vu (t. Ier, p.~20) que ce fut l'abbé Dubois, comment il y
parvint, combien il s'introduisit avant dans l'amitié et la confiance
d'un enfant qui ne connaissoit personne, et l'énorme usage qu'il en sut
faire pour espérer fortune et acquérir du pain. Le précepteur sentait
qu'il ne tiendrait pas longtemps par cette place, et tout le poids
d'avoir été l'instrument du consentement qu'il surprit au jeune prince
pour son mariage, lequel ne lui avait pas rendu ce qu'il en avait
espéré, et qui l'avait même perdu auprès du roi par la folie qu'il eut,
dans une audience secrète qu'il en obtint, de lui demander pour prix de
son service la nomination au chapeau. Il se vit donc réduit à M. de
Chartres, et ne pensa plus qu'à le gouverner. Il a fait un si grand
personnage depuis la mort du roi, qu'il est nécessaire de le faire
connaître. On y reviendra bientôt.

Monsieur, qui était fort glorieux et gâté encore par avoir eu un
gouverneur devenu duc et pair dans sa maison, et dont la postérité
successive, décorée de la même dignité, était demeurée dans la charge de
premier gentilhomme de sa chambre, et par celle de dame d'honneur de
Madame, remplie par la duchesse de Ventadour, voulut des gens titrés
pour gouverneurs de M. son fils. Cela n'était pas aisé, mais il en
trouva, et ne considéra guère autre chose. M. de Navailles fut le
premier qui accepta. Il était duc à brevet et maréchal de France, plein
de vertu, d'honneur et de valeur, et avait figuré autrefois, mais ce
n'était pas un homme à élever un prince. Il y fut peu et mourut en
février 1684, à soixante-cinq ans. Le maréchal d'Estrades lui succéda,
qui en aurait été fort capable, mais il était fort vieux, et mourut en
février 1686, à soixante-dix-neuf ans. M. de La Vieuville, duc à brevet,
le fut après, qui mourut en février 1689, un mois après avoir été fait
chevalier de l'ordre. Il n'avait rien de ce qu'il fallait pour cet
emploi, mais ce fut une perte pour Monsieur, qui ne trouva plus de gens
titrés qui en voulussent. Saint-Laurent, qui avait toute sa confiance,
avait aussi toute l'autorité effective, et suppléait à ces messieurs,
qui n'étaient que \emph{ad honores}. Les deux sous-gouverneurs étaient
La Bertière, brave et honnête gentilhomme, mais dont le prince ne
s'embarrassait guère, quoiqu'il l'estimât, et Fontenay, qui en était
extrêmement capable, mais qui avait au moins quatre-vingts ans. Il avait
élevé le comte de Saint-Paul tué au passage du Rhin, sur le point d'être
élu roi de Pologne, dont le fameux Sobieski profita. Le marquis d'Arcy
fut le dernier gouverneur. Il avait passé par des ambassades avec
réputation, et servi de même. C'était un homme de qualité\,; qui le
sentait fort, chevalier de l'ordre de 1688. Son frère aîné l'avait été
en 1661. D'Arcy était aussi conseiller d'État d'épée. On a vu ailleurs
comment il se conduisit dans cet emploi, surtout à la guerre. Sa mort
arrivée à Maubeuge, en juin 1694, fut le plus grand malheur qui pût
arriver à son élève, sur qui il avait pris non seulement toute autorité,
mais toute confiance, et à qui toutes ses manières et sa conduite
plaisaient et lui inspiraient une grande estime, qui en ce genre ne va
point sans déférence.

Le prince n'ayant plus ce sage mentor, qu'on a vu qu'il a toujours
regretté, ainsi que le maréchal d'Estrades, et qui l'a toute sa vie
marqué à tout ce qui est resté d'eux, tomba tout à fait entre les mains
de l'abbé Dubois et des jeunes débauchés qui l'obsédèrent. Les exemples
domestiques de la cour de Monsieur, et ce que de jeunes gens sans
réflexion, las du joug, tout neufs, sans expérience, regardent comme le
bel air dont ils sont les esclaves, et souvent jusque malgré eux,
effacèrent bientôt ce que Saint-Laurent et le marquis d'Arcy lui avaient
appris de bon. Il se laissa entraîner à la débauche et à la mauvaise
compagnie, parce que la bonne, même de ce genre, craignait le roi, et
l'évitait. Marié par force et avec toute l'inégalité qu'il sentit trop
tard, il se laissa aller à écouter des plaisanteries de gens obscurs
qui, pour le gouverner, le voulaient à Paris\,; il en fit à son tour, et
se croyant autorisé par le dépit que Monsieur témaignait de ne pouvoir
obtenir pour lui ni gouvernement qui lui avait été promis, ni
commandement d'armée, il ne mit plus de bornes à ses discours ni à ses
débauches, partie facilité, partie ennui de la cour, vivant comme il
faisait avec M\textsuperscript{me} sa femme, partie chagrin de voir M.
le Duc, et bien plus M. le prince de Conti en possession de ce qu'il y
avait de plus brillante compagnie, enfin dans le ruineux dessein de se
moquer du roi, de lui échapper, de le piquer à son tour, et de se venger
ainsi de n'avoir ni gouvernement ni armée à commander. Il vivait donc
avec des comédiennes et leurs entours, dans une obscurité honteuse, et à
la cour tout le moins qu'il pouvait. L'étrange est que Monsieur le
laissait faire par ce même dépit contre le roi, et que Madame, qui ne
pouvait pardonner au roi ni à M\textsuperscript{me} sa belle-fille son
mariage, désapprouvant la vie que menait M. son fils, ne lui en parlait
presque point intérieurement ravie des déplaisirs de
M\textsuperscript{me} sa belle-fille, et du chagrin qu'en avait le roi.

La mort si prompte et si subite de Monsieur changea les choses. On a vu
tout ce qui arriva M. le duc d'Orléans, content et n'ayant plus Monsieur
pour bouclier, vécut quelque temps d'une façon plus convenable, et avec
assiduité à la cour, mieux avec M\textsuperscript{me} sa femme par les
mêmes raisons, mais toujours avec un éloignement secret qui ne finit que
quand je les raccommodai, lorsque je le séparai de M\textsuperscript{me}
d'Argenton\,: l'amour et l'oisiveté l'attachèrent à cette maîtresse qui
l'éloigna de la cour. Il voyait chez elle des compagnies qui le
voulaient tenir, de concert avec elle, dont l'abbé Dubois était le grand
conducteur. En voilà assez pour marquer les tristes routes qui ont gâté
un si beau naturel. Venons maintenant aux effets qu'a produits ce long
et pernicieux poison, ce qui ne se peut bien entendre qu'après avoir
fait connaître celui à qui il le dut presque en entier.

L'abbé Dubois était un petit homme maigre, effilé, chafouin, à perruque
blonde, à mine de fouine, à physionomie d'esprit, qui était en plein ce
qu'un mauvais français appelle un sacre, mais qui ne se peut guère
exprimer autrement. Tous les vices combattaient en lui à qui en
demeurerait le maître. Ils y faisaient un bruit et un combat continuel
entre eux. L'avarice, la débauche, l'ambition étaient ses dieux\,; la
perfidie, la flatterie, les servages, ses moyens\,; l'impiété parfaite,
son repos\,; et l'opinion que la probité et l'honnêteté sont des
chimères dont on se pare, et qui n'ont de réalité dans personne, son
principe, en conséquence duquel tous moyens lui étaient bons. Il
excellait en basses intrigues, il en vivait, il ne pouvait s'en passer,
mais toujours avec un but où toutes ses démarches tendaient, avec une
patience qui n'avait de terme que le succès, ou la démonstration
réitérée de n'y pouvoir arriver, à moins que, cheminant ainsi dans la
profondeur et les ténèbres, il ne vit jour à mieux en ouvrant un autre
boyau. Il passait ainsi sa vie dans les sapes. Le mensonge le plus hardi
lui était tourné en nature avec un air simple, droit, sincère, souvent
honteux. Il aurait parlé avec grâce et facilité, si, dans le dessein de
pénétrer les autres en parlant, la crainte de s'avancer plus qu'il ne
voulait ne l'avait accoutumé à un bégayement factice qui le déparait, et
qui, redoublé quand il fut arrivé à se mêler de choses importantes,
devint insupportable, et quelquefois inintelligible. Sans ses contours
et le peu de naturel qui perçait malgré ses soins, sa conversation
aurait été aimable. Il avait de l'esprit, assez de lettres, d'histoire
et de lecture, beaucoup de monde, force envie de plaire et de
s'insinuer, mais tout cela gâté par une fumée de fausseté qui sortait
malgré lui de tous ses pores et jusque de sa gaieté, qui attristait par
là. Méchant d'ailleurs avec réflexion et par nature, et, par
raisonnement, traître et ingrat, maître expert aux compositions des plus
grandes noirceurs, effronté à faire peur étant pris sur le fait\,;
désirant tout, enviant tout, et voulant toutes les dépouilles. On connut
après, dès qu'il osa ne se plus contraindre, à quel point il était
intéressé, débauché, inconséquent, ignorant en toute affaire, passionné
toujours, emporté, blasphémateur et fou, et jusqu'à quel point il
méprisa publiquement son maître et l'État, le monde sans exception et
les affaires, pour les sacrifier à soi tous et toutes, à son crédit, à
sa puissance, à son autorité absolue, à sa grandeur, à son avarice, à
ses frayeurs, à ses vengeances. Tel fut le sage à qui Monsieur confia
les moeurs de son fils unique à former, par le conseil de deux hommes
qui ne les avaient pas meilleures, et qui en avaient bien fait leurs
preuves.

Un si bon maître ne perdit pas son temps auprès d'un disciple tout neuf
encore, et en qui les excellents principes de Saint-Laurent n'avaient
pas eu le temps de prendre de fortes racines, quelque estime et quelque
affection qu'il ait conservée toute sa vie pour cet excellent homme. Je
l'avouerai ici avec amertume, parce que tout doit être sacrifié à la
vérité, M. le duc d'Orléans apporta au monde une facilité, appelons les
choses par leur nom, une faiblesse qui gâta sans cesse tous ses talents,
et qui fut à son précepteur d'un merveilleux usage toute sa vie. Hors de
toute espérance du côté du roi depuis la folie d'avoir osé lui demander
sa nomination au cardinalat, il ne songea plus qu'à posséder son jeune
maître par la conformité à soi. Il le flatta du côté des mœurs pour le
jeter dans la débauche, et lui en faire un principe pour se bien mettre
dans le monde, jusqu'à mépriser tous devoirs et toutes bienséances, ce
qui le ferait bien plus ménager par le roi qu'une conduite mesurée\,; il
le flatta du côté de l'esprit, dont il le persuada qu'il en avait trop
et trop bon pour être la dupe de la religion, qui n'était, à son avis,
qu'une invention de politique, et de tous les temps, pour faire peur aux
esprits ordinaires et retenir les peuples dans la soumission. Il
l'infatua encore de son principe favori que la probité dans les hommes
et la vertu dans les femmes ne sont que des chimères sans réalité dans
personne, sinon dans quelques sots en plus grand nombre qui se sont
laissé imposer ces entraves comme celles de la religion, qui en sont des
dépendances, et qui pour la politique sont du même usage, et fort peu
d'autres qui ayant de l'esprit et de la capacité se sont laissé
raccourcir l'un et l'autre par les préjugés de l'éducation. Voilà le
fond de la doctrine de ce bon ecclésiastique, d'où suivait la licence de
la fausseté, du mensonge, des artifices, de l'infidélité, de la
perfidie, de toute espèce de moyens, en un mot, tout crime et toute
scélératesse tournés en habileté, en capacité, en grandeur, liberté et
profondeur d'esprit, de lumière et de conduite, pourvu qu'on sût se
cacher et marcher à couvert des soupçons et des préjugés communs.

Malheureusement tout concourut en M. le duc d'Orléans à lui ouvrir le
coeur et l'esprit à cet exécrable poison. Une neuve et première
jeunesse, beaucoup de force et de santé, les élans de la première sortie
du joug et du dépit de son mariage et de son oisiveté, l'ennui qui suit
la dernière, cet amour, si fatal en ce premier âge, de ce bel air qu'on
admire aveuglément dans les autres, et qu'on veut imiter et surpasser,
l'entraînement des passions, des exemples et des jeunes gens qui y
trouvaient leur vanité et leur commodité, quelques-uns leurs vues à le
faire vivre comme eux et avec eux. Ainsi il s'accoutuma à la débauche,
plus encore au bruit de la débauche jusqu'à n'avoir pu s'en passer, et
qu'il ne s'y divertissait qu'à force de bruit, de tumulte et d'excès.
C'est ce qui le jeta à en faire souvent de si étranges et de si
scandaleuses, et comme il voulait l'emporter sur tous les débauchés, à
mêler dans ses parties les discours les plus impies et à trouver un
raffinement précieux à faire les débauches les plus outrées, aux jours
les plus saints, comme il lui arriva pendant sa régence plusieurs fois
le vendredi saint de choix et les jours les plus respectables. Plus on
était suivi, ancien, outré en impiété et en débauche, plus il
considérait cette sorte de débauchés, et je l'ai vu sans cesse dans
l'admiration poussée jusqu'à la vénération pour le grand prieur, parce
qu'il y avait quarante ans qu'il ne s'était couché qu'ivre, et qu'il
n'avait cessé d'entretenir publiquement des maîtresses et de tenir des
propos continuels d'impiété et d'irréligion. Avec de tels principes et
la conduite en conséquence, il n'est pas surprenant qu'il ait été faux
jusqu'à l'indiscrétion de se vanter de l'être, et de se piquer d'être le
plus raffiné trompeur.

Lui et M\textsuperscript{me} la duchesse de Berry disputaient
quelquefois qui des deux en savait là-dessus davantage, et quelquefois à
sa toilette devant M\textsuperscript{me} de Saint-Simon, et ce qui y
était avant le public, et M. le duc de Berry même, qui était fort vrai
et qui en avait horreur, et sans que M\textsuperscript{me} de
Saint-Simon, qui n'en souffrait pas moins et pour la chose et pour
l'effet, pût la tourner en plaisanterie, ni leur faire sentir la porte
pour sortir d'une telle indiscrétion. M. le duc d'Orléans en avait une
infinie dans tout ce qui regardait la vie ordinaire et sur ce qui le
regardait lui-même. Ce n'était pas injustement qu'il était accusé de
n'avoir point de secret. La vérité est qu'élevé dans les tracasseries du
Palais-Royal, dans les rapports, dans les redits dont Monsieur vivait et
dont sa cour était remplie, M. le duc d'Orléans en avait pris le
détestable goût et l'habitude, jusqu'à s'en être fait une sorte de
maxime de brouiller tout le monde ensemble, et d'en profiter pour
n'avoir rien à craindre des liaisons, soit pour apprendre par les aveux,
les délations et les piques, et par la facilité encore de faire parler
les uns contre les autres. Ce fut une de ses principales occupations
pendant tout le temps qu'il fut à la tête des affaires, et dont il se
sut le plus de gré, mais qui, tôt découverte, le rendit odieux et le
jeta en mille fâcheux inconvénients. Comme il n'était pas méchant, qu'il
était même fort éloigné de l'être, il demeura dans l'impiété et la
débauche où Dubois l'avait premièrement jeté, et que tout confirma
toujours en lui par l'habitude, dans la fausseté, dans la tracasserie
des uns aux autres, dont qui que ce soit ne fut exempt, et dans la plus
singulière défiance qui n'excluait pas en même temps et pour les mêmes
personnes, de la plus grande confiance\,; mais il en demeura là sans
avoir rien pris du surplus des crimes familiers à son précepteur.

Revenu plus assidûment à la cour, à la mort de Monsieur, l'ennui l'y
gagna et le jeta dans les curiosités de chimie dont j'ai parlé ailleurs,
et dont on sut faire contre lui un si cruel usage. On a peine à
comprendre à quel point ce prince était incapable de se rassembler du
monde, je dis avant que l'art infernal de M\textsuperscript{me} de
Maintenon et du duc du Maine l'en eût totalement séparé\,; combien peu
il était en lui de tenir une cour\,; combien avec un air désinvolte il
se trouvait embarrassé et importuné du grand monde, et combien dans son
particulier, et depuis dans sa solitude au milieu de la cour quand tout
le monde l'eut déserté, il se trouva destitué de toute espèce de
ressource avec tant de talents, qui en devaient être une inépuisable
d'amusements pour lui. Il était né ennuyé, et il était si accoutumé à
vivre hors de lui-même, qu'il lui était insupportable d'y rentrer, sans
être capable de chercher même à s'occuper. Il ne pouvait vivre que dans
le mouvement et le torrent des affaires, comme à la tête d'une armée, ou
dans les soins d'y avoir tout ce dont il aurait besoin pour les
exécutions de la campagne, ou dans le bruit et la vivacité de la
débauche. Il y languissait dès qu'elle était sans bruit et sans une
sorte d'excès et de tumulte, tellement que son temps lui était pénible à
passer. Il se jeta dans la peinture après que le grand goût de la chimie
fut passé ou amorti par tout ce qui s'en était si cruellement publié. Il
peignait presque toute l'après-dînée à Versailles et à Marly. Il se
connaissoit fort en tableaux, il les aimait, il en ramassait et il en
fit une collection qui en nombre et en perfection ne le cédait pas aux
tableaux de la couronne. Il s'amusa après à faire des compositions de
pierres et de cachets à la merci du charbon, qui me chassait souvent
d'avec lui, et des compositions de parfums les plus forts qu'il aima
toute sa vie, et dont je le détournais, parce que le roi les craignait
fort, et qu'il sentait presque toujours. Enfin jamais homme né avec tant
de talents de toutes les sortes, tant d'ouverture et de facilité pour
s'en servir, et jamais vie de particulier si désœuvrée ni si livrée au
néant et à l'ennui. Aussi Madame ne le peignit-elle pas moins
heureusement qu'avait fait le roi par l'apophtegme qu'il répondit sur
lui à Maréchal, et que j'ai rapporté.

Madame était pleine de contes et de petits romans de fées. Elle disait
qu'elles avaient toutes été conviées à ses couches, que toutes y étaient
venues, et que chacune avait doué son fils d'un talent, de sorte qu'il
les avait tous\,; mais que par malheur on avait oublié une vieille fée
disparue depuis si longtemps qu'on ne se souvenait plus d'elle, qui,
piquée de l'oubli, vint appuyée sur son petit bâton et n'arriva qu'après
que toutes les fées eurent fait chacune leur don à l'enfant\,; que,
dépitée de plus en plus, elle se vengea en le douant de rendre
absolument inutiles tous les talents qu'il avait reçus de toutes les
autres fées, d'aucun desquels, en les conservant tous, il n'avait jamais
pu se servir. Il faut avouer qu'à prendre la chose en gros le portrait
est parlant.

Un des malheurs de ce prince était d'être incapable de suite dans rien,
jusqu'à ne pouvoir comprendre qu'on en pût avoir. Un autre, dont j'ai
déjà parlé, fut une espèce d'insensibilité qui le rendait sans fiel dans
les plus mortelles offenses et les plus dangereuses\,; et comme le nerf
et le principe de la haine et de l'amitié, de la reconnaissance et de la
vengeance est le même, et qu'il manquait de ce ressort, les suites en
étaient infinies et pernicieuses. Il était timide à l'excès, il le
sentait et il en avait tant de honte qu'il affectait tout le contraire,
jusqu'à s'en piquer. Mais la vérité était, comme on le sentit enfin dans
son autorité par une expérience plus développée, qu'on n'obtenait rien
de lui, ni grâce ni justice, qu'en l'arrachant par crainte, dont il
était infiniment susceptible, ou par une extrême importunité. Il tâchait
de s'en délivrer par des paroles, puis par des promesses, dont sa
facilité le rendait prodigue, mais que qui avait de meilleures serres
lui faisait tenir. De là tant de manquements de paroles qu'on ne
comptait plus les plus positives pour rien, et tant de paroles encore
données à tant de gens pour la même chose qui ne pouvait s'accorder qu'à
un seul, ce qui était une source féconde de discrédit et de mécontents.
Rien ne le trompa et rie lui nuisit davantage que cette opinion qu'il
s'était faite de savoir tromper tout le monde. On ne le croyait plus,
lors même qu'il parlait de la meilleure foi, et sa facilité diminua fort
en lui le prix de toutes choses. Enfin la compagnie obscure, et pour la
plupart scélérate, dont il avait fait sa société ordinaire de débauche,
et que lui-même ne feignait pas de nommer publiquement ses \emph{roués},
chassa la bonne jusque dans sa puissance et lui fit un tort infini.

Sa défiance sans exception était encore une chose infiniment dégoûtante
avec lui, surtout lorsqu'il fut à la tête des affaires, et le monstrueux
unisson à ceux de sa familiarité hors de débauche. Ce défaut, qui le
mena loin, venait tout à la fois de sa timidité, qui lui faisait
craindre ses ennemis les plus certains, et les traiter avec plus de
distinctions que ses amis\,; de sa facilité naturelle\,; d'une fausse
imitation d'Henri IV, dont cela même n'est ni le plus beau ni le
meilleur endroit\,; et de cette opinion malheureuse que la probité était
une parure fausse, sans réalité, d'où lui venait cette défiance
universelle. Il était néanmoins très persuadé de la mienne, jusque-là
qu'il me l'a souvent reprochée comme un défaut et un préjugé d'éducation
qui m'avait resserré l'esprit et accourci les lumières\,; et il m'en a
dit autant de M\textsuperscript{me} de Saint-Simon, parce qu'il la
croyait vertueuse. Je lui avais aussi donné des preuves d'attachement
trop fortes, trop fréquentes, trop continuelles dans les temps les plus
dangereux, pour qu'il en pût douter\,; et néanmoins voici ce qui
m'arriva dans la seconde ou troisième année de la régence, et je le
rapporte comme un des plus forts coups de pinceau, et si\footnote{\emph{Quoique}
  mon désintéressement lui eût été mis en évidence.} dès lors mon
désintéressement lui avait été mis en évidence par les plus fortes
coupelles\footnote{Épreuves. Le mot \emph{coupelle} désigne, au sens
  propre, un vase dont on se sert pour purifier, par l'action du feu,
  les métaux de tout alliage.}, comme on le verra par la suite.

On était en automne. M. le duc d'Orléans avait congédié les conseils
pour une quinzaine. J'en profitai pour aller passer ce temps à la
Ferté\,; je venais de passer une heure seul avec lui, j'en avais pris
congé et j'étais revenu chez moi, où, pour être en repos, j'avais fermé
ma porte. Au bout d'une heure au plus, on me vint dire que Biron était à
la porte, qu'il ne se voulait point laisser renvoyer, et qu'il disait
qu'il avait ordre de M. le duc d'Orléans, qui l'envoyait, de me parler
de sa part. Il faut ajouter que mes deux fils avaient chacun un régiment
de cavalerie, et que tous les colonels étaient lors par ordre à leurs
corps. Je fis entrer Biron avec d'autant plus de surprise, que je ne
faisais que de quitter M. le duc d'Orléans. Je demandai donc avec
empressement ce qu'il y avait de si nouveau. Biron fut embarrassé, et à
son tour s'informa où était le marquis de Ruffec. Ma surprise fut encore
plus grande\,; je lui demandai ce que cela voulait dire. Biron, de plus
en plus empêtré, m'avoua que M. le duc d'Orléans en était inquiet, et
l'envoyait à moi pour le savoir. Je lui dis qu'il était à son régiment
comme tous les autres, et logé dans Besançon chez M. de Lévi, qui
commandait en Franche-Comté. «\, Mais, me dit Biron, je le sais bien\,;
n'auriez-vous point quelque lettre de lui ? --- Pourquoi faire\,?
répondis-je. --- C'est que franchement, puisqu'il vous faut tout dire,
M. le duc d'Orléans, me répondit-il, voudrait voir de son écriture. » Il
m'ajouta que peu après que je l'eus quitté, il était descendu dans le
petit jardin de M\textsuperscript{me} la duchesse d'Orléans, laquelle
était à Montmartre\,; que la compagnie ordinaire, c'est-à-dire les roués
et les p\ldots\ldots{} s'y promenaient avec lui\,; qu'il était venu un
commis de la poste avec des lettres, à qui il avait parlé quelque temps
en particulier\,; qu'après cela il avait appelé lui Biron, lui avait
montré une lettre datée de Madrid du marquis de Ruffec à sa mère, et que
là-dessus il lui avait donné sa commission de me venir trouver.

À ce récit je sentis un mélange de colère et de compassion, et je ne
m'en contraignis pas avec Biron. Je n'avais point de lettres de mon
fils, parce que je les brûlais à mesure comme tous papiers inutiles. Je
chargeai Biron de dire à M. le duc d'Orléans une partie de ce que je
sentais\,; que je n'avais pas la plus légère connaissance avec qui que
ce fût en Espagne, et le lieu où mon fils était\,; que je le priais
instamment de dépêcher sur-le-champ un courrier à Besançon, pour le
mettre en repos par ce qu'il lui rapporterait. Biron, haussant les
épaules, me dit que tout cela était bel et bon, mais que, si je
retrouvais quelque lettre du marquis de Ruffec, il me priait de la lui
envoyer sur-le-champ, et qu'il mettrait ordre qu'elle lui parvint même à
table, malgré l'exacte clôture de leurs soupers. Je ne voulus pas
retourner au Palais-Royal pour y faire une scène, et je renvoyai Biron.
Heureusement M\textsuperscript{me} de Saint-Simon rentra quelque temps
après\,; je lui contai l'aventure. Elle trouva une dernière lettre du
marquis de Ruffec que nous envoyâmes à Biron. Elle perça jusqu'à table
comme il me l'avait dit. M. le duc d'Orléans se jeta dessus avec
empressement. L'admirable est qu'il ne connaissoit point son écriture.
Non seulement il la regarda, mais il la lut\,; et comme il la trouva
plaisante, il en régala tout haut sa compagnie, dont elle devint
l'entretien, et lui tout à coup affranchi de ses soupçons. À mon retour
de la Ferté, je le trouvai honteux avec moi, et je le rendis encore
davantage par ce que je lui dis là-dessus.

Il revint encore d'autres lettres de ce prétendu marquis de Ruffec. Il
fut arrêté longtemps après à Bayonne, à table chez Dadoncourt, qui y
commandait, et qui en prit tout à coup la résolution sur ce qu'il lui
vit prendre des olives avec une fourchette. Il avoua au cachot qui il
était\,; et ses papiers décelèrent le libertinage du jeune homme qui
court le pays, et qui, pour être bien reçu et avoir de l'argent, prit le
nom de marquis de Ruffec, se disait brouillé avec moi, écrivait à
M\textsuperscript{me} de Saint-Simon pour se raccommoder par elle et la
prier de payer ce qu'on lui prêtait, le tout pour qu'on vît ses lettres,
et que cela, joint à ce qu'il disait de la famille, le fît croire mon
fils et lui en procurât les avantages. C'était un grand garçon, bien
fait, avec de l'esprit, de l'adresse et de l'effronterie, qui était fils
d'un huissier de Madame, qui connaissoit toute la cour, et qui, dans le
dessein qu'il avait pris de passer pour mon fils, s'était bien informé
de la famille pour en parler juste et n'être point surpris. On le fit
enfermer pour quelque temps. Il avait auparavant couru monde sons
d'autres noms. Il crut que celui de mon fils, de l'âge auquel il se
trouvait à peu près, lui rendrait davantage.

La curiosité d'esprit de M. le duc d'Orléans, jointe à une fausse idée
de fermeté et de courage, l'avait occupé de bonne heure à chercher à
voir le diable, et à pouvoir le faire parler. Il n'oubliait rien,
jusqu'aux plus folles lectures, pour se persuader qu'il n'y a point de
Dieu, et il croyait le diable jusqu'à espérer de le voir et de
l'entretenir. Ce contraste ne se peut comprendre, et cependant il est
extrêmement commun. Il y travailla avec toutes sortes de gens obscurs,
et beaucoup avec Mirepoix, mort en 1699, sous-lieutenant des
mousquetaires noirs, frère aîné du père de Mirepoix, aujourd'hui
lieutenant général et chevalier de l'ordre. Ils passaient les nuits dans
les carrières de Vanvres et de Vaugirard à faire des invocations. M. le
duc d'Orléans m'a avoué qu'il n'avait jamais pu venir à bout de rien
voir ni entendre, et se déprit enfin de cette folie. Ce ne fut d'abord
que par complaisance pour M\textsuperscript{me} d'Argenton, mais après
par un réveil de curiosité, qu'il s'adonna à faire regarder dans un
verre d'eau le présent et le futur, dont j'ai rapporté sur son récit des
choses singulières\,; et il n'était pas menteur. Faux et menteur,
quoique fort voisins, ne sont pas même chose\,; et quand il lui arrivait
de mentir, ce n'était jamais que, lorsque pressé sur quelque promesse ou
sur quelque affaire, il y avait recours malgré lui pour sortir d'un
mauvais pas.

Quoique nous nous soyons souvent parlé sur la religion, où, tant que
j'ai pu me flatter de quelque espérance de le ramener, je me tournais de
tous sens avec lui pour traiter cet important chapitre sans le
rebuter\,; je n'ai jamais pu démêler le système qu'il pouvait s'être
forgé, et j'ai fini par demeurer persuadé qu'il flottait sans cesse sans
s'en être jamais pu former. Son désir passionné, comme celui de ses
pareils en moeurs, était qu'il n'y eût point de Dieu\,; mais il avait
trop de lumière pour être athée, qui sont une espèce particulière
d'insensés bien plus rare qu'on ne croit. Cette lumière l'importunait\,;
il cherchait à l'éteindre et n'en put venir à bout. Une âme mortelle lui
eût été une ressource\,; il ne réussit pas mieux dans les longs efforts
qu'il fit pour se la persuader. Un Dieu existant et une âme immortelle
le jetaient en un fâcheux détroit, et il ne se pouvait aveugler sur la
vérité de l'un et de l'autre. Le déisme lui parut un refuge, mais ce
déisme trouva en lui tant de combats, que je ne trouvai pas grand'peine
à le ramener dans le bon chemin, après que je l'eus fait rompre avec
M\textsuperscript{me} d'Argenton. On a vu avec quelle bonne foi de sa
part par ce qui en a été raconté. Elle s'accordait avec ses lumières
dans cet intervalle de suspension de débauche. Mais le malheur de son
retour vers elle le rejeta d'où il était parti. Il n'entendit plus que
le bruit des passions qui s'accompagna pour l'étourdir encore des mêmes
propos d'impiété, et de la folle affectation de l'impiété. Je ne puis
donc savoir que ce qu'il n'était pas, sans pouvoir dire ce qu'il était
sur la religion. Mais je ne puis ignorer son extrême malaise sur ce
grand point, et n'être pas persuadé qu'il ne se fût jeté de lui-même
entre les mains de tous les prêtres et de tous les capucins de la ville,
qu'il faisait trophée de tant mépriser, s'il était tombé dans une
maladie périlleuse qui lui en aurait donné le temps. Son grand faible en
ce genre était de se piquer d'impiété et d'y vouloir surpasser les plus
hardis.

Je me souviens qu'une nuit de Noël à Versailles, où il accompagna le roi
à matines et aux trois messes de minuit, surprit la cour par sa
continuelle application à lire dans le livre qu'il avait apporté, et qui
partit un livre de prière. La première femme de chambre de
M\textsuperscript{me} la duchesse d'Orléans, ancienne dans la maison,
fort attachée et fort libre, comme le sont tous les vieux bons
domestiques, transportée de joie de cette lecture, lui en fit compliment
chez M\textsuperscript{me} la duchesse d'Orléans le lendemain, où il
avait du monde. M. le duc d'Orléans se plut quelque temps à la faire
danser, puis lui dit\,: «\,Vous êtes bien sotte, madame Imbert\,;
savez-vous donc ce que je lisais\,? C'était Rabelais que j'avais porté
de peur de m'ennuyer.\,» On peut juger de l'effet de cette réponse. La
chose n'était que trop vraie, et c'était pure fanfaronnade. Sans
comparaison des lieux ni des choses, la musique de la chapelle était
fort au-dessus de celle de l'Opéra et de toutes les musiques de
l'Europe\,; et comme les matines, laudes et les trois messes basses de
la nuit de Noël duraient longtemps, cette musique s'y surpassait encore.
Il n'y avait rien de si magnifique que l'ornement de la chapelle et que
la manière dont elle était éclairée. Tout y était plein\,; les travées
de la tribune remplies de toutes les dames de la cour en déshabillé,
mais sous les armes. Il n'y avait donc rien de si surprenant que la
beauté du spectacle, et les oreilles y étaient charmées. M. le duc
d'Orléans aimait extrêmement la musique\,; il la savait jusqu'à
composer, et il s'est même amusé à faire lui-même une espèce de petit
opéra, dont La Fare fit les vers, et qui fut chanté devant le roi\,;
cette musique de la chapelle était donc de quoi l'occuper le plus
agréablement du monde, indépendamment de l'accompagnement d'un spectacle
si éclatant, sans avoir recours à Rabelais\,; mais il fallait faire
l'impie et le bon compagnon.

M\textsuperscript{me} la duchesse d'Orléans était une autre sorte de
personne. Elle était grande et de tous points majestueuse\,; le teint,
la gorge, les bras admirables, les yeux aussi\,; la bouche assez bien
avec de belles dents, un peu longues\,; des joues trop larges et trop
pendantes qui la gâtaient, mais qui n'empêchaient pas la beauté. Ce qui
la déparait le plus étaient les places de ses sourcils qui étaient comme
pelés et rouges, avec fort peu de poils\,; de belles paupières et des
cheveux châtains bien plantés. Sans être bossue ni contrefaite, elle
avait un côté plus gros que l'autre, une marche de côté, et cette
contrainte de taille en annonçait une autre qui était plus incommode
dans la société, et qui la gênait elle-même. Elle n'avait pas moins
d'esprit que M. le duc d'Orléans, et de plus que lui une grande suite
dans l'esprit\,; avec cela une éloquence naturelle, une justesse
d'expression, une singularité dans le choix des termes qui coulait de
source et qui surprenait toujours, avec ce tour particulier à
M\textsuperscript{me} de Montespan et à ses soeurs, et qui n'a passé
qu'aux personnes de sa familiarité ou qu'elle avait élevées\,;
M\textsuperscript{me} la duchesse d'Orléans disait tout ce qu'elle
voulait et comme elle le voulait, avec force délicatesse et agrément\,;
elle disait même jusqu'à ce qu'elle ne disait pas, et faisait tout
entendre selon la mesure et la précision qu'elle y voulait mettre\,;
mais elle avait un parler gras si lent, si embarrassé, si difficile aux
oreilles qui n'y étaient pas fort accoutumées, que ce défaut, qu'elle ne
paraissait pourtant pas trouver tel, déparait extrêmement ce qu'elle
disait.

La mesure et toute espèce de décence et de bienséance étaient chez elle
dans leur centre, et la plus exquise superbe dans son trône. On sera
étonné de ce que je vais dire, et toutefois rien n'est plus exactement
véritable\,: c'est qu'au fond de son âme elle croyait avoir fort honoré
M. le duc d'Orléans en l'épousant. Il lui en échappait des traits fort
souvent qui s'énonçaient dans leur imperceptible. Elle avait trop
d'esprit pour ne pas sentir que cela n'eût pu se supporter, trop
d'orgueil aussi pour l'étouffer\,; impitoyable avec cela jusqu'avec ses
frères sur le rang qu'elle avait épousé, et petite-fille de France
jusque sur sa chaise percée. M. le duc d'Orléans, qui en riait souvent,
l'appelait M\textsuperscript{me} Lucifer en parlant à elle, et elle
convenait que ce nom ne lui déplaisait pas. Elle ne sentait pas moins
tous les avantages et toutes les distinctions que son mariage avait
valus à M. le duc d'Orléans à la mort de Monsieur\,; et ses déplaisirs
de la conduite de M. le duc d'Orléans avec elle, où toutefois l'air
extérieur était demeuré convenable, ne venaient point de jalousie, mais
du dépit de n'en être pas adorée et servie comme une divinité, sans que
de sa part elle eût voulu faire un seul pas vers lui, ni quoi que ce fût
qui pût lui plaire et l'attacher, ni se contraindre en quoi que ce soit
qui le pouvait éloigner, et qu'elle voyait distinctement qui
l'éloignait. Jamais de sa part en aucun temps rien d'accueillant, de
prévenant pour lui, de familier, de cette liberté d'une femme qui vit
bien avec son mari, et toujours recevant ses avances avec froid, et une
sorte de supériorité de grandeur. C'est une des choses qui avaient le
plus éloigné M. le duc d'Orléans d'elle, et dont tout ce que M. le duc
d'Orléans y mit de son côté après leur vrai raccommodement put moins
{[}triompher{]} que la politique, que les besoins d'une part, les vues
de l'autre amenèrent, laquelle encore ne réussit qu'à demi. Pour sa
cour, car c'est ainsi qu'il fallait parler de sa maison et de tout ce
qui allait chez elle, c'était moins une cour qu'elle voulait qu'un
culte\,; et je crois pouvoir dire avec vérité qu'elle n'a jamais trouvé
en sa vie que la duchesse de Villeroy et moi qui ne le lui en ayons
jamais rendu, et qui lui ayons toujours dit et fait ordinairement faire
tout ce qu'il nous paraissait à propos. La duchesse de Villeroy était
haute, franche, libre, sûre, et le lien\,; comme on l'a vu, entre
M\textsuperscript{me} la duchesse de Bourgogne et elle, et moi le lien
entre elle et M. son mari\,; cela pouvait bien entrer pour beaucoup dans
une pareille exception. M\textsuperscript{me} de Saint-Simon, qui ne la
gâtait pas non plus, n'avait pas les mêmes occasions avec elle, jusqu'au
mariage de M\textsuperscript{me} la duchesse de Berry.

La timidité de M\textsuperscript{me} la duchesse d'Orléans était en même
temps extrême. Le roi l'eût fait trouver mal d'un seul regard un peu
sévère, et M\textsuperscript{me} de Maintenon peut-être aussi\,; du
moins tremblait-elle devant elle, et sur les choses les plus communes,
et en public, elle ne leur répondait jamais qu'en balbutiant et la
frayeur sur le visage. Je dis répondait, car de prendre la parole avec
le roi surtout, cela était plus fort qu'elle. Sa vie, au reste, était
fort languissante dans une très ferme santé\,; solitude et lecture
jusqu'au dîner seule, ouvrage le reste de la journée, et du monde depuis
cinq heures du soir qui n'y trouvait ni amusement ni liberté, parce
qu'elle n'a jamais su mettre personne à son aise. Ses deux frères furent
tour à tour ses favoris. Jamais de commerce que de rare et sérieuse
bienséance avec M\textsuperscript{me} la duchesse du Maine\,; avec ses
sœurs, on a vu ailleurs comme elles étaient ensemble, c'est-à-dire point
du tout. Lorsque je commençai à la voir, le favori était son petit
frère. C'est ainsi que par amitié et âge elle appelait le comte de
Toulouse. Il la voyait tous les jours avec la compagnie, assez souvent
seul dans son cabinet avec elle. M. du Maine, ce n'était alors que par
visites peu fréquentes, et encore moins avec la compagnie. Ses vues l'en
rapprochèrent après le mariage de M. le duc de Berry\,; et depuis la
mort de ce prince, il la ménageait, mais pour s'en faire ménager, et de
M. le duc d'Orléans par elle avec un manège merveilleux. Pour moi je ne
la voyais jamais quand la compagnie avait commencé. C'était presque
toujours tête à tête, souvent avec M. le duc d'Orléans, quelquefois,
mais rarement surtout avant la mort du roi, avec M. le comte de
Toulouse, jamais avec M. du Maine. Ni l'un ni l'autre ne mettaient
jamais le pied chez M. le duc d'Orléans qu'aux occasions\,; ni l'un ni
l'autre ne l'aimaient. Le duc du Maine avait peu de disposition, intérêt
à part, à aimer personne. Il épousa ensuite les sentiments de
M\textsuperscript{me} de Maintenon, et on a vu après ce qu'il sut faire
pour éloigner M. le duc d'Orléans des droits de sa naissance, et se
saisir du souverain pouvoir. Le comte de Toulouse froid, menant une vie
toute différente, et n'approuvant pas celle de M. le duc d'Orléans,
touché des déplaisirs de sa sœur, et retenu par les mécontentements du
roi. Je n'ai remarqué depuis en lui dans tous les temps que vérité,
honneur, conduite sage, et devoirs de lui à M. le duc d'Orléans, sans
que ces choses se soient poussées jusqu'à liaison et amitié.

M\textsuperscript{me} la duchesse d'Orléans avait une maison dont elle
ne faisait d'usage que pour leurs fonctions et grossir sa cour. Elle
n'en faisait pas davantage de ce qui la remplissait le plus souvent.
Ainsi je ne m'arrêterai qu'à ce très peu de personnes qui avaient pris
du crédit sur son esprit. Celui de Saint-Pierre, son premier écuyer, lui
avait imposé par un flegme de sénateur, et un impérieux silence qu'il ne
rompait guère que pour prononcer des sentences et des maximes. C'était
un intrigant d'un esprit fort dangereux, duquel elle se devait d'autant
plus défier que, pour son coup d'essai, ce sage l'avait brouillée avec
M. le duc d'Orléans sur la compagnie de ses Cent-Suisses qu'eut Nancré,
et qu'il voulut emporter de haute lutte, jusqu'à commettre ainsi
M\textsuperscript{me} la duchesse d'Orléans qui l'en dédommagea, non de
la promesse mais de la prétention, par la charge de son premier écuyer
que la mort de Fontaine-Martel fit vaquer peu après. M. le duc d'Orléans
avait défendu à Saint-Pierre de mettre le pied chez lui. Saint-Pierre
s'en moquait, et parlait de lui avec la dernière insolence, traitant la
chose de couronne à couronne. Il ne daigna en aucun temps faire un seul
pas vers ce prince, dont la faiblesse trouva plus commode de le
mépriser. Ce fut un pernicieux ouvrier entre le mari et la femme, et en
tout ce qu'il put au dehors contre M. le duc d'Orléans. Sa femme, bonne
demoiselle de Bretagne, qui avait été fort jolie et fort aventurière,
l'air et le jeu fort étourdis, mais avec de l'esprit et de l'art,
apaisait M. le duc d'Orléans à force de badinages et de manèges. C'était
elle qui avait introduit son mari, lequel avait été cassé de capitaine
de vaisseau, pour avoir mis la sédition dans la marine, lorsque le roi y
voulut établir l'école du petit Renault. Comme cela est ancien et
chétif, je n'ai jamais su comment M\textsuperscript{me} de Saint-Pierre
s'était introduite elle-même, mais en peu de temps M\textsuperscript{me}
la duchesse d'Orléans ne s'en put passer ni lui rien refuser\,; cela a
duré bien des années, et l'amitié et la familiarité toujours. Elle était
gaie, libre, plaisante, savait toutes les galanteries de la cour\,; et
la meilleure créature du monde. Marly les tenta, M\textsuperscript{me}
la duchesse d'Orléans y fit l'impossible, et ne se rebuta point pendant
plusieurs années. Elle y échoua toujours. Saint-Pierre était un très
petit gentilhomme de basse Normandie, si tant est qu'il le fût bien, et
le roi qui s'en informa n'en voulut pas ouïr parler pour Marly, pour
manger ni pour entrer dans les carrosses. Ce fut le ver rongeur des
Saint-Pierre qui, non contents de s'être enrichis et placés, voulaient
faire les seigneurs.

J'ai dit ailleurs un mot de M\textsuperscript{me} de Jussac, qui était
une femme du premier mérite en tout genre et du plus aimable\,; ainsi je
n'en redirai rien ici.

La duchesse Sforce était celle qui possédait le plus le cœur et l'esprit
de M\textsuperscript{me} la duchesse d'Orléans. C'était sa cousine
germaine, seconde fille de M\textsuperscript{me} de Thianges, sœur de
M\textsuperscript{me} de Montespan, qui l'avait mariée fort jeune à Rome
au duc Sforce en 1678, qui mourut sans enfants en 1685 à soixante-sept
ans, veuf en premières noces d'une Colonne, fille du prince de
Carbognano. Il était chevalier de l'ordre, qu'il avait reçu en septembre
1675 par les mains du duc de Nevers à Rome, avec le duc de Bracciano. Sa
mère était fille du duc de Mayenne, chef de la Ligue, et il était le
neuvième descendant de père en fils de ce fameux Attendolo, qui de
laboureur de Cotignola devint un des plus grands capitaines de l'Europe,
seigneur et comte de sa patrie, avec d'autres grands États, gonfalonier
de l'Église et connétable de Naples sous la reine Jeanne, et qui établit
une puissante maison. Il prit le nom de Sforza d'un sobriquet sur la
force de corps, sur ce que, résistant avec insolence à son général
Alberic Balbiano sur le partage du butin, Balbiano lui demanda s'il
voulait \emph{usar meco forza}, et qu'il ferait bien de prendre le nom
de \emph{Sforza} qu'il prit en effet, et le fit passer à sa postérité.
De Bosio, son puîné, est venu le duc Sforce qui donne lieu à cette
remarque, dont le frère aîné fut duc de Milan, par son mariage avec
l'héritière fille du duc Philippe-Marie Visconti. Son fils Galeas-Marie,
successivement gendre du marquis de Mantoue et du duc de Savoie, fut tué
jeune et laissa le duché de Milan à son fils Jean-Galeas tout enfant
sous la tutelle de son frère Ludovic, si connu par le surnom de More,
qui le maria à la fille d'Alphonse, duc de Calabre, depuis roi de
Naples, l'empoisonna après et usurpa le duché de Milan sur son
petit-neveu François qui ne fut point marié\,; et tous deux moururent en
France\,: celui-ci, abbé de Marmoutier\,; Louis le More à Loches, dans
une cage où il vécut plusieurs années, et où Louis XII l'avait fait
enfermer, après l'avoir fait prisonnier. Son fils aîné rentra ensuite
dans le duché de Milan, et en fut encore dépouillé, et vint achever sa
vie à Paris sans alliance. Son frère François fut plus heureux. Il fut
rétabli à Milan, et mourut sans enfants de la fille de ce Christiern,
roi de Danemark, fameux par ses insignes cruautés et sa catastrophe, et
d'une sœur de Charles-Quint. Il y a eu d'autres branches tant légitimes
que bâtardes de ces Sforce qui ont eu en Italie des établissements et
des alliances considérables. Je n'ai pu refuser ce petit écart de
curiosité avant d'en venir à la duchesse Sforce.

Elle était belle, sage et spirituelle, et plut assez au roi à son retour
pour donner lieu à M\textsuperscript{me} de Maintenon de l'écarter.
C'était encore assez qu'elle fût nièce de M\textsuperscript{me} de
Montespan, et qu'elle en eût ce langage singulier dont j'ai parlé plus
d'une fois. Il se forma dans les suites une liaison de convenance entre
elle et M\textsuperscript{me} la duchesse d'Orléans, qui parvint au
dernier point d'intimité et de confiance, jusqu'à ne pouvoir se passer
l'une de l'autre, qui a duré tant que la duchesse Sforce a vécu, dont
M\textsuperscript{me} de Castries, leur cousine germaine, fille de M. de
Vivonne et dame d'atours de M\textsuperscript{me} la duchesse d'Orléans,
qui avait bien plus d'esprit, et le même tour que M\textsuperscript{me}
Sforce, mourait de jalousie. M\textsuperscript{me} Sforce avait de
l'esprit, comme il a été remarqué, mais sage, sensé, avisé, réfléchi\,;
bonne et honnête par nature, éloignée de tout mal, et se portant à tout
bien, et cette intimité avec M\textsuperscript{me} la duchesse d'Orléans
fut un bonheur pour cette princesse, pour M. le duc d'Orléans et pour
toute cette branche royale. Elles passaient leur vie ensemble, et
dînaient presque tous les jours tête à tête. Son extérieur droit, sec,
froid et haut, avait du rebutant. Elle aimait à gouverner. Tout montrait
en elle une rinçure de la princesse des Ursins. Mais perçant cet
épiderme, vous ne trouviez que sagesse, mesure, bonté, politesse,
raison, désir d'obliger, de concilier, surtout vérité, sincérité,
droiture, sûreté entière, secret inviolable, assemblage si précieux et
si rare, surtout à la cour, et dans une femme. Elle était glorieuse sans
orgueil et sans bassesse, c'est-à-dire qu'elle se sentait fort, et
qu'elle se conduisait avec réserve et dignité loin de toute prostitution
de cour, où avec cela elle se faisait compter, quoique en allant tort
peu.

La parenté que j'avais avec elle par sa mère, soeur de
M\textsuperscript{me} de Montespan, m'en attira des honnêtetés, rares
parce que nous ne nous rencontrions guère, plus ordinaires à
M\textsuperscript{me} de Saint-Simon, qu'elle voyait souvent chez
M\textsuperscript{me} la duchesse d'Orléans. Aussitôt qu'après le congé
donné à M\textsuperscript{me} d'Argenton, je fus en commerce particulier
avec M\textsuperscript{me} la duchesse d'Orléans, M\textsuperscript{me}
Sforce me fit des avances de liaison auxquelles je répondis à son gré.
Je ne la connaissois point assez pour être prévenu de tout son mérite,
mais sur ce que j'en avais appris, et sur ce que je savais de son
intimité avec M\textsuperscript{me} la duchesse d'Orléans et sans
partage, je crus utile au maintien du raccommodement que je venais de
faire avec tant de peine, et à tout ce qui pourrait survenir de vues et
d'affaires à M. le duc d'Orléans, de vivre dans l'intelligence qui
m'était offerte. Bientôt après nous être un peu connus, et
M\textsuperscript{me} de Saint-Simon quelquefois en tiers, ou seule avec
elle, quoique rarement depuis cette époque, elle nous plut tant et nous
à elle que l'amitié et la confiance suivirent bientôt, que rien depuis
n'a pu affaiblir. Je ne parle point de la duchesse de Villeroy, dont
j'ai suffisamment fait mention ailleurs, et qui mourut peu de jours
avant Monseigneur. Ainsi au temps où nous sommes, il n'était plus
question que de la regretter il y avait longtemps.

\hypertarget{chapitre-vi.}{%
\chapter{CHAPITRE VI.}\label{chapitre-vi.}}

~

{\textsc{Vie ordinaire de M. {[}le duc{]} et de M\textsuperscript{me} la
duchesse d'Orléans.}} {\textsc{- Caractère de M\textsuperscript{me} la
duchesse de Berry.}} {\textsc{- Caractère de la Mouchy et de son mari.}}
{\textsc{- Caractère de Madame.}} {\textsc{- Embarras domestiques de M.
le duc d'Orléans.}} {\textsc{- Singulier manège du maréchal de Villeroy
avec moi.}} {\textsc{- Caractère du maréchal de Villeroy.}}

~

L'abandon total qui faisait de la cour la plus parfaite solitude pour M.
le duc d'Orléans, la paresse de M\textsuperscript{me} la duchesse
d'Orléans qui ne croyait pas devoir faire un pas vers personne, et en
qui l'orgueil et la paresse étaient au dernier point, et parfaitement
d'accord pour attendre tout sur son trône sans se donner la moindre
peine, rendait leur vie languissante, honteuse, indécente et méprisée.
Ce fut une des premières choses à quoi il fallut remédier. Tous deux le
sentirent, et il faut pourtant dire que M\textsuperscript{me} la
duchesse d'Orléans, une fois convaincue et résolue, s'y porta avec plus
de courage et de suite que M. le duc d'Orléans. Je dis de courage, par
les mortifications continuelles que son orgueil eut à essuyer dans de
longs essais pour sortir de cet état. Marly, où se passait presque la
moitié de l'année, et où les dames ne mangeaient plus depuis longtemps
avec le roi qu'à souper, et où la table de M\textsuperscript{me} la
duchesse de Bourgogne, et les fréquents retours de chasse de Monseigneur
et des deux princes ses fils étaient disparus avec eux, donna moyen à
M\textsuperscript{me} la duchesse d'Orléans de rechercher du monde pour
ses dîners. C'est ce qu'elle entreprit dès avant la mort de M. le duc de
Berry avec peu de succès. Les dames qu'elle invitait, ou par les siennes
ou le plus souvent par elle-même, étaient fertiles en excuses. On
redoutait la compagnie de M. le duc d'Orléans. Les plus avisées épiaient
ses tours à Paris pour dîner chez M\textsuperscript{me} sa femme, et
s'en tenir quittes après pour longtemps. On craignait le roi,
c'est-à-dire M\textsuperscript{me} de Maintenon, et les plus au fait, M.
du Maine\,; et ces refus se soutinrent longtemps, comme à la mode,
jusque-là qu'on cherchait à se disculper et d'y être laissé entraîner,
par la presse qu'on en avait essuyée, et qui ne pouvait plus donner lieu
à de plus longs refus. Les hommes étaient encore plus embarrassants que
les femmes, parce que le rang de petite-fille de France n'en permettait
à leur table que de titrés.

M\textsuperscript{me} la duchesse d'Orléans, qui sentit enfin
l'importance de rompre une si indécente barrière qui la séparait du
monde, à cause de M. son mari, et qu'elle ne pouvait rapprivoiser avec
elle sans le rapprocher de lui, ne se rebuta point, et prit les manières
les plus convenables autant qu'il fut en elle pour fondre ces glaces et
faire fleurir sa table et son appartement. Le travail fut également
dégoûtant et opiniâtre, mais enfin il réussit. On s'enhardit enfin les
uns à l'exemple des autres, et le nombre qui s'augmenta peu à peu
s'appuya sur le nombre même pour s'appuyer et s'augmenter de plus en
plus. La table était exquise, et la contrainte à la fin, tout respect et
décence gardés, y devint peu perceptible. M. le duc d'Orléans y contint
la liberté de ses discours, il s'y mit peu à peu à converser quand il
n'y trouvait point de véritable contrebande, mais de choses publiques,
générales, convenables, incapables d'embarrasser personne ni lui-même.
Souvent des tables de jeu suivaient le repas, et retenaient la compagnie
avec celle qui survenait jusqu'à l'heure du salon. On se loua enfin
beaucoup de ces dîners\,; on s'étonna de la répugnance qu'on y avait
eue\,; on se trouva à l'aise de ce que le roi {[}et{]}
M\textsuperscript{me} de Maintenon y paraissaient indifférents, on eut
honte d'avoir mal à propos appréhendé de leur déplaire. Mais le salon,
pour tout cela, n'en devint pas plus favorable à M. le duc d'Orléans. À
ces dîners, c'était chez une bâtarde du roi\,; on n'y était avec M. le
duc d'Orléans que par occasion, on était invité, rien de tout cela dans
le salon, où le très grand nombre en hommes qui n'était point de ces
dîners était demeuré dans la même réserve avec lui, où il était même
évité de presque tous ceux qui sortaient de sa table, sans que cela ait
pu changer à son égard, jusqu'à l'extrémité de la maladie du roi.

Son ennui le menait souvent à Paris faire des soupers et des parties de
débauche. On tâchait de les éloigner par d'autres parties avec
M\textsuperscript{me} la duchesse d'Orléans à Saint-Cloud et à l'Étoile,
la plus gentille petite maison que le roi avait donnée il y avait
longtemps à M\textsuperscript{me} la duchesse d'Orléans, dans le parc de
Versailles, qu'elle avait accommodée le mieux du monde, en quoi elle
avait le goût fort bon. Elle aimait la table, les conviés l'aimaient
tous, et à table c'était tout une autre personne, libre, gaie excitante
charmante. M. le duc d'Orléans n'aimait que le bruit\,; et comme il se
mettait en pleine liberté dans ces sortes de parties, on était fort
contraint sur le choix des convives, dont les oreilles et la politique
auraient été également embarrassées du peu de mesure de ses propos, et
leurs yeux fort étonnés de le voir s'enivrer tout seul dès les
commencements du repas au milieu de tous gens qui ne songeaient qu'à
s'amuser et à se réjouir honnêtement, et dont pas un n'y approcha jamais
de l'ivresse. Parmi cette vie qui fut la même jusqu'à la fin du roi, les
attentions et les embarras ne manquaient pas\,; c'est {[}ce{]} qu'on
tâchera de développer après que, pour le mieux entendre, on aura exposé
l'état intérieur de la famille de M. le duc d'Orléans, qui alors ne
consistait qu'en M\textsuperscript{me} la duchesse de Berry et Madame.

On a pu sentir quelle était M\textsuperscript{me} la duchesse de Berry
en plusieurs endroits de ces Mémoires, mais on la verra bientôt faire un
personnage si singulier en soi, et par rapport à M. son père, devenu
régent du royaume, que je ne craindrai point quelque légère répétition
pour la faire connaître autant qu'il est nécessaire. Cette princesse
était grande, belle, bien faite, avec toutefois assez peu de grâce, et
quelque chose dans les yeux qui faisait craindre ce qu'elle était. Elle
n'avait pas moins que père et mère le don de la parole, d'une facilité
qui coulait de source, comme en eux, pour dire tout ce qu'elle voulait
et comme elle le voulait dire avec une netteté, une précision, une
justesse, un choix de termes et une singularité de tour qui surprenait
toujours. Timide d'un côté en bagatelles, hardie d'un autre jusqu'à
effrayer, hardie jusqu'à la folie, basse aussi jusqu'à la dernière
indécence, il se peut dire qu'à l'avarice près, elle était un modèle de
tous les vices, qui était d'autant plus dangereux qu'on ne pouvait pas
avoir plus d'art ni plus d'esprit. Je n'ai pas accoutumé de charger les
tableaux que je suis obligé de présenter pour l'intelligence des choses,
et on s'apercevra aisément combien je suis étroitement réservé sur les
dames, et sur toute galanterie qui n'a pas une relation indispensable à
ce qui doit s'appeler important. Je le serais ici plus que sur qui que
ce soit par amour-propre, quand ce ne serait pas par respect du sexe et
dignité de la personne. La part si considérable que j'ai eue au mariage
de M\textsuperscript{me} la duchesse de Berry, et la place que
M\textsuperscript{me} de Saint-Simon, quoique bien malgré elle et malgré
moi, a occupée et conservée auprès d'elle jusqu'à la mort de cette
princesse, seraient pour moi de trop fortes raisons de silence, si ce
silence ne jetait pas des ténèbres sur toute la suite de ce qui fait
l'histoire de ce temps, dont l'obscurité couvrirait la vérité. C'est
donc à la vérité que je sacrifie ce qu'il en va coûter à l'amour-propre,
et avec la même vérité aussi que je dirai que si j'avais connu ou
seulement soupçonné dans cette princesse une partie dont le tout ne
tarda guère à se développer après son mariage, et toujours de plus en
plus depuis, jamais elle n'eût été duchesse de Berry.

Il est ici nécessaire de se souvenir de ce souper de Saint-Cloud si
immédiat après ses noces (t. VIII, p.~417) et de ce qui est légèrement,
mais intelligiblement touché du voyage de Marly qui le suivit de si
près\,; de cet emportement contre l'huissier qui par ignorance avait
chez elle (t. IX, p. 162) ouvert les deux battants de la porte à
M\textsuperscript{me} sa mère\,; de son désespoir et de sa cause à la
mort de Monseigneur\,; des fols et effrayants aveux qu'elle en fit à
M\textsuperscript{me} de Saint-Simon, de sa haine pour Mgr {[}le duc{]}
et surtout pour M\textsuperscript{me} la duchesse de Bourgogne, et de sa
conduite avec elle, à qui elle devait tout, et qui ne se lassa jamais
d'aller au-devant de tout avec elle\,; du désespoir de lui donner la
chemise et le service lorsqu'elle fut devenue Dauphine, de tout ce qu'il
fallut employer pour l'y résoudre, et de tout ce qu'elle avait fait pour
en empêcher M. le duc de Berry malgré lui, et pour le brouiller contre
son coeur et tout devoir avec Mgr {[}le duc{]} et M\textsuperscript{me}
la duchesse de Bourgogne (t. IX, p.~166-169)\,; des causes de l'orage
qu'elle essuya du roi et de M\textsuperscript{me} de Maintenon (t. IX,
p.~165), et qui ne fut pas le dernier\,; de la matière et du succès de
l'avis que la persécution de M\textsuperscript{me} la duchesse d'Orléans
et le cri public, tout indigne qu'il était, me força de donner à M. le
duc d'Orléans sur elle (t. IX, p.~392)\,; de l'étrange éclat arrivé
entre elle et M\textsuperscript{me} sa mère sur le procédé des perles de
la reine mère, et sur une pernicieuse femme de chambre qu'on lui chassa
(t. X, p.~61)\,; de celui qu'elle eut sur les places de premier écuyer
de M. le duc de Berry, et de future gouvernante de ses enfants (t. X,
p.~76)\,; et enfin de ce qui a été touché (t. XI, p.~88), le plus
succinctement qu'il a été possible, de la façon dont elle était avec M.
le duc de Berry, et des sentiments de ce prince pour elle, lorsqu'il
mourut, de toutes lesquelles choses M\textsuperscript{me} de Saint-Simon
a vu se passer d'étranges scènes en sa présence, et reçu et calmé
d'étranges confidences de M. le duc de Berry\,; enfin de ce qu'on a vu
combien elle se piquait d'une fausseté parfaite, et de savoir
merveilleusement tromper, en quoi elle excellait même sans aucune
occasion.

Elle fit ce qu'elle put pour ôter toute religion à M. le duc de Berry,
qui en avait un véritable fond et une grande droiture. Elle le
persécutait sur le maigre et sur le jeûne, qu'il n'aimait point, mais
qu'il observait exactement. Elle s'en moquait jusqu'à lui en avoir fait
rompre, quoique rarement, à force d'amour, de complaisance et d'embarras
de ses aigres plaisanteries, et comme cela n'arrivait point sans combat
et sans qu'on ne vît avec quelle peine et quel scrupule il se laissait
aller, c'était encore sur cela même un redoublement de railleries qui le
désolaient. Son équité naturelle n'avait pas moins à souffrir des
emportements avec lesquels elle exigeait des injustices criantes dans sa
maison à lui, car pour la sienne il n'eût osé rien dire. D'autres sujets
plus intéressants mettaient sans cesse sa patience à bout, et plus d'une
fois sur le dernier bord du plus affreux éclat. Elle ne faisait guère de
repas libres, et ils étaient fréquents, qu'elle ne s'enivrât à perdre
connaissance, et à rendre partout ce qu'elle avait pris, et si rarement
elle demeurait en pointe, c'était marché donné. La présence de M. le duc
de Berry, de M. {[}le duc{]} et de M\textsuperscript{me} la duchesse
d'Orléans, ni les dames avec qui elle n'avait aucune familiarité, ne la
retenaient pas le moins du monde. Elle trouvait même mauvais que M. le
duc de Berry n'en fît pas autant. Elle traitait souvent M. son père avec
une hauteur qui effrayait sur toutes sortes de chapitres. La crainte du
roi l'empêchait de s'échapper si directement avec M\textsuperscript{me}
sa mère, mais ses manières avec elle y suppléaient, de manière que pas
un des trois n'osait hasarder la moindre contrariété, beaucoup moins le
moindre avis, et si quelquefois quelque raison forte et pressante les y
forçait, c'était des scènes étranges, et le père et le mari en venaient
aux soumissions et au pardon, qu'ils achetaient chèrement.

Les galanteries difficiles dans sa place n'avaient pas laissé d'avoir
plusieurs objets, et avec assez peu de contrainte. À la fin elle se
rabattit sur La Haye, qui de page du roi était devenu écuyer particulier
de M. le duc de Berry. C'était un grand homme sec, à taille contrainte,
à visage écorché, l'air sot et fat, peu d'esprit et bon homme de cheval,
à qui elle fit faire pour son état une rapide fortune en charges par son
maître. Les lorgneries dans le salon de Marly étaient aperçues de tout
ce qui y était, et nulle présence ne les contenait. Enfin il faut le
dire, parce que ce trait renferme : tout elle voulut se faire enlever
dans Versailles par La Haye, M. le duc de Berry et le roi pleins de vie,
et gagner avec lui les Pays-Bas. La Haye pensa mourir d'effroi de la
proposition qu'elle lui en fit elle-même, et elle de la fureur où la
mirent ses représentations. Des conjurations les plus pressantes elle en
vint à toutes les injures que la rage lui put suggérer, et que les
torrents de larmes lui purent laisser prononcer. La Haye n'en fut pas
quitte pour une attaque, tantôt tendre, tantôt furieuse. Il était dans
le plus mortel embarras. Enfin la terreur de ce que pouvait enfanter une
folie si démesurée força sagement sa discrétion pour que rien ne lui fût
imputé, si elle se portait à quelque extravagance. Le secret fut
fidèlement gardé, et on prit les mesures nécessaires. La Haye cependant
n'avait osé disparaître, à cause de M. le duc de Berry d'une part et du
monde de l'autre qui, sans être au fait de cette incroyable folie, y
était de la passion. Quand à la fin M\textsuperscript{me} la duchesse de
Berry, ou rentrée en quelque sens, ou hors de toute espérance de
persuader La Haye, vit bien clairement que cette persécution n'allait
qu'à se tourmenter tous deux, elle cessa ses poursuites, mais la passion
continua jusqu'à la mort de M. le duc de Berry et quelque temps après.
Voilà quelle fut la dépositaire du coeur et de l'âme de M. le duc
d'Orléans, qui sut pleinement toute cette histoire, qui en fut dans les
transes les plus extrêmes, non d'un enlèvement impossible, et auquel La
Haye n'avait garde de se commettre, mais des éclats et des aventures
dont tout était à craindre de cet esprit hors de soi, et qui devant et
après n'en fut pas moins la dépositaire des secrets de M. son père tant
qu'elle vécut, et qui lui en donna d'autres encore qui se trouveront en
leur temps.

Jamais elle n'avait reçu que douceur, amitié, présents de
M\textsuperscript{me} la duchesse d'Orléans. Elle n'avait d'ailleurs
presque jamais été auprès d'elle. Elle n'avait donc point été à portée
de ces petites choses qui fâchent quelquefois les enfants. Mais son
orgueil était si extrême qu'elle regardait en soi, comme une tache
qu'elle en avait reçue, d'être fille d'une bâtarde, et en avait conçu
pour elle une aversion et un mépris qu'elle ne contraignit plus après
son mariage, et que devant et après elle prit sans cesse à tâche
d'attiser dans le coeur et dans l'esprit de M. le duc d'Orléans.
L'orgueil de M\textsuperscript{me} sa mère n'était rien en comparaison
du sien. Elle se figura devant et depuis son mariage qu'il n'y avait
qu'elle en Europe que M. le duc de Berry pût épouser, et qu'ils étaient
tous deux but à but. On a vu en son temps que M. le duc d'Orléans lui
confiait à mesure tout ce qui se passait sur son mariage, parce qu'il ne
pouvait lui rien cacher\,; qu'elle m'en raconta mille choses à
Saint-Cloud lorsqu'il fut déclaré, pour que je ne pusse ignorer cette
dangereuse confiance qu'elle ne put donc douter de tout ce qu'il y avait
eu à surmonter, et tout ce qu'elle me témoigna de sa reconnaissance.
Elle ne fut pas trois mois mariée qu'elle montra sa parfaite ingratitude
à tout ce qui y avait eu part, et que lors de la scène qu'elle eut avec
M\textsuperscript{me} de Lévi qu'elle avait si cruellement trompée et
jouée, de propos délibéré, sur la charge de premier écuyer de M. le duc
de Berry, elle ne put se tenir de lui dire qu'elle était indignée de
sentir qu'une personne comme elle pût avoir obligation à quelqu'un,
qu'aussi elle haïssait de tout son cœur tout ce qui avait eu part à son
mariage jusqu'à ne leur pouvoir pardonner\,; sur quoi
M\textsuperscript{me} de Lévi, perdant tout respect et toutes mesures,
la traita comme elle le méritait, et vécut depuis avec elle en
conséquence, et en public, dont M\textsuperscript{me} la duchesse de
Berry, timide en petites choses, comme on l'a dit, et glorieuse au
suprême, était dans le dernier embarras, et lui fit faire mille avances
inutiles pour se délivrer de ce dont elle n'osait se plaindre.

Sa conduite rebuta enfin le roi, et M\textsuperscript{me} de Maintenon,
de s'en soucier après tant de réprimandes et de menaces si fortes et si
inutiles, surtout depuis la mort de M. le duc de Berry\,; et
M\textsuperscript{me} la Dauphine, longtemps avant la sienne, ne s'en
mêlait plus. Le roi, à l'extérieur, vivait honnêtement mais froidement
avec elle\,; lui et M\textsuperscript{me} de Maintenon la méprisaient.
Le roi la souffrait par nécessité\,; pour M\textsuperscript{me} de
Maintenon, elle ne la voyait plus, et avec toute cette conduite, elle
les craignait tous deux comme le feu, muette et embarrassée au dernier
point avec eux, même en public avec le roi. Tous ces mécontentements de
l'un et de l'autre retombaient à plomb sur M. le duc d'Orléans, qu'ils
comptaient qui les avait trompés en leur donnant sa fille qu'il devait
connaître, et qu'ils haïssaient et méprisaient de la faiblesse qu'il
avait pour elle, et de ce que cette amitié si suivie n'était bonne à
rien pour opérer aucun changement en elle.

L'unique personne de son entière confiance était M\textsuperscript{me}
de Mouchy dont il a été parlé, et dont les mœurs et le caractère en
étaient parfaitement dignes. Outre la galanterie et la licence de la
table, elle avait un talent et des ressources d'inventions tout entières
de la plus horrible noirceur, une effronterie sans pareille et une
avidité d'intérêt à lui faire tout entreprendre, avec tout l'esprit,
l'art et le manège propre à réussir\,; toujours un but, et ne disant et
ne faisant jamais rien sans un dessein, pour léger et indifférent que
parût ce qu'elle disait ou faisait. Son mari, qui avait de la naissance,
n'était pas moins bassement intéressé, et trouvait tout bon d'elle,
pourvu que cela lui rapportât\,; de ces officiers d'ailleurs, quoique
mort lieutenant général de la régence, bons au plus à placer quelque
part capitaines des portes.

Madame était une princesse de l'ancien temps, attachée à l'honneur, à la
vertu, au rang, à la grandeur\,; inexorable sur les bienséances. Elle ne
manquait point d'esprit, et ce qu'elle voyait elle le voyait très bien.
Bonne et fidèle amie, sûre, vraie, droite, aisée à prévenir et à
choquer, fort difficile à ramener\,; grossière, dangereuse à faire des
sorties publiques, fort Allemande dans toutes ses moeurs, et franche,
ignorant toute commodité et toute délicatesse pour soi et pour les
autres, sobre, sauvage et ayant ses fantaisies. Elle aimait les chiens
et les chevaux, passionnément la chasse et les spectacles, n'était
jamais qu'en grand habit ou en perruque d'homme, et en habit de cheval,
et avait plus de soixante ans que, saine ou malade, et elle ne l'était
guère, elle n'avait pas connu une robe de chambre. Elle aimait
passionnément M. son fils, on peut dire follement le duc de Lorraine et
ses enfants, parce que cela avait trait à l'Allemagne, et singulièrement
sa nation et tous ses parents, qu'elle n'avait jamais vus. On a vu, à
l'occasion de la mort de Monsieur, qu'elle passait sa vie leur écrire et
ce qu'il lui en pensa coûter. Elle s'était à la fin apprivoisée, non
avec la naissance de M\textsuperscript{me} sa belle-fille, mais avec sa
personne, qu'elle traitait fort bien dès avant le renvoi de
M\textsuperscript{me} d'Argenton.

Elle estimait, elle plaignait, elle aimait presque M\textsuperscript{me}
la duchesse d'Orléans. Elle blâmait fort la vie désordonnée que M. le
duc d'Orléans avait menée\,; elle était suprêmement indignée de celle de
M\textsuperscript{me} la duchesse de Berry, et s'en ouvrait quelquefois
avec la dernière amertume et toute confiance à M\textsuperscript{me} de
Saint-Simon, qui dès les premiers temps qu'elle fut à la cour avait
trouvé grâce dans son estime et dans son amitié, qui demeurèrent
constantes. Elle n'avait donc de sympathie avec M\textsuperscript{me} la
duchesse de Berry que la haine parfaite de M. du Maine, des bâtards et
de leur grandeur, et elle était blessée de ce que M. son fils n'avait
point de vivacité là-dessus. Avec ces qualités elle avait des
faiblesses, des petitesses, toujours en garde qu'on ne lui manquât. Je
me souviens que s'étant mise dans un petit appartement, au Palais-Royal,
pendant un hiver de la régence, où elle n'était guère, car elle haïssait
Paris et était toujours à Saint-Cloud, M. le duc d'Orléans me dit un
jour qu'il avait un plaisir et une complaisance à me demander. C'était
d'aller quelquefois chez Madame, qui lui avait fait ses plaintes qu'elle
ne me voyait jamais et que je la méprisais\,: on peut juger de mes
réponses. Le dernier était, comme on peut penser, sans aucune apparence,
et ce n'était pas un sentiment que personne pût avoir pour Madame\,;
l'autre était vrai, je ne lui faisais ma cour à Versailles qu'aux
occasions, et j'avais alors, quand il n'y en avait point d'aller chez
elle, tout autre chose à faire. Depuis cela j'allais à sa toilette une
fois en quinze jours ou trois semaines, quand elle était à Paris, et j'y
étais toujours fort bien reçu.

M. le duc d'Orléans était le meilleur père, le meilleur fils et, depuis
sa rupture avec M\textsuperscript{me} d'Argenton, le meilleur mari du
monde. Il aimait fort Madame et lui rendait de grands et de continuels
devoirs. Il la craignait aussi, n'avait pas grande idée de ses
ressources. Ainsi son ouverture pour elle et sa confiance étaient
médiocres\,; et quoiqu'on fût sûr du secret avec elle, il s'en fallait
tout qu'il lui fit part des siens, il se contentait de lui rendre compte
en gros des choses de famille, comme sur le mariage de ses enfants, et
quand il fut le maître de ce qui allait être public, le moins qu'il
pouvait auparavant. Elle influa donc fort peu dans sa conduite privée et
publique, se mêla peu de lui rien demander, quoique point refusée sur
les grâces, et ne fut de rien du tout sur aucune affaire. Cela me
dispensera de faire mention du peu de personnes qui pouvaient le plus
sur elle. J'ajouterai seulement que Madame fut toujours d'avec le roi et
d'avec M\textsuperscript{me} la duchesse d'Orléans contre la conduite de
M\textsuperscript{me} la duchesse de Berry, à qui elle faisait
quelquefois d'étranges sorties, que le roi lui en parlait avec
confiance, qu'il la mit un temps sous sa direction, qu'elle s'en lassa
bientôt comme le roi avait fait, et qu'elle ne trouvait pas meilleur que
lui cet attachement et ce particulier continuel de M. le duc d'Orléans
avec M\textsuperscript{me} la duchesse de Berry, si inutile au
changement de sa conduite.

Avant d'entrer dans les embarras du dehors, il faut expliquer les
domestiques {[}de M. le duc d'Orléans{]}. Il n'y avait sans doute
personne dont les intérêts dussent être si fort les mêmes que les siens,
personne encore de meilleur conseil, et dont il fût plus à portée à tous
les instants, que M\textsuperscript{me} la duchesse d'Orléans. Il était
vrai aussi qu'à un article près, leurs intérêts étaient effectivement
les mêmes, et qu'elle le pensait et le sentait ainsi. Mais cet article
était tel qu'il influait très nécessairement sur tout autre, et qu'il
opérait la plus embarrassante séparation. On entend bien, sans qu'il
soit besoin de l'expliquer, que cet article fatal regardait M. du
Maine\,; mais ce qu'on ne peut entendre sans le dernier étonnement,
c'est que l'intérêt de M. du Maine effaçait tout autre dans son cœur et
dans son esprit, et ce qui va jusqu'à l'incroyable en même temps qu'il
est dans la plus étroite vérité, c'est que la béatitude anticipée de
l'autre monde eût été pour elle en celui-ci, si elle avait pu voir le
duc du Maine établi roi de France au préjudice de son mari et de son
fils, beaucoup plus si elle avait pu y contribuer. Que si on y ajoute
qu'elle connaissoit très bien le duc du Maine, qu'elle en éprouvait des
artifices et des tromperies qu'elle ressentait beaucoup, qu'elle ne
l'aimait point du tout et l'estimait beaucoup moins encore\,; que ce que
j'en avance ici, elle me l'a dit à moi-même sans colère, mais en parlant
et en raisonnant avec poids et avec réflexion, on sentira jusqu'à quel
point elle était possédée du démon de la bâtardise, et que la superbe,
poussée jusqu'au fanatisme, était devenue sa suprême divinité.

De là suivait que tout ce qui non seulement allait, mais pouvait tourner
aux avantages, à l'élévation, à la puissance du duc du Maine, elle n'y
était pas moins ardente que lui\,; que tous moyens de l'exalter et de
l'affermir, je dis seulement ceux qui se peuvent proférer, lui étaient
bons, et que cet aveuglement la portait à être de moitié de tout avec le
duc du Maine pour tout ce qu'il pouvait désirer de M. le duc d'Orléans
pour sa solide grandeur contre la sienne, et que les panneaux qu'il lui
tendait sans cesse pour le tromper et l'écraser sous ses pieds, elle les
trouvait des propositions raisonnables, sensées, pour le moins très
plausibles, qui méritaient d'être examinées, et dont l'examen allait
toujours à tout ce que le duc du Maine pouvait souhaiter. Ce que M. du
Maine n'osait par lui-même, il le faisait insinuer par Saint-Pierre, qui
ayant reconnu de bonne heure jusqu'à quel point la bâtardise était le
point capital par lequel il pouvait gouverner cette princesse, s'était
dévoué à eux sans y paraître, et était eu intime liaison avec d'O\,; et
celui-ci, qui était au comte de Toulouse, et qui ne paraissait pas avoir
grande liaison avec le duc du Maine, était tout à lui là-dessus, et se
maintint par là dans la faveur et la confiance du roi et de
M\textsuperscript{me} de Maintenon, à quoi la conduite du comte de
Toulouse ne pouvait plus servir de nourriture, après qu'il fut parvenu à
un certain âge.

M\textsuperscript{me} la duchesse d'Orléans ainsi conduite, et sans
cesse recordée et pressée sur des choses qu'elle-même ne souhaitait pas
moins, était donc une épine fort dangereuse dans le sein de M. le duc
d'Orléans. Il fallait bien vivre avec elle, ne lui montrer aucun
soupçon, et pour cela l'écouter, raisonner et discuter avec elle, sans
rien montrer qui la pût mettre en garde sur les gardes continuelles où
on devait être avec elle, et très souvent l'amuser d'espérances, de
prétextes et de délais sur des choses positives qu'il aurait été
périlleux de rejeter et pernicieux au dernier point d'accepter. Tout
cela était mêlé d'avis fréquents donnés à M\textsuperscript{me} la
duchesse d'Orléans, de bagatelles vraies ou fausses de l'intérieur du
roi et de M\textsuperscript{me} de Maintenon sur M. le duc d'Orléans, de
conseils là-dessus, et des services que M. du Maine lui rendait en ces
occasions, services que M\textsuperscript{me} la duchesse d'Orléans
faisait valoir à merveilles, et qui ne tendaient qu'à persuader M. le
duc d'Orléans de l'attachement du duc du Maine pour lui, et de la
confiance qu'il y devait mettre, en même temps de payer ces services par
un concert et une union solidement prouvés pour entretenir un secours si
nécessaire. J'étais le plastron de ces sortes d'entretiens qui me
faisaient suer à trouver des défaites, et qui coûtaient au delà de toute
expression à mon naturel franc et droit. C'était après, entre M. le duc
d'Orléans et moi, à nous rendre compte l'un à l'autre de ces
conversations que nous avions eues chacun en particulier, et à nous
diriger et à convenir des propos que nous aurions à tenir chacun à part
à M\textsuperscript{me} la duchesse d'Orléans. «\,Nous sommes dans un
bois, me disait souvent ce prince, nous ne saurions trop prendre garde à
nous.\,»

Quoique M\textsuperscript{me} la duchesse d'Orléans ne pût ignorer mes
sentiments sur la bâtardise et tout ce qu'elle avait obtenu, elle ne
laissait pas de me parler sur toutes ces choses, parce qu'elles ne
regardaient pas le rang, mais la liaison avec M. du Maine et ce qui y
était nécessaire, fondée selon elle sur le besoin qu'en avait M. le duc
d'Orléans, et l'attachement pour lui du duc du Maine, continuellement
marqué par les avis qu'elle en recevait, et les services qu'il rendait,
chose dont nul autre que lui n'était à portée. Ce qui nous donna le plus
de peine fut le mariage du prince de Dombes avec M\textsuperscript{lle}
de Chartres, que M. du Maine voulait ardemment, et que
M\textsuperscript{me} la duchesse d'Orléans ne s'était pas mis moins
avant dans la tête, tout aussitôt que le roi eut accordé au duc du Maine
et au comte de Toulouse tous les mêmes rangs et honneurs qu'ils avaient
{[}pour{]} leur postérité. On aperçait du premier coup d'oeil tout
l'avantage que le duc du Maine tirait, pour la solidité des prodiges
qu'il avait entassés, de faire son fils gendre et beau-frère du seul
petit-fils et du seul fils de France, et frère du Dauphin, et de les
forcer par cette alliance à en devenir les protecteurs et les boucliers.
Je n'y trouvai d'issue que dans une approbation qui me donnât créance
pour les délais, car le refus eût été la perte de M. le duc d'Orléans.
Je montrai donc à M\textsuperscript{me} la duchesse d'Orléans, qui m'en
parla avant de l'oser proposer à M. son mari, que je goûtais cette
pensée, mais que je n'en pouvais approuver la précipitation. J'insistai
sur l'âge des parties, je m'étendis sur l'effroi que les princes du sang
et toute la cabale de Meudon prendraient de cette union si fort à
découvert, et tous les ennemis et les jaloux de M. le duc d'Orléans et
de M. du Maine. On peut juger que M\textsuperscript{me} la duchesse
d'Orléans ne se rendit pas, et que cette matière fut souvent débattue
entre nous.

Je ne me cachai pas à elle, dès la première fois qu'elle m'en parla, que
j'en dirais mon avis à M. le duc d'Orléans, s'il me le demandait\,; et
ce que j'eus de plus pressé fut de lui en rendre promptement compte. Il
approuva fort ce que j'avais répondu, il s'expliqua lui-même dans le
même sens, et nous coulâmes le temps de la sorte jusqu'à la mort de
Monseigneur. Alors, la cabale de Meudon n'étant plus à craindre, les
instances qui s'étaient un peu ralenties reprirent de nouveau. L'âge des
parties et les autres inconvénients déjà allégués furent le bouclier
dont nous parâmes, avec grand travail, jusqu'à la mort de M. {[}le
Dauphin{]} et de M\textsuperscript{me} la Dauphine. L'intérêt alors du
duc du Maine devint bien plus grand. Le roi vieillissait et changeait,
la régence regardait de plein droit M. le duc de Berry\,; l'avoir
contraire et M. le duc d'Orléans, ou pour protecteurs nécessaires comme
beau-frère et gendre, quelle immense différence\,! par conséquent, quels
manèges et quelles presses ne furent-ils pas employés\,! Je soutins tous
les assauts avec les mêmes armes dont je m'étais déjà servi, car
toujours j'étais le premier et le plus vivement attaqué, et M. le duc
d'Orléans y tint bon de son côté\,; mais c'était des recharges
continuelles. La mort de M. le duc de Berry fit une telle augmentation
d'intérêt qu'elle causa aussi les instances les plus violentes. M. du
Maine sentait le poids de ses crimes, du moins à l'égard de M. le duc
d'Orléans qui vivait, et ce prince était sur le point d'être régent, et
en plein état de se venger. Le duc du Maine en tremblait, et cela
n'était pas difficile à imaginer par tout ce que la peur des ducs lui
fit faire pour les mettre aux mains, comme on l'a vu, avec le parlement,
et comme on le verra en son lieu avec tout le monde.

Il ne s'agissait pas encore du testament ni des mesures qui ont été
racontées. Il ne voyait donc que ce mariage qui pût le rassurer. Aussi
dès qu'il eut mis la dernière main à sa grandeur héréditaire par s'être
fait déclarer lui, son frère et leur postérité, capables de succéder à
la couronne, il se servit de ce dernier comble comme d'une nouvelle
raison pour la prompte conclusion du mariage. Je fus encore attaqué
là-dessus le premier par M\textsuperscript{me} la duchesse d'Orléans,
qui comprenait apparemment qu'il fallait me persuader, sans quoi elle
n'arriverait point à faire ce mariage. Mes premières armes étaient
usées, les parties à marier avaient pris des années depuis que cette
affaire était sur le tapis. Les princes du sang étaient des enfants, et
M\textsuperscript{me} la Duchesse tombée depuis la mort de Monseigneur.
Les ennemis, les jaloux, le monde, c'était des mots et non des choses,
et cela, qui était vrai, m'avait été souvent répondu. Je m'avisai donc
d'une autre barrière, derrière laquelle je me retranchai. Je dis à
M\textsuperscript{me} la duchesse d'Orléans que j'étais surpris comment
avec tout son esprit, et M. du Maine avec tout le sien, et les
connaissances qu'ils avaient du caractère du roi l'un et l'autre, ils
pouvaient songer à faire alors ce mariage, qui était le moyen sûr et
prompt de perdre M. du Maine auprès du roi, jusqu'à un point dont
personne ne pouvait prévoir jusqu'où les suites en pourraient être
portées.

Ce début parut à M\textsuperscript{me} la duchesse d'Orléans infiniment
étrange\,; elle m'interrompit pour me le témoigner modestement. Je
m'expliquai ensuite, et lui dis que pour M. le duc d'Orléans, il
n'aurait guère à y perdre à la façon dont malheureusement il était avec
le roi, et à couvert de tout par sa naissance qui lui assurait la
régence sans qu'il fût possible de l'empêcher, et que l'âge du roi
laissait apercevoir d'assez près\,; que ce n'était donc pas par rapport
à lui que j'allais lui exposer ce que je pensais du mariage, mais par
rapport à M. du Maine. Je la priai de bien considérer comment le roi
était fait, combien il était jaloux, jusqu'où il portait la délicatesse
sur son autorité, à quel point il était susceptible d'indignation contre
toute pensée, et plus encore contre toutes mesures pour après lui\,; que
faire actuellement le mariage attaquait jusqu'au vif toutes ces
dispositions du roi, lequel, plus il avait fait pour M. du Maine et plus
grièvement se trouverait-il offensé, et qu'il ne lui pardonnerait jamais
que le premier pas qu'il ferait après le comble de l'habilité à la
couronne qui ne faisait que d'éclore, fût de lui faire sentir qu'il
comptait peu son autorité et sa puissance, s'il ne la soutenait par
celui qui y allait succéder, en conséquence de quoi il n'avait rien de
si pressé que de s'unir à ce successeur par les liens les plus étroits
et les plus publics\,; que c'était lui déclarer une persuasion entière
de sa mort prochaine, et en l'attendant, le vouloir tenir dans la
dépendance, établi, comme il était par cette union, avec le soleil
levant. Je paraphrasai ces propos avec tant de force que
M\textsuperscript{me} la duchesse d'Orléans en demeura étourdie, et
convint que ces considérations méritaient des réflexions.

Au sortir de cet entretien, qui fut long, je me hâtai d'en aller rendre
compte à M. le duc d'Orléans, qui fut charmé de l'invention, qui
l'adopta, et qui, non sans rire un peu de l'adresse, résolut de ne point
sortir de ce retranchement. J'eus encore des combats à essuyer tête à
tête, et avec M. le duc d'Orléans en tiers, qui avait la bonté de m'y
laisser la parole, dont je prenais la liberté de le bien quereller
après, et que cela n'en corrigeait point, parce qu'il lui était plus
commode d'applaudir à ce que je disais que de parler et de produire.
M\textsuperscript{me} la duchesse d'Orléans, qui avait eu le temps de
reprendre ses sens, et peut-être aussi d'être recordée, entra en quelque
débat sur l'impression que le roi recevrait de ce mariage. Comme tout ce
que j'y répondis ne pouvait être que le même thème en plusieurs façons,
auquel j'ajoutais ce que la crainte et la jalousie lui ferait ressentir
après coup et revenir même par les rapports du dehors, je n'allongerai
point cette matière par les dits et redits de nos fréquentes
conversations. J'ajouterai seulement que je la maintins toujours dans la
croyance que je trouvais le mariage très bon à faire aussitôt après la
mort du roi et que, si nous différions elle et moi de sentiment, ce
n'était que sur le temps et non sur la chose. Ce ne fut pas tout. Voyant
qu'ils ne pouvaient nous rassurer sur le crédit de M. du Maine, qui se
chargeait sans cesse de faire goûter au roi ce mariage, et qui répondait
de tout, et ce n'était pas là aussi de quoi nous doutions, mais dont
nous voulions absolument douter et demeurer incapables d'être rassurés
sur nos craintes, ils se rejetèrent à proposer un engagement et des
articles de mariage signés. Ce fut encore à moi à qui
M\textsuperscript{me} la duchesse d'Orléans en parla, avant d'en avoir
rien dit à M. le duc d'Orléans.

Le piège était grossier, mais il était difficile de ne se pas découvrir
en l'éludant. Toutefois je ne perdis pas la présence d'esprit. Je
m'écriai que ce serait pis que faire le mariage si le roi venait à
découvrir l'engagement, et qu'il y aurait de la folie à le hasarder dans
la sécurité qu'il lui demeurât caché à la longue\,; qu'elle se souvint
de ce qui lui était arrivé à elle-même, depuis si peu, de rengagement
pris entre elle et M\textsuperscript{me} la princesse de Conti pour le
mariage de leurs enfants\,; qu'encore que personne n'eût ici l'intérêt
personnel qu'avait eu M\textsuperscript{lle} de Conti à la trahison
qu'elle avait faite, il était vrai pourtant que tout bon sens répugnait
à se persuader que la connaissance de l'engagement pris et signé entre
M. le duc d'Orléans et M. du Maine pût demeurer caché au roi si curieux,
si attentif, si jaloux d'être instruit de ce qui se passait de plus
indifférent dans sa cour, dans Paris, et parmi tout ce qui pouvait être
connu de lui ou même l'amuser, à plus forte raison de ce qui pouvait se
passer d'important et d'intéressant dans sa plus intime famille\,; que
d'ailleurs c'était là une précaution tout à fait inutile dans un mariage
où la dot et les conventions n'étaient d'aucune considération pour le
faire ou pour le rompre, et que, quand le temps de liberté serait venu,
il n'y aurait ni plus de difficulté ni plus de longueur à le faire tout
de suite qu'à achever alors ce qui aurait été commencé aujourd'hui. Ce
fut un retranchement souvent attaqué, mais où je fis si belle défense,
et M. le duc d'Orléans aussi, que rien ne le put forcer. Vint après
l'affaire du bonnet après laquelle M\textsuperscript{me} la duchesse
d'Orléans sentit bien apparemment qu'il ne me fallait plus parler sur ce
mariage, et qui cessa en même temps aussi d'en plus rien dire à M. le
duc d'Orléans. D'entrer dans le détail journalier des panneaux tendus
par le duc du Maine, et de l'occupation de M\textsuperscript{me} la
duchesse d'Orléans à faire valoir l'importance de cultiver par toute
sorte de complaisance l'amitié du duc du Maine et ses soins pour M. le
duc d'Orléans, cela serait infini, et il suffit de dire une fois pour
toutes que ce fut le fléau domestique qui occupa M. le duc d'Orléans et
moi, jusqu'à la mort du roi, avec M\textsuperscript{me} la duchesse
d'Orléans. De cette adoration pour M. du Maine vint le danger extrême de
rien communiquer à M\textsuperscript{me} la duchesse d'Orléans sur le
présent et sur l'avenir, et ce secret continuel n'était pas un petit
embarras. Le prince le secouait, mais je n'avais pas la même ressource.

M\textsuperscript{me} la duchesse d'Orléans était bien persuadée que M.
le duc d'Orléans me confiait tout sans réserve, et que j'influais fort
dans tout ce qu'il pensait et pouvait pour le présent et pour le futur.
Elle en avait l'entière expérience, et elle voyait, plus distinctement
encore que le dehors, que j'étais l'unique avec qui il pût s'ouvrir sur
des matières si importantes, quoique le dehors ne le vît aussi que trop
clairement. Elle n'était pas moins persuadée que je n'étais pas sans
réflexion et sans projets sur ce qui devait suivre le présent règne.
Elle était donc fort attentive à découvrir ce que je pensais, et à me
promener dans nos fréquents tête-à-tête, quelquefois la duchesse Sforce
en tiers, quoique rarement, sur les personnes et les choses. J'étais
également en garde sur les unes et sur les autres, moins exactement
fermé sur les personnes, quoique fort circonspect, parce qu'elle
n'ignorait pas mes sentiments sur plusieurs\,; et pour les choses je me
sauvais par des généralités. Je me jetai aussi, à mesure que le terme se
découvrait de plus près, sur l'incurie, la légèreté, la paresse de M. le
duc d'Orléans, qui vivait comme si le temps présent devait toujours
durer\,; et quoique j'exagérasse fort ces plaintes, qui me servaient
encore à protester que de dépit je ne pensais plus à rien moi-même dans
l'inutilité où il était de penser tout seul, n'était que trop vrai,
comme on le verra dans son temps, que ces plaintes n'étaient que trop
fondées.

M\textsuperscript{me} la duchesse d'Orléans n'était pas la seule qui fût
dans la curiosité et dans l'inquiétude là-dessus. On a pu voir en
différents endroits que mon intime amitié avec la maréchale et la
duchesse de Villeroy jusqu'à leur mort, ni ma liaison particulière avec
le duc de Villeroy jusqu'à l'époque de ma préséance sur le duc de La
Rochefoucauld, n'avait pu vaincre mon éloignement pour le maréchal de
Villeroy, jusque-là que je ne m'en cachais pas avec elles\,; et qu'elles
se sont quelquefois diverties à m'enfermer dans un recoin par la
compagnie pour m'empêcher de sortir quand il entrait chez sa femme, et
de la mine qu'elles me voyaient faire. Je n'avais pas changé depuis\,;
hors de me faire écrire aux occasions chez le maréchal, ce qui ne s'omet
qu'en brouillerie ouverte, jamais il n'entendait parler de moi, et
jamais je ne l'abordais dans les lieux où je le rencontrais. Nous en
étions donc là ensemble, lorsque, aussitôt après la mort de M. {[}le
Dauphin{]} et de M\textsuperscript{me} la Dauphine,
M\textsuperscript{me} de Maintenon le tira de la plus profonde disgrâce,
et le fit subitement paraître à Marly en favori. Ses amis, ceux qui lui
avaient été le plus contraires, et le très grand nombre qui était les
plus indifférents, s'empressèrent à l'envi auprès de lui. Pour moi, je
ne m'en émus pas le moins du monde, et je laissai bouillonner la cour
autour de lui.

Ma surprise fut grande lorsqu'au bout d'une quinzaine je reçus de lui
les avances de politesse qu'il aurait pu attendre de moi, et
qu'incontinent après je ne pus paraître en aucun lieu où il fût, comme
les lieux de cour et d'autres par hasard, qu'il ne m'accostât et qu'il
ne liât conversation. Je le laissais toujours venir à moi le premier,
souvent même je l'évitais adroitement. Je répondais avec civilité aux
siennes, mais avec une mesure qui tenait fort de la sécheresse. Rien ne
le rebuta. Il cherchait à la messe du roi à Marly à partager mon
carreau, ou à me faire partager le sien, à mettre le sien auprès du
mien, à m'en faire apporter un par le suisse de la chapelle qui était
chargé de ce soin-là\,; surtout de m'entretenir pendant toute la messe,
et, suivant sa manière, de me faire des questions. Ce manège ne dura pas
longtemps sans me jeter sur les affaires et sur les personnages en
effleurant, à quoi il avait beau jeu avec moi qui me gardais de lui, et
qui me tenais nageant sur les superficies. Peu à peu il se mit, comme à
l'impromptu\,; à pousser plus avant, avec sa façon de conversation sans
suite et rompue\,; et de là, se rendant de plus en plus familier, je le
vis venir me demander à dîner comme nous nous mettions à table, et
bientôt après venir dîner ou souper très ordinairement, et quelquefois,
même arriver à la fin du premier service ou après. J'en étais désolé.
J'ai toujours eu partout un très gros ordinaire pour un nombre d'amis et
de connaissances familières qui y venaient sans prier\,; j'aimais et eux
aussi à y être libres\,; le maréchal de Villeroy nous pesait
cruellement. J'en étais extrêmement importuné, parce que je voyais
clairement qu'il ne venait que pour me pomper\,; et comme son esprit
était court sans être pourtant bête, et qu'il était plein de vent, il me
disait des riens du roi et de M\textsuperscript{me} de Maintenon pour me
faire parler, parmi lesquels il ne s'apercevait pas qu'il y avait
quelquefois des choses qui me manifestaient sa mission et ce qu'il se
proposait de découvrir. Quelquefois il me louait M. le duc d'Orléans,
beaucoup plus souvent le blâmait, se lâchait là-dessus à des confidences
sur le roi et M\textsuperscript{me} de Maintenon, et ne se contraignit
point de me faire les questions les plus fortes et les plus redoublées,
et retournées en cent façons, sur les projets de M. le duc d'Orléans
pour l'avenir et ce que j'en pensais moi-même\,; toujours
s'interrompant, me regardant entre deux yeux, raisonnant lui-même, et se
portant sur l'avenir avec une liberté qui me surprenait, quoique au
métier qu'il faisait avec moi il n'avait rien à craindre, quand j'aurais
voulu abuser cette confiance qu'il me voulait persuader qui
s'établissait entre nous. Il passait de la sorte des heures entières, et
souvent plus, dans ma chambre, à toutes les heures, tête à tête, parce
que tout en entrant, il me priait que nous ne fussions point
interrompus, et avec cela me prenait très souvent en particulier chez le
roi ou dans les jardins à sa suite. C'était un homme qui croyait
toujours vous circonvenir et vous découvrir.

Je profitais du peu de suite et des ressauts ordinaires de sa
conversation\,: force crainte et respect du roi, parfaite inutilité de
penser à rien pour après lui, chose de soi peu décente et peu permise,
et matière si dépendante de tant de circonstances qui ne se pouvaient ni
prévoir ni peut-être imaginer\,; que bâtir des projets pour ces temps,
c'était bâtir des châteaux en Espagne. C'étaient là mes réponses, avec
force louanges du roi, et le cercle de généralités et défaites tournées
en tous sens dont je ne me laissais point tirer. Jamais je n'allais chez
lui, jamais je ne l'attaquais, jamais ne parut s'en apercevoir. Nous
riions, M. le duc d'Orléans et moi, d'un tel personnage. Ce commerce
forcé dura jusqu'à la querelle du duc d'Estrées et du comte d'Harcourt,
que je me lâchai fortement contre tout ce qui se passa de sa part, sur
la prétention des maréchaux de France de soumettre les ducs à leur
tribunal, où je ne l'épargnai pas. Cela nous brouilla ouvertement. Je ne
me contraignis de là en avant ni sur les propos ni sur les procédés.
Quelque temps après il s'en alla à Lyon, d'où il arriva triomphant
successeur des places de M. de Beauvilliers dans le conseil, et plus
brillant que jamais. Ce veau d'or n'eut point mon encens ni aucun
compliment de ma part\,; et nous en demeurâmes en ces termes jusqu'après
la mort du roi.

Le maréchal de Villeroy a tant figuré, devant et depuis, qu'il est
nécessaire de le faire connaître. C'était un grand homme bien fait, avec
un visage fort agréable, fort vigoureux, sain, qui sans s'incommoder
faisait tout ce qu'il voulait de son corps. Quinze et seize heures à
cheval ne lui étaient rien, les veilles pas davantage. Toute sa vie
nourri et vivant dans le plus grand monde\,; fils du gouverneur du roi,
élevé avec lui dans sa familiarité dès leur première jeunesse, galant de
profession, parfaitement au fait des intrigues galantes de la cour et de
la ville, dont il savait amuser le roi qu'il connaissoit à fond, et des
faiblesses duquel il sut profiter, et se maintenir en osier de cour dans
les contre-temps qu'il essuya avant que je fusse dans le monde. Il était
magnifique en tout, fort noble dans toutes ses manières, grand et beau
joueur sans se soucier du jeu, point méchant gratuitement, tout le
langage et les façons d'un grand seigneur et d'un homme pétri de la
cour\,; glorieux à l'excès par nature, bas aussi à l'excès pour peu
qu'il en eût besoin, et à l'égard du roi et de M\textsuperscript{me} de
Maintenon valet à tout faire. On a vu un crayon de lui à propos de son
subit passage de la disgrâce à la faveur.

Il avait cet esprit de cour et du monde que le grand usage donne, et que
les intrigues et les vues aiguisent, avec ce jargon qu'on y apprend, qui
n'a que le tuf, mais qui éblouit les sots, et que l'habitude de la
familiarité du roi, de la faveur, des distinctions, du commandement
rendait plus brillant, et dont la fatuité suprême faisait tout le fond.
C'était un homme fait exprès pour présider à un bal, pour être le juge
d'un carrousel, et, s'il avait eu de la voix, pour chanter à l'Opéra les
rôles de rois et de héros\,; fort propre encore à donner les modes et à
rien du tout au delà. Il ne se connaissoit ni en gens ni en choses, pas
même en celles de plaisir, et parlait et agissait sur parole\,; grand
admirateur de qui lui imposait, et conséquemment dupe parfaite, comme il
le fut toute sa vie de Vaudemont, de M\textsuperscript{me} des Ursins et
des personnages éclatants\,; incapable de bon conseil, comme on l'a vu
sur celui que lui donna le chevalier de Lorraine\,; incapable encore de
toute affaire, même d'en rien comprendre par delà l'écorce, au point
que, lorsqu'il fut dans le conseil, le roi était peiné de cette ineptie,
au point d'en baisser la tête, d'en rougir et de perdre sa peine à le
redresser, et à tâcher de lui faire comprendre le point dont il
s'agissait. C'est ce que j'ai su longtemps après de Torcy, qui était
étonné au dernier point de la sottise en affaires d'un homme de cet âge
si rompu à la cour. Il y était en effet si rompu qu'il en était
corrompu. Il se piquait néanmoins d'être fort honnête homme\,; mais
comme il n'avait point de sens, il montrait la corde fort aisément, aux
occasions même peu délicates, où son peu de cervelle le trahissait, peu
retenu d'ailleurs quand ses vues, ses espérances et son intérêt, même
l'envie de plaire et de flatter, ne s'accordaient pas avec la probité.
C'était toujours, hors des choses communes, un embarras et une confiance
dont le mélange devenait ridicule. On distinguait l'un d'avec l'autre,
on voyait qu'il ne savait où il en était\,; quelque \emph{sproposito}
prononcé avec autorité, étayé de ses grands airs, était ordinairement sa
ressource Il était brave de sa personne\,; pour la capacité militaire on
en a vu les funestes fruits. Sa politesse avait une hauteur qui
repoussait\,; et ses manières étaient par elles-mêmes insultantes quand
il se croyait affranchi de la politesse par le caractère des gens. Aussi
était-ce l'homme du monde le moins aimé, et dont le commerce était le
plus insupportable, parce qu'on n'y trouvait qu'un tissu de fatuité, de
recherche et d'applaudissement de soi, de montre de faveur et de
grandeur de fortune, un tissu de questions qui en interrompaient les
réponses, qui souvent ne les attendaient pas, et qui toujours étaient
sans aucun rapport ensemble. D'ailleurs nulle chose que des contes de
cour, d'aventures, de galanteries\,; nulle lecture, nulle instruction,
ignorance crasse surtout, plates plaisanteries, force vent et parfait
vide. Il traitait avec l'empire le plus dur les personnes de sa
dépendance. Il est incroyable les traitements continuels que jusqu'à sa
mort il a faits continuellement à son fils qui lui rendait des soins
infinis et une soumission sans réplique, et j'ai su par des amis de
Tallard, dont il était fort proche et {[}qu'il{]} a toujours protégé,
qu'il le mettait sans cesse au désespoir, même parvenu à la tête de
l'armée. Enfin la fausseté, et la plus grande et la plus pleine opinion
de soi en tout genre, mettent la dernière main à la perfection de ce
trop véritable tableau.

\hypertarget{chapitre-vii.}{%
\chapter{CHAPITRE VII.}\label{chapitre-vii.}}

~

{\textsc{Quels, à l'égard de M. le duc d'Orléans, étaient le maréchal de
Villeroy, Tallard, le cardinal et le prince de Rohan, la duchesse de
Ventadour, Vaudemont, ses nièces, Harcourt, Tresmes, le duc de Villeroy,
Liancourt, La Rochefoucauld, Charost, Antin, Guiche, Aumont, le premier
écuyer, M. de Metz, Huxelles, le maréchal et l'abbé d'Estrées, les
ministres, les secrétaires d'État, le P. Tellier.}} {\textsc{-
Inquiétude et manège du P. Tellier avec moi.}} {\textsc{- Caractère du
duc de Noailles.}} {\textsc{- Inquiétude du duc de Noailles sur les
desseins de M. le duc d'Orléans.}} {\textsc{- Contade\,; sa fortune\,;
son caractère.}} {\textsc{- Liaison du duc de Noailles et de Maisons.}}
{\textsc{- Caractère de Canillac.}} {\textsc{- Liaison du duc de
Noailles avec Canillac par Maisons.}} {\textsc{- Noailles et l'abbé
Dubois anciennement liés.}} {\textsc{- Liaison de Noailles et
d'Effiat.}} {\textsc{- Extraction et caractère d'Effiat\,; ses
liaisons.}} {\textsc{- Effiat bien traité du roi\,; fort considéré de M.
le duc d'Orléans.}} {\textsc{- Noailles raccroche Longepierre, lequel
s'abandonne après à l'abbé Dubois.}}

~

Monsieur avait passé toute sa vie, depuis son enfance jusqu'à sa mort,
dans l'amitié et la confiance pour le maréchal de Villeroy. L'habitude,
dès la plus tendre jeunesse jamais interrompue, et soutenue par le
chevalier de Lorraine et par Effiat, ses amis intimes, avait mis à
portée de tout avec lui. Il était l'entremetteur de toutes les petites
querelles qui arrivaient entre le roi et Monsieur, dont il m'a conté des
aventures étranges sur le vilain goût de Monsieur que le roi ne pouvait
souffrir, dont il lui faisait porter des romancines par le maréchal,
jusqu'à ne vouloir pas que La Carte, devenu capitaine de ses gardes, fût
avec lui des voyages de Marly, et à charger le maréchal de dire à
Monsieur que, s'il l'amenait, il le ferait jeter par les fenêtres\,; et
les peines que le maréchal avait entre eux deux sur ce fâcheux chapitre
qui recommençait souvent, et tantôt à empêcher Monsieur de mener cet
homme, tantôt d'obtenir du roi qu'il accompagnât Monsieur à Marly. Je
rapporte ces détails pour faire voir que M. le duc d'Orléans était
accoutumé, depuis qu'il était au monde, à considérer et à compter le
maréchal de Villeroy, et que le maréchal de Villeroy, en ayant été
toujours traité avec toute sorte de distinctions, lui devait, par
rapport à feu Monsieur et à lui-même, beaucoup d'attachement. Ce ne fut
pas là sa conduite.

Le bel air et la mode, dont il était esclave, ne lui permirent pas
d'abord de suivre à cet égard ce que le devoir, l'honneur et la
reconnaissance demandaient de lui. Bientôt après il n'eut garde de ne
s'éloigner pas de plus en plus d'un prince dont le roi était pas
content, et qui en était encore moins content lui-même. Enfin, dès que
M\textsuperscript{me} de Maintenon l'eut pris en aversion, il était trop
vil courtisan pour ne se pas piquer d'en épouser tous les sentiments. Il
était de plus lié en dupe avec les Rohan, les Tallard, qui se moquèrent
de lui quand ils n'en eurent plus besoin, {[}avec{]} M. de Vaudemont et
ses nièces, qui tous unis à M\textsuperscript{me} la Duchesse avaient eu
grand soin d'entretenir Monseigneur dans sa haine, et depuis sa mort
avait pu pardonner à M. le duc d'Orléans tout ce qu'ils avaient fait
contre lui, et trouvaient en même temps à plaire à M\textsuperscript{me}
de Maintenon. Je mets ici Tallard avec les autres, parce que depuis le
mariage de son fils il était qu'un avec les Rohan, et qu'auparavant il
suivait le gros et le torrent. Ils avaient entraîné la duchesse de
Ventadour qui, comblée par Monsieur et par Madame de tout ce qui peut
témoigner l'amitié et la plus grande considération, et qui ayant
toujours été traitée avec les mêmes égards par M. le duc d'Orléans, ne
devait pas devenir son ennemie, et qui toutefois s'y laissa emporter. Il
y avait plus de cinquante ans que le maréchal de Villeroy et elle se
faisaient fort publiquement l'amour, sans toutefois s'en contraindre de
part et d'autre pour ce qu'ils trouvaient à leur gré, et sans que cette
liberté réciproque altérât le moins du monde leur commerce, sur lequel
la plus intime amitié et confiance était entée.

M\textsuperscript{me} de Ventadour avait été charmante\,; elle conserva
toujours un grand air et un air de beauté, et parfaitement bien faite.
Nul esprit, de la bonté, mais gouvernée toute sa vie, et faite pour
l'être. D'ailleurs esclave de la cour par ses aventures et ses besoins
domestiques, et quand elle en fut à l'abri, par habitude et par rage de
places et d'être. Il fallait donc suivre les impressions des Rohan qui
en faisaient tout ce qu'ils voulaient, et celles de son ancien galant,
surtout se conformer à ce qu'on lui montrait du roi et de
M\textsuperscript{me} de Maintenon. Harcourt était trop avant ancré avec
elle et avec M\textsuperscript{me} des Ursins, trop fin courtisan
d'ailleurs, et trop habile politique pour prendre d'autres brisées que
les siennes\,; et le duc de Tresmes, trop plat pour ne pas suivre la
mode et la grande volée de la cour à l'égard de M. le duc d'Orléans. Le
duc de Villeroy, accoutumé au joug de son père, ne pouvait penser
autrement que lui, lié d'ailleurs de toute sa vie et le plus intimement
avec M. de Luxembourg, M. de La Rochefoucauld, et le marquis de
Liancourt, son frère, qui avait de l'esprit et du sens pour eux tous.
Ils ne étaient pu défaire de cet éloignement de M. le duc d'Orléans,
pour en parler modérément, qu'ils avaient puisé dans la société intime
de M. le prince de Conti, dont ils avaient à la fin comme hérité. La
probité singulière du maréchal de Boufflers avait soutenu contre ce
torrent, mais il ne vivait plus, et Charost qui avait eu sa charge était
tout à moi, mais ce était pas un homme à exister, par conséquent à
compter. D'Antin, tout à M\textsuperscript{me} la Duchesse, et qui,
établi dans l'intérieur des cabinets, ne pouvait ignorer les sentiments
du roi et de M\textsuperscript{me} de Maintenon, se tenait à l'écart
dans la douleur, sur l'avenir, de ne pouvoir se partager. Villars moins
empêtré, plus frivole en apparence, ne prenait point parti, se tenait
habilement entre deux, et gardait toutes sortes de mesures, qu'il
prétextait même de la place de chevalier d'honneur de
M\textsuperscript{me} la duchesse d'Orléans, dans laquelle son père
était mort.

Berwick rarement fixé en place, habitant Saint-Germain, quoique fort
avant dans la cour, imitait cette conduite, et gardait tout à fait celle
d'un homme qui avait commandé en Espagne sous M. le duc d'Orléans et qui
en avait été content. Huxelles, vil esclave de la faveur, qu'on a vu se
déshonorer publiquement à l'apothéose des bâtards, et valet du premier
président, ainsi que son cousin, le premier écuyer, avec qui il était
qu'un, était au duc du Maine, et à tous les ennemis de M. le duc
d'Orléans, mais en tapinais, et dans le doute de l'avenir le plus
sourdement qu'il lui était possible, sans se rapprocher jamais de ce
prince, mais se faisant vanter à lui par Maisons. Le duc d'Aumont,
beau-frère du premier écuyer, et lié à lui, conduits tous deux par
M\textsuperscript{me} de Beringhen, méchante, intrigante, avec beaucoup
d'esprit, fausse, basse, et dangereuse au dernier point. On a vu, à
l'occasion du bonnet, quel était cet homme qui voulait être de tous les
côtés, et qui devint bientôt le mépris de tous. Le maréchal d'Estrées et
l'abbé son frère étaient honnêtes gens, et tout à fait portés à M. le
duc d'Orléans, mais si faibles, si courtisans, si timides, qu'il y avait
à rire de leurs frayeurs. Pour le duc de Guiche, c'était un homme sans
consistance, sans esprit, qui avait que des airs et une charge
importante, qui était gueux, avare, dépensier, qui serait à qui lui
donnerait davantage, et qui était gouverné par Contade, major du
régiment des gardes, et par un aide-major appelé Villars, qui faisait de
l'important, et qui était qu'un avec Contade. Je différerai peu à parler
du duc de Noailles. En attendant, voilà le principal des gens qui
méritaient d'être comptés. On ne finirait pas à traiter de ce qui
figurait moins, et des subdivisions des femmes.

Pour les ministres, la discussion en sera bientôt faite, par rapport à
M. le duc d'Orléans. On a déjà vu Voysin âme damnée de
M\textsuperscript{me} de Maintenon et de M. du Maine, et le maréchal de
Villeroy. Desmarets, gendre de Béchameil mort surintendant de Monsieur,
et beau-frère de Nointel que Monsieur, avant le retour de Desmarets,
avait fait faire conseiller d'État, semblait devoir un attachement
marqué pour M. le duc d'Orléans. Son ami intime le maréchal de Villeroy
était son guide sur la politique de la cour\,; et Desmarets compta pour
tout le roi et M\textsuperscript{me} de Maintenon, et qu'ils ne
finiraient point\,; tout le reste pour rien, et se conduisait en
conséquence. Torcy, dont la sœur Bouzols avait grand crédit sur lui par
confiance en son esprit dont elle avait comme un démon, et de laideur et
de méchanceté espèce de démon elle-même, et tout à M\textsuperscript{me}
la Duchesse de tous les temps, l'aurait volontiers tourné de ce côté-là.
Il avait une égale horreur de M. du Maine, et de ce qui se disait de M.
le duc d'Orléans. Il connaissait bien le roi, et n'aimait point
M\textsuperscript{me} de Maintenon, qui aussi lui était fort contraire,
mais il était assez ami du maréchal de Villeroy et des Estrées. C'était
en ce genre les deux contraires. Il était, mais intimement, de Castries
et de sa femme, tous deux à M\textsuperscript{me} la duchesse d'Orléans,
et il était aussi de M. de Metz qui, sans savoir pourquoi, était fort
contraire à M. le duc d'Orléans. De tant de contrastes rien ne
résultait. Torcy, enveloppé dans sa sagesse et dans ses fonctions, ne
montra rien, et ne fit aucun pas d'un côté ni d'un autre. Voilà tous les
ministres. Restaient deux secrétaires d'État qui ne étaient point\,:
Pontchartrain fort contraire à M. le duc d'Orléans, pour se faire de
fête auprès de M\textsuperscript{me} de Maintenon et des importants\,;
et La Vrillière, dont la charge et l'emploi était la cinquième roue d'un
chariot. Je remets à faire connaître plus particulièrement ceux des
personnages sur qui je ne me suis pas encore étendu à mesure qu'on les
verra arriver aux places, ou qu'il sera question d'eux pour cela entre
M. le duc d'Orléans et moi.

Le P. Tellier ne doit pas être oublié. On a vu son caractère, et depuis,
qu'il servit fort utilement M. le duc d'Orléans pour le mariage de M. le
duc de Berry. Quoiqu'il ait eu la discrétion de ne jamais rien dire sur
l'odieux chapitre du poison, je suis persuadé qu'il n'y servit pas moins
bien M. le duc d'Orléans. Il voulait le repos du roi, il haïssait
M\textsuperscript{me} de Maintenon qui ne le haïssait pas moins\,; il
voulait trouver le roi tranquille, et de bonne humeur, pour toutes les
choses qu'il voulait insinuer ou obtenir\,; et au peu qu'il m'a dit,
j'ai soupçonné qu'il connaissait M. du Maine. Il ne s'est trouvé de
contrebande en rien sur M. le duc d'Orléans, et il n'a paru par rien
qu'il ait eu nulle part au testament du roi, ni aux dispositions qu'il a
faites outre celles de son testament, comme les grandeurs des bâtards,
quoique je croie aussi qu'il ne s'y est pas opposé si le roi l'a
consulté. Il en voulait et en attendait trop pour le contredire sur un
point si chéri, moins encore à se mettre au hasard d'être congédié. On a
vu en plus d'un endroit à quel point lui et moi en étions ensemble\,:
cela dura jusqu'à la mort du roi.

Pendant la dernière année de sa vie, surtout vers les fins, ce père me
promenait sur tous les personnages, et me pressait de lui dire ce que
j'en pensais, enfin de les lui dépeindre. Je me mettais à rire, et je
lui disais qu'il les connaissait mieux que moi. Il insistait encore
davantage, et me disait qu'il avait pu connaître que ses livres, occupé
dans l'intérieur, comme il avait toujours été avant d'être appelé à la
cour, et que depuis qu'il y était, les affaires que lui donnait sa place
ne lui avaient pas donné un moment de loisir pour pouvoir être informé
des personnes ni des choses qui étaient pas de son ministère\,; puis en
m'accablant de cajoleries et de louanges, il me disait qu'il n'y avait
que moi avec qui il pût s'ouvrir avec confiance, et avoir celle que je
voudrais bien répondre à la sienne en répondant à ses questions, et le
mettant au fait des personnes. Il n'y en eut aucune sur qui il m'en fit,
et réitérât tant et me pressât davantage que sur M\textsuperscript{me}
de Maintenon, M. du Maine, et M\textsuperscript{me} la Duchesse. J'étais
d'autant plus embarrassé que je n'étais pas persuadé de son ignorance,
et que néanmoins je l'avais vu souvent, et le voyais encore tomber, et
vraiment, dans des lourdises là-dessus d'un paysan de basse Normandie
qu'il était, qui n'en serait jamais sorti. Outre que je ne me fiais à
lui que de bonne sorte, je craignais que le roi ne se servit de lui,
d'autant plus que cela redoubla depuis que j'eus cessé tout commerce
avec le maréchal de Villeroy. Je n'avais rien à perdre du côté de
M\textsuperscript{me} de Maintenon, de M. du Maine, de
M\textsuperscript{me} la Duchesse, du maréchal de Villeroy, de
Pontchartrain, et de quelques autres. Ceux-là me servirent à satisfaire
sa vraie ou feinte confiance, et à me donner moyen de réserve sur qui je
ne voulus pas m'expliquer avec lui.

Le duc de Noailles, auquel il en faut enfin venir, est un homme dont la
description et ses suites coûteront encore plus à mon amour-propre que
n'a fait le tableau de M\textsuperscript{me} la duchesse de Berry. Quand
je n'avouerais pas que je ne le connaissais point au temps dont j'écris,
et que je croyais le connaître, qu'on ne se trompa jamais plus
lourdement que je fis, et qu'on ne peut pas être plus complètement sa
dupe et en tous points, on le verrait clairement par le récit de ce qui
s'est passé depuis en tous genres, de cour, d'affaires, d'État, de mon
particulier. Je ne chercherai point à diminuer ma sottise ni à charger
le tableau. La vérité la plus pure et la plus exacte sera ici, comme
partout, mon guide unique et ma maîtresse. Je demande seulement grâce
pour quelque répétition de ce qui se trouve peut-être répandu sur lui à
propos de ses premières recherches pour moi, mais la vue d'un tout
ensemble mérite ici cette indulgence.

Le serpent qui tenta Ève, qui renversa Adam par elle, et qui perdit le
genre humain, est l'original dont le duc de Noailles est la copie la
plus exacte, la plus fidèle, la plus parfaite, autant qu'un homme peut
approcher des qualités d'un esprit de ce premier ordre, et du chef de
tous les anges précipités du ciel. La plus vaste et la plus insatiable
ambition, l'orgueil le plus suprême, l'opinion de soi la plus confiante,
et le mépris de tout ce qui n'est point soi, le plus complet\,; la soif
des richesses, la parade de tout savoir, la passion d'entrer dans tout,
surtout de tout gouverner\,; l'envie la plus générale, en même temps la
plus attachée aux objets particuliers, et la plus brûlante, la plus
poignante\,; la rapine hardie jusqu'à effrayer, de faire sien tout le
bon, l'utile, l'illustrant d'autrui\,; la jalousie générale,
particulière et s'étendant à tout\,; la passion de dominer tout la plus
ardente, une vie ténébreuse, enfermée, ennemie de la lumière, tout
occupée de projets et de recherches de moyens d'arriver à ses fins, tous
bons pour exécrables, pour horribles qu'ils puissent être, pourvu qu'ils
le fassent arriver à ce qu'il se propose\,; une profondeur sans fond,
c'est le dedans de M. de Noailles. Le dehors, comme il vit et qu'il
figure encore, on sait comme il est fait pour le corps\,: des pieds, des
mains, une corpulence de paysan et la pesanteur de sa marche,
promettaient la taille où il est parvenu. Le visage tout dissemblable\,:
toute sa physionomie est esprit, affluence de pensées, finesse et
fausseté, et n'est pas sans grâces. Une éloquence naturelle, une
élocution facile, une expression telle qu'il la veut\,; un homme
toujours maître de soi, qui sait parler toute une journée et avec
agrément sans jamais rien dire, qui en conversation est tout à celui à
qui il veut plaire, et qui pense et sent si naturellement comme lui, que
c'est merveille qu'une fortuite conformité si semblable. Jamais
d'humeur, égalité parfaite, insinuation enchanteresse, langage de
courtisan, jargon des femmes, bon convive, sans aucun goût quand il le
faut, revêtu sur-le-champ des goûts de chacun\,; égale facilité à louer
et à blâmer le même homme ou la même chose, suivant la personne qui lui
parle\,; grand flatteur avec un air de conviction et de vérité qui
l'empêche d'y être prodigue, et une complaisance de persuasion factice
qui l'entraîne à propos malgré lui dans votre opinion, ou une persuasion
intime tout aussi fausse, mais tout aussi parée, quand il lui convient
de vous résister, ou de tâcher, comme malgré lui, de vous entraîner où
il est entraîné lui-même. Toujours à la mode, dévot, débauché, mesuré,
impie tour à tour selon qu'il convient\,; mais ce qui ne varie point,
simple, détaché, ne se souciant que de faire le bien, amoureux de
l'État, et citoyen comme on était à Sparte. Le front serein, l'air
tranquille, la conversation aisée et gaie, lorsqu'il est le plus agité
et le plus occupé\,; aimable, complaisant, entrant avec vous quand il
médite de vous accabler des inventions les plus infernales, et quelque
long délai qui arrive entre l'arrangement de ses machines et leur effet,
il ne lui coûte pas la plus légère contrainte de vivre avec vous en
liaison, en commerce continuel d'affaires et de choses de concert, enfin
en apparences les plus entières de l'amitié la plus vraie et de la
confiance la plus sûre\,; infiniment d'esprit et toutes sortes de
ressources dans l'esprit, mais toutes pour le mal, pour ses désirs, pour
les plus profondes horreurs, et les noirceurs les plus longuement
excogitées, et pourpensées de toutes ses réflexions pour leur succès.
Voilà le démon, voici l'homme.

Il est surprenant qu'avec tant d'esprit, de grâces, de talents, tant de
désir d'en faire le plus énorme usage, tant d'application à y parvenir,
et tant de moyens par sa position particulière, de charges, d'emplois,
de famille, d'alliances et de fortune, il n'eût pas su faire un ami, non
pas même parmi ses plus proches Il n'y ménagea jamais que sa sœur, la
duchesse de Guiche, par le goût déterminé de M\textsuperscript{me} de
Maintenon pour elle, et le duc de Guiche, à cause de sa charge pour
avoir crédit sur lui, qui, de son côté, était en respect devant l'esprit
du duc de Noailles. Il n'est pas moins étonnant encore que cet homme si
enfermé, et en apparence si appliqué, qui se piquait de tout savoir, de
se connaître en livres, et d'amasser une nombreuse bibliothèque, qui
carrossait les gens de lettres et les savants pour en tirer, pour s'en
faire honneur, pour s'en faire préconiser, n'ait jamais passé l'écorce
de chaque matière, et que le peu de suite de son esprit, excepté pour
l'intrigue, ne lui ait pu permettre d'approfondir rien, ni de suivre
jamais, quinze jours, le même objet pour lequel tour à tour il avait
abandonné tous les autres. Ce fut même légèreté en affaires, par
conséquent la même incapacité. Jamais il n'a pu faire un mémoire sur
rien\,; jamais il n'a pu être content de ceux qu'il a fait faire,
toujours corriger, toujours refondre, c'était son terme favori\,; on l'a
vu dans la surprise que nous lui fîmes à Fontainebleau. Ce n'est pas
tout\,: il n'a jamais pu tirer de soi une lettre d'affaires. Ses
changements d'idées désolaient ceux qu'il employait, et les accablaient
d'un travail toujours le même, toujours à recommencer. C'est une maladie
incurable en lui, et qui éclate encore par le désordre qu'elle a mis
dans les expéditions, les amas en divers lieux, les ordres réitérés et
changés dix, douze, quinze fois dans le même jour, et tous
contradictoires, aux troupes qu'il a commandées dans ces derniers temps,
et à son armée entière pour marcher ou demeurer, qui l'a rendu le fléau
des troupes et des bureaux. Je ne parlerai point de sa capacité
militaire, dont il vante volontiers les hauts faits\,; je me tairai
pareillement sur sa valeur personnelle\,; j'en laisse le public juge\,;
je m'en rapporte à lui, et même aux armées ennemies opposées à la sienne
en Italie, en Allemagne et en Flandre, et aux événements qui en ont
résulté jusqu'en cette année 1745, en septembre.

Si cette partie a été si complètement dévouée, je puis m'assurer que le
reste ne le sera pas moins clairement par les faits publics que j'ai à
rapporter dans ce qui a accompagné et suivi la mort du roi, si j'ai le
temps d'achever ces Mémoires, et que ceux que ce portrait aura
épouvantés jusqu'à être tentés de le croire imaginaire se trouveront
saisis d'horreur et d'effroi quand les faits auront prouvé, et des faits
clairs, et quant à leur vérité manifestes, que les paroles n'ont pu
atteindre la force de ce qu'elles ont voulu annoncer, et quelle
surprise, de plus de n'y pouvoir méconnaître un coin très déclaré de
folie.

M. de Noailles jeté à moi par les raisons qui ont été expliquées alors,
et reçu par celles que j'ai exposées, n'oublia rien pour m'enchanter à
lui. Il fit sa cour à ceux de mes amis qu'il crut les plus intimes, et
en qui il jugea que j'avais le plus de confiance\,; il fit sa cour à
M\textsuperscript{me} de Saint-Simon avec le plus grand soin. Point de
semaines qu'il ne mangeât plusieurs fois chez moi, quelquefois nous chez
lui. Il n'y eut recherche, soins, industrie oubliés. Tous mes sentiments
avaient toujours été les siens, jusqu'à mes goûts et pour gens et pour
choses, l'identité ne pouvait être plus parfaite. Je n'ai peut-être que
trop répété de choses qui se trouvent t. X, p.~35 et suivantes, du
contenu entier desquelles il est nécessaire de se souvenir
distinctement. Le commerce étroit, continuel, plein de confiance établi
comme on l'a vu, et soutenu entre le duc de Noailles et moi, lui donnait
beau jeu à me sonder sur le futur. C'était sur ces temps, qui désormais
semblaient prochains, qu'il déployait tous ses raisonnements, et qu'il
ne cessait de me donner des attaques pour découvrir mes pensées et
celles de M. le duc d'Orléans. Mon plan était fait, il y avait
longtemps, et je n'en étais pas à avoir bien tout discuté avec ce
prince. Mais outre que ce qui se passait entre lui et moi était son
secret plus que le mien, étais bien éloigné de m'ouvrir de rien à
personne.

Cette réserve colorée comme je le pus ne rebuta point le duc de
Noailles, mais il languit longtemps dans son impatience et dans son
inquiétude là-dessus. Son agitation ne était pas bornée à moi seul par
rapport à M. le duc d'Orléans. Il était d'ailleurs, et pour des vues
différentes et plus anciennes, attaché Contade qui était, comme je l'ai
dit, major du régiment des gardes, qui gouvernait le duc de Guiche, et
qu'on a vu en plus d'une occasion ici dans toute la confiance du
maréchal de Villars, et dépêché plusieurs fois par lui de l'armée, et
après, de Rastadt, pour traiter directement avec le roi des choses de
confiance.

Contade était un gentilhomme d'Anjou, qui avait été beau et bien fait,
qui avait été fort à la mode en galanteries nombreuses et distinguées,
qui s'en mêlait encore, qui par d'excellentes chiennes couchantes que
son père et lui donnaient au roi de temps en temps, s'en était fait
connaître, puis goûter dans le détail de son emploi qui l'approchait
souvent de lui. Il était aimé et considéré à la cour de ce qu'il y avait
de meilleur et de plus distingué\,; il avait pris tout le soin possible
de l'être aussi du régiment des gardes, de toute l'infanterie dont il
faisait le détail à l'armée, et de ce qui y servait de plus marqué en
naissance, entours ou grades, surtout en mérite pour les officiers
particuliers. Il avait peu d'esprit, mais tout tourné à la conduite, du
sens, du secret, du jugement, une modestie qui le tenait plus qu'à sa
place, et dont on lui savait gré, beaucoup de sagesse et une discrétion
qui lui avait dévoué les dames, en sorte que, d'amant heureux il était
devenu ami de confiance. Il l'était de M\textsuperscript{me} de Maisons,
et Maisons qui le voyait un personnage en son genre, et qui ne
négligeait rien, en avait fait le sien. Contade fut donc employé pour la
liaison de Noailles et de Maisons, et elle était déjà étroite lors de la
scène dont j'ai parlé, qui se passa chez Maisons, entre lui, le duc de
Noailles et moi, qu'il avait envoyé chercher à Marly, le jour de la
déclaration de l'habilité des bâtards à la couronne.

Maisons qui, tout courtisan qu'il était, n'était pas au fait toujours de
l'intrinsèque, était ravi de s'accrocher au duc de Noailles par vanité,
et plus encore par intérêt dans la position présente du duc dont il
ignorait l'état avec le roi et M\textsuperscript{me} de Maintenon, et
pour le futur encore, où il comptait qu'un homme aussi établi, et avec
autant d'esprit, figurerait grandement. Noailles, de son côté, qui
voulait gouverner le parlement et s'en servir à ses usages, ne pouvait
s'associer mieux qu'à Maisons pour cette vue, parce qu'il comptait tout
persuader. Il n'ignorait pas peut-être ses liaisons avec M. du Maine, et
il était instruit de toutes celles qu'il prenait avec M. le duc
d'Orléans. Il se flattait d'enchanter assez Maisons, non seulement pour
se faire préconiser par lui à M. le duc d'Orléans, mais pour le
persuader qu'il était de son intérêt de le faire pour le gouverner
ensemble, et savoir tout ce que Maisons pourrait découvrir des desseins
de gouvernement, sur lesquels M. le duc d'Orléans pourrait s'ouvrir à
lui, soit par confiance, soit par consultation. De cette façon, sûr de
moi, à mon insu concerté avec Maisons, et s'assurant du parlement par ce
magistrat, on peut juger quel essor prit son ambitieuse imagination.
Mais tant de cordes ne lui suffirent pas\,: il y en avait une autre plus
délicate à toucher pour lui que pour personne, et je ne démêlai tout
cela que longtemps après. Cette corde était le marquis de Canillac, qui
paraîtra tant, et en tant de façons, dans la régence, que c'est un homme
qui dès à présent doit être connu.

C'était un grand homme, bien fait, maigre, châtain, d'une physionomie
assez agréable, qui promettait beaucoup d'esprit, et qui n'était pas
trompeuse. L'esprit était orné\,; beaucoup de lecture et de mémoire\,;
le débit éloquent, naturel, choisi, facile, l'air ouvert et noble\,; de
la grâce au maintien et à la parole toujours assaisonnée d'un sel fin,
souvent piquant, et d'expressions mordantes qui frappaient par leur
singularité, souvent par leur justesse. Sa gloire, sa vanité, car ce
sont deux choses, la bonne opinion de soi, l'envie et le mépris des
autres, étaient en lui au plus haut point. Sa politesse était extrême,
mais pour s'en faire rendre autant, et il était plus fort que lui de le
cacher. Paresseux, voluptueux en tout genre, et dans un goût étrange
aussi\,; d'une santé délicate qu'il ménageait\,; particulier, et par
hauteur difficile à apprivoiser\,; avare aussi, mais sans se refuser ce
qu'il y avait de meilleur goût dans ce qu'il se permettait, toujours sur
les échasses pour la morale, l'honneur, la plus rigide probité, le débit
des sentences et des maximes\,; toujours le maître de la conversation,
et souvent des compagnies qu'il voyait choisies, relevées, et les
meilleures\,; comptant faire honneur partout. Il parlait beaucoup, et
beaucoup trop, mais si agréablement qu'on le lui passait. Il savait
toutes les histoires de la cour où il n'allait plus, et de la ville, les
anciennes, les modernes, les courantes de toutes les sortes. Il contait
à ravir, et il était le premier homme du monde pour saisir le ridicule
et pour le rendre comme sans y toucher. Méchant et, comme on le verra,
un des plus malhonnêtes hommes du monde. Il discutait volontiers les
nouvelles, volontiers tournait tout en mauvaise part, n'approuvait
guère, blâmait cruellement et grand frondeur. Il avait eu assez
longtemps le régiment de Rouergue, avait servi assez négligemment, fait
sa cour de même, et comme plus du tout depuis longtemps qu'il avait
quitté le service. Il haïssait le roi, M\textsuperscript{me} de
Maintenon, les ministres en perfection, et ravissant en liberté sur tous
ces chapitres, dont autrefois j'étais souvent témoin chez un ami commun
dont il était intime et moi aussi. Ils rompirent au commencement de 1710
une amitié de toute leur vie, à ne s'être jamais revus depuis, sans que
jamais personne en ait pénétré la cause, ni la manière d'une rupture si
brusque et si nette. Je voyais déjà beaucoup moins Canillac dès lors
chez notre ami par le peu que j'allais à Paris, et je le perdis tout à
fait de vue depuis cette brouillerie, parce que je ne le voyais que chez
cet ami, avec lequel je suis toujours demeuré en la même intimité
jusqu'à aujourd'hui. Cela n'empêcha pas, que rencontrant bien rarement
Canillac depuis, lui et moi ne nous fissions non seulement politesse,
mais même conversation particulière qui me divertissait. Son ambition
était si peu éteinte par sa retraite de la guerre et de la cour, qu'il
ne prît en aversion quiconque y faisait fortune. Il était occupé de tout
savoir, et de se lier avec des gens de la cour et de Paris
considérables. Il était souvent à l'hôtel de La Rochefoucauld, et ami de
tous les temps intime de La Feuillade, qui s'en laissait maîtriser par
habitude et par complaisance, et il était presque tous les jours chez M.
et M\textsuperscript{me} de Maisons, avec lesquels il politiquait sur le
futur, avec toute liberté de part et d'autre, et une liaison de
plusieurs années.

Canillac était un homme qui se prenait aux louanges et aux déférences
avec la dernière faiblesse qui allait à la duperie\,: Il faisait
profession ouverte de haïr les Noailles, dont il disait pis que pendre,
surtout du duc de Noailles, comme neveu de M\textsuperscript{me} de
Maintenon, quoique assez bien avec le duc de Guiche. De tout temps il
avait vu M. le duc d'Orléans à Paris. Il y était souvent de ses parties,
mais sobrement pour sa part, et presque toujours de sang-froid. Le sel
de ses blâmes et de ses plaisanteries amusait un prince mécontent, et
dans les suites ennuyé, puis embarrassé de sa personne. Sa morale
mondaine, débitée avec autorité, lui avait imposé\,; son esprit et
l'ornement qui y était avait achevé l'opinion que M. le duc d'Orléans en
avait prise, en sorte qu'il en résultait une considération qui allait
même à quelque chose de plus. L'amitié de ce prince avait été jalouse
des liaisons que Canillac avait eues autrefois avec M. le prince de
Conti, auxquelles, malgré cela, il avait tenu bon jusqu'à sa mort, et y
était demeuré avec les amis particuliers de ce prince. Sa mort avait
terminé la jalousie et la pique de M. le duc d'Orléans. La liberté
ensuite lui en avait plu, et l'estime et la considération en était
augmentée, et se nourrissait par tous ses voyages de Paris, où il voyait
toujours Canillac qu'il en faisait avertir. Au caractère de celui-ci, on
peut juger qu'il ne s'en cachait pas, qu'il bâtissait de grandes
espérances sur la régence de ce prince, et qu'en attendant il ne
manquait pas à se faire valoir.

Le duc de Noailles était trop attentif et trop instruit pour ignorer
cette position de Canillac, et pour être tranquille sur l'aversion qu'il
lui portait. Les brocards les plus cruels et les mieux assenés coulaient
sur lui comme sur toile cirée, pour peu qu'il crût avoir intérêt à les
secouer. Canillac ne les lui avait pas épargnés, il s'en piquait même,
et s'en faisait un jeu et un divertissement aux compagnies qu'il
fréquentait. Cette habitude lui durait encore alors, et ne fut pas
capable de rebuter Noailles de captiver Canillac et d'en faire sa
conquête. Il n'ignorait pas son faible\,; les bassesses et les
prostitutions ne lui coûtaient rien\,; il espéra tout de cette voie et
ne s'y trompa point. Mais l'affaire était d'approcher Canillac, et de le
réduire à se laisser apprivoiser. Maisons fut celui à qui il s'adressa
par Contade, qui lui fit goûter l'avantage d'être leur lien et leur
modérateur. Maisons ne travailla pas en vain. Il lui fit comprendre de
quelle force serait leur triumvirat bien uni sur un prince faible et
timide\,; car Canillac, qui le connaissait bien, l'avait bien détaillé à
Maisons. Il fallut quelque temps et quelques cérémonies pour accorder
l'orgueil de Canillac avec un changement trop subit\,; mais sa déférence
pour Maisons abrégea tout. Il le regardait comme l'oracle du parlement,
qui le deviendrait de la cour, où il se conduirait d'autant mieux qu'il
ne se gouvernerait que par ses conseils, et il se considérait ainsi
comme l'âme et le moteur du triumvirat qui s'allait former.

Maisons, qui le regardait comme une linotte qui parlait bien et
beaucoup, et qui ne faisait nul cas de son jugement, ainsi qu'il s'en
est maintes fois expliqué avec moi, comptait de son côté le jouer sous
jambe, et gouverner le duc de Noailles qu'il n'estimait guère davantage
et dont il connaissait fort bien, je ne dis pas la scélératesse, mais
les défauts\,; et celui-ci, rempli de ses talents et perché sur ses
établissements et ses alliances, content de m'avoir gagné, ne doutait
pas de mener deux hommes qui ne connaissaient pas la

court comme lui, qui n'en étaient point, à qui il ferait perdre terre
toutes les fois que cela lui conviendrait, et qu'il aurait cependant en
main pour les machines qu'il voudrait faire jouer auprès de M. le duc
d'Orléans. Une affaire où chacun se persuade de trouver si bien son
compte ne tarde pas à se conclure. Canillac s'excusa de n'avoir pu
résister aux recherches du duc de Noailles et aux personnes qu'il avait
su y employer. Il s'éventa là-dessus tant qu'il lui plut, et Noailles et
Maisons n'en firent que rire. Noailles n'épargna point les moyens qu'il
avait projetés\,; il écouta parler Canillac tant qu'il voulut, l'admira,
l'encensa, le pria de le redresser, de le conduire. Canillac trouva que
ce garçon-là avait bien du bon et bien de l'esprit, et, moyennant un air
de déférence, pour ne pas dire de respect, Noailles en fit tout ce qu'il
voulut.

Il avait saisi une autre avenue\,: c'était l'abbé Dubois. Les scélérats
du premier ordre se sentent de loin, homogènes jusqu'à un certain point,
se connaissent, se lient jusqu'à ce qu'à la fin le plus adroit étrangle
l'autre\,: c'est ce qui arriva à ceux-ci. Je fus surpris, lorsque la
maison de M\textsuperscript{me} la duchesse de Berry se fit pièce à
pièce, que le duc de Noailles me pressât avec les plus vives instances
et les plus réitérées de faire obtenir à l'abbé Dubois la charge de
secrétaire des commandements de M\textsuperscript{me} la duchesse de
Berry. Le roi n'en voulut point, M. du Maine et M\textsuperscript{me} la
duchesse d'Orléans y mirent Longepierre. J'en ai parlé ailleurs.
Noailles et Dubois se cultivèrent l'un l'autre, et je crois, car ce
n'est qu'opinion, que ce fut par Dubois que Noailles se lia avec Effiat,
car je n'ai pu découvrir d'autre point de réunion. Dubois avait toujours
cultivé avec une grande dépendance le chevalier de Lorraine tant qu'il
avait vécu, et son ami d'Effiat, ses anciens protecteurs, à qui, en tant
de choses principales, il était homogène\,; et je me suis toujours
persuadé qu'il avait été l'instrument dont Noailles s'était servi pour
se lier avec Effiat, liaison qui demeura longtemps dans les ténèbres.

On a vu (t. X, p.~155) quel était le marquis d'Effiat et en lui-même et
à l'égard de M. le duc d'Orléans, à quoi j'aurai peu de chose à ajouter.
Son nom était Coiffier, son origine d'Auvergne\,; l'illustration,
d'avoir été contrôleur de la maison de MM. de Montpensier, enfin
receveur des tailles du bas Limousin\,; les alliances à l'avenant. Ces
emplois n'appauvrissent pas. Ce receveur des tailles fit son fils
général des finances, trésorier et maître des comptes en Piémont, Savoie
et Dauphiné. Tous les vilains n'ont pas toujours peur. Il se fourra aux
premiers rangs à la bataille de Cérisoles, et fut fait chevalier le
lendemain par le comte d'Enghien, prince du sang, déjà héros à son âge,
que les Guise déjà pointants et projetants assommèrent d'un coffre en se
jouant avec lui à la Rocheguyon. Il était frère d'Antoine, roi de
Navarre, père d'Henri IV, et du prince de Condé, tué à Jarnac, etc. Ce
beau chevalier s'enrichit, acheta Effiat d'Antoine de Neuville, frère du
père de M. de Villeroy, secrétaire d'État, lequel vécut et mourut
secrétaire du roi sans s'être marié. Coiffier épousa Bonne Rusé, fille
du receveur de Touraine et sœur de Beaulieu qui devint secrétaire
d'État, et qui, se trouvant sans postérité, fit son héritier Antoine
Coiffier, fils du fils de cette sœur, à la rare condition pour un homme
de cette espèce de prendre son nom et ses armes, condition aussi aisée à
accepter pour un autre homme de même sorte tel qu'était ce petit-neveu,
qui par là se trouva fort riche. Ce même petit-neveu est le maréchal
d'Effiat dont la fortune est connue, et qui n'est pas de mon sujet. Il
eut de Marie de Fourcy, sa femme, trois fils et deux filles. L'aîné fut
gendre de Sourdis, chevalier de l'ordre, vécut obscur et pas longtemps,
et ne laissa que le marquis d'Effiat qui cause cette petite
digression\,; le second fut le grand écuyer Cinq-Mars, dont la fortune
et la catastrophe sont aussi bien connues\,; le troisième, l'abbé
d'Effiat, mort aveugle, de qui on a parlé en son lieu. L'aînée des
filles, mariée et démariée d'avec d'Alègre, seigneur de Beauvoir, épousa
le maréchal de Meilleraye, et fut mère du duc Mazarin\,; l'autre,
religieuse et fondatrice du couvent de la Croix au faubourg
Saint-Antoine à Paris.

Comment d'Effiat devint premier écuyer de Monsieur, cela est trop ancien
pour moi, et en soi peu important. Comment, après avoir empoisonné
Madame, et le roi l'ayant su, comme on a vu d'original, et étant outré
de cette mort, il a laissé d'Effiat en charge, qui lui a valu l'ordre à
la présentation de Monsieur, en 1688, c'est encore ce que je ne puis
expliquer. Mais on a vu aussi que le chevalier de Lorraine et lui
s'étaient bien mis avec le roi, M\textsuperscript{me} de Maintenon et
les bâtards, en leur vendant Monsieur, et M. le duc de Chartres pour son
mariage\,; qu'Effiat s'entretint toujours depuis bien avec
M\textsuperscript{me} la duchesse d'Orléans, et sourdement avec M. du
Maine\,; que de moitié, inséparable avec le chevalier de Lorraine, il
gouverna Monsieur jusqu'à sa mort, très souvent avec insolence, et se
mêlait avec autorité de ses affaires, de sa cour, de sa famille\,; et
que cela avait accoutumé M. le duc d'Orléans à une estime de son esprit
et de sa capacité, qui passait souvent la considération et la déférence,
et que d'Effiat sut bien maintenir et s'y aider de Dubois, et lui
réciproquement. Il était veuf, sans enfants, depuis longues années,
d'une Leuville que Monsieur fit gouvernante de ses enfants, quand il
chassa la maréchale de Clérembault\,; et à M\textsuperscript{me}
d'Effiat succéda la maréchale de Grancey, mère de M\textsuperscript{me}
de Maré, qui la fut sous et après elle. Effiat vivait garçon, fort
riche, fort peu accessible, aimant fort la chasse, et disposant de la
meute de Monsieur, et après lui {[}de celle{]} de M. le duc d'Orléans,
qui ne s'en servaient point\,; six ou sept mois de l'année à Montargis,
ou dans ses terres presque seul, et ne voyant que des gens obscurs, fort
particulier, et obscur aussi à Paris, avec des créatures de même
espèce\,; débuchant parfois en bonne compagnie courtement, car il
n'était bien qu'avec ses grisettes et ses complaisants. C'était un assez
petit homme, sec, bien fait, droit, propre, à perruque blonde, à mine
rechignée, fort glorieux, poli avec le monde, et qui en avait fort le
langage et le maintien\,; ami intime du maréchal de Villeroy par leur
ancien ami commun le chevalier de Lorraine\,; presque jamais à la cour,
et encore en apparition, et ne voyant presque personne de connu, si ce
n'était quelques gens du Palais-Royal, encore assez subalternes. Il
donnait quelquefois de fort bonnes chiennes couchantes au roi, et il en
était toujours reçu avec une sorte de distinction, et que M. du Maine
ménageait lui-même pour être son pigeon privé auprès de M. le duc
d'Orléans, comme il l'était déjà et le fut toujours. On se souviendra
ici du pernicieux conseil où il engagea ce prince à la mort de M. {[}le
Dauphin{]} et de M\textsuperscript{me} la Dauphine, et de l'infâme trait
qu'il me fit depuis, lorsque M\textsuperscript{me} la duchesse d'Orléans
me força de parler à M. le duc d'Orléans devant lui de ses affaires
domestiques.

Rien ne manquait au duc de Noailles avec de telles mesures pour
favoriser tous ses desseins. Mais rien ne lui suffisait. Le bel esprit,
les vers, le dos des livres lui servirent à raccrocher Longepierre, rat
de cour, pédant, à qui un homme comme le duc de Noailles tournait la
tête, et qui se trouva heureux qu'il eût oublié, ou voulu oublier, qu'il
avait eu, malgré ses soins et ses services, une charge chez
M\textsuperscript{me} la duchesse de Berry. Longepierre se fourrait où
il pouvait à l'ombre du grec et des pièces de théâtre. Il était fort
bien avec M\textsuperscript{me} la duchesse d'Orléans et avec M. du
Maine. Noailles voulait tirer d'eux par lui, et par lui être vanté à
eux\,; la voie était fort sourde et immédiate, et il en sut tirer parti,
parce que Longepierre avait plus d'esprit que d'honneur, et qu'il
voulait faire fortune. C'est ce qui le jeta dans la suite à l'abbé
Dubois, qui en fit le même usage que Noailles, et à l'égard des mêmes
personnes, et qui, pour cela, pardonna sans peine à ce poète, orateur,
géomètre et musicien, pédant d'ailleurs fort maussade, d'avoir emporté
sur lui une charge qu'il ne pouvait déjà plus regretter. Malgré tant de
soins, de devants et d'entours, rien ne transpirait encore. Noailles ne
put rien tirer de tous ces gens-là, parce que tous étaient dans la même
ignorance. J'étais le seul à qui M. le duc d'Orléans s'ouvrait, et avec
qui tout se discutait sans réserve.

\hypertarget{chapitre-viii.}{%
\chapter{CHAPITRE VIII.}\label{chapitre-viii.}}

1715

~

{\textsc{Réflexions sur le gouvernement présent et sur celui à
établir.}} {\textsc{- Je propose à M. le duc d'Orléans les divers
conseils et l'ordre à y tenir.}} {\textsc{- L'établissement des conseils
résolu\,; discussion de leurs chefs.}} {\textsc{- Marine.}} {\textsc{-
Finances et guerre.}} {\textsc{- Affaires ecclésiastiques et feuille des
bénéfices.}} {\textsc{- Constitution.}} {\textsc{- Jésuites.}}
{\textsc{- P. Tellier.}} {\textsc{- Rome et le nonce.}} {\textsc{-
Évêques\,; leur assemblée.}} {\textsc{- Commerce du clergé de France à
Rome, et à Paris avec le nonce.}} {\textsc{- Affaires étrangères.}}
{\textsc{- Affaires du dedans du royaume.}} {\textsc{- Je m'excuse de me
choisir une place, et je refuse obstinément l'administration des
finances.}} {\textsc{- État forcé des finances.}} {\textsc{- Banqueroute
préférable à tout autre parti.}} {\textsc{- Je persiste au refus des
finances, malgré le chagrin plus que marqué de M. le duc d'Orléans.}}
{\textsc{- Je propose le duc de Noailles.}} {\textsc{- Résistance et
débat là-dessus.}} {\textsc{- M. le duc d'Orléans y consent à la fin.}}
{\textsc{- Je suis destiné au conseil de régence.}}

~

Il y avait longtemps que je pensais à l'avenir, et que j'avais fait bien
des réflexions sur un temps aussi important et aussi critique. Plus je
discutais en moi-même tout ce qu'il y avait à faire, plus je me trouvais
saisi d'amertume de la perte d'un prince qui était né pour le bonheur de
la France et de toute l'Europe, et avec lequel tout ce qui y pouvait le
plus contribuer était projeté, et pour la plupart résolu et arrangé avec
un ordre, une justesse, une équité, non seulement générale et en gros,
mais en détail autant qu'il était possible, et avec la plus sage
prévoyance. C'était un bien dont nous n'étions pas dignes, qui ne nous
avait été montré que pour nous faire voir la possibilité d'un
gouvernement juste et judicieux, et que le bras de Dieu n'était pas
raccourci pour rendre ce royaume heureux et florissant, quand nous
mériterions de sa bonté un roi véritablement selon son cœur. Il s'en
fallait bien que le prince à qui la régence allait échoir fût dans cet
état si heureux pour soi et pour toute la France\,; il s'en fallait bien
aussi que, quelque parfait que pût être un régent, il pût exécuter comme
un roi. Je sentais l'un et l'autre dans toute son étendue, et j'avais
bien de la peine à ne me pas abandonner au découragement.

J'avais affaire à un prince fort éclairé, fort instruit, qui avait toute
l'expérience que peut donner une vie de particulier fort éloigné du
trône, et du cas de la régence, fort au fait de tant de grandes fautes
qu'il avait vues, et quelques-unes senties de si près, et des malheurs
par lesquels lui-même avait tant passé, mais prince en qui la paresse,
la faiblesse, l'abandon à la plus dangereuse compagnie, mettaient des
défauts et des obstacles aussi fâcheux que difficiles, pour ne pas dire
impossibles à corriger, même à diminuer. Mille fois nous avions raisonné
ensemble des défauts du gouvernement, et des malheurs qui en
résultaient. Chaque événement, jusqu'à ceux de la cour, nous en
fournissait sans cesse la matière. Lui et moi n'étions pas d'avis
différents sur leurs causes et sur les effets. Il ne s'agissait donc que
d'en faire une application juste et suivie pour gouverner d'une manière
qui fût exempte de ces défauts, et en arranger la manière selon la
possibilité qu'en peut avoir un régent, et dans la vue aussi d'élever le
roi dans de bonnes et raisonnables maximes, de les lui faire goûter
quand l'âge lui permettrait, et de lui ouvrir les yeux et la volonté à
perfectionner en roi, après sa majorité, ce que la régence n'aurait pu
achever ni atteindre. Ce fut là mon objet et toute mon application, pour
insinuer à M le duc d'Orléans tout ce que je crus propre à l'y conduire,
dès la vie même de M. le duc de Berry, dont il devait tendre à être le
vrai conseil, beaucoup plus encore lorsqu'il n'y eut plus personne entre
M. le duc d'Orléans et la régence. À mesure que, par l'âge et la
diminution de la santé du roi, je la voyais s'approcher, j'entrais plus
en détail, et c'est ce qu'il faut expliquer.

Ce que j'estimai le plus important à faire, et le plus pressé à
exécuter, fut l'entier renversement du système de gouvernement intérieur
dont le cardinal Mazarin a emprisonné le roi et le royaume. Un étranger
de la lie du peuple, qui ne tient à rien et qui n'a d'autre dieu que sa
grandeur et sa puissance, ne songe à l'État qu'il gouverne que par
rapport à soi. Il en méprise les lois, le génie, les avantages\,; il en
ignore les règles et les formes\,; il ne pense qu'à tout subjuguer, à
tout confondre, à faire que tout soit peuple\,; et comme cela ne se peut
exécuter que sous le nom du roi, il ne craint pas de rendre le prince
odieux, ni de faire passer dans son esprit sa pernicieuse politique. On
l'a vu insulter au plus proche sang royal, se faire redouter du roi,
maltraiter la reine mère en la dominant toujours\footnote{Voy. notes à
  la fin du volume.}, abattre tous les ordres du royaume, en hasarder la
perte à deux différentes reprises par ses divisions à son sujet, et
perpétuer la guerre au dehors pour sa sûreté et ses avantages, plutôt
que de céder le timon qu'il avait usurpé. Enfin on l'a vu régner en
plein par lui-même par son extérieur et par son autorité, et ne laisser
au roi que la figure du monarque. C'est dans ce scandaleux éclat qu'il
est mort avec les établissements, les alliances et l'immense succession
qu'il a laissée, monstrueuse jusqu'à pouvoir enrichir seule le plus
puissant roi de l'Europe.

Rien n'est bon ni utile qu'il ne soit en sa place. Sans remonter
inutilement plus haut, la Ligue qui n'en voulait pas moins qu'à la
couronne, et le parti protestant, avaient interverti tout ordre sous les
enfants d'Henri II. Tout ce que put Henri IV avec le secours de la
noblesse fidèle fut, après mille travaux, de se faire reconnaître pour
ce qu'il était de plein droit, en achetant, pour ainsi dire, la couronne
de ses sujets par les traités et les millions qu'il lui en coûta avec
eux, les établissements prodigieux et les places de sûreté aux chefs
catholiques et huguenots. Des seigneurs ainsi établis, et qui se
croyaient pourtant bien déchus après les chimères que chacun d'eux
s'était faites, n'étaient pas faciles à mener. L'union subsistait entre
la plupart. La plupart avait conservé ses intelligences étrangères\,; le
roi était obligé de les ménager, et même de compter avec eux. Rien de
plus destructif du bon ordre, du droit du souverain, de l'état de sujet,
quelque grand qu'il puisse être, de la sûreté, de la tranquillité du
royaume. La régence de Marie de Médicis ne fit qu'augmenter ce mal, qui
s'était affaibli depuis la mort du maréchal de Biron. Le pouvoir et la
grandeur du maréchal d'Ancre, de sa femme et de ce tas de misérables
employés sous leurs ordres, révoltèrent les grands, les corps, les
peuples. La mort de ce maire du palais étranger, l'anéantissement de ses
créatures, l'éloignement d'une mère altière qui n'avait point d'yeux par
elle-même, mais une humeur, un caprice, une jalousie de domination, dont
des confidents infimes profitaient pour régner sous son nom, rendirent
le calme à la France pour quelque temps, mais en ménageant les grands
dont la puissance et les dangereux établissements rendaient l'obéissance
arbitraire.

Le cardinal de Richelieu sentit également les maux du dedans et du
dehors, et avec les années y apporta les remèdes. Il abattit peu à peu
cette puissance et cette autorité des grands qui balançait et qui
obscurcissait celle du roi, et peu à peu les réduisit à leur juste
mesure d'honneurs, de distinction, de considération et d'une autorité
qui leur étaient dus, mais qui ne pouvait plus soutenir à remuer, ni
parler haut au roi qui n'en avait plus rien à craindre\,: Ce fut la
suite d'une longue conduite sagement et sans interruption dirigée vers
ce but, et de l'abattement entier du parti protestant par la ruine de la
Rochelle et de ses autres places, qui faisant auparavant un État dans
l'État, étaient d'une sûre et réciproque ressource aux ennemis du dehors
et aux séditieux du dedans, même catholiques, si souvent excités par
Marie de Médicis et par Gaston son fils bien-aimé, réduit enfin à la
soumission comme les autres. Louis XIII ne vécut pas assez pour le
bonheur de la France, pour la félicité des bons, pour l'exemple des
meilleurs et des plus grands rois la soumission et la tranquillité du
dedans, la mesure, la règle, le bon ordre, la justice, qu'il avait
singulièrement adoptés, ne durèrent que huit ou neuf ans.

La minorité, qui est un temps de faiblesse, excita les grands et les
corps à se remettre en possession des usurpations qui leur avaient été
arrachées, et que la vile et l'étrangère extraction du maître que la
régente leur avait donné et à elle-même, et les fourbes, les bassesses,
les pointes, les terreurs et les \emph{sproposito} de son gouvernement
également avare, craintif et tyrannique, semblaient rendre, sinon
nécessaires, au moins supportables. Il n'en fallut pas tant que ce que
Mazarin en éprouva pour lui faire jurer la perte de toute grandeur et de
toute autorité autre que la sienne. Tous ses soins, toute son
application se tourna à l'anéantissement des dignités et de la naissance
par toutes sortes de voies, à dépouiller les personnes de qualité de
toute sorte d'autorité, et pour cela de les éloigner, par état, des
affaires\,; d'y faire entrer des gens aussi vils d'extraction que lui\,;
d'accroître leurs places en pouvoir, en distinctions, en crédit, en
richesses\,; de persuader au roi que tout seigneur était naturellement
ennemi de son autorité, et de préférer, pour manier ses affaires en tout
genre, des gens de rien, qu'au moindre mécontentement on réduisait au
néant en leur ôtant leur emploi avec la même facilité qu'on les en avait
tirés en le leur donnant\,; au lieu que des seigneurs déjà grands par
leur naissance, leurs alliances, souvent par leurs établissements,
acquéraient une puissance redoutable par le ministère et les emplois qui
y avaient rapport, et devenaient dangereux à cesser de s'en servir, par
les mêmes raisons. De là l'élévation de la plume et de la robe, et
l'anéantissement de la noblesse par les degrés qu'on pourra voir
ailleurs, jusqu'au prodige qu'on voit et qu'on sent aujourd'hui, et que
ces gens de plume et de robe ont bien su soutenir, et chaque jour
aggraver leur joug, en sorte que les choses sont arrivées au point que
le plus grand seigneur ne peut être bon à personne, et qu'en mille
façons différentes il dépend du plus vil roturier. C'est ainsi que les
choses passent d'un comble d'extrémité à un autre tout opposé..

Je gémissais depuis que j'avais pu penser à cet abîme de néant par état
de toute noblesse. Je me souviens que, dès avant que d'être parvenu à la
confiance des ducs de Beauvilliers et de Chevreuse, mais déjà fort libre
avec eux, je ne m'y contraignis pas un jour sur cette plainte. Ils me
laissèrent dire quelque temps. À la fin le rouge prit au duc de
Beauvilliers, qui d'un ton sévère me demanda\,: «\,Mais que
voudriez-vous donc pour être content\,? --- Je vais, monsieur, vous le
dire, lui répondis-je vivement\,: je voudrais être né de bonne et
ancienne maison, je voudrais aussi avoir quelques belles terres et en
beaux droits, sans me soucier d'être fort riche. J'aurais l'ambition
d'être élevé à la première dignité de mon pays, et je souhaiterais aussi
un gouvernement de place, jouir de cela, et je serais content.\,» Les
deux ducs m'entendirent, se regardèrent, sourirent, ne répondirent rien,
et un moment après changèrent de propos. Eux-mêmes, comme je le vis dans
les suites, pensaient absolument comme moi, et je n'en pus douter par le
concert entre eux et moi uniquement et ce prince dont je ne puis me
souvenir sans larmes.

Quelque abattu que je fusse de sa perte, mes pensées et mes désirs
n'avaient pu changer\,; et quelque disproportion que je sentisse de ce
prince unique à celui qui allait gouverner, et des moyens d'un roi ou
d'un régent, je ne pus renoncer à une partie de ce tout qui m'était
échappé. Mon dessein fut donc de commencer à mettre la noblesse dans le
ministère avec la dignité et l'autorité qui lui convenait, aux dépens de
la robe et de la plume, et de conduire sagement les choses par degrés et
selon les occurrences, pour que peu à peu cette roture perdît toutes les
administrations qui ne sont pas de pure judicature, et que seigneurs et
toute noblesse et peu à peu substituée à tous leurs emplois, et toujours
supérieurement à ceux que la nature ferait exercer par d'autres mains,
pour soumettre tout à la noblesse en toute espèce d'administration, mais
avec les précautions nécessaires contre les abus. Son abattement, sa
pauvreté, ses mésalliances, son peu d'union, plus d'un siècle
d'anéantissement, de cabales, de partis, d'intelligences au dehors,
d'associations au dedans, rendaient ce changement sans danger, et les
moyens ne manquaient pas d'empêcher sûrement qu'il n'en vînt dans la
suite.

L'embarras fut l'ignorance, la légèreté, l'inapplication de cette
noblesse accoutumée à n'être bonne à rien qu'à se faire tuer, à
n'arriver à la guerre que par ancienneté, et à croupir du reste dans la
plus mortelle inutilité, qui l'avait livrée à l'oisiveté et au dégoût de
toute instruction hors de guerre, par l'incapacité d'état de s'en
pouvoir servir à rien. Il était impossible de faire le premier pas vers
ce but sans renverser le monstre qui avait dévoré la noblesse,
c'est-à-dire le contrôleur général et les secrétaires d'État, souvent
désunis, mais toujours parfaitement réunis contre elle. C'est dans ce
dessein que j'avais imaginé les conseils dont j'ai parlé, et qui
longtemps après, au commencement de 1709, surprirent si fort le duc de
Chevreuse qui, m'entretenant chez moi pour la première fois de ce même
dessein qu'il me confia pour en avoir mon avis, le trouva sur-le-champ
écrit de ma main tel qu'il l'avait conçu, ainsi que cela se voit plus au
long t. VII, p.~99 et suiv. Mgr le duc de Bourgogne l'avait adopté dans
le même dessein, et ce sont ces conseils que M. le duc d'Orléans en
appuya\textsuperscript{{[}Voy. notes à la fin du volume.{]}}{[}Voy.
notes à la fin du volume.{]} , lorsqu'il nous proposa l'établissement au
parlement, en déclarant qu'ils avaient été trouvés dans la cassette de
Mgr le duc de Bourgogne, sur quoi je remarquerai que ce n'était pas
celle dont j'ai parlé et qui me donna tant d'inquiétude.

La formation de ces conseils fut donc une des premières choses dont je
parlai à M. le duc d'Orléans. Il n'était pas moins blessé que moi de la
tyrannie que ces cinq rois de France exerçaient à leur gré sous le nom
du roi véritable, et presque en tout à son insu, et l'insupportable
hauteur où ils étaient montés. Je proposai donc d'éteindre deux charges
de secrétaires d'État, celui de la guerre et celui des affaires
étrangères, qui seraient gérées par les conseils, expédiées par les
secrétaires de ces conseils\,; de diminuer autant qu'il serait possible
la multiplicité des signatures en commandement, poussées à l'infini par
l'intérêt dos secrétaires d'État de faire passer tout par leurs mains\,;
et que ce qu'il serait indispensable de signer en commandement, le
serait par les deux secrétaires d'État restants, qui en auraient tout le
loisir en toutes matières, parce qu'il ne leur en resterait aucune à
expédier ni à répondre, sinon les ordres secrets du régent qui
n'appartiennent en particulier à nulle matière. Ainsi de la marine,
ainsi de toutes les provinces du royaume qui font la matière du conseil
des dépêches, que j'appelais conseil des affaires du dedans. Ce n'était
pas que j'eusse dessein de conserver un second secrétaire d'État à la
longue\,; un seul suffisait à l'expédition des choses les plus secrètes,
que je voulais rendre aussi les plus rares, et aux signatures en
commandement absolument nécessaires, que j'avais dessein aussi
d'éclaircir beaucoup en substituant celle du chef du conseil, en la
joignant pour lors à celle du secrétaire du même conseil. On n'ignore
pas que la prétendue signature du roi, mise au bas de chaque expédition
qui sort des bureaux par le sous-commis qui écrit l'expédition même, n'a
de force et d'autorité que celle qu'elle reçoit de la signature du
secrétaire d'État. Il n'était donc pas difficile de supprimer cette
prétendue signature du roi dont personne n'était la dupe, et qui n'était
qu'une prostitution très indécente, et de transporter aux chefs des
conseils, pour les matières de leurs conseils, le poids et l'autorité de
celles des secrétaires d'État. Ce sont de ces choses que le temps amène
comme de soi-même, en ne perdant pas les occasions de les établir sans
entreprendre tout à la fois, mais se contenter d'abord du renversement
de l'arbre pour en arracher après les racines à propos, et en empêcher
radicalement la funeste reproduction.

Je proposai en même temps que les secrétaires d'État n'entrassent dans
aucun des conseils, où l'ombre de ce qu'ils ne feraient que cesser
d'être les rendrait dangereux\,; mais d'admettre sans voix ni
délibérative ni consultative même, surtout sans faculté de rapporter
quoi que ce fût, un des deux secrétaires d'État au conseil de régence
pour en tenir le registre exactement, qui serait vérifié exactement tous
les mois par celui des membres de ce conseil qui, à tour de rôle, se
trouverait en mois pour recevoir les placets que le seul secrétaire
d'État de la guerre était en usage de recevoir sur toutes matières,
lesquels lui seraient rapportés chez lui par deux maîtres des requêtes
qui l'auraient accompagné en les recevant derrière la table dressée pour
cela dans l'antichambre du roi, comme faisait seul le secrétaire d'État
de la guerre\,; et les rapporter ensuite à M. le duc d'Orléans,
accompagné des mêmes deux maîtres des requêtes. C'était rendre à ces
charges leur droit primitif, et se servir de leurs lumières pour mille
choses en ce genre qui avaient souvent trait à des choses que des gens
d'épée ne pouvaient savoir, surtout en ces commencements. On comprend
bien que je proposai en même temps d'éteindre l'emploi de contrôleur
général et d'en faire passer l'emploi et l'autorité au conseil des
finances, et substituer la signature du chef de ce conseil à celle du
contrôleur général.

À ce plan général il en fallait ajouter de particuliers. Je proposai
donc celui de ces conseils que j'avais faits autrefois, et qu'on
trouvera parmi les Pièces, tels que je les fis pour lors, mais j'en
supprimai qui ne convenaient plus ni au moment présent ni au temps d'une
régence. Ils furent, pour leur matière et pour leur nom, tels que M. le
duc d'Orléans les établit, mais avec une confusion, un nombre de
membres, un désordre que je n'y aurais pas mis, et dont la cause se
découvrira en son temps. Je ne m'y arrêterai donc pas davantage à cette
heure. Vint après la discussion des gens à admettre ou à exclure, puis
celle de la destination de chacun de ceux qui seraient employés.

Je représentai à M. le duc d'Orléans que cet établissement flatterait
extrêmement les seigneurs et toute la noblesse, éloignée des affaires
depuis près d'un siècle, et qui ne voyait point d'espérance de se
relever de l'abattement où elle se trouvait plongée\,; que ce retour
inespéré et subit du néant à l'être toucherait également ceux qui en
profiteraient par leurs nouveaux emplois, et ceux encore à qui il n'en
serait point donné, parce qu'ils en espéreraient dans la suite par
l'ouverture de cette porte, et qu'en attendant ils s'applaudiraient d'un
bien commun et de la jouissance de leurs pareils\,; en même temps que
c'était à lui à balancer si bien l'inclusion, l'exclusion, la
distribution des emplois, que son autorité, bien loin d'en souffrir,
n'en fût que plus confirmée, et d'éviter aussi des mécontentements
dangereux\,; que par cette raison je ne croyais pas qu'il pût sagement
exclure certaines gens qui bien ou mal à propos avaient acquis un
certain poids dans le monde, dont l'estime et l'opinion avantageuse
prise d'eux s'était tournée en mode, dont le choix le ferait applaudir
et donnerait réputation au nouveau genre de gouvernement, dont
l'exclusion produirait un sentiment contraire, et capable d'enhardir ces
gens-là, pour la plupart fort établis, à cabaler et à le traverser, au
contraire de l'intérêt qu'ils prendraient en lui, et au succès de ce à
quoi ils se trouveraient employés\,; et qu'il recevrait un double gré du
public et d'eux-mêmes d'un choix auquel ils ne devaient pas s'attendre
par le peu, et souvent tout le contraire de ce qu'ils avaient mérité de
lui\,; qu'aussi, tant pour le bon ordre des affaires que pour ne pas
tenter par la facilité des gens peu sûrs pour lui qui en pourraient
abuser, il était très essentiel d'établir et de maintenir dans chacun
des conseils une égalité parfaite d'autorité de fonctions entre tous les
membres, et une balance exacte entre eux et le chef, pour que le chef
n'y prenne pas une autorité qui non seulement absorbe celle du conseil,
mais même qui l'obscurcisse, et qu'il jouisse aussi de sa qualité sans
une dépendance qui l'y rende un fantôme.

Pour arriver à ce tempérament, mon sentiment fut que le chef ne pût
parler que le dernier\,; qu'il partageât les différentes affaires à
chacun, toujours en plein conseil\,; qu'il n'y en pût rapporter
aucune\,; qu'il n'eût que sa voix en quelque cas que ce pût être\,; qu'y
ayant partage, le membre de la régence en mois y fût appelé pour
départager, sans pouvoir y entendre parler d'aucune autre affaire, et
que le chef de chaque conseil venant rapporter à la régence les affaires
de son conseil, qui toutes, hors les bagatelles du courant, y devaient
être exactement portées et définitivement réglées, y fût accompagné de
l'un des conseillers d'avis contraire au chef dans les choses
principales, choisi par la pluralité des conseillers du même avis que
lui\,; enfin que toutes les délibérations de chaque conseil, surtout de
celui de régence, fussent écrites à mesure par le secrétaire séant au
bas bout de la table, lues par lui à la fin du conseil, signées de lui
et du conseiller de semaine\,; {[}ce{]} qui serait son modèle pour son
registre plus étendu, qui, à la fin de chaque mois, serait relu au
conseil et y serait signé du chef et du secrétaire. Avec ces précautions
je crus la balance bien observée, et bien difficile de rien expédier à
l'insu ou contre l'avis du conseil, et cela dans celui des affaires
étrangères comme les autres, pour les instructions, les lettres, les
réponses, les ordres, et toute autre matière, excepté les choses
également secrètes, importantes et rares, qui demeureraient entre le
régent et le chef de ce conseil, mais qu'il serait pernicieux et
destructif d'étendre au delà d'une invincible nécessité.

Je voulais aussi des jours réglés pour tenir les différents conseils,
tous dans la maison du roi, et des jours marqués à la régence pour y
entendre les affaires de chaque conseil\,; et, s'il s'en trouvait de
nature à ne pouvoir y être vues au jour ordinaire, les y porter seules
au commencement ou à la fin du conseil de régence, sans que le chef d'un
autre conseil, étant en son jour ordinaire à la régence, pût être de
l'affaire extraordinaire qui y serait portée, non plus que celui qui l'y
porterait en entendre aucune de celles qui y seraient naturellement
traitées ce jour-là. J'insistai encore à séparer chaque département de
conseil d'une manière si nette, si distincte et si précise, et à décider
si promptement et si clairement les questions et les prétentions
réciproques qui pourraient naître là-dessus dans les commencements, que
chaque conseil ne pût empiéter ni lutter contre un autre, et que dans le
public on n'eût aucun embarras pour savoir à qui s'adresser sur toute
sorte d'affaire. Il fallait pourvoir avec la même précision à séparer
bien distinctement les fonctions particulières de chaque membre de
chaque conseil, et pourvoir ainsi à l'union des membres, en retranchant
toute cause de prétention et de jalousie, ainsi qu'aux conseils, même
respectivement\,; et en même temps au mûr examen et à la prompte
expédition des affaires.

J'en fis sentir l'utilité et la facilité par l'exemple continuel de la
cour de Vienne, où rien ne s'étrangle ni ne languit parmi tant de
différents conseils qui y sont établis, et que si le contraire a paru en
Espagne, c'est que sous les derniers rois de la maison d'Autriche on n'y
opinait que par écrit\,; et ces votes, qui couraient des uns aux autres,
portés au roi, renvoyés par lui à d'autres encore, devenaient des
plaidoyers à longue distance sur les moindres affaires, dont grand
nombre de pareilles n'auraient tenu qu'une matinée en opinant de vive
voix ensemble\,; au lieu qu'une seule affaire ne finissant point, il se
faisait un engorgement qui arrêtait et perdait toutes les affaires par
des lenteurs qui n'avaient point de fin. J'ajoutai qu'à l'égard du règne
de Philippe V, M. le duc d'Orléans savait mieux que personne ce qui y
avait rendu les conseils inutiles et ridicules, qui n'avaient pu se
soutenir contre l'adresse et le crédit de M\textsuperscript{me} des
Ursins ayant M\textsuperscript{me} de Maintenon en croupe, qui voulait
tirer à soi seule toute l'autorité du gouvernement, dont les deux
monarchies ne s'étaient pas bien trouvées.

M. le duc d'Orléans goûta extrêmement ce projet, qui fut maintes fois
rebattu et discuté entre lui et moi. Il sentit l'importance du secret et
le garda, et sur la chose et sur toutes ses dépendances. La résolution
prise, il fallut débattre les sujets. Je lui représentai qu'il n'avait
point à choisir pour les chefs des conseils des affaires
ecclésiastiques, de la guerre, de la marine et des finances\,; qu'il n'y
avait aucune apparence de faire l'affront à M. le comte de Toulouse,
amiral, qui avait commandé des flottes, qui avait gagné une bataille
navale, qui tenait tous les jours le conseil des prises, qui les allait
juger définitivement au conseil devant le roi, et qui était admis à
l'examen des promotions qui se taisaient dans la marine, de l'exclure de
la place de chef de ce conseil\,; que le comte de Toulouse était à son
égard très différent du duc du Maine, et d'un caractère sage et modéré,
et aussi aimé et estimé en général que celui de son frère était méprisé
et abhorré parmi la crainte et la servitude qui réduisaient là-dessus au
silence. Je conclus donc qu'il était juste, sans péril, et nécessaire de
le faire chef de ce conseil, et très dommageable et même dangereux de ne
le pas faire, mais que je croyais aussi qu'il n'était pas moins à propos
de ne lui pas tellement abandonner ce conseil qu'il en devînt une
chimère, et que le comte se rendît maître de la marine, qu'il n'y avait
pour cela qu'à y faire entrer le maréchal d'Estrées, homme droit,
d'honneur, sachant et connaissant bien la marine, qui en était estimé et
considéré par sa valeur, ses actions, sa probité, ses talents d'homme de
mer, qui par son expérience, sa charge de vice-amiral, son office de
maréchal de France se rallierait et étayerait ce conseil\,; qu'il
pouvait compter sur lui qu'en l'y mettant il ne ferait que le mettre à
sa place, qu'il serait extraordinaire même qu'il ne l'y mît pas\,; qu'il
était bien avec le comte de Toulouse, et de longue main accoutumés l'un
à l'autre, pour avoir été souvent à la mer, ensemble et dans les ports,
et unis tous deux, et avec d'O, dans la même querelle et dans la même
inimitié contre Pontchartrain. Tout cela fut encore approuvé, et M. le
duc d'Orléans remit au temps où il pourrait parler, à voir avec le
maréchal d'Estrées, et après avec le comte de Toulouse, les marins les
plus convenables à composer ce conseil, avec quelque intendant de marine
pour ce qui y demandait nécessairement de la plume.

Venant après au conseil des finances, je lui dis que je connaissais très
bien le maréchal de Villeroy, et quel il était à son égard, mais qu'il
était chef de ce conseil et ministre d'État\,; que ne lui pas laisser
cette place, quoique autrement tournée, c'était le plus sanglant affront
qu'il se pût faire, et à un homme tel que celui-là\,; que son incapacité
et sa futilité le rendait un personnage fort indifférent à la tête
d'affaires qu'il n'entendait ni n'entendrait jamais\,; qu'il ne
s'agissait pour parer à tout que d'y joindre un président comme à la
marine, qui imposât tacitement à ses grands airs de supériorité, et qui
en ôtât la peur à des gens de robe, dont d'ici à quelque temps on ne
pourrait s'y passer comme intendants des finances, qui en avaient fait
un grimoire pour qu'il ne pût être connu que d'eux, jusqu'à ce que
l'autorité et l'application l'eût fait mettre au net, et mis la matière
à portée de gens d'épée\,; et passant tout de suite à la guerre, je fis
comprendre à M. le duc d'Orléans que le premier maréchal de France étant
placé ailleurs, la place de ce conseil ne pouvait être remplie que par
Villars, second maréchal de France, qui avait commandé les armées
jusqu'à la paix qu'il avait faite depuis lui-même à Rastadt et à Bade,
et qui ne lui était pas suspect. Villars m'avait prié, il y avait déjà
quelque temps, d'assurer M. le duc d'Orléans de son attachement. Je
l'avais fait, et j'en avais rapporté un remerciement et des compliments,
dont le maréchal me parut fort content.

Ces trois points arrêtés de la sorte, vint celui des affaires
ecclésiastiques, qui fut plus longtemps à peser. Je dis à M. le duc
d'Orléans qu'il n'avait pas plus de liberté dans ce choix que pour les
trois autres qu'il avait faits, avec cette différence que le cardinal de
Noailles, que la place de chef de ce conseil regardait uniquement, ne
lui pouvait être suspect, et que Villars, le moins sans proportion des
trois autres, avait des coins de folie auxquels il fallait prendre
garde\,; que l'âge, les mœurs, la suite d'une vie apostolique et sans
reproche du cardinal de Noailles, son ancienneté, qui le mettait à la
tête du clergé, indépendamment des autres droits, sa qualité
d'archevêque de la capitale et de diocésain de la cour, celle du plus
ancien de nos cardinaux, les établissements et les alliances de sa
famille la plus proche, le savoir et la modération qu'il avait montrés
en tant d'occasions particulières et publiques, formaient un groupe de
raisons transcendantes qui en emportaient la démonstration\,; qu'à
l'égard de l'affaire de la constitution, c'était à lui-même à qui
j'aurais voulu demander ce qu'il en pensait, ou plutôt que je n'en avais
pas besoin, parce qu'il me l'avait dit bien des fois, avec l'indignation
qu'en méritaient les artifices, les friponneries, les violences dont
toute cette affaire n'était qu'un tissu\,; que ce n'était pas à un
prince éclairé comme il l'était à se laisser imposer par une odieuse
cabale détestée de tous les honnêtes gens, même de ceux que la faiblesse
ou l'intérêt y avait engagés\,; que c'était la partie saine, savante,
pieuse du royaume avec qui il avait à compter sur les affaires
ecclésiastiques, qui demandaient des mains pures et reconnues
universellement pour telles, au péril de perdre toute réputation et
toute confiance dès ce premier faux pas. J'ajoutai que je ne voyais
point de prélat qui fût tout ensemble assez marqué, assez distingué par
les lumières, assez porté par la vénération publique, pour entrer en
aucune comparaison avec le cardinal de Noailles\,; et qu'à l'égard des
cardinaux de Rohan et de Bissy, c'était à lui-même à voir si les
affaires ecclésiastiques seraient sûrement en remettant leur direction
principale et la feuille des bénéfices à deux ambitieux esclaves de la
cour de Rome\,: le premier qui ne respirait que la grandeur de sa maison
et de ses chimères, l'autre d'en faire une, tous deux de dominer le
clergé et la cour, et d'être chefs de parti, tous deux liés et livrés à
ce qui lui était le plus contraire autour du roi et dans le public\,;
sur quoi il devait de plus savoir à quoi s'en tenir sur les Rohan.

Passant de là aux partis que formait la constitution, je lui fis sentir
toute la différence de la réputation de tout temps et publique des
prélats unis au cardinal de Noailles d'avec les autres\,; le poids de la
Sorbonne, des autres écoles, des curés de Paris, si importants et si
fort à ménager dans des temps jaloux, de la foule du second ordre, des
corps réguliers illustres par leur science et leur piété\,; enfin celui
des parlements, surtout de celui de Paris, ouvertement déclarés pour la
cause et pour la personne du cardinal de Noailles, qui avait tous les
cœurs, et vers lequel tout courrait en foule, dès que la terreur
présente finirait avec la vie du roi\,; enfin, que ce serait faire le
plus signalé affront au premier prélat du royaume, au plus établi, au
plus universellement chéri, et en vénération entière, et se livrer au
cri et au ressentiment universel, et cela pour des gens qui, méprisés
aujourd'hui qu'ils disposaient de toutes les foudres, et détestés par
l'abus de leur pouvoir, combien plus honnis quand la liberté s'en
trouverait rendue.

M. le duc d'Orléans n'eut rien à répondre à un raisonnement qui ne
tirait sa force que des choses mêmes par leur évidence fondée sur la
vérité. Il m'avoua qu'il n'y avait que le cardinal de Noailles à qui il
pût donner cette place, mais il était embarrassé de l'affaire de la
constitution, et pour Rome, et pour la France même. Le raisonnement
là-dessus se reprit à plusieurs fois. Le mien ne varia point. Mon
sentiment fut qu'il avait pour en sortir, et bien, et promptement, le
plus beau jeu du monde s'il voulait bien ne se point laisser éblouir\,;
qu'il n'était point roi, se piquant d'une autorité sans bornes, et qu'il
n'avait pris sur cette affaire aucun engagement avec Rome, avec
personne, ni avec lui-même, par l'engagement de son pouvoir déjà
compromis\,; que le roi se trouvait dans tous ces termes, dont ceux qui
l'y avaient su pousser savaient aussi bien profiter pour le conduire où
jamais il n'avait pu imaginer d'être mené\,; que lui, régent, devait
aussi en profiter en sa manière, et profiter de sa liberté, et des
limites de son autorité, pour éviter ce même écueil, et ne se pas livrer
à des gens vendus et engagés en toutes les façons du monde, dont les
artifices, l'ambition, les manèges, les fourberies, les violences
n'étaient ignorées désormais de personne, qui ne seraient jamais
contents, voudraient toujours aller en avant, immoler tout à leurs vues,
surtout entretenir cette guerre pour se rendre nécessaires et
importants, pour se faire courtiser et redouter, et parce qu'il n'y a
plus de parti, et dès lors plus de chefs, ni de principaux de parti,
quand l'affaire qui l'avait fait est finie\,; qu'il comprît donc qu'en
leur prêtant l'oreille, il ne la terminerait jamais, qu'il en serait
plus tourmenté que d'aucune autre du gouvernement, et qu'il se
trouverait peu à peu entraîné à plus de violences, et tout aussi peu
utiles à la protection même qu'il voudrait donner, qu'il n'en avait vu
commettre au roi, qui de sa part seraient bien plus odieuses\,; qu'à mon
avis, il n'avait qu'un parti à prendre, mais à s'y tenir bien
fermement\,: déclarer qu'il n'en prendrait aucun dans cette affaire,
mander le cardinal de Noailles dès l'instant que le roi ne serait plus,
le présenter au nouveau roi lui-même, avec quelque propos gracieux mais
sans affectation, lui faire valoir tête à tête ce premier pas et la
place où il l'allait mettre, et s'assurer ainsi de lui\,; déclarer
aussitôt après le conseil entier des affaires ecclésiastiques, pour
éviter d'être obligé de refuser le pape si on lui donnait le temps de
faire des démarches là-dessus\,; traiter avec distinction Rohan et
Bissy\,; leur faire sentir que vous voulez résolument une fin très
prompte à cette affaire\,; que vous avez toujours été ennemi de toute
violence, surtout en matière qui a rapport à la religion\,; qu'ils se
doivent attendre qu'il n'en sera plus fait aucune\,; que les prisons
vont même être ouvertes à ceux que cette affaire y a conduits, et toutes
les lettres de cachet à cette occasion révoquées, et l'exécuter en même
temps\,; les assurer que vous ne prenez aucun parti, et que c'est même
en preuve de cette neutralité que vous rendez la liberté à ceux à qui
cette affaire l'a fait perdre\,; que vous laissez donc une égale liberté
de part et d'autre, mais que vous ne souffrirez d'aucun côté la licence,
ni pas plus les longueurs à terminer\,; couper court ensuite, et s'ils
abusent de votre politesse pour s'engager en longs discours, faire la
révérence et les laisser, en les assurant que vous n'avez ni n'aurez
jamais assez de loisir pour vous noyer en ces disputes\,; s'ils osaient
s'échapper tant soit peu, leur dire poliment, mais avec une fermeté
sèche, de songer à qui ils ont l'honneur de parler\,; et sur-le-champ la
pirouette, et les laisser là. Rien n'est pis que de se laisser manquer
ni entamer le moins du monde, et le moyen de l'éviter pour toujours est
dès la première fois une pareille leçon. Tout de suite faire enlever les
jésuites Lallemant, Doucin et Tournemine, et leurs papiers\,; mettre le
dernier au donjon de Vincennes, sans papier, ni encre, ni plumes, ni
parler à personne, du reste bien logé et nourri à cause de sa condition
personnelle\,; les deux autres au cachot, en des prisons différentes,
avec le traitement du cachot\,; qu'on ne sût où ils sont, et les y
laisser mourir\,; ce sont les boute-feu de toute cette affaire, et de
très dangereux scélérats. Mander en même temps le provincial et les
trois supérieurs des maisons de Paris, leur témoigner estime, amitié,
désir de les marquer à leur compagnie, de l'obliger, de la distinguer,
de la servir\,; que ce n'est que dans ce dessein que vous vous êtes cru
obligé de les délivrer de trois brouillons très pernicieux, dont vous
êtes bien instruit qu'ils ne l'ont pas été moins chez eux en choses
domestiques (ce qui est très vrai) qu'ils l'ont été très criminellement
au dehors\,; que vous ne voulez pas pousser à leur égard les choses plus
loin\,; que sans entrer en aucun détail avec ceux à qui vous parlez,
vous vous contentez de leur dire que vous aimez la paix, et, poussant un
peu le ton, que vous la voulez, que vous comptez assez sur eux, par la
manière dont vous avez parlé d'eux, et usé en toutes les occasions qui
s'en sont présentées, pour leur demander d'y contribuer effectivement,
et vous donner moyen par cette conduite de leur vouloir et faire tout le
plaisir et le bien dont les occasions se pourront présenter, et dont le
désir en vous se nourrira et s'augmentera à la mesure de ce que vous
verrez qu'ils feront efficacement pour remplir en cela votre désir. Cela
dit, interrompre leurs remontrances, supplications sur les prisonniers,
protestations, etc., par des compliments et des persuasions qui feront
merveilles pour leur couper la parole, et tout aussitôt vous retirer et
les laisser\,; et s'ils hasardaient de vous suivre, ou de vous faire
demander à vous parler, leur faire dire civilement que l'accablement
d'affaires ne vous le permet pas.

Mander un moment après le P. Tellier, lui dire que vous n'oubliez point
les services qu'il vous a rendus\,; que vous désireriez avec ardeur que
le bien des affaires se pût accorder avec tout ce que vous voudriez
faire pour lui, mais que la place que vous tenez vous impose des mesures
auxquelles vous ne pouvez manquer\,; qu'ainsi vous êtes forcé à lui dire
que le roi veut qu'il soit conduit sur-le-champ à la Flèche, où il lui
défend très expressément d'écrire ou de recevoir aucune lettre de
personne que vues par celui qui en sera chargé, et qui les rendra ou
enverra, ou non, comme il le jugera à propos\,; que du reste le roi lui
donne six mille livres de pension, et que, s'il en désire davantage, il
n'a qu'à parler, avec certitude de l'obtenir sur-le-champ\,; que le roi
veut que rien ne lui manque en bois, en meubles, en logement, en
nourriture, en livres, en tout ce qui peut servir à sa santé, à sa
commodité, à son amusement\,; qu'il ait deux valets et un frère que le
roi payera, à condition qu'il les choisira et changera comme il lui
plaira, sans dépendance que de l'intendant de la province, qui aura
ordre de tenir la main à ce que rien ne lui manque\,; qu'il soit libre
et indépendant des jésuites, du collège, et qu'ils aient pour lui tous
les égards, les attentions et les déférences possibles\,; qu'il se
puisse promener et dîner dans les environs, mais sans découcher\,; et
que le roi est disposé à lui accorder d'ailleurs tout ce qui pourra lui
convenir, et même, en sa considération, des grâces, quand elles ne
seront point préjudiciables.

Cela dit, le congédier sans écouter trop de discours\,; et avoir pourvu
que, en l'absence des supérieurs de la maison professe étant chez vous
et du P. Tellier y venant, on prenne tout ce que lui et son secrétaire
auront de papiers chez eux, et deux hommes sûrs, mais polis, qui
paquetteront, au sortir de chez vous, le P. Tellier et son compagnon
dans un carrosse, y monteront avec eux, et les conduiront tout de suite
à la Flèche, où ils remettront six mille livres au P. Tellier, et le
livreront à l'intendant de la province, qu'on aura pourvu d'y faire
trouver avec les ordres du roi pour lui et pour les jésuites de la
Flèche concernant le P. Tellier. C'est ce qui se doit exécuter à
Versailles, pour que l'aller et venir, tant des supérieurs que du P.
Tellier, donne le temps nécessaire de saisir les papiers en leur
absence, et faire la capture des trois prisonniers en même temps. Je
crus pouvoir sans témérité assurer M. le duc d'Orléans d'une joie et des
bénédictions publiques de cette conduite, et que, bien loin d'emporter
aucun danger, elle accélérerait la paix. Je l'avertis qu'il se fallait
bien garder de rien dire sur tout cela, avant ni après l'exécution, aux
cardinaux de part ni d'autre, ni à personne des leurs\,: à l'un, parce
que cela lui ferait prendre trop de force, et, lui, paraîtrait s'enrôler
avec lui\,; aux autres, parce que cela sentirait l'excuse et la crainte.
Si les uns ou les autres voulaient lui en parler en louange ou en
plaintes, leur fermer la bouche poliment\,; mais leur dire tout court,
d'un ton à se faire sentir, que vous voulez la paix, et que vous êtes
résolu de l'avoir sans prendre aucun parti que celui de la paix. S'ils
passent outre, la révérence, leur dire que vous êtes fâché de n'avoir
pas le loisir d'être plus longtemps avec eux, et vous retirer.
Assurez-vous, dis-je à M. le duc d'Orléans, qu'avec cette conduite,
l'étourdissement de la mort du roi, et les affaires ecclésiastiques,
surtout la feuille des bénéfices entre les mains du cardinal de
Noailles, fera tomber les armes des mains à Rohan et Bissy, qui, étant
ce qu'ils sont, n'ont plus de fortune personnelle à faire, qui
hasarderaient leur crédit pour leur famille et leur considération en se
raidissant, et qui dès lors ne songeront qu'à vous gagner et à finir
pour vous plaire\,; et c'est ce qu'il faudra saisir brusquement, et
finir solidement, à quelque prix que ce soit, ayant toujours les écoles,
les corps ecclésiastiques et les parlements en croupe, pour finir
convenablement.

Tout cela longuement discuté et à bien des reprises, M. le duc d'Orléans
me parla de Rome et du nonce Bentivoglio, qu'il gardait pour la fin, et
sur quoi il m'expliqua ses craintes. Je l'écoutai longuement, puis je
lui dis que cet objet, si principal dans la matière que nous traitions,
ne m'était pas échappé\,; que je trouvais fort aisé de couper court avec
Rome, sans qu'elle put s'en offenser, et d'éconduire son ministre qui
était un fou et un furieux par ambition, sans religion ni honneur, et
qui entretenait publiquement une fille de l'Opéra, dont il avait déjà un
enfant qui n'était pas ignoré\,; que jusqu'à ce que les conseils fussent
entièrement formés et déclarés, les ministres du roi subsisteraient\,;
qu'ainsi il ne devait jamais se commettre avec le nonce, mais lui
refuser toute audience sous prétexte de la multitude d'affaires et
d'ordres à donner. S'il vous attaque lorsqu'il vous rencontrera, voyant
tout le monde, l'interrompre, lui dire poliment que ce n'est pas le lieu
de parler d'affaires, et le renvoyer à Torcy\,; s'il insiste, lui
tourner le dos, et vous retirer\,; charger Torcy de se rendre peu
visible au nonce et de battre la campagne, le lasser ainsi, et se moquer
de lui.

À l'égard du pape, se bien garder que rien de sa part, ni verbal et bien
moins par écrit, vienne à vous sans que Torcy l'ait ouï ou lu
auparavant, pour refuser de vous en rendre compte, comme il est souvent
arrivé au roi de refuser de recevoir des brefs, etc., ou pour vous en
rendre compte si la chose le comporte\,; ne rien répondre que des choses
générales au nonce\,; au pape force respects, désirs, soumissions, puis
lui écrire ou taire dire pathétiquement que le roi le plus craint, le
plus absolu, le plus obéi qui ait jamais régné en France, n'ayant pu
opérer ce que Sa Sainteté désire, et à quoi Sa Majesté s'était engagée à
elle, et y ayant vainement employé les soins, les grâces, les menaces et
jusqu'à la violence, pendant quatre ou cinq ans sans relâche, il ne faut
pas espérer d'un temps de minorité, par conséquent de faiblesse, ni de
l'autorité limitée et précaire d'un régent, ce que n'a pu le plus
puissant et le plus redouté des rois de France\,; qu'il est également de
la sagesse de Sa Sainteté de n'y pas compter, et de sa charité
paternelle de ne pas exiger l'impossible\,; que le régent se croit de
plus en droit d'espérer d'un si grand et si saint pape qu'il serait le
premier à chercher tous les moyens possibles d'arrêter les divisions et
les troubles dans le royaume d'un enfant, fils aîné de l'Église, aux
ancêtres de qui l'Église universelle, celle de Rome en particulier, sont
si particulièrement redevables, plutôt que de les augmenter en exigeant
l'impossible\,; étendre et paraphraser ce thème au mieux et avec les
expressions les plus touchantes et les plus soumises, mais en montrant
aussi une fermeté à s'y tenir qui ôte toute espérance de l'ébranler\,;
surtout ne se point lasser des recharges, et d'y répondre toujours sur
ce même ton.

En même temps, faire revenir au nonce que s'il n'est sage, on ne sera
pas retenu d'informer le pape de sa conduite scandaleuse, de la répandre
à Rome et de lui fermer le chemin au cardinalat par cela même qu'il
emploie à le hâter\,; avertir sous main les jésuites qu'on est attentif
à leur conduite dans toutes les provinces, qu'on n'est pas moins
instruit de celle de leur général et des principaux de leur compagnie à
Rome, qu'ils s'apercevront par un traitement attentif, suivi,
proportionné, du mécontentement ou de la satisfaction qu'on en recevra.
Tout d'une main séparer et finir l'assemblée actuelle des évêques qui
n'est bonne ni occupée qu'à brouiller, n'accorder sur cela ni délai ni
audience, dire aux cardinaux de Rohan et de Bissy qu'on n'a affaire qu'à
eux, et qu'on n'écoutera rien qu'après qu'on aura su par les intendants
des provinces que tous les évêques sont arrivés chacun dans son diocèse.
Empêcher après qu'aucun ne revienne à Paris, les renvoyer subitement,
s'ils l'osent, par le ministère naturel du procureur général, et tenir
la main par les procureurs généraux des autres parlements qu'ils ne se
courent point les uns les autres, qu'ils se tiennent chacun chez eux\,;
les y faire avertir d'être sages, et si quelqu'un de part ou d'autre ne
l'était pas, le pincer tout aussitôt ou sourdement ou avec éclat,
suivant sa faute en dessous ou publique, et le châtier aussi dans sa
parenté, moyen très sensible et d'autant plus efficace que des parents
d'évêques, et surtout tels qu'ils sont pour la plupart, n'ont pas les
ressources des évêques dans le public ni dans le particulier, et qui,
vexés par rapport à eux, les réduisent bientôt à la raison pour leur
délivrance.

Ce qui est de très principal et que j'appuyai bien à M. le duc
d'Orléans, c'est la nouvelle licence de leur correspondance à Rome et de
leurs liaisons avec le nonce. Jamais ni l'un ni l'autre ne s'était
toléré avant l'affaire de la constitution, témoin celle dont j'eus tant
de peine à tirer Mailly, archevêque d'Arles, dont j'ai parlé en son
temps, où il ne s'agissait uniquement que d'un présent au pape de
quelques reliques de saint Trophime, qui lui en avait attiré un bref de
pur remerciement, sans qu'il y eût pour lors l'ombre de rien autre
chose, pas même dans aucun lointain. Il n'était permis à aucun évêque ni
à aucun ecclésiastique d'écrire à qui que ce fût de la cour de Rome, ni
d'en recevoir de lettres, sans la permission expresse du roi sur chaque
chose, et sans que le secrétaire d'État des affaires étrangères ne les
vît et en pût répondre. Autrement c'était un crime, et ces lettres mêmes
étaient infiniment rares, parce qu'elles se permettaient fort
difficilement, et qu'elles laissaient toujours ombrage et démérite,
tellement qu'elles étaient tombées tout à fait hors d'usage, parce que
le commerce nécessaire des bulles, des dispenses, etc., se faisait
uniquement par les banquiers\footnote{On appelait autrefois
  \emph{banquiers en cour de Rome} ou \emph{banquiers expéditionnaires}
  ceux qui avaient le privilège de faire obtenir les grâces, bulles,
  dispenses, etc., de la cour de Rome. Ils étaient devenus officiers
  publics Par un édit de 1673 et une déclaration du mois de janvier
  1675. Ils étaient au nombre de douze pour Paris. Les actes expédiés
  par la chancellerie romaine devaient être revêtus de leur signature
  pour avoir un caractère authentique devant les tribunaux.}.

À l'égard des nonces, ni commerce ni visites\,; un évêque, un
ecclésiastique simple, un moine même eût été sévèrement tancé, et après
longuement éclairé, qui aurait vu le nonce sans que le ministre des
affaires étrangères eût su pourquoi, et en eût parlé au roi, et même
avec cela jamais au delà de l'étroit nécessaire. Le P. Tellier avait le
premier osé rompre cette barrière, et que n'osa-t-il pas\,? Aussitôt
grand nombre et de prélats et de gens du second ordre s'empressèrent à
se faire de fête, et se proposèrent des chimères. Rome et le nonce
entretinrent soigneusement leur vanité et leur espérance, et peu à peu
s'attachèrent ainsi une grande partie du clergé, pour se faire valoir
des deux côtés, ce qui, depuis la vue du cardinalat qui en enivra
beaucoup jusqu'aux moindres objets, débaucha un clergé vain, oisif,
avare, ambitieux, ignorant, et pour la plupart pris de la lie du peuple
ou de la plus abjecte bourgeoisie. On sent aisément ce que deviennent
alors ces précieuses libertés de l'Église gallicane, les droits du roi,
le lien à la patrie\,; et c'est ce qu'il était si important de
redresser, en privant Rome de tant et de si dangereux transfuges, en
remettant les anciennes règles en vigueur, dont Rome même n'eût osé se
plaindre, puisqu'elles y étaient encore, et sans interruption, lors des
premiers progrès de l'affaire qui fit naître celle de la constitution,
c'est-à-dire, il y a cinq ou six ans, et de plus qui n'étaient violées
que par simple et tacite tolérance, sans aucune sorte de révocation, ni
même de consentement formel. C'était donc bien assez de laisser le
commerce de Rome libre aux cardinaux de Noailles, Rohan et Bussy
uniquement, et celui du nonce à cinq ou six prélats ou gens du second
ordre, bien choisis et nommés pour cela par M. le duc d'Orléans, et
châtier sévèrement et irrémissiblement tous prélats et gens du second
ordre qui oseraient transgresser la défense le moins du monde, en
quelque manière, et sous quelque prétexte et protection que ce pût être.
Nous fûmes souvent et longuement sur cette matière M. le duc d'Orléans
et moi, et à la fin je le laissai persuadé.

Restaient les conseils des affaires étrangères et des dépêches ou du
dedans du royaume. Je dis à M. le duc d'Orléans qu'il restait aussi deux
hommes sur qui il ne devait pas compter, mais qui outre leurs
établissements étaient dans le public, l'un bien moins à propos que
l'autre, à ne pouvoir laisser\,: Harcourt et Huxelles\,; que j'estimais
qu'il fallait les mettre à la tête de ces deux conseils, mais que je ne
voyais pas qu'il eût à contraindre son goût sur leurs places. La
situation où M. le duc d'Orléans avait été si longtemps avec l'Espagne,
et les liaisons étroites d'Harcourt en ce pays-là, et avec
M\textsuperscript{me}s de Maintenon et des Ursins, le déterminèrent aux
affaires étrangères pour Huxelles, et à celles du dedans du royaume pour
Harcourt. Cela fut bientôt décidé. Mais avant que la résolution en fût
prise\,: «\,Mais vous, me dit M. le duc d'Orléans, vous me proposez tout
le monde, et ne me parlez point de vous\,; à quoi donc voulez-vous
être\,?» Je lui répondis que ce n'était à moi ni de me proposer ni moins
encore de choisir, mais à lui-même à voir s'il voulait m'employer, s'il
m'en croyait capable, et en ce cas de déterminer la place qu'il me
voudrait faire occuper. C'était à Marly, dans sa chambre, et il m'en
souviendra toujours.

Après quelque petit débat, qu'entre pareils on appellerait compliments,
il me proposa la présidence du conseil des finances, c'est-à-dire de les
diriger avec un imbécile en ce genre tel que le maréchal de Villeroy, et
me dit que c'était ce qui convenait le mieux à lui et à moi. Je le
remerciai de l'honneur et de la confiance, et je le refusai
respectueusement\,: c'était la place que je destinais au duc de
Noailles. M. le duc d'Orléans fut fort étonné, et se mit sur son
bien-dire pour me persuader. Je lui répondis que je n'avais nulle
aptitude pour les finances, que c'était un détail devenu science et
grimoire qui me passait\,; que le commerce, les monnaies, le change, la
circulation, toutes choses essentielles à la gestion des finances, je
n'en connaissais que les noms\,; que je ne savais pas les premières
règles de l'arithmétique\,; que je ne m'étais jamais mêlé de
l'administration de mon bien, ni de ma dépense domestique, parce que je
m'en sentais incapable, combien plus des finances de tout un royaume, et
embarrassées comme elles l'étaient. Il me représenta l'instruction et le
soulagement que je trouverais dans les divers membres du conseil des
finances, et dans ceux d'ailleurs que je voudrais consulter. Il ajouta
tout ce qui pouvait me flatter\,; il appuya sur ma probité et sur mon
désintéressement, chose si capitale au maniement des finances. Sur quoi
je lui répondis que peu importerait à la chose publique que je volasse
les finances, ou que mon incapacité les laissât voler\,; qu'à la vérité
je croyais bien me pouvoir répondre à lui et à moi-même de ma fidélité
là-dessus, mais qu'avec la même sincérité, je ne sentais aucune des
lumières nécessaires pour m'apercevoir même des friponneries grossières,
combien moins des panneaux infinis dont cette matière est si
susceptible. La fin de plus d'une heure de ce débat fut de se fâcher
contre moi, puis de me prier de faire bien mes réflexions, et que nous
en parlerions le lendemain.

Il y avait longtemps qu'elles étaient toutes faites. Je n'étais pas,
depuis la mort de cet admirable Dauphin, et plus encore depuis celle de
M. le duc de Berry, à m'être occupé des diverses places du gouvernement
à venir, avec ce projet des conseils, et à penser, je le dirai avec
simplicité, non à celles qui me conviendraient, mais à celles à qui je
conviendrais moi-même, qui est l'unique façon de bien placer les hommes,
et pour la chose publique et pour eux-mêmes. Celle des finances s'était
présentée à moi comme les autres\,; je n'aurai pas la grossièreté de
dire que je ne crusse pas bien que M. le duc d'Orléans ne me laisserait
pas sans me donner part au gouvernement, et je ne pensai pas qu'il y eût
de la présomption à m'en persuader, et à réfléchir en conséquence. La
matière des finances me répugnait par les raisons que je venais
d'alléguer à M. le duc d'Orléans, et par bien d'autres encore, dont
celle du travail était la moindre. Mais les injustices que les
nécessités y attachent me faisaient peur\,; je ne pouvais m'accommoder
d'être le marteau du peuple et du public, d'essuyer les cris des
malheureux, les discours faux, mais quelquefois vraisemblables, surtout
en ce génie, des fripons, des malins, des envieux\,; et ce qui me
détermina plus que tout, la situation forcée où les guerres et les
autres dépenses prodigieuses avaient réduit l'État, en sorte que je n'y
voyais que le choix de l'un de ces deux partis\,; de continuer et
d'augmenter même autant qu'il serait possible toutes les impositions
pour pouvoir acquitter les dettes immenses, et conséquemment achever de
tout écraser, ou de faire banqueroute publique par voie d'autorité, en
déclarant le roi futur quitte de toutes dettes et non obligé à celles du
roi son aïeul et son prédécesseur, injustice énorme et qui ruinerait une
infinité de familles et directement et par cascades.

L'horreur que je conçus de l'une et de l'autre de ces iniquités ne me
permit pas de m'en charger, et quant à un milieu qui ne peut être qu'une
liquidation des différentes sortes de dettes pour assurer l'acquittement
des véritables, et rayer les fausses, et l'examen des preuves, et celui
des parties payées, et jusqu'à quel point, cela me parut une mer sans
fond où mes sondes ne parviendraient jamais. Et d'ailleurs quel vaste
champ à pièges et à friponneries\,! Oserais-je avouer une raison encore
plus secrète\,? Me trouvant chargé des finances, j'aurais été trop
fortement tenté de la banqueroute totale, et c'était un paquet dont je
ne me voulais pas charger devant Dieu ni devant les hommes. Entre deux
effroyables injustices, tant en elles-mêmes que par leurs suites, la
banqueroute me paraissait la moins cruelle des deux, parce qu'aux dépens
de la ruine de cette foule de créanciers, dont le plus grand nombre
l'était devenu volontairement par l'appât du gain, et dont beaucoup en
avaient fait de grands, très difficiles à mettre au jour, encore plus en
preuves, tout le reste du public était au moins sauvé, et le roi au
courant, par conséquent diminution d'impôts infinie, et sur-le-champ.
C'était un avantage extrême pour le peuple tant des villes que de la
campagne qui est sans proportion, le très grand nombre, et le nourricier
de l'État. C'en était un aussi extrêmement avantageux pour tout commerce
au dehors et au dedans, totalement intercepté et tari par cette
immensité de divers impôts.

Ces raisons qui se peuvent alléguer m'entraînaient\,; mais j'étais
touché plus fortement d'une autre que je n'explique ici qu'en tremblant.
Nul frein possible pour arrêter le gouvernement sur le pied qu'il est
enfin parvenu. Quelque disproportion que la découverte des trésors de
l'Amérique ait mise à la quantité de l'or et de l'argent en Europe
depuis que la mer y en apporte incessamment, elle ne répond en nulle
sorte à la prodigieuse différence des revenus de nos derniers rois
{[}qui n'allaient pas{]} à la moitié de ceux de Louis XIV. Nonobstant
l'augmentation jusqu'à l'incroyable, j'avais bien présenté la situation
déplorable de la fin d'un règne si long, si abondant, si glorieux, si
naïvement représentée par ce qui causa et se passa au voyage de Torcy à
la Haye, et depuis à Gertruydemberg, dont il ne fallut pas moins que le
coup du ciel le plus inattendu pour sauver la France par l'intrigue
domestique de l'Angleterre\,; ce qui se voit dans les Pièces par les
dépêches originales et les récits qui les lient, que j'ai eus de M. de
Torcy. Il résulte donc par cet exposé qu'il n'y a point de trésors qui
suffisent à un gouvernement déréglé, que le salut d'un État n'est
attaché qu'à la sagesse de le conduire, et pareillement sa prospérité,
son bonheur, la durée de sa gloire et de sa prépondérance sur les
autres.

Louvois, pour régner seul et culbuter Colbert, inspira au roi l'esprit
de conquête. Il forma des armées immenses, il envahit les Pays-Bas
jusqu'à Amsterdam, et il effraya tellement toute l'Europe par la
rapidité des succès, qu'il la ligua toute contre la France, et qu'il mit
les autres puissances dans la nécessité d'avoir des armées aussi
nombreuses que celles du roi. De là toutes les guerres qui n'ont comme
point cessé depuis, de là l'épuisement d'un royaume, quelque vaste et
abondant qu'il soit, quand il est seul sans cesse contre toute
l'Europe\,; de la cette situation désespérante où le roi se vit en fin
réduit de ne pouvoir ni soutenir la guerre ni obtenir la paix à quelques
cruelles conditions que ce pût être. Que ne pourrait-on pas ajouter en
bâtiments immenses de ce règne et plus qu'inutiles de places ou de
plaisirs, et de tant d'autres sortes de dépenses prodigieuses et
frivoles, toutes voies dans un autre règne pour se retrouver au même
point, ce qui n'est pas difficile, après y avoir été une fois. On dépend
donc pour cela, non seulement d'un roi, de ses maîtresses, de ses
favoris, de ses goûts, mais de ses propres ministres, comme on le doit
originairement à Louvois.

On conviendra, je m'assure, qu'il n'est rien qui demande plus
pressamment un remède, et que ce remède est dissous il y a longtemps.
Que substituer donc, pour garantir les rois et le royaume de cet
abîme\,? L'incomparable Dauphin l'a bien senti et l'avait bien résolu.
Mais pour l'exécuter, il fallait être roi, non régent, et plus que roi,
car il fallait être roi de soi-même et divinement supérieur à son propre
trône. Qui peut espérer un roi de cette sorte, après s'en être vu
enlever le modèle formé des mains de Dieu même, sur le point de parvenir
à la couronne et d'exécuter les merveilles qui avaient été inspirées à
son esprit, et que le doigt de Dieu avait gravées si profondément dans
son cœur. C'est donc la forte considération de raisons si
prégnantes\footnote{Vieux mot que les anciens éditeurs ont remplacé par
  \emph{concluantes}, mais qui se traduirait mieux par
  \emph{pressantes}.} et si fort au-dessus de toutes autres
considérations qui me persuada que le plus grand service qui pût être
rendu à l'État pour lequel les rois sont faits, et non l'État pour les
rois, comme ce Dauphin le sentait si bien, et ne craignait pas de le
dire tout haut, et le plus grand service encore qui pût être rendu aux
rois mêmes était de les mettre hors d'état de tomber dans l'abîme qui
s'ouvrit de si près sous les pieds du roi, ce qui ne se peut exécuter
qu'en les mettant à l'abri des ambitieuses suggestions des futurs
Louvois, et de la propre séduction des rois mêmes par l'entraînement de
leurs goûts, de leurs passions, l'ivresse de leur puissance et de leur
gloire, et l'imbécillité des vues et des lumières dont la vaste étendue
n'est pas toujours attachée à leur sceptre. C'est ce qui se trouvait par
la banqueroute et par les motifs de l'édit qui l'aurait déclarée, qui se
réduisent à ceux-ci. La monarchie n'est point élective et n'est point
héréditaire. C'est un fidéicommis, une substitution faite par la nation
à une maison entière, pour en jouir et régner sur elle de mâle en mâle,
né et à naître, en légitime mariage, graduellement, perpétuellement, et
à toujours, d'aîné en aîné, tant que durera cette maison, à l'exclusion
de toute femelle, et dans quelque ligne et degré que ce puisse être.

Suivant cette vérité qui ne peut être contestée, un roi de France ne
tient rien de celui à qui il succède, même son père\,; il n'en hérite
rien, car il n'est ici question que de la couronne, et de ce qui y est
inhérent, non de joyaux et de mobilier. Il vient à son tour à la
couronne, en vertu de ce fidéicommis, et du droit qu'il lui donne par sa
naissance, et nullement par héritage ni représentation. Conséquemment
tout engagement pris par le roi prédécesseur périt avec lui, et n'a
aucune force sur le successeur, et nos rois payent le comble du pouvoir
qu'ils exercent pendant leur vie par l'impuissance entière qui les suit
dans le tombeau. Mineurs, à quelque âge qu'ils se trouvent, pour revenir
de ce qu'ils font eux-mêmes contre leurs intérêts, ou du préjudice
qu'ils y reçoivent par le fait d'autrui qu'ils auront consenti et
autorisé, auront-ils moins de privilèges d'être libres et quittes de ce
qui leur nuit, à quoi ils n'ont contribué ni par leur fait, ni par leur
engagement, ni par leur autorisation\,? et de condition tellement
distinguée en mieux que leurs sujets par cette minorité qui les relève
de tout ce qui leur préjudicie, à quelque âge qu'ils l'aient fait ou
ratifié, peuvent-ils devenir de pire condition que tous leurs sujets,
dont aucun n'est tenu que de son propre fait, ou du fait de celui dont
il hérite ou qu'il représente, et qui ne le peut être du fait
particulier de celui dont le bien lui échait à titre de substitution\,?
Ces raisons prouvent donc avec évidence que le successeur à la couronne
n'est tenu de rien de tout ce que son prédécesseur l'était\,; que tous
les engagements que le prédécesseur a pris sont éteints avec lui, et que
le successeur reçoit, non de lui, mais de la loi primordiale qui
l'appelle à la couronne par le fidéicommis et la substitution, qu'elle
lui a réservée à son tour pure, nette, franche, libre et quitte de tout
engagement précédent.

Un édit bien libellé, bien serré, bien ferme et bien établi sur ces
maximes et sur les conséquences qui en résultent si naturellement, et
dont l'évidence ne peut être obscurcie non plus que la vérité et la
solidité des principes dont elles se tirent, peut exciter des murmures,
des plaintes, des cris, mais ne peut recevoir de réponse solide ni
d'obscurcissement le plus léger. Il est vrai que bien des gens en
souffriraient beaucoup, mais il n'est pas moins vrai, dans la plus
étroite exactitude, que si un tel édit manque à la miséricorde en une
partie pour la faire entière au véritable public, c'est sans commettre
d'injustice, parce qu'il n'y en eut jamais à s'en tenir à son droit, et
à ne se pas charger de ce dont il est exactement vrai qu'on n'est pas
tenu, et à ce raisonnement je ne vois aucune réponse vraie, solide,
exacte, effective\,; conséquemment je ne vois que justice étroite et
irrépréhensible dans cet édit. Or l'équité mise à couvert, et du côté du
roi successeur, un tel édit deviendra le supplément des barrières qui ne
se peuvent plus envoyer. Plus il excitera de plaintes, de cris, de
désespoirs par la ruine de tant de gens et de tant de familles, tant
directement que par cascade, conséquemment de désordres et d'embarras
dans les affaires de tant de particuliers, plus il rendra sage chaque
particulier pour l'avenir. On a beau courir aux charges, aux rentes, aux
loteries, aux tontines de nouvelle création après y avoir été trompé
tant de lois, et toujours excité par des appâts trompeurs, mais qui
n'ont pu l'être pour tous, et qui en ont enrichi tant aux dépens des
autres que chacun à part se flatte toujours d'avoir la fortune ou
l'industrie de ces heureux, la banqueroute sans exception causée et
fondée en principes et en droit par l'exposé de l'édit dessille tous les
yeux et ne laisse à personne aucune espérance d'échapper à sa ruine, si,
prenant des engagements avec le roi de quelque nature qu'ils puissent
être, ils viennent à perdre ce roi avant d'en être remplis. Voilà donc
une raison précise, juste, efficace, à la portée de tout le monde, des
plus ignorants, des plus grossiers, qui resserre toutes les bourses, qui
rend tout leurre, tout fantôme, toute séduction inutiles, qui guérit,
par la crainte d'une perte certaine et au-dessus de ses forces,
l'orgueil de s'élever par des charges de nouvelle érection ou de nouveau
rétablissement, et de la soif du gain qu'on trouve dans les traités de
longue durée, par l'avarice même, ou plutôt par la juste crainte qu'on
vient d'exposer.

De là deux effets d'un merveilleux avantage\,: impossibilité au roi de
tirer ces sommes immenses pour exécuter tout ce qui lui plaît, et
beaucoup plus souvent ce qu'il plaît à d'autres de lui mettre dans la
tête pour leur intérêt particulier\,; impossibilité qui le force à un
gouvernement sage et modéré, qui ne fait pas de son règne un règne de
sang et de brigandages et de guerres perpétuelles contre toute l'Europe
bandée sans cesse contre lui, armée par la nécessité de se défendre, et
à la longue, comme il est arrivé à Louis XIV, pour l'humilier, le mettre
à bout, le conquérir, le détruire, car ce ne fut pas à moins que ses
ennemis visèrent à la fin\,; impossibilité qui l'empêche de se livrer à
des entreprises romaines du côté des bâtiments militaires et civils, à
une écurie qui aurait composé toute la cavalerie de ses prédécesseurs, à
un luxe d'équipages de chasses, de fêtes, de profusions, de luxe de
toute espèce qui se voilent du nom d'amusements, dont la seule dépense
excède de beaucoup les revenus d'Henri IV et des commencements de Louis
XIII\,; impossibilité enfin qui n'empêche pas un roi de France d'être et
de se montrer le plus puissant roi de l'Europe, de fournir avec
abondance à toutes les parties du gouvernement, qui le rendent non
seulement considérable mais redoutable à tous les potentats de l'Europe,
dont aucun n'approche de ses revenus, ni de l'étendue suivie, ni de
l'abondance des terres de sa domination, et qui ne lui ôte pas les
moyens de tenir une cour splendide digne d'un aussi grand monarque, et
de prendre des divertissements et des amusements convenables à sa
grandeur, enfin de pourvoir sa famille avec une abondance raisonnable et
digne de leur commune majesté.

L'autre effet de cette impossibilité délivre la France d'un peuple
ennemi, sans cesse appliqué à la dévorer par toutes les inventions que
l'avarice peut imaginer et tourner en science fatale par cette foule de
différents impôts, dont la régie, la perception et la diversité, plus
funestes que le taux des impôts mêmes, forme ce peuple nombreux dérobé à
toutes les fonctions utiles à la société, qui n'est occupé qu'à la
détruire, à piller tous les particuliers, à intervertir commerce de
toute espèce, régimes intérieurs de famille, et toute justice, par les
entraves que le contrôle des actes et tant d'autres cruelles inventions
y ont mises\,; encourage le laboureur, le fermier, le marchand,
l'artisan, qui désormais travaillera plus pour soi et pour sa famille
que pour tant d'animaux voraces qui le sucent avant qu'il ait recueilli,
qui le consomment en frais de propos délibéré, et avec qui il est
toujours en reste\,; cause une circulation aisée qui fait la richesse,
parce qu'elle décuple l'argent effectif qui court de main en main sans
cesse, inconnue depuis tant d'années\,; facilite et donne lieu à toute
espèce de marchés entre particuliers, les délivre du poids également
accablant et insultant de ce nombre immense d'offices et d'officiers
nouveaux et inutiles, multiplie infiniment les taillables et soulage
chaque taillable du même coup, fait rentrer ce peuple immense, oisif,
vorace, ennemi, dans l'ordre de la société, dont il multiplie tous les
différents États\,; ressuscite la confiance, l'attachement au roi,
l'amour de la patrie, éteint parce qu'on ne compte plus de patrie\,;
rend supportables les situations qui étaient forcées, et celles qui ne
l'étaient pas, heureuses\,; redonne le courage et l'émulation détruits,
parce qu'on ne profite de rien, et que plus vous avez et plus on vous
prend\,; enfin rend aux pères de famille ce soin domestique qui
contribue si principalement, quoique si imperceptiblement, à l'harmonie
générale et à l'ordre public presque universellement abandonné par le
désespoir de rien conserver, et de pouvoir élever, moins encore
pourvoir, chacun sa famille.

Tels sont les effets de la banqueroute qui ne sauraient être contestés,
et qui ne sont préjudiciables (je ne parle pas des créanciers) qu'à un
très petit nombre de particuliers de bas lieu, jusqu'à cette heure, qui
abusent de la confiance de leur maître pour s'élever à tout sur les
ruines de tous les ordres du royaume, et qui pour leur grandeur
particulière comptent pour rien d'exposer ce maître à qui ils doivent
tout, au précipice qu'on vient de voir, et toute la France aux derniers
et aux plus irrémédiables malheurs. Balancez après cet exposé les
inconvénients et les fruits de la banqueroute avec ceux de continuer et
de multiplier les impôts pour acquitter les dettes du roi, ou ce milieu
de liquidation si ténébreux, et si peu fructueux, même si peu
praticable. Voyez quelle suite d'années il faudra nourrir toute la
France de larmes et de désespoir pour achever le remboursement de ces
dettes\,; et j'ose m'assurer qu'il n'est point d'homme, sans intérêt
personnel au maintien des impôts jusqu'à se préférer à tout, qui, dans
la malheureuse nécessité d'une injustice, ne préfère de bien loin celle
de la banqueroute. En un mot, c'est le cas d'un homme qui est dans le
malheur d'avoir à choisir de passer douze ou quinze années dans son lit,
dans les douleurs continuelles du fer et du caustique et le régime qui y
est attaché, ou de se faire couper la jambe qu'il sauverait par cet
autre parti. Qui peut douter qu'il ne préférât l'opération plus
douloureuse et la privation de sa jambe, pour se trouver deux mois après
en pleine santé, exempt de douleur, et dans la jouissance de soi-même et
des autres par la société, et le libre exercice de ce qui l'occupait
auparavant son mal\,? Reste à finir par l'autorité du roi.

Un mot seul suppléera à tout ce qui se pourrait dire, et à ce que les
flatteurs et les empoisonneurs des rois se voudraient donner la licence
de critiquer. Reportons-nous à ces temps malheureux où le plus absolu et
le plus puissant de tous nos rois, le plus maître aussi de son maintien
et de son visage, et dont le règne a été tel qu'on l'a vu, ne put
retenir ses larmes en présence de ses ministres dans l'affreuse
situation où il se voyait de ne pouvoir plus soutenir la guerre ni
obtenir la paix. Remettons-nous devant les yeux l'éclat où il avait
porté ses ministres, et l'humiliation plus que servile où il avait
autrefois réduit les Hollandais. Entrons après dans l'esprit et dans le
cœur de ce monarque de bonheur, de gloire, de majesté, ne craignons pas
d'ajouter d'apothéose après les monuments que nous en avons vus, et
voyons ce prince ennemi implacable du prince d'Orange, pour avoir refusé
d'épouser sa bâtarde, envoyer son principal ministre en ce genre courir
en inconnu en Hollande avec pour tout passeport celui d'un courrier,
descendre chez un banquier de Rotterdam et se faire mener par lui à la
Haye chez le pensionnaire Heinsius, créature et confident de ce même
prince d'Orange et héritier de sa haine, implorer la paix comme à ses
genoux. Suivons par les Pièces tout ce que Torcy y essaya, poursuivons
tous les sacrifices offerts et méprisés, qui, dans cette extrémité, ne
rebutèrent pas le roi d'envoyer ses plénipotentiaires à
Gertruydemberg\,; continuons, par les Pièces, de repasser les
traitements indignes et les propositions énormes dont on se joua d'eux
et du roi, et l'état de ce prince à la rupture d'une négociation où, en
lui prescrivant jusqu'à l'inhumanité qu'il n'osa refuser en partie, on
exigea encore qu'il se soumît à s'engager à ce qu'ils ne déclareraient
que quand il leur plairait, et aux augmentations vagues qu'ils
pourraient ajouter. Réfléchissons sur une situation si forcée et si
cruelle, fruit déplorable de cette ancienne conquête de la Hollande, et
de tant d'autres exploits. Qui après ne demeurera pas, je ne dis pas
persuadé, mais convaincu que le roi n'eût donné tout ce qu'on eût voulu,
pour n'avoir jamais connu Louvois ni les flatteurs, moins encore les
moyens de franchir ce qu'il avait encore trouvé de barrières à un
pouvoir illimité, dont toutefois il s'était montré si jaloux, et ne se
pas trouver, et inutilement encore, aux genoux et à la merci de ceux
dont il avait triomphé, et qu'il avait insultés par tant de monuments et
de médailles\,? Tenons-nous-en donc à cette réflexion transcendante,
pour ne pas craindre la banqueroute par rapport à l'autorité des rois.

Tranchons une dernière objection possible. Que diront les étrangers sur
un édit qui, sur des fondements aussi bien établis, rend le successeur à
la couronne pleinement libre de tout engagement de son prédécesseur, et
que deviendront leurs traités et les engagements réciproques\,? La
réponse est aisée. Les rois ne traitent point par édits avec les
puissances étrangères. Il y a des traités, et c'est le plus grand
nombre, qui ont des temps limités, ou qui ne sont que pour le règne des
princes qui les font. S'il s'en trouve qui les outrepassent, alors ce
n'est plus le roi seulement, mais sa couronne qui est engagée avec un
autre État, ce qui n'a point d'application aux sujets de la couronne, et
alors les traités subsistent dans leur vigueur. De plus, quand, ce qui
ne peut tomber dans ce cas, le successeur ne serait pas obligé de tenir
les traités de son prédécesseur, le bien de l'État voudrait qu'il le fit
peut-être pour le fruit du traité même, certainement pour le maintien de
la confiance et de la sûreté des traités. Ainsi nulle comparaison des
sujets avec les puissances étrangères, ni d'un traité avec elles et
l'effet d'un édit qui, remontant à la source du droit de la maison
régnante, le montre tel qu'il est, d'où suit ce qui vient d'être
expliqué qui n'a trait ni application quelconque aux puissances
étrangères, ni aux traités subsistants, avec lesquels il ne s'agit ni
d'héritage, ni de substitution, ni des différents effets de ces deux
manières de succéder. Cette réponse paraît péremptoire, sans s'arrêter
plus longtemps à cette spécieuse mais frivole objection.

M. le duc d'Orléans ne me trouva donc pas plus disposé à me charger des
finances après le loisir qu'il m'avait donné pour y penser. Même
empressement, mêmes prières, mêmes raisonnements de sa part\,; mêmes
réponses, même fermeté de la mienne. Il se fâcha, il n'y gagna rien. La
fâcherie se tourna en mécontentement si marqué que je le vis moins
assidûment, et beaucoup plus courtement, sans qu'il montrât sentir cette
réserve, et sans que lui et moi nous parlassions plus que des choses
courantes, publiques, indifférentes, en un mot, de ce qui s'appelle la
pluie et le beau temps. Cette bouderie froide de sa part, tranquille de
la mienne, dura bien trois semaines. Il s'en lassa le premier. Au bout
de ce temps, au milieu d'une conversation languissante, mais où je
remarquai plus d'embarras de sa part qu'à l'ordinaire\,: «\,Hé bien\,!
donc, s'interrompit-il lui-même, voilà qui est donc fait\,? Vous
demeurez déterminé à ne point vouloir des finances\,?» me dit-il en me
regardant.

Je baissai respectueusement les yeux, et je répondis d'une voix assez
basse que je comptais qu'il n'était plus question de cela. Il ne put
retenir quelques plaintes, mais sans aigreur et sans se fâcher\,; puis
se levant et se mettant à faire des tours de chambre, sans dire mot et
la tête basse, comme il faisait toujours quand il était embarrassé, il
se tourna tout à coup brusquement à moi en s'écriant\,: «\,Mais qui donc
y mettrons-nous\,?» Je le laissai un peu se débattre, puis je lui dis
qu'il en avait un tout trouvé, s'il le voulait tout au meilleur, et qui
à mon avis ne refuserait pas. Il chercha sans trouver\,; je nommai le
duc de Noailles. À ce nom il se fâcha et me répondit que cela serait bon
pour remplir les poches de la maréchale de Noailles, de la duchesse de
Guiche, qui de profession publique vivaient des affaires qu'elles
faisaient à toutes mains, et enrichir une famille la plus ardente et la
plus nombreuse de la cour, et qui se pouvait appeler une tribu. Je le
laissai s'exhaler, après quoi je lui représentai que pour le personnel
il ne me pouvait nier que le duc de Noailles n'eût plus d'esprit qu'il
n'en fallait pour se bien acquitter de cet emploi, ni toute la fortune
la plus complète en biens, en charges, en gouvernements, en alliances,
pour y être à l'abri de toute tentation, et donner à son administration
tout le crédit et toute l'autorité nécessaire, en sorte que, dès que son
Altesse Royale convenait qu'il y fallait mettre un seigneur, il n'y en
avait point qui y fût plus convenable. Quant à ses proches, parmi
lesquels ses enfants ne se pouvaient compter par leur enfance, ni sa
femme par le peu qu'elle avait su se faire considérer dans la famille,
et par sa tante même qui avait été la première à lui ôter toute
considération, il n'y avait rien à craindre de ses sœurs ni de ses
beaux-frères, excepté l'aînée, par la façon d'être de presque tous, et
par la manière de vivre du duc de Noailles avec eux, en liaison et en
familiarité, mais hors de portée de s'en laisser entamer. Quant à sa
mère et à la duchesse de Guiche, il était vrai ce qu'il m'en disait,
mais il fallait aussi lui apprendre à quel titre\,; que la maréchale
chargée de ce grand nombre de filles et de dots pour les marier toutes,
et le duc de Guiche, qui n'avait rien et à qui son père ne donnait rien,
hors d'état de soutenir la dépense des campagnes, avaient l'un et
l'autre obtenu un ordre du roi au contrôleur général, dès le temps que
Pontchartrain l'était, de faire pour la mère et pour la fille toutes les
affaires qu'elles protégeraient, et de chercher à leur donner part dans
le plus qu'il pourrait\,; que Chamillart avait reçu le même ordre en
succédant à Pontchartrain\,; que je le savais de l'un et de l'autre,
parce que tous deux me l'avaient dit, et qu'on m'avait assuré que le
même ordre avait été renouvelé lorsque Desmarets fut fait contrôleur
général\,; que de cette sorte ce n'était plus avidité ni ténébreux
manège à redouter d'elles auprès du duc de Noailles, mais des grâces
pécuniaires que le roi voulait et comptait leur faire sans bourse
délier, et qu'il ne dépendait plus des contrôleurs généraux de
refuser\,; qu'au reste, il ne fallait pas croire que la maréchale de
Noailles eût grand crédit sur son fils, ni que la duchesse de Guiche fit
ce qu'elle voulait de son frère\,; qu'il ne se trouvait personne sans
quelque inconvénient, et que celui-là semblait trop peu fondé pour
l'exclusion d'un homme qui, étant tout ce que celui-là était, ne pouvait
avoir d'autre ambition que de se faire une réputation par son
administration, bien supérieure à toute faiblesse pour sa famille, à
l'égard de laquelle il n'avait pas témoigné jusqu'ici y avoir de
disposition. Cette discussion souffrit bien des répliques en plus d'une
conversation de part et d'autre, et finit enfin par laisser M. le duc
d'Orléans déterminé à faire le duc de Noailles président du conseil des
finances. J'étais en effet persuadé qu'il y ferait fort bien, surtout
étudiant comme il faisait assidûment sous Desmarets, ainsi que je l'ai
dit en son lieu, et j'étais bien aise aussi d'appuyer le cardinal de
Noailles par cette place de son neveu, et si propre à accroître le
crédit réel et la considération extérieure.

Le moment d'après que cela fut résolu entre M. le duc d'Orléans et
moi\,: «\,Et vous enfin, me dit-il, que voulez-vous donc être\,?» et me
pressa tant de m'expliquer que je le fis enfin, et, dans l'esprit que
j'ai exposé plus haut, je lui dis que s'il voulait me mettre dans le
conseil des affaires du dedans, qui est celui des dépêches, je croyais y
pouvoir faire mieux qu'ailleurs. «\, Chef donc, répondit-il avec
vivacité. --- Non pas cela, répliquai-je, mais une des places de ce
conseil.\,» Nous insistâmes tous deux, lui pour, moi contre. Je lui
témoignai que ce travail en soi et celui de rapporter au conseil de
régence toutes les affaires de celui du dedans m'effrayait, et
qu'acceptant cette place, je n'en voyais plus pour Harcourt. «\,Une
place dans le conseil du dedans, me dit-il, c'est se moquer, et ne se
peut entendre. Dès que vous n'en voulez pas absolument être chef, il n'y
a plus qu'une place qui vous convienne et qui me convient fort aussi\,:
c'est que vous soyez du conseil où je serai, qui sera le conseil suprême
ou de régence.\,» Je l'acceptai et le remerciai. Depuis ce moment cette
destination demeura invariable, et il se détermina tout à fait à donner
la place de chef au maréchal d'Harcourt du conseil du dedans. Il n'y fut
point question de président, parce que les affaires n'y étaient pas
assez jalouses pour donner ce contrepoids au chef. Il n'en fut point
parlé pour celui des affaires étrangères pour n'y pas multiplier le
secret, ni dans celui de guerre, qui en temps de paix n'était que de
simple courant d'administration intérieure, ni dans celui des affaires
ecclésiastiques pour y relever davantage le chef, qui était le cardinal
de Noailles. Cette invention de présidence ne dut alors avoir lieu que
pour les conseils de marine et de finances, pour contrebalancer la trop
grande autorité des deux chefs, et suppléer à l'ineptie en finance du
maréchal de Villeroy.

\hypertarget{chapitre-ix.}{%
\chapter{CHAPITRE IX.}\label{chapitre-ix.}}

1715

~

{\textsc{Précautions que je suggère à M. le duc d'Orléans.}} {\textsc{-
Résolution que je propose à M. le duc d'Orléans sur l'éducation du roi
futur.}} {\textsc{- Je lui propose le duc de Charost pour gouverneur du
roi futur, et Nesmond, archevêque d'Alby, pour précepteur.}} {\textsc{-
Discussion entre M. le duc d'Orléans et moi sur le choix des membres du
conseil de régence et l'exclusion des gens à écarter.}} {\textsc{-
Villeroy à conserver, Voysin à chasser, et donner les sceaux au bonhomme
d'Aguesseau.}} {\textsc{- Torcy.}} {\textsc{- Desmarets et Pontchartrain
à chasser.}} {\textsc{- Je sauve La Vrillière à grand'peine, et lui
procure une place principale et unique.}} {\textsc{- Discussion de la
mécanique et de la composition du conseil de régence.}} {\textsc{- Je
propose à M. le duc d'Orléans de convoquer, aussitôt après la mort du
roi, les états généraux, qui sont sans danger, utiles sur les finances,
avantageux à M. le duc d'Orléans.}} {\textsc{- Grand parti à tirer
délicatement des états généraux sur les renonciations.}} {\textsc{- Rien
de répréhensible par rapport au roi dans la conduite proposée à M. le
duc d'Orléans, par rapport à la tenue des états généraux.}} {\textsc{-
Usage possible à faire des états généraux à l'égard du duc du Maine.}}
{\textsc{- Mécanique à observer.}}

~

Les conseils, leurs chefs, leurs présidents réglés, je représentai à M.
le duc d'Orléans qu'il devait profiter du reste de ce règne pour bien
examiner les choix qu'il ferait pour les remplir. Je l'exhortai à se
tenir au plus petit nombre que la nature de chaque conseil pourrait
souffrir, de les remplir tous dès lors comme s'ils existaient par une
liste sous sa clef, dont les noms ne seraient connus que de lui. Que de
ceux qu'il y aurait écrits, il rayât ceux qui mourraient avant le roi et
ceux qu'il reconnaîtrait avoir mal choisis, par l'examen qu'il ferait
secrètement de leur conduite, et qu'à mesure qu'il en rayerait un, il en
mît un autre en sa place, comme si la chose existait et qu'il remplît
une vacance\,; de régler ainsi tout ce qui pouvait l'être d'avance, afin
de n'avoir que les déclarations à en faire à la mort du roi, parce que,
lorsque cela arriverait, il se trouverait tout à coup accablé de tant et
de diverses sortes de choses, affaires, ordres, cérémonial, disputes,
demandes, règlements, décisions, inondation de monde qu'il n'aurait le
temps de rien, à peine même de penser, et qu'il pouvait compter encore
qu'il se verrait forcé de donner son temps aux bagatelles préférablement
aux affaires, parce qu'en ces occasions les bagatelles sont les affaires
du lendemain, souvent du jour même et de l'instant, qu'il faut régler
sur l'heure, et qui se succèdent sans cesse les unes aux autres,
tellement qu'il pouvait s'assurer que, s'il n'avait alors tous ses
arrangements d'affaires et ses choix tout prêts sur son papier, sous sa
clef, ils demeureraient noyés dans ce chaos, et en arrière à n'avoir
plus le temps ni de les faire ni de les différer, tellement que ce
serait le hasard et les instances des demandeurs qui en disposeraient,
et qui les lui arracheraient sans égard au mérite ni à l'utilité,
beaucoup moins à lui et à ses intérêts\,; qu'alors, outre l'embarras et
le rompement de tête, l'affluence de tout ce qui lui tomberait tout à la
fois, il ne pourrait ni peser, ni comparer, ni discuter, ni raisonner
sur rien, ni faire un choix par lui-même, emporté qu'il serait par le
temps, le torrent, la nécessité\,; et que de choses et de choix réglés
dans ce tumulte de gens et d'affaires de toutes sortes il éprouverait un
long et cuisant repentir, s'il n'éprouvait pis encore. C'est ce que je
lui répétai sans cesse tout le reste du temps que le roi vécut\,; c'est
ce qu'il m'assura toujours qu'il ferait, et quelquefois à demi qu'il
faisait, et qu'il ne fit jamais, par paresse.

Je ne voulais pas lui demander ni ses choix ni ses règlements, pour
ménager sa défiance. Je m'étais contenté de lui indiquer les choses en
gros, et les chefs et présidents des conseils comme le plus important.
Pour les détails et les places des conseils, je ne crus pas devoir lui
faire naître le soupçon que je cherchasse à disposer de tout en lui
proposant choses en détail, et gens pour remplir ces places. C'était
lui-même qui m'avait mis en consultation la forme du futur gouvernement,
et à portée de lui parler de tout ce qui vient d'être exposé. J'attendis
sagement qu'il me mît dans la nécessité de lui parler de tout le reste,
comme on verra qu'il arriva quelquefois.

Toutes ces choses se passaient entre lui et moi, longtemps avant qu'il
fût question du testament du roi. Assez près de ce qui vient d'être
rapporté, je lui parlai de l'éducation du roi futur. Je lui dis qu'il me
paraissait difficile que le roi n'y pourvût de quelque façon que ce pût
être\,; que si cela arrivait, quelque mal qu'il le fît, soit pour
l'éducation même, soit par rapport à Son Altesse Royale, ce lui devait
être une chose à jamais sacrée par toutes sortes de considérations, mais
surtout par celles des horreurs dont on avait voulu l'accabler, et dont
la noirceur se renouvelait sans cesse\,; que, par cette même raison, si
le roi venait à mourir sans y avoir pourvu, il devait bien fermement
exclure, moi tout le premier, et tout homme qui lui était
particulièrement attaché, éviter aussi d'en choisir de contraires et de
dangereux, et que pour peu qu'on différât à rien déclarer là-dessus, je
croyais très important qu'il en usât là-dessus comme pour les conseils,
par une liste à lui seul connue de toute cette éducation, pour avoir le
loisir de la bien pourpenser, de rayer et de remplacer, enfin, lorsqu'il
en serait temps, de n'avoir qu'à la déclarer. Nous agitâmes le
gouverneur, sur quoi il me dit force choses sur moi que je ne
rapporterai pas. Cette discussion finit par lui conseiller le duc de
Charost. Ce n'était pas que lui ni moi l'en crussions capable. Tel est
le malheur des princes et de la nécessité des combinaisons\,; mais nous
n'en trouvâmes guère qui le fussent, et ce très peu d'ailleurs
dangereux.

Charost avait la naissance, la dignité, le service militaire, l'habitude
de la cour, de la guerre, du grand monde où partout il était bienvoulu.
Il était plein d'honneur, avait de la valeur, de la vertu, une piété de
toute sa vie, à sa mode à la vérité, mais vraie, qui n'avait rien de
ridicule ni d'empesé, qui n'avait pas empêché la jeune et brillante
compagnie de son temps de vivre avec lui, même de le rechercher\,; nulle
relation particulière avec M. le duc d'Orléans, ni avec rien de ce qui
lui était contraire, intimement lié, aux affaires près, avec feu MM. de
Chevreuse et de Beauvilliers, mon ami particulier et ancien, enfin, ce
qui faisait beaucoup, capitaine des gardes par le choix et le désir du
Dauphin, père du roi futur. Ces raisons déterminèrent M. le duc
d'Orléans, qui se résolut à chercher soigneusement deux sous-gouverneurs
qui pussent suppléer à ce qui manquerait au gouverneur, dont la douceur
et la facilité n'apporterait ni obstacle ni ombrage à l'utilité de leurs
fonctions. Je proposai pour précepteur Nesmond, archevêque d'Alby,
avouant très franchement que je ne le connaissais point du tout\,; et ce
qui me faisait penser à lui, c'était la harangue qu'il fit au roi pour
la clôture de l'assemblée du clergé, et en même temps sur la mort de
Monseigneur. Je ne répéterai rien de ce que j'en ai dit en son temps.

La respectueuse mais généreuse liberté de cette harangue, d'ailleurs
très belle et très touchante, à un roi tel que le nôtre, à qui ce
langage était inconnu depuis tant d'années, me donna une grande idée de
ce prélat pour une éducation dont les lettres et la science ne pouvait
faire une grande partie. Il était en réputation d'honneur et de mœurs,
et sa capacité en ce genre, je ne sais quelle elle était, se pouvait
aisément suppléer par les sous-précepteurs. Ce choix n'était guère plus
aisé que celui du gouverneur, tant l'épiscopat allait tombant de plus en
plus, depuis que M. de Chartres, Godet, l'avait rempli des ordures des
séminaires, surtout depuis que le P. Tellier l'avait si effrontément
vendu à ses desseins. Il fallait donc un prélat de bonne réputation, qui
ne fût ni de la lie du peuple ni de celle des séminaires, qui n'eût
point d'attachement particulier à M. le duc d'Orléans, ni de liaison
avec ce qui lui était contraire, et qui n'eût levé aucun étendard pour
ni contre la constitution. Tout cela se trouvait en celui-ci. M. le duc
d'Orléans en fut fort ébranlé\,; mais, comme je ne le connaissais point
ni lui non plus, il se réserva à s'en informer davantage.

Il passa de là à raisonner avec moi sur le conseil de régence. Mon avis
fut différent de celui que je viens d'expliquer sur l'éducation, au cas
que le roi disposât de la formation de ce conseil. S'il le réglait, il
n'y avait point à douter que, pour les choses et pour le choix des
personnes, ce ne fût au pis pour M. le duc d'Orléans. Ce prince n'avait
point à cet égard les entraves qu'il avait sur l'éducation, par les
horreurs qu'on avait répandues contre lui, et qu'on ne cessait de
renouveler. Il ne fallait donc pas se laisser museler par les
dispositions que le roi ferait à cet égard, qui par sa personne, ni par
leur valeur, ne pouvaient être plus vénérables que celles de Charles V,
et en dernier lieu de Louis XIII, où la prudence et la sagesse avaient
si essentiellement présidé, et dont l'autorité mort-née fut abrogée
aussitôt après la mort de ces deux grands et admirables rois, quoiqu'ils
n'eussent point de monstres à rendre formidables. Je crus donc possible
et indispensable d'aller tête levée aussitôt après la mort du roi contre
les dispositions de gouvernement qu'il aurait faites, soit secrètes,
jusqu'après ce moment, soit déclarées, soit même exécutées par la
formation de ce conseil et de cette forme de gouvernement de son vivant,
pendant lequel il ne fallait que soumission et silence, mais sans cesser
de se préparer à le renverser.

La discussion du choix des personnes pour composer le conseil de régence
fut difficile. Il fallut traiter le conseil présent et les exclusions
pour balayer la place, éclaircir, et rendre après le choix plus aisé. De
tous les ministres actuels, je ne voulus conserver que le maréchal de
Villeroy, non par estime ni aucune amitié, mais par la considération de
ses établissements, de ses emplois, de ses alliances. Le chancelier
était un homme de néant en tout genre, incapable, ignorant, intéressé,
sans amis que ceux de sa faveur et de ses places, haï à la cour et
détesté des troupes par sa sécheresse, son orgueil, sa hauteur, méprisé
par le tuf qu'il montrait en toute affaire, enfin qui n'avait de mérite
que celui d'esclave de M\textsuperscript{me} de Maintenon et de M. du
Maine, de valet du cardinal de Bissy et de Rome, du nonce et des furieux
de la constitution, pour lesquels tous sa prostitution ne trouvait rien
de difficile\,; ennemi de plus de M. le duc d'Orléans, à proportion
qu'il était vendu au duc du Maine et à M\textsuperscript{me} de
Maintenon. Ainsi je proposai à M. le duc d'Orléans d'éteindre sa charge
de secrétaire d'État, de le reléguer quelque part, comme à Moulins ou à
Bourges, et de donner les sceaux au bonhomme d'Aguesseau, magistrat de
l'ancienne roche, qui ne tenait à rien qu'à l'honneur, à la justice, a
la vraie et solide piété, dont la réputation avait toujours été sans
tache, la capacité reconnue dans les premiers emplois de sa profession
qu'il avait exercés, qui touchait au décanat du conseil, qui était
depuis longtemps l'ancien des deux conseillers au conseil royal des
finances, doux, éclairé, d'un facile accès, avec de l'esprit et une
grande expérience dans les affaires de son état, universellement aimé,
estimé, considéré, d'une modestie fort approchante de l'humilité, et
père du procureur général qui avait aussi une grande réputation et une
grande considération dans le parlement, où il avait longtemps brillé
avocat général.

M. le duc d'Orléans sentit qu'il n'y avait rien de meilleur à faire que
de se délivrer d'un ennemi à la chute duquel tout applaudirait, et qui
ne serait regretté que de la cabale du duc du Maine et de celle de la
constitution, et de se faire en même temps tout l'honneur possible d'un
choix qui d'ailleurs lui serait avantageux, et qui enlèverait
l'applaudissement général, sans qu'aucun osât se montrer mécontent ni
compétiteur. Il y trouvait encore l'avantage d'un âge qui laissait
l'espérance ouverte de succéder aux sceaux, espérance qui tiendrait les
principaux prétendants dans une dépendance qui lui faciliterait beaucoup
l'intérieur des affaires qui ont à passer par les mains des magistrats.

Torcy était ami particulier des maréchaux de Villeroy, de Tallard et de
Tessé. Sa sœur, qui avait grand crédit sur lui, était de tout temps à
M\textsuperscript{me} la Duchesse\,; il n'avait point de liaison avec M.
du Maine, et n'était pas bien avec M\textsuperscript{me} de Maintenon.
Sa société était contraire à M. le duc d'Orléans, ainsi que ses amis
particuliers. J'en concluais qu'il lui était aussi contraire qu'eux. Je
n'avais pas oublié ce qu'il avait dit au roi de moi sur les
renonciations que j'ai rapportées. Je n'avais jamais eu avec lui ni
commerce, ni la plus légère relation. Les ducs de Chevreuse et de
Beauvilliers ne l'aimaient point du tout, quoique amis intimes de
Pomponne, son beau-père, parce qu'ils le croyaient janséniste, et qu'ils
n'avaient jamais fait grand cas de Croissy, ni de sa femme, pensant à
leur égard comme Seignelay, leur beau-frère, avec qui ils avaient été
intimement liés jusqu'à sa mort. Je ne connaissais donc Torcy que par
avoir pensé me perdre, et par un extérieur emprunté, embarrassé et
timide, que je prenais pour gloire\,; je voulais donc l'écarter comme
les autres ministres, en supprimant sa charge de secrétaire d'État. Je
lui donnai force attaques auprès de M. le duc d'Orléans, et je
m'irritais en moi-même du peu de progrès que j'y faisais. Voilà, il faut
l'avouer, comment la passion et l'ignorance séduisent, et conduisent en
aveugles\,; il n'est pas temps encore de dire combien j'ai été aise
depuis de n'avoir pas réussi à l'exclure.

Pour Desmarets, j'avais juré sa perte, et j'y travaillais il y avait
longtemps. C'était le prix de son ingratitude et de sa brutalité à mon
égard, dont j'ai parlé. Sa conservation était incompatible avec un
conseil de finances tel que je l'avais proposé et qu'il avait été
résolu, et c'était une délivrance publique que celle de son humeur, de
l'avarice de sa femme, de la hauteur et du pillage de Bercy, leur
gendre, qui avait pris le montant sur eux et sur les finances, et dont
l'esprit et la capacité, dont il avait beaucoup, étaient fort dangereux.
J'en vins à bout, et son exclusion ne varia point. À ce que l'on a vu en
divers endroits de Pontchartrain, on jugera aisément qu'il y avait
longtemps que j'employais tout ce qui était en moi pour lui tenir la
parole que j'avais donnée de le perdre. Son caractère et sa conduite m'y
donnaient beau jeu\,; c'était faire une vengeance publique du plus
détestable et du plus méprisable sujet, et regardé comme tel sans
exception, par toute la France, et par tous les pays étrangers avec qui
sa place l'avait mis en relation. On a vu comment et pourquoi, de propos
délibéré, il avait perdu la marine, et on verra en son temps combien il
l'avait pillée. Il était trop misérable pour ne pas chercher à se
distinguer auprès de M\textsuperscript{me} de Maintenon, de M. du Maine,
du torrent à la mode, et du bel air contre M. le duc d'Orléans\,; en un
mot, c'était, tout vil qu'il fût, un ennemi public, dont le sacrifice
était dû au public, et fort agréable, un homme sans nul ami, et sans
aucune qualité regrettable parmi toutes celles qui font abhorrer. Sa
perte était résolue dès longtemps, et je m'applaudissais secrètement de
l'avoir faite.

La Vrillière, son cousin, qui ne l'en aimait pas mieux, avait mérité des
sentiments tout contraires C'était un homme dont la taille différait peu
d'un nain, grosset, monté sur de hauts talons, d'une figure assez
ridicule. Il avait de l'esprit, trop de vivacité, des expédients, de la
vanité beaucoup trop poussée, entendant bien sa besogne, qui n'était
pourtant que la matière du conseil des dépêches sans aucun
département\,; mais bon ami, très obligeant, et capable de rendre des
services avec adresse, même avec hasard, mais sans préjudice de
l'honneur et de la probité. À l'égard du public, obligeant, honnête,
d'un accès aisé et ouvert, cherchant à plaire et à se faire des amis.
Son grand-père et son père, secrétaires d'État comme lui, ayant Blaye et
la Guyenne dans leur département, avaient été amis particuliers de mon
père, et l'avaient servi en tout ce qu'ils avaient pu. J'ai rapporté en
leur lieu des services essentiels que j'ai reçus de La Vrillière. Je
m'étais donc fait un point capital de le sauver, et de le mettre, de
plus, seul en place et en fonction de secrétaire d'État. M. le duc
d'Orléans qui se prenait assez aux figures, quoique la mienne ne fût pas
avantageuse, mais il y était accoutumé d'enfance, me répondait sans
cesse\,: «\,Mais on se moquera de nous avec ce bilboquet,\,» --- en
sorte que je fus plus d'un an à mettre tout ce que j'eus de force et
d'industrie à le poulier. J'en vins enfin à bout, à force de bras, et
cette destination ne varia plus.

Il fut question après de la composition du conseil de régence et de sa
mécanique. Cette mécanique était bien plus aisée que le choix de ses
membres. C'était là où toutes les affaires de toute espèce avaient à
être portées et décidées eu dernier ressort à la pluralité des voix, et
où celle du régent ne devait être qu'une comme les autres, excepté au
cas de partage égal, où, à l'exemple du chancelier au conseil des
parties, elle serait prépondérante. Établis comme l'étaient les bâtards,
comment pouvoir les en exclure\,? et qu'était-ce qu'y avoir le duc du
Maine, qui même y tiendrait le comte de Toulouse de fort près et de fort
court\,? L'âge d'aucun prince du sang ne leur en permettait l'entrée, et
quand on aurait franchi toute règle en faveur de M. le Duc, le plus âgé
de tous, qu'attendre d'un prince, né le 28 août 1692, encore sous l'aile
de M\textsuperscript{me} la Duchesse, et sous la tutelle de d'Antin, qui
n'avait ni instruction ni lumières, et qui ne montrait que de
l'opiniâtreté et de la brutalité, sans la moindre étincelle d'esprit\,?
Un tour de force était un début dangereux parmi tant de sortes
d'affaires, et qui n'était pas dans le caractère de faiblesse de M. le
duc d'Orléans.

L'abus énorme de leur grandeur par-dessus toute mesure, et au mépris de
toutes les lois divines et humaines, était bien un crime, et leur
attentat au rang, aux droits, à l'état des princes du sang, et à la
succession à la couronne, en était bien un de lèse-majesté, et qui en
emportait toute la punition sur le duc du Maine qui seul l'avait commis,
et de notoriété publique, à l'insu du comte de Toulouse, qui depuis ne
l'avait jamais approuvé. Mais quelle corde à remuer dans ces premiers
moments de régence, sans l'appui et la juridique réquisition des princes
du sang tous enfants\,! c'était donc une chose à laquelle il ne fallait
pas penser pour lors, et qu'il fallait réserver aux temps et aux
occasions qu'on ferait naître, selon que le duc du Maine se conduirait,
trop grand pour l'attaquer sans avoir bien pris les plus justes mesures,
trop établi pour l'attaquer sans être en certitude et en volonté bien
déterminée de le pousser par delà les dernières extrémités, et ses
enfants à ne pouvoir se relever, ni avoir jamais aucune existence,
châtiment trop juste et mille fois trop mérité de ce Titan de nos jours,
et leçon si nécessaire à la faiblesse et à la séduction des rois, et à
l'ambition effrénée de leurs bâtards pour toute la suite de la durée de
la monarchie. Je ne pus donc conseiller l'exclusion du duc du Maine,
dont M. le duc d'Orléans sentit bien toute la difficulté. Lui et le
maréchal de Villeroy dans le conseil de régence, c'était y mettre deux
ennemis certains, et encore deux ennemis d'un parfait concert, qui
mettaient dans la nécessité de les contre-balancer, d'autant plus grande
qu'il était presque également difficile de n'y pas mettre le comte de
Toulouse et de pouvoir compter sur lui. On le pouvait sur d'Aguesseau,
mais son naturel était faible et timide, et il était d'ailleurs tout
neuf en tout ce qui n'était pas de son métier, et en la plus légère
connaissance des choses de la cour et du monde. Nous parlâmes de
l'archevêque de Cambrai, et la discussion ne fut pas longue. Toute
l'inclination de M. le duc d'Orléans l'y portait, comme je l'ai déjà
remarqué ailleurs\,; et comme je l'ai aussi raconté en son temps,
j'avais travaillé à entretenir ce goût et cette estime. Nous cherchâmes
après à bien des reprises. L'un n'était pas sûr, un autre pas assez
distingué, celui-ci manquait de poids, celui-là ne serait pas approuvé
du public, sans compter l'embarras de trouver sûreté, fermeté et
capacité dans un même sujet. À chaque discussion cet embarras nous fit
quitter prise et remettre à plus de réflexions et d'examen, et pour le
dire tout de suite, ces remises, devant et depuis le testament, nous
conduisirent jusqu'à la mort du roi, tant sur le choix que sur la
mécanique, ce qui me fait remettre d'expliquer l'un et l'autre au temps
où M. le duc d'Orléans les décida, ainsi que les membres de tous les
conseils.

Il y avait longtemps que je pensais à une assemblée d'états généraux, et
que je repassais dans mon esprit le pour et le contre d'une aussi
importante résolution. J'en repassai dans ma mémoire les occasions, les
inconvénients, les fruits de leurs diverses tenues\,; je les combinai,
je les rapprochai des moeurs et de la situation présente. Plus j'y
sentis de différence, plus je me déterminai à leur convocation. Plus de
partis dans l'État, car celui du duc du Maine n'était qu'une cabale
odieuse qui n'avait d'appui que l'ignorance, la faveur présente, et
l'artifice dont le méprisable et timide chef, ni les bouillons insensés
d'une épouse qui n'avait de respectable que sa naissance, qu'elle-même
tournait contre sol, ne pouvaient effrayer qu'à la faveur des ténèbres,
leurs utiles protectrices\,; plus de restes de ces anciennes factions
d'Orléans et de Bourgogne\,; personne dans la maison de Lorraine dont le
mérite, l'acquêt, les talents, le crédit, la suite ni la puissance fit
souvenir de la Ligue\,; plus d'huguenots et point de vrais personnages
en aucun genre ni état, tant ce long règne de vile bourgeoisie, adroite
à gouverner pour soi et à prendre le roi par ses faibles, avait su tout
anéantir, et empêcher tout homme d'être des hommes, en exterminant toute
émulation, toute capacité, tout fruit d'instruction, et en éloignant et
perdant avec soin tout homme qui montrait quelque application et quelque
sentiment.

Cette triste vérité qui avait arrêté M. le duc d'Orléans et moi sur la
désignation de gens propres à entrer dans le conseil de régence, tant
elle avait anéanti les sujets, devenait une sécurité contre le danger
d'une assemblée d'états généraux. Il est vrai aussi que les personnes
les plus séduites par ce grand nom auraient peine à montrer aucun fruit
de leurs diverses tenues, mais il n'est pas moins vrai que la situation
présente n'avait aucun trait de ressemblance avec toutes celles où on
les avait convoqués, et qu'il ne s'était encore jamais présenté aucune
conjoncture où ils pussent l'être avec plus de sûreté, et où le fruit
qu'on s'en devait proposer fût plus réel et plus solide. C'est ce que me
persuadèrent les longues et fréquentes délibérations que j'avais faites
là-dessus en moi-même, et qui me déterminèrent à en faire la proposition
à M. le duc d'Orléans. Je le priai de ne prendre point d'alarme, avant
d'avoir ouï les raisons qui m'avaient convaincu, et après lui avoir
exposé celles qui viennent d'être expliquées, je lui mis au meilleur
jour que je pas les avantages qu'il en pourrait tirer. Je lui dis que
jetant à part les dangers que je venais de lui mettre devant les yeux,
mais qui n'ont plus d'existence, le seul péril d'une assemblée d'états
généraux ne regardait que ceux qui avaient eu l'administration des
affaires, et si l'on veut, par contre-coup, ceux qui les y ont employés.
Que ce péril ne regardait point Son Altesse Royale, puisqu'il était de
notoriété publique qu'il n'y avait jamais eu la moindre part, et qu'il
n'en pouvait prendre aucune en pas un des ministres du roi, ni en qui
que ce soit qui les ait choisis ni placés. Que cette raison, si les
suivantes le touchaient, lui devait persuader de ne pas laisser écouler
une heure après la mort du roi sans commander aux secrétaires d'État les
expéditions nécessaires à la convocation, exiger d'eux qu'elles fussent
toutes faites et parties avant vingt-quatre heures, les tenir de près
là-dessus, et, du moment qu'elles seraient parties, déclarer
publiquement la convocation. Qu'elle devait être fixée au terme le plus
court, tant pour les élections des députés par bailliages que pour
l'assemblée de ces députés pour former les états généraux \footnote{Voy.
  notes à la fin du volume.}, pour qu'on vit qu'il n'y avait point de
leurre, et que c'est tout de bon et tout présentement que vous les
voulez, et pour n'avoir à toucher à rien en attendant leur prompte
ouverture, et n'avoir, par conséquent, à répondre de rien. Que le
Français, léger, amoureux du changement, abattu sous un joug dont la
pesanteur et les pointes étaient sans cesse montées jusqu'au comble
pendant ce règne, après la fin duquel tout soupirait, serait saisi de
ravissement à ce rayon d'espérance et de liberté proscrit depuis plus
d'un siècle, vers lequel personne n'osait plus lever les yeux, et qui le
comblerait d'autant plus de joie, de reconnaissance, d'amour,
d'attachement pour celui dont il tiendrait ce bienfait, qu'il partirait
du pur mouvement de sa bonté, du premier instant de l'exercice de son
autorité, sans que personne eût eu le moment d'y songer, beaucoup moins
le temps ni la hardiesse de le lui demander. Qu'un tel début de régence,
qui lui dévouait tous les cœurs sans aucun risque, ne pouvait avoir que
de grandes suites pour lui, et désarçonner entièrement ses ennemis,
matière sur laquelle je reviendrai tout à l'heure. Que l'état des
finances étant tel qu'il était, n'étant ignoré en gros de personne, et
les remèdes aussi cruels à choisir, parce qu'il n'y en pouvait avoir
d'autres que l'un des trois que j'avais exposés à Son Altesse Royale
lorsqu'elle me pressa d'accepter l'administration des finances, ce lui
était une chose capitale de montrer effectivement et nettement à quoi
elle en est là-dessus, avant qu'elle-même y eût touché le moins du
monde, et qu'elle en tirât d'elle un aveu public par écrit, qui serait
pour Son Altesse Royale une sûreté pour tous les temps plus que
juridique, et la plus authentique décharge, sans tenir rien du bas des
décharges ordinaires, ni rien de commun avec l'état des ordonnateurs
ordinaires, ni avec le besoin qu'ils ont d'en prendre, et le titre le
plus sans réplique et le plus assuré pour canoniser à jamais les
améliorations et les soulagements que les finances pourront recevoir
pendant la régence, peu perceptibles et peu crus sans cela, ou de pleine
justification de l'impossible, si elles n'étaient pas soulagées dans
l'état où il constait d'une manière si solennelle que le roi les avait
mises, et laissées en mourant\,: avantage essentiel pour Son Altesse
Royale dans tous les temps, et d'autant plus pur qu'il ne s'agit que de
montrer ce qui est, sans charger ni accuser personne, et avec la grâce
encore de ne souffrir nulle inquisition là-dessus, mais uniquement de
chercher le remède à un si grand mal. Déclarer aux états que ce mal
étant extrême, et les remèdes extrêmes aussi, Son Altesse Royale croit
devoir à la nation de lui remettre le soin de le traiter elle-même\,; se
contenter de lui en découvrir toute la profondeur, lui proposer les
trois uniques moyens qui ont pu être aperçus d'opérer dans cette
maladie, de lui en laisser faire en toute liberté la discussion et le
choix, et de ne se réserver qu'à lui fournir tous les éclaircissements
qui seront en son pouvoir, et qu'elle pourra désirer pour se guider dans
un choix si difficile, ou à trouver quelque autre solution, et, après
qu'elle aura décidé seule et en pleine et franche liberté, se réserver
l'exécution fidèle et littérale de ce qu'elle aura statué par forme
d'avis sur cette grande affaire\,; l'exhorter à n'y pas perdre un
moment, parce qu'elle n'est pas de nature à pouvoir demeurer en suspens
sans que toute la machine du gouvernement soit aussi arrêtée\,; finir
par dire un mot, non pour rendre un compte qui n'est pas dû, et dont il
se faut bien garder de faire le premier exemple, mais légèrement avec un
air de bonté et de confiance, leur parler, dis-je, en deux mots, de
l'établissement des conseils, déclarés et en fonction entre la
convocation et la première séance des états généraux, et sous prétexte
de les avertir que le conseil établi pour les finances n'a fait et ne
fera que continuer la forme du gouvernement précédent, sans innover ni
toucher à rien jusqu'à la décision de l'avis des états, qui est remise à
leur sagesse, pour se conformer après à celle qu'on en attend.

«\,Je ne crois pas, ajoutai-je, qu'il faille recourir à l'éloquence pour
vous persuader du prodigieux effet que ce discours produira en votre
faveur. La multitude ignorante, qui croit les états généraux revêtus
d'un grand pouvoir, nagera dans la joie et vous bénira comme le
restaurateur des droits anéantis de la nation. Le moindre nombre, qui
est instruit que les états généraux sont sans aucun pouvoir par leur
nature, et que ce n'est que les députés de leurs commettants pour
exposer leurs griefs, leurs plaintes, la justice et les grâces qu'ils
demandent, en un mot, de simples plaignants et suppliants, verront votre
complaisance comme les arrhes du gouvernement le plus juste et le plus
doux\,; et ceux qui auront l'œil plus perçant que les autres apercevront
bien que vous ne faites essentiellement rien de plus que ce qu'ont
pratiqué tous nos rois en toutes les assemblées tant d'états généraux
que de notables, qu'ils ont toujours consultés principalement sur la
matière des finances, et que vous ne faites que vous décharger sur eux
du choix de remèdes qui ne peuvent être que cruels et odieux, desquels,
après leur décision, personne n'aura plus à se plaindre, tout au moins à
se prendre à vous de sa ruine et des malheurs publics.\,»

Je vins ensuite à ce qui touchait M. le duc d'Orléans d'une façon encore
plus particulière\,: je lui parlai des renonciations. Je lui remis
devant les yeux combien elles étaient informes et radicalement
destituées de tout ce qui pouvait opérer la force et le droit d'un tel
acte, le premier qu'on eût vu sous les trois races de nos rois pour
intervertir l'ordre, jusque-là si sacré, à l'aînesse masculine,
légitime, de mâle en mâle, à la succession nécessaire à la couronne.
Cette importante matière avait tant été discutée en son temps entre M.
le duc de Berry, lui surtout et moi, qu'il l'avait encore bien présente.
Je partis donc de là pour lui faire entendre de quelle importance il lui
était de profiter de la tenue des états généraux pour les capter, comme
il était sûr qu'il se les dévouerait par tout ce qui vient d'être
exposé, et d'en saisir les premiers élans d'amour et de reconnaissance
pour se faire acclamer en conséquence des renonciations, et en tirer
brusquement un acte solennel en forme de certificat du vœu unanime.

Je lui fis sentir la nécessité de suppléer au juridique par un populaire
de ce poids, et de profiter de l'erreur si répandue du prétendu pouvoir
des états généraux\,; que, après ce qu'ils auraient fait en sa faveur,
la nation se croirait engagée à le soutenir à jamais, par cette chimère
même de ce droit qui lui était si précieuse, ce qui lui donnait toute la
plus grande sûreté et la plus complète de succéder, le cas arrivant, en
quelque temps que ce pût être, à l'exclusion de la branche d'Espagne,
par l'intérêt essentiel que la nation commise se croirait dans tous les
temps y avoir. En même temps je lui fis remarquer qu'en tirant pour soi
le plus grand parti qu'il était possible, et l'assurance la plus
certaine d'avoir à jamais la nation pour soi et pour sa branche contre
celle d'Espagne, ce qui faisait également pour tous les princes du sang,
et qui en augmentait la force par le nombre et le poids des intéressés,
il n'acquerrait ce suprême avantage que par un simple leurre auquel la
nation se prendrait, et qui ne donnait rien aux états généraux. Alors je
lui fis sentir l'adresse et la délicatesse, à laquelle sur toutes choses
il fallait bien prendre garde à s'attacher à coup sûr, que les états ne
prononceraient rien, ne statueraient rien, ne confirmeraient rien\,; que
leur acclamation ne serait jamais que ce qu'on appelle \emph{verba et
voces}, laquelle pourtant engagerait la nation à toujours par des liens
d'autant moins dissolubles, qu'elle se tiendrait intéressée pour son
droit le plus cher qu'elle croirait avoir exercé et qu'elle
soutiendrait, le cas avenant, en quelque temps que ce put être, par ce
motif le plus puissant sur une nation, pour légère qu'elle puisse être,
qui est de se croire en pouvoir de se donner un maître, et de régler la
succession à la couronne, tandis qu'elle n'aura fait qu'une acclamation.
Je fis prendre garde aussi à M. le duc d'Orléans à la même adresse et
délicatesse pour l'acte par écrit en forme de simple certificat de
l'acclamation, parce que le certificat pur et simple qu'une chose a été
faite n'est qu'une preuve qu'elle a été faite, n'en peut changer l'être
et la nature, ni avoir plus de force et d'autorité que la chose qu'il ne
fait que certifier\,; or cette chose n'étant ni loi, ni ordonnance, ni
simple confirmation même, l'acte qui la certifie avoir été faite ne lui
donne rien de plus qu'elle n'a\,; ainsi le leurre est entier, tout y est
vide, les états généraux n'en acquièrent aucun droit, et néanmoins M. le
duc d'Orléans en a tout l'essentiel par cette erreur spécieuse et si
intéressant toute la nation, qui, pour son plus cher intérêt à
elle-même, la lie à lui pour jamais et a tous les autres princes du
sang, pour l'exclusion de la branche d'Espagne de succéder à la
couronne.

Le moyen après de contenir les états, après les avoir si puissamment
excités, me parut bien aisé\,: Protester avec confiance et modestie
qu'on ne veut que leurs cœurs, qu'on compte leur parole donnée par cette
acclamation pour si sacrée et si certaine, qu'on ne croirait pas la
mériter si on souffrait qu'ils donnassent plus\,; qu'on le déchirerait
même, et qu'on regardait recevoir plus comme un crime. Qu'on acceptait
cette parole uniquement pour l'extrême plaisir de recevoir une telle
marque de l'affection publique, et pour la considération éloignée du
repos de la France, mais dans le désir passionné et la ferme espérance
que le cas prévu n'arrivera jamais, par la longue vie du roi, et la
grande bénédiction de Dieu sur sa postérité\,; qu'aller plus loin que
cette parole si flatteuse, et le très simple certificat qui en fait foi,
ne peut convenir au respect des circonstances, qui sont un régent qui,
pour le présent, ne peut encore rien voir de longtemps entre le roi et
lui. Se tenir à ces termes de confiance et de reconnaissance, de
modestie, de respect, de raisons, inébranlablement, avec la plus extrême
attention à n'en pas laisser soupçonner davantage\,; paraphraser ces
choses et les compliments\,; surtout brusquer l'affaire, couper court,
finir, et ne manquer pas après de bien imposer silence sur l'acclamation
et le certificat et toute cette matière, et y bien tenir la main, sous
prétexte que sous un roi hors d'état de régner par lui-même, et de se
marier de longtemps, c'est une matière qui, passé la nécessité remplie,
est odieuse, et n'est propre qu'à des soupçons, à donner lieu aux
méchants, et à qui aime le désordre, de troubler l'harmonie, le concert,
la bonté et la confiance du roi pour le régent\,; mais ne dire cela, et
avec fermeté, qu'après la chose entièrement faite, de peur d'y jeter des
réflexions et de l'embarras. Outre le fruit infini de rejeter sur les
états les suites douloureuses du remède auquel ils auront donné la
préférence pour les finances, d'avoir acquis par leur tenue, et cette
marque de déférence, l'amour et la confiance de la nation, et de l'avoir
liée par son acclamation, à l'exclusion de la branche d'Espagne de la
succession à la couronne, par les liens les plus sûrs, les plus forts et
les plus durables, quelle force d'autorité et de puissance cette union
si éclatante et si prompte du corps de la nation avec M. le duc
d'Orléans, à l'entrée de sa régence, ne lui donne-t-elle pas au dedans,
pour contenir princes du sang, grands corps, et quelle utile réputation
au dehors pour arrêter les puissances qui pourraient être tentées de
profiter de la faiblesse d'une longue minorité, et quel contre-coup sur
ses ennemis domestiques, et sur l'Espagne même, dont l'appui et les
liaisons n'auraient plus d'objet pour elle, ni de prétexte et
d'assurance pour eux\,!

Une réflexion naturelle découvre que les états généraux sont presque
tous composés de gens de province des trois ordres, surtout du premier
et du dernier\,; que presque tous ceux, corps et particuliers, sur qui
porte cet immense faix de dettes du roi sont de Paris\,; que la noblesse
des provinces, quoique tombée par sa pauvreté dans les mésalliances,
n'en a point ou presque point fait hors de son pays, et ne tient point
aux créanciers du roi, qui sont tous des financiers établis à Paris, et
des corps de roturiers richards de la même ville, comme secrétaires du
roi, trésoriers de France, et toute espèce de trésoriers, fermiers
généraux, etc., gens à n'être point députés pour le tiers état\,; par
conséquent, que la grande pluralité des députés des trois ordres aura un
intérêt personnel, et pour leurs commettants, à préférer la banqueroute
à la durée et à toute augmentation possible des impositions, et comptera
pour peu les ruines et les cris que causera la banqueroute, en
comparaison de la délivrance de tant de sortes d'impôts qui révèlent le
secret des familles, en troublent l'économie, et les dispositions
domestiques, livrent chacun à la malice et à l'avidité des financiers de
toute espèce, ôtent toute liberté au commerce intérieur et extérieur, et
le ruinent avec tous les particuliers. Cette vue de liberté, d'impôts
médiocres, et encore aux choix des états, en connaissance de cause par
l'expérience de leurs effets, l'aise de se voir au courant leur fera
voir une nouvelle terre et de nouveaux cieux, et ne les laissera pas
balancer entre leur propre bonheur et le malheur des créanciers. Les
rentes sur l'hôtel de ville, où beaucoup de députés se pourront trouver
intéressés, auront peut-être quelque exception par cet intérêt\,;
peut-être encore le comparant avec celui d'abroger un plus grand nombre
d'impôts, la modification serait-elle légère, ou même n'y en aurait-il
point, et c'est à la banqueroute, si flatteuse par elle-même pour le
gros, qu'il faudrait tourner les états avec adresse. J'ajoutai que ce
serait perdre presque tout le fruit que M. le duc d'Orléans
recueillerait de tout ce qui vient d'être dit, s'il ne se faisait pas
une loi, qu'aucune considération ne pût entamer dans la suite, de se
conformer inviolablement au choix du remède porté par l'avis formé par
les états. Y manquer, ce serait se déshonorer par la plus publique et la
plus solennelle de toutes les tromperies, tourner l'amour et la
confiance de la nation en haine et en désir de vengeance, je ne craignis
pas d'ajouter, s'exposer à une révolution, sans être plaint ni secouru
de personne, et donner beau jeu aux étrangers d'en profiter, et à
l'Espagne de le perdre.

À l'égard du jeune roi, je priai M. le duc d'Orléans de considérer qu'il
n'y avait rien dans toute cette conduite qui en aucun temps lui pût être
rendu suspect avec la plus légère apparence, et dont il ne fût en état
de lui rendre le compte le plus exact. Son Altesse Royale trouve en
arrivant à la régence les finances dans un désordre et dans un état
désespéré, les peuples au delà des derniers abois, le commerce ruiné,
toute confiance perdue, nul remède que les plus cruels. Il n'accuse
personne, personne aussi n'est accusé, mais lui, qui n'a jamais eu la
moindre part aux affaires, a raison de n'y vouloir pas toucher du bout
du doigt sans avoir exposé leur situation au public, et ne présume pas
assez de soi pour de son chef y apporter des remèdes. Il n'en aperçait
que de cruels, c'est le public qui en portera tout le poids et toute la
souffrance, soit d'une manière ou de l'autre\,; n'est-il pas de la
sagesse et de l'équité de lui en laisser le choix\,? C'est aux états
généraux qu'il le défère. Il ne fait en cela qu'imiter les rois
prédécesseurs, et Louis XIII lui-même, qui les assembla et les consulta
à Paris, en 1614. Il a suivi l'avis des états généraux. On ne peut donc
lui imputer de présomption dans une affaire si générale et si
principale\,; on ne peut aussi l'accuser de faiblesse, ni d'avoir fait
la plus petite brèche à l'autorité royale, puisqu'il n'a fait qu'imiter
à la lettre ce que les rois prédécesseurs, jusqu'au pénultième, ont tous
fait, majeurs et mineurs, et pour des cas bien moins importants. Si les
états touchés de cette confiance loi en ont marqué leur reconnaissance
par cette acclamation sur les renonciations, outre qu'il ne la leur a
jamais demandée, ils n'ont rien fait que montrer des vœux, et une
disposition de leurs cœurs conforme à celle du feu roi et de toute
l'Europe, et pour ainsi dire, canoniser ses volontés, les fondements de
la paix, ceux du repos de la France en quelque cas que ce puisse être,
dont lui et eux espèrent, et ont en même temps montré leurs plus
sincères désirs et espérance qu'il puisse n'arriver jamais, en quoi il
n'a paru que de la bonne et franche volonté, et rien qui puisse toucher,
le plus légèrement même, ni aux droits sacrés de l'autorité royale, ni à
ceux d'aucun ordre, corps, ni particulier, pas même, ce qui est tout
dire, de la branche d'Espagne, puisque elle-même a solennellement et
volontairement fait, en pleins cortès assemblés à Madrid, ses
renonciations, avant même que M. le duc de Berry et Son Altesse Royale
eussent fait les leurs en plein parlement, dans l'assemblée et en
présence des pairs, tous mandés par le roi pour s'y trouver. Où y a-t-il
dans tout cela quoi que ce soit de tant soit peu répréhensible, en
quelque sens qu'il puisse être pris, et de quelque côté qu'on le puisse
tourner\,?

Outre tant de grands et de si avantageux partis qu'on vient de voir que
M. le duc d'Orléans pouvait si aisément tirer de la tenue des états
généraux, je ne crus pas dangereux d'y en tenter encore un autre, ni
fort difficile d'y réussir, en profitant de leur premier enthousiasme de
se revoir assemblés, et déférer l'important choix du remède aux
finances, et de leur acclamation sur les renonciations. Il fallait
qu'elle fût faite avant de remuer ce qui va être exposé, mais le leur
présenter aussitôt après avec la même délicatesse, afin de profiter,
pour les y engager, des idées flatteuses dont ces actes leur auraient
rempli la tête, et ne pas perdre le temps jusqu'à ce qu'ils eussent
réglé leur avis sur les finances, ce qui aurait trop long trait, et
donnerait le temps d'intriguer et de les manier à celui qu'il s'agirait
d'attaquer. Dans quelque servitude que tout fût réduit en France, il
restait des points sur lesquels la terreur pouvait retenir les discours,
mais n'avait pas atteint à corrompre les esprits. Un de ces points était
celui des bâtards, de leurs établissements, surtout de leur apothéose.

Tout frémissait en secret, jusqu'au milieu de la cour, de leur
existence, de leur grandeur, de leur habilité de succéder à là couronne.
Elle était regardée comme le renversement de toutes les lois divines et
humaines, comme le sceau de tout joug, comme un attentat contre Dieu
même, et le tout ensemble, comme le danger le plus imminent de l'État et
de tous les particuliers. C'était alors le sentiment intime et général
des princes du sang et des grands, par indignation et par intérêt, je
dis de ceux même qui devaient le plus au roi, à la faveur de
M\textsuperscript{me} de Maintenon, et qui paraissaient le plus en
mesures étroites avec le duc du Maine. Je le sais par ce que m'en ont
dit à moi-même, et en divers temps et toujours, les maréchaux
d'Harcourt, de Villars et de Tessé, et cela du fond du cœur, de dépit,
de colère, de raisonnement, point pour me sonder et me faire parler, car
ils savaient de reste ce que j'en pensais et sentais\,; et je cite
ceux-là comme étant avec eux en quelque commerce, beaucoup moins
pourtant avec Tessé qui ne s'en expliquait pas moins librement devant
moi, mais lesquels, surtout en ce temps-là, n'étaient avec moi en aucune
liaison particulière. Jusqu'au maréchal de Villeroy ne s'en est pas tu
avec moi depuis la mort du roi, et fut un des plus vifs lorsqu'il fut
question d'agir contre leur rang en toutes les occasions qui s'en sont
présentées, ainsi que les deux autres que j'ai cités, car Tessé n'étant
pas duc ne put qu'applaudir. Les gens de qualité n'étaient pas alors
moins irrités, et j'en étais informé de plusieurs immédiatement\,; et
par cette bricole, de bien d'autres.

Le parlement si attaché aux règles anciennes, si hardi en usurpations,
comme on l'a vu à propos du bonnet, jusque sur la reine régente, si
tenace à les soutenir, n'avait pas caché son indignation de la violence
faite à tout ce qu'il y a de plus fort, de plus fixe, de plus ancien, de
plus vénérable parmi les lois, en faveur des bâtards, ni le dépit des
honneurs qu'ils avaient forcé cette compagnie de leur rendre. Le gros du
monde de tous états était irrité d'une grandeur inouïe en tout genre, et
jusqu'au peuple ne s'en cachait pas en les voyant passer, ou en
entendant parler. Cette disposition universelle n'avait point cessé. Les
artifices et la cabale ne l'avaient point attaquée, et par ce qui en
sera expliqué en son temps, on verra que ces ruses n'auraient pu avoir
le moindre succès s'il y avait eu des états généraux. Je crus donc que
l'objet des bâtards leur pouvait être présenté comme le plus dangereux
colosse, et le plus digne de toute leur attention.

Outre ce qui vient d'être dit de l'impression que cette monstrueuse
élévation avait faite sur les esprits, leur montrer le groupe de leurs
richesses, de leurs gouvernements, de leurs charges, de cette multitude
de gens de guerre et de soldats sous leurs ordres et d'importantes
provinces sous leur commandement, avec cette différence que tous autres
gouverneurs et chefs de troupes ne l'étaient que de nom, impuissants
avec des titres qui n'étaient que de vains noms, eux-mêmes inconnus aux
lieux et aux troupes que leurs patentes semblaient leur soumettre,
tandis que la marine, l'artillerie, les carabiniers, tous les Suisses et
Grisons, sept ou huit régiments sous leur nom, outre toutes ces troupes,
étaient dans leur très effective dépendance de tout temps, parce que le
roi l'avait ainsi voulu, et qu'encore que leur assiduité près de lui les
eût empêchés d'aller en Guyenne, en Languedoc, en Bretagne, ils ne
laissaient pas d'y être très puissants, par l'autorité et les
dispositions des grâces que le roi leur y avait soigneusement données.
Faire sentir aux états généraux de quel danger était une si formidable
puissance entre les mains de deux frères, surtout quand elle était
jointe au nom, rang, droits, état de prince du sang, capables de
succéder à la couronne, vis-à-vis des princes du sang tous enfants, et
sans établissement entre eux tous que le gouvernement de Bourgogne, une
belle charge mais uniquement domestique, et sept ou huit régiments sur
lesquels ils n'avaient jamais eu l'autorité que les bâtards avaient sur
les leurs, et sans contre-poids encore d'aucun seigneur, dont les
gouvernements et les charges n'étaient que des noms vides de choses, et
qui n'opéraient que des appointements. Faire envisager aux états la
facilité qu'avaient les bâtards de tout entreprendre, et les horreurs de
leur joug et des guerres civiles pour l'établir et pour s'en défendre.
Enfin leur faire toucher l'évidence du crime de lèse-majesté dans
l'attentat d'oser prétendre à la couronne, et d'avoir abusé de la
faiblesse d'un père qui n'aurait jamais dû reconnaître de doubles
adultérins, et qui est le premier qui l'ait osé par la surprise, qu'on a
vue ailleurs, pour escalader tous les degrés par lesquels ils sont
parvenus à une si effrayante grandeur, et ne s'en faire encore qu'un
échelon pour s'assimiler en tout aux princes du sang, jusqu'au monstre
incroyable de se rendre comme eux habiles à succéder à la couronne.
Exciter les uns par le renversement des familles, et la tentation de
devenir mères de semblables géants, les autres par les motifs de la
religion, ceux-ci par le mépris et l'anéantissement de toutes les lois,
ceux-là par celui de tout ordre, tous par l'exemple qui serait suivi des
rois successeurs, dont naîtrait une postérité qui envahirait tout, et ne
laisserait rien aux vrais princes du sang, dont ils craindraient et
haïraient la naissance, et au-dessous d'eux tout ordre légitime et
légal. Surtout leur exposer bien clairement jusqu'où entraîne l'ambition
de régner avec un droit tel qu'il puisse être\,; que tout ce que ces
bâtards ont obtenu, surtout les rangs et droits de princes du sang et
d'habilité à la couronne, est l'ouvrage du seul duc du Maine\,; les
propos de la duchesse du Maine aux ducs de La Force et d'Aumont à
Sceaux\,; la facilité à tout que leur donnent leurs établissements\,;
enfin combien moins de distance entre eux et la couronne aujourd'hui
qu'à être parvenus à y être déclarés habiles\,; et que le motif exprimé
et enregistré de ces derniers degrés de rang d'état de princes du sang,
d'habilité à succéder à la couronne, est l'honneur qu'ils ont d'être
fils et petits-fils du roi. Conduire les états à en conclure que
l'adultère étant par là tacitement mis au niveau du mariage par cette
énorme expression de l'honneur qu'ils ont d'être fils et petits-fils du
roi, il n'y a plus qu'un pas à faire, et dont tout le chemin se trouve
frayé, pour les déclarer fils de France, ce qu'on aurait peut-être vu si
le roi eût vécu quelque peu davantage, et à quoi même il y a toute
apparence, au degré de puissance où le roi s'était mis, à l'état de
disgrâce où l'art préparatoire avait réduit M. le duc d'Orléans, à
l'enfance de tous les princes du sang, à l'anéantissement et à
l'impuissance de tous les ordres du royaume, à l'ambition démesurée du
duc du Maine, et à son pouvoir sans bornes sur la faiblesse du roi à son
égard.

Tels sont les motifs à remuer les états généraux, sans que M. le duc
d'Orléans y parût en aucune sorte. Exciter tristement, timidement,
plaintivement la fermentation des esprits\,; s'assurer de leur volonté,
exciter leur courage en leur montrant péril, justice, religion,
patrie\,; leur faire sentir que ces grandes choses se trouvaient
naturellement en leurs mains, les piquer d'honneur d'immortaliser leur
tenue et leurs personnes par se rendre les libérateurs de tout ce qui
est le plus sacré et le plus cher aux hommes\,; conduire de l'œil
l'effet résultant de ce souffle\,; inculquer le secret sur l'impression
et la résolution, non qu'il se pût espérer tel qu'il serait nécessaire,
mais pour contenir au moins et procéder par chefs accrédités, qui mènent
le gros sur parole, sans trop s'expliquer avec eux. Si la mollesse, les
délais, les embarras font craindre nul succès, ou un succès équivoque,
s'arrêter doucement, laisser évaporer le projet en fumée, où personne
n'aurait paru directement. Discours, propos, réflexions en l'air, rien
de M. le duc d'Orléans ni d'aucuns personnages\,; tous, occupés de
l'accablement d'affaires, ont ignoré ces raisonnements, ou n'en ont ouï
parler qu'à bâtons rompus et faiblement, et n'ont seulement pas pris la
peine de les ramasser. Que fera M. du Maine\,? À qui s'en
prendra-t-il\,? Que peut-il de pis que ce qu'il a fait\,? Au contraire,
timide comme il est, il sera souple, tremblant\,; et pourvu qu'il
échappe, prendra tout pour bon, et sera le premier à se moquer de propos
chimériques, à les dire tels dans la frayeur qu'ils ne se réalisent, et
que le cas qu'il en ferait par ses plaintes ne l'engageât plus loin
qu'il n'oserait. Si, au contraire, on voyait bien distinctement les
états prendre résolument le mors aux dents, les induire à ne donner pas
aux bâtards cet avantage, par l'entreprise de se rendre leurs juges, de
revenir dans la suite en inspirant au roi majeur de défaire un ouvrage
entrepris sur son autorité, et dont l'exemple toléré et laissé en son
entier la menace des plus dangereuses entreprises\,; mais à suivre leur
objet par les moyens les plus respectueux qui ne donnent que plus de
force aux plus fermes, et se garder de la honte de donner dans un piège
tendu pour leur faire manquer le principal en haine de l'accessoire. Les
porter à s'adresser au roi par une requête en leur nom où tout ce qui
vient d'être exposé soit expliqué d'une manière concise, forte,
pressante, où il soit bien exprimé que le roi, même à la tête de toute
la nation, n'a pas droit de donner à qui que ce soit, ni en aucun cas,
le droit de succéder à la couronne acquis aux mâles, de mâles en mâles,
d'aîné en aîné, à la maison régnante, à laquelle personne, tant qu'il en
peut exister un, ne peut être subrogé. Montrer que ce pas une fois
franchi ne reçoit plus de bornes\,; que tous les bâtards futurs
remueront tout pour atteindre ceux d'aujourd'hui\,; qu'un favori peut
devenir assez puissant, plus aisément encore un premier ministre, pour
se proposer et pour arriver au même but, et qui auront encore pour eux
une naissance illustre, du moins honnête et légitime, non adultérine,
réprouvée de Dieu et des hommes, et qui, jusqu'à ces doubles adultérins
appelés à la couronne, ne l'avaient pas seulement pu être aux droits les
plus communs de la société, et n'avaient jamais été tirés du néant et
des ténèbres\,; enfin qu'il n'y a pas plus loin, et peut-être beaucoup
moins, dès que tout pouvoir est reconnu en ce genre par l'admission de
son exercice, à intervertir l'ordre de la succession entre ceux qui sont
reconnus habiles à succéder à la couronne, qu'à donner cette habilité à
ceux que leur naissance n'y appelle pas, encore plus à ceux dont le vice
infamant de la naissance les enterre nécessairement dans la plus épaisse
obscurité du non-être, sans état et sans droit à nulle succession, ni
donation même la plus ordinaire, pas même de faire passer la leur à
leurs enfants légitimes s'ils ont acquis quelque bien. S'arrêter à la
réflexion de ce qui serait arrivé de la France et de toute la maison
régnante, si ce droit de disposer de la couronne avait été par l'usage
reconnu dans les rois, si les fils de Philippe le Bel avaient préféré
leur soeur à un parent aussi éloigné que Philippe de Valois, et si les
fils d'Henri II, gouvernés par Catherine de Médicis, par sa haine pour
Henri IV, par sa prédilection pour sa fille de Lorraine, par un prétexte
de religion qui avait les plus grands appuis, eussent préféré cette sœur
à un parent aussi éloigné qu'Henri IV, qui sans cela eut tant de peines
et de travaux à essuyer pour se mettre à coups d'épée en possession du
royaume qui lui appartenait, et qu'il acheta encore par tant de traités,
de millions et d'établissements de la Ligue, qui lui avait pensé
arracher la couronne tant de fois pour la porter dans une maison
étrangère\,; enfin ce qui serait arrivé de l'État et de la maison de
France, si ce droit reconnu de disposer de la couronne eût eu la force
des exemples, du temps de Charles VI et d'Isabeau de Bavière, qui
déshéritèrent le Dauphin et toute leur maison, et firent couronner dans
Paris le roi d'Angleterre leur gendre et {[}le{]} reconnaître roi
régnant de France, sans droit aucun, ni même idée de ce droit.

On sait les suites d'une telle entreprise, qui fit verset tant de sang,
qui épuisa tant de trésors, qui mit si longtemps la France à deux doigts
de sa perte et de son entier renversement. La richesse, l'importance, la
réalité effective d'une matière qui, pour ainsi dire, comprend tout, ne
doit rien perdre par le lâche et le diffus d'une vaine éloquence. Tout y
doit faire voûte et se contre-tenir par toute la force dont elle est si
grandement susceptible\,; rien d'inutile, tout concis, tout serré, tout
en preuve, et en chaîne sans interruption.

l est donc important d'avoir cette requête toute prête pour ne la pas
laisser au différent génie de tant de gens qui ne s'accorderaient qu'en
des longueurs très périlleuses, mais en forme de canevas, pour ménager
leur vanité, et s'avantager de leur paresse et des jalousies en leur
proposant ce canevas à mettre en forme à leur gré, ce qu'ils
retoucheront sans peine et en peu de temps, assez pour compter qu'entre
leurs mains il est devenu leur ouvrage, ce qu'il est très important
qu'ils se persuadent bien. Il y a toujours dans ces nombreuses
assemblées des chefs effectifs à divers étages qui, sans en avoir le nom
ni le caractère, en ont la confiance et l'autorité par l'estime, par
l'adresse, par une mode que le hasard établit, et que la conduite
soutient jusqu'à les rendre presque maîtres de tourner les esprits et
les délibérations où ils veulent. C'est ceux-là qu'il faut de bonne
heure reconnaître et persuader, pour avoir par eux toute l'assemblée, et
certes on n'eut jamais plus beau jeu qu'à mettre de telles vérités en
évidence, et à toucher les hommes par ce qui est tout à la fois le plus
intéressant par toutes les parties les plus sensibles, le plus important
et le plus raisonnable par tout ce qu'il s'y peut faire de sages
réflexions, de plus odieux et de plus périlleux en soi et par ses
suites, enfin de plus juste, de plus nécessaire, de plus instant, de
plus essentiel à arrêter pour jamais par une punition qui, proportionnée
aux attentats, mette pour jamais à l'abri de Titans et d'usurpateurs
possibles la nation, la couronne, et l'unique maison qui, tant qu'elle
dure, y a un droit unique et exclusif acquis, qui assure à jamais le
repos et la tranquillité publique à cet égard, et la prééminence si
distinctive de cette maison sur toutes les autres maisons du monde.

On ne peut donc donner trop d'adresse, de délicatesse et de soins pour
dignement et nerveusement dresser ce canevas, le faire promptement
tourner et adopter par les états en requête, la leur rendre leur et
comme le chef-d'oeuvre de leur sagesse et de leur poids, surtout la leur
montrer sans danger, par l'impuissance de ceux qu'elle regarde contre
une multitude qui représente le corps de la nation. Ne point laisser
d'intervalle entre l'adoption de la requête et sa présentation, pour
éviter les mouvements et les artifices du duc du Maine, en quoi il s'est
montré si grand maître\,; et par les

mêmes moyens qu'on sera parvenu à l'adoption de la requête, et à la
résolution de la présenter, n'y pas perdre un seul instant, et, s'il est
possible, sans mettre une seule nuit entre-deux. Cette présentation est
l'engagement, par conséquent le premier coup de partie et celui qui
entraîne le reste. Arrivés à ce point, la mécanique est aisée. Je
comtois que Meudon serait prêté à la reine d'Angleterre pour s'y tenir
avec sa cour et sa suite, et laisser Saint-Germain libre aux états
généraux, où, à tous égards, ils auraient été fort bien, ni trop loin ni
trop près de Paris, et M. le duc d'Orléans en liberté de tenir le roi à
Paris, à Versailles, à Marly, comme il l'aurait voulu, pour en
différents temps s'approcher ou s'éloigner davantage de Saint-Germain.
C'est dans le salon de Marly où il aurait fallu destiner les audiences à
donner par le roi aux états, comme un lieu vaste, commode, dégagé de
quatre côtés, joignant l'appartement du roi et celui du régent, un corps
de maison isolé, et toutefois enfermé et gardé, et à une lieue de
Saint-Germain.

Aussitôt donc que la requête par le vœu des états serait prête à être
présentée, partir tous en corps, et ne prendre que le temps, toujours
assez long, d'un pareil embarquement dans les carrosses qu'on aurait
pris partout où on aurait pu, mais dont sous main on aurait fait
rencontrer sous divers prétextes le plus qu'on aurait pu sans rien
marquer\,; prendre, dis-je, ce temps pour envoyer devant quelques
députés au régent, l'avertir de la résolution prise de venir en corps
trouver le roi, {[}de la part{]} desquels {[}états{]} ils sont chargés
de supplier Son Altesse Royale de les conduire à Sa Majesté pour lui
demander audience, et lui dire qu'ils sont en chemin et qu'ils vont
arriver. Il ne sera pas inutile qu'il y ait quelque dispute entre le
régent et eux sur l'affaire qui les amène, dont les députés éviteront de
s'expliquer clairement et même devant le roi. C'est à l'adresse du
régent à s'y conduire avec délicatesse, entre trop d'inquiétude et trop
de froideur, sur une explication plus précise qu'il se faut bien garder
de causer pour éviter l'embarras qu'elle ferait naître, et qu'il
faudrait pourtant surmonter, et pour ne pas émousser l'effet de la
surprise et de tout ce qui l'accompagne, qui ne pourra qu'être grand,
quelque chose qu'il est impossible qu'il n'en ait transpiré alors. Les
états arrivant vers la chapelle où on met pied à terre seront conduits
au roi, rencontrés en chemin dans le petit salon par le régent, non par
cérémonial, mais voulant savoir plus clairement ce qui les amène, ne
laissant pas de s'avancer toujours et d'arriver avec eux jusqu'au roi,
sans avoir été plus satisfait.

Une très courte et très respectueuse harangue annoncera l'excès de
l'importance de ce qui les amène ainsi aux pieds du roi, et finira par
lui demander la permission de lui présenter leur très humble requête, et
de leur permettre d'attendre à Marly qu'il lui ait plu de la faire
examiner par son conseil, persuadés qu'elle y sera trouvée si simple, si
importante, si juste, que l'examen n'en pourra être long et qu'il leur
sera favorable. La recevoir et la faire examiner n'est pas chose qui se
puisse refuser. Le roi se retirera dans son appartement et le régent
dans le sien avec les députés à la suite de l'affaire, qui alors s'en
expliqueront nettement. Débat entre eux et le régent, qui ne trouvera
pas que ce soit chose à répondre ainsi sur-le-champ, et eux qui ne se
laisseront point persuader de quitter prise, et qui protesteront que les
états sont résolus de ne pas sortir du salon, aux portes duquel il sera
bon qu'il y ait plus que les Suisses ordinaires, pour empêcher l'entrée
aux gens suspects. Les députés ne manqueront pas de récuser ceux du
conseil que leur requête regarde\,; et finalement le conseil sera mandé
et assemblé sur-le-champ. M. le duc d'Orléans y marquera sa surprise
sans s'engager en grands discours, et plus encore son étonnement et son
embarras de l'opiniâtre résolution des états à demeurer dans le salon
jusqu'à la réponse à leur requête, pour communiquer au conseil le même
embarras et le même étonnement. Ce sera après à son adresse, à sa
délicatesse, à son esprit, à son poids à ne s'ouvrir sur rien que sur
l'importance de la requête, l'état violent et plus qu'embarrassant qui
naît de cette attente opiniâtre des états généraux dans le salon, la
nécessité extrême de les ménager, profiter de l'absence de ceux que la
requête regarde, nécessairement abstenus du conseil, et de l'intérêt et
de la bonne volonté qu'il peut trouver dans les autres membres, et faire
conclure que la requête sera renvoyée par le roi au parlement pour y
être jugée, les pairs mandés de s'y trouver par le roi, comme étant
cause très majeure. Laisser les portes fermées, passer par le petit
salon avec le conseil dans le cabinet du roi, lui rendre compte de la
résolution, repasser chez lui avec le conseil, mander dans le salon les
députés commis à la suite de l'affaire, leur remettre le résultat du
conseil signé de lui, de tout le conseil et du secrétaire d'État qui en
tient le registre, et en leur présence lui ordonner d'aller expédier
sur-le-champ le renvoi de leur requête et de la leur envoyer à
Saint-Germain. Les députés demanderont que le roi veuille bien recevoir
le très humble remerciement des états, ajouteront que cependant le
renvoi pourra être expédié, et déclareront que les états ne partiront
point de Marly qu'ils n'aient toutes les lettres et expéditions
nécessaires. Altercation encore là-dessus, fermeté d'une part,
complaisance enfin de l'autre sur une chose qui n'emporte rien de plus
que ce qui est accordé.

Les députés retourneront dans le salon rendre compte du succès de leur
requête, tandis que le régent, suivi du conseil, passera chez le roi
pour le suivre à l'audience de remerciement qu'il ira donner aux états.
Ce remerciement sera pathétique sur l'importance de l'affaire, énergique
sur la fidélité et l'attachement. Le roi, le régent et le conseil à sa
suite retirés, les états iront par leurs députés remercier le régent et
le conseil retournés chez lui, attendront leurs expéditions, les
examineront bien en les recevant des mains du secrétaire d'État, et s'en
retourneront avec à Saint-Germain.

Le premier président, le doyen du parlement et les gens du roi seront
mandés le lendemain pour recevoir du roi, en présence et par la bouche
du régent, les ordres conformes au renvoi, et pour leur recommander
l'importance de l'affaire, tant en elle-même que par la dignité des
états et la considération de ceux qu'elle regarde. C'est après à M. le
duc d'Orléans à se savoir lestement tirer d'intrigue dans sa famille\,:
surprise, force, embarras de pareille démarche et si opiniâtre, et de
savoir adroitement profiter de la gravité des raisons, des dispositions
des juges, du poids de ce grand nom d'états généraux, et de la nature
d'une affaire qui n'est embarrassée ni de lois diverses, ni
d'ordonnances, ni de coutumes, ni d'arrêts, ni de procédures, et qui
s'offre tout entière de première vue, pour accélérer et terminer au gré
de pleine et entière justice et de barrière inaltérable à l'avenir\,;
enfin, dans le jugement et après le jugement, de distinguer entre les
deux frères l'innocent d'avec le coupable, suivant leur mérite à chacun.
La suite a bien fait voir combien j'avais eu raison de concevoir ce
dessein, et combien celui à qui il était si nécessaire et à qui il
devait être si doux, en était peu capable en effet, quoiqu'il eût paru
le goûter et le sentit.

Une idée sans exécution est un songe et son développement dans tout ce
détail un roman. Je l'ai compris avant de l'écrire. Mais j'ai cru me
devoir à moi-même de montrer que je n'enfante pas des chimères\,; la
nécessité, l'importance, l'équité de la chose par la foule des plus
fortes et des plus évidentes raisons\,; la possibilité et peut-être la
facilité en présentant la disposition des esprits générale alors, et une
suite de mécanique qu'il faut en tous projets se rendre à soi-même
claire et faisable par un mûr examen des obstacles et des difficultés
d'une part, et de l'autre des moyens de réussir. Un roman serait un nom
bien impropre à donner au rétablissement d'un gouvernement sage et
mesuré\,; au relèvement de la noblesse anéantie, ruinée, méprisée,
foulée aux pieds\,; à celui du calme dans l'Église\,; à l'allégement du
joug, sans attenter quoi que ce soit à l'autorité royale, joug qu'on
sent assez sans qu'il soit besoin de l'expliquer, et qui a conduit Louis
XIV aux derniers bords du précipice\,; à laisser au moins à la nation le
choix du genre de ses souffrances, puisqu'il n'est plus possible de l'en
délivrer, enfin de préserver la couronne des attentats ambitieux,
conserver à la maison régnante l'éclat de sa prérogative si uniquement
distinctive, et la tranquillité intérieure de l'état du péril du
titanisme, et des dangereuses secousses qu'il ne peut manquer d'en
recevoir, puisque pour des choses si monstrueusement nouvelles on est
contraint de les exprimer par des mots faits pour les pouvoir exprimer.
Si des projets de cette qualité, et dont l'exécution est rendue
sensible, n'ont pas réussi, c'est qu'ils n'ont pas trouvé dans le temps
le plus favorable un régent assez ferme et qui eût en soi assez de
suite. On en verra d'autres dans le cours de cette année et des
suivantes qui ont eu le même sort. Dois-je me repentir pour cela de les
avoir pensés et proposés\,? J'ai toujours cru que ce n'était pas le
succès qui décidait de la valeur des choses qui se proposent, beaucoup
moins quand il dépend d'un autre qui néglige de les suivre ou qui ne
veut pas même les entreprendre. Ce qui va suivre est de ce dernier
genre.

\hypertarget{chapitre-x.}{%
\chapter{CHAPITRE X.}\label{chapitre-x.}}

1715

~

{\textsc{Discussion entre M. le duc d'Orléans et moi sur la manière
d'établir et de déclarer sa régence.}} {\textsc{- Aveu célèbre du
parlement par la bouche du premier président de La Vacquerie y séant, de
l'entière incompétence de cette compagnie de toute matière d'État et de
gouvernement.}} {\textsc{- Deux uniques et modernes exemples de régences
faites au parlement.}} {\textsc{- Causes de cette nouveauté.}}
{\textsc{- Raisons de se passer du parlement pour la régence, comme
toujours avant ces deux derniers exemples.}} {\textsc{- Observation à
l'occasion de la majorité de Charles IX et de l'interprétation de l'âge
de la majorité des rois.}} {\textsc{- Mesures et conduite à tenir pour
prendre la régence.}} {\textsc{- Conduite à tenir sur les dispositions
du roi indifférentes, et sur le traitement à faire à
M\textsuperscript{me} de Maintenon.}} {\textsc{- Prévoyances à avoir.}}
{\textsc{- Faiblesse de M. le duc d'Orléans à l'égard du parlement.}}
{\textsc{- État et caractère de Nocé.}}

~

Après de longs et de fréquents tête-à-tête sur toutes ces différentes
matières, entre M. le duc d'Orléans et moi, nous vînmes à celle de la
régence. Je l'avais fort examinée, et voici comme je lui en parlai et ce
que je lui proposai. Je lui dis qu'il ne s'agissait point ici de ces
régences réglées par les rois pendant l'absence qu'ils vont faire hors
de leur royaume et qui finissent par leur retour, mais de celles
uniquement que la mort d'un roi et la minorité de son successeur rendent
nécessaires. Je n'eus pas de peine à montrer que celles-là tombent de
droit tellement au plus proche du roi mineur, que les mères et les sœurs
y sont admises, quoique les femelles soient exclues de la couronne, et
que par conséquent ni les cabales ni quelque disposition que le roi pût
faire, il n'était pas dans le possible de la lui ôter. Qu'à l'égard de
la brider, ce qui ne se pouvait tenter que par des dispositions du roi
odieuses, il savait ce que les plus sages et les plus solennelles
étaient devenues aussitôt après la mort de Charles V et de Louis XIII
qui les avaient faites sur lesquelles il n'y avait point à craindre que
celles du roi eussent de l'avantage par toutes sortes de raisons\,; que
néanmoins il fallait penser à s'en garantir en ne se commettant pas avec
imprudence\,; que si le roi faisait des dispositions là-dessus, il n'y
avait point à douter qu'elles ne tendissent à le diminuer pour accroître
le duc du Maine\,; que sans me départir de ce que je lui avais dit de la
disposition des esprits, et en particulier du parlement sur la grandeur
des bâtards, surtout sur leur apothéose, il fallait songer que le
premier président était l'âme damnée de M. et de M\textsuperscript{me}
du Maine, qui pour leur intérêt l'avaient mis à la tête du parlement
dont il épouserait aveuglément toutes les volontés, parce que, brouillé
par cet attachement avec M\textsuperscript{me} la Duchesse et les
princes du sang, ne pouvant par cela même s'assurer de Son Altesse
Royale, et mal au dernier point par l'affaire du bonnet avec tant de
gens considérables, il n'avait de ressource que la protection du duc du
Maine, et par conséquent le plus vif intérêt à toute sa grandeur, et son
pouvoir\,; que tel que fût le premier président, il avait acquis à force
de manèges du crédit dans sa compagnie, éblouie de son jargon, de sa
politesse, de l'attachement qu'il leur avait persuadé avoir pour tous
les avantages de la compagnie et de ses magistrats, enfin par ses grands
airs, sa table, sa dépense, et l'union que l'affaire du bonnet avait si
bien rétablie entre lui et les présidents à mortier, dont quelques-uns
auparavant le tenaient en brassière\,; que les cabales et les bassesses
qui ne coûtaient rien à M. ni à M\textsuperscript{me} du Maine, qui
avaient tant fait leurs preuves en artifices et en noires inventions,
étaient indignes de tout homme et impraticables pour Son Altesse Royale,
dans le degré surtout où elle se trouverait\,; qu'autre chose était de
présenter un colosse dangereux à abattre et les plus saintes lois à
préserver d'une ambition démesurée et toute-puissante, autre chose
d'entrer en concurrence avec ce colosse sur des dispositions du roi en
sa faveur à la diminution de l'autorité d'un régent\,; qu'indépendamment
d'équité, le parlement est toujours porté à se croire et à faire, autant
qu'il en trouve les occasions, le modérateur de la puissance, puisqu'il
a si souvent tenté de le faire sentir même aux rois, à plus forte raison
dans une entrée de régence, temps de faiblesse dont ce corps a toujours
su se prévaloir\,; que le même amour-propre qui le flatterait d'avoir à
prononcer sur le renversement du colosse, si la cause lui en était
déférée, et lui ferait goûter la justice et les raisons d'user du
pouvoir de le renverser, ce même amour-propre trouvera sa satisfaction à
prononcer entre le régent et ce colosse\,; et comme il ne s'agira pas
alors de le détruire, le même amour-propre le portera à le favoriser
sous différents prétextes pour faire naître une suite de divisions dont
il espérera se mêler et en profiter, et pour avoir un puissant soutien
de sa considération et de son autorité qui, en minorité, a si souvent
entrepris sur l'autorité royale qui est celle dont le régent est revêtu
et qu'il ne doit pas laisser entamer. De ce raisonnement, qui n'a rien
de contraire à la disposition du parlement contre les bâtards et leurs
grandeurs, où il ne s'agit pas ici de les remettre dans les bornes, il
sera aisé aux manèges du duc du Maine et de Mesmes de le tourner
favorablement aux prétentions du duc du Maine. Ainsi lutte indécente et
inégale et publique\,; et si elle bâte mal suivant ces apparences, quels
embarras et peut-être quels désordres\,! Certainement, quel lustre et
quel degré de continuelles entreprises du parlement, qui se voudra mêler
de tout avec autorité\,! Quel triomphe et quelle dangereuse victoire du
duc du Maine\,! quelle honte pour le régent\,! et quelle situation
pendant tout le cours de la régence\,! On tremble donc avec raison en
pensant jusqu'où tout cela peut porter.

Je proposai donc à M. le duc d'Orléans de ne s'y pas commettre et de
prendre un autre tour. Je lui fis observer qu'il ne s'était fait au
parlement que les deux dernières régences. On n'y avait jamais songé
auparavant. Le duc d'Orléans, dépité de voir la régence entre les mains
de M\textsuperscript{me} de Beaujeu, femme du frère du duc de Bourbon,
connétable de France, et sœur fort aînée de Charles VIII, pendant sa
minorité, tenta la voie de se plaindre, et de demander au parlement
justice du tort qu'il prétendait être fait à son droit sur la régence.
La réponse célèbre que le premier président de La Vacquerie lui fit en
plein parlement n'est ignorée de personne, et se trouve la même dans
toutes les histoires\,: «\,La cour, lui dit ce magistrat, n'est établie
que pour juger au nom et à la décharge du roi les procès entre ses
sujets, et nullement pour se mêler d'aucune affaire d'État ni du
gouvernement, où elle n'a pas droit d'entrer, sinon par un commandement
exprès de Sa Majesté.\,» Le duc d'Orléans, lors héritier présomptif de
la couronne, et qui y succéda à Charles VIII, sous le nom de Louis XII,
ne put tirer autre chose du parlement. Il prit les armes, il n'y fut pas
heureux\,; M\textsuperscript{me} de Beaujeu demeura régente sans
question ni difficulté, et son administration fut bonne et heureuse,
jusqu'à la majorité de Charles VIII. Je passe M\textsuperscript{me}
d'Angoulême qui n'a été régente que pendant deux absences du roi
François Ier son fils, qui l'établit en partant, et la reine
Marie-Thérèse que le roi établit deux fois régente en partant pour ses
conquêtes. Ainsi, jusqu'à la mort d'Henri IV, nulle mention du parlement
à cet égard.

Personne n'ignore de quelle manière le parricide fut commis, ni les
ténèbres qui ont couvert un si grand crime. Il est difficile aussi de se
refuser d'en deviner la cause que ces ténèbres mêmes indiquent, et que
les histoires et les mémoires de ces temps-là font sentir et même
quelque chose de plus. Cette remarque était nécessaire, on s'en
contentera. Le cas était unique. Le roi mort à l'instant au milieu des
seigneurs, qui étaient dans son carrosse qu'ils firent retourner au
Louvre avec le corps du roi, peu de grands à Paris, le prince de Condé
hors du royaume, le comte de Soissons chez lui mécontent de ce qui
s'était passé sur la duchesse de Vendôme au couronnement de la reine\,;
l'intérieur intime du Louvre, peu étonné et gardant moins que
médiocrement les bienséances, tout occupé d'assurer toute l'autorité à
la reine pour établir la leur et leur fortune\,; cette princesse élevée
au-dessus de toute faiblesse, et sans distraction sur tout ce qui
pouvait établir sa pleine et entière régence, on courut au parlement
pour avoir un lieu public et solennel, et un corps intéressé à soutenir
ce qui se ferait dans son sein\,; un corps encore qu'on avait à ménager
par d'autres raisons plus ténébreuses, et qui n'étaient pas moins
importantes. Le duc d'Épernon environna de son infanterie le dehors et
le dedans des Grands Augustins où le parlement tenait ses séances depuis
que le palais était occupé des préparatifs qui s'y faisaient pour les
fêtes qui devaient suivre le couronnement de la reine. Tout cela se fit
sur-le-champ, M. de Guise et lui entrèrent en séance, et la reine y fut
aussitôt déclarée régente, en présence de trois ou quatre autres pairs
ou officiers de la couronne, qui y arrivèrent l'un après l'autre.

Le murmure fut grand d'une nouveauté si subite et si précipitée\,; force
mouvements ranimés par la prompte arrivée et les plaintes de M. le comte
de Soissons, et depuis encore par le retour du prince de Condé, et ses
prétentions. Mais la chose était faite, et la déprédation des trésors
d'Henri IV, déposés à la Bastille pour l'exécution de ses grands
desseins et la guerre de Clèves, achevèrent d'affermir l'autorité de la
régente, ou plutôt des gens qui la gouvernaient. C'est le premier
exemple d'une régence faite au parlement. On laisse à juger et des
causes et de la manière et du droit qu'il peut avoir acquis au
parlement.

Le second exemple est tout de suite, lorsque la mort la plus sainte et
la plus héroïque couronna la vie la plus illustre et la plus juste, et
en fit à tous les rois la plus sublime leçon. La valeur de Louis XIII,
si utilement brillante lors du malheur de Corbie\footnote{Il s'agit de
  la prise de Corbie par les Espagnols, en 1636.}, aux îles de la
Rochelle, au siège de cette ville, et à tant d'autres exploits, au
célèbre Pas de Suse, en Roussillon, et partout où sa conduite ne fut pas
moins admirable\,; la sagesse de son gouvernement, le discernement de
ses choix, l'équité de son règne, la piété de sa belle vie, tant de
vertus enfin si relevées par sa rare modestie, et le peu qu'il comptait
tout ce qui n'est point Dieu\,; ses victoires, ses succès qui arrêtèrent
ceux de la maison d'Autriche, et qui anéantirent le parti protestant,
qui faisait un État dans l'État, au point que le roi son fils n'a plus
eu besoin que de la simple révocation d'un édit pour le proscrire\,;
l'utile protection donnée à ses alliés, et sa fidélité à ses traités,
tant de grandes choses n'avaient pu le préserver des malheurs
domestiques, augmentés sans cesse par vingt ans de stérilité de la
reine.

Arrivé lentement à sa fin, pour le malheur de la France et de l'Europe
entière, à un âge qui n'est souvent que la moitié de celui des hommes,
il ne la regarda que comme sa délivrance pour s'envoler à son Dieu, et
il profita de la tranquillité, de la paix, de la liberté de l'esprit que
lui conserva si parfaitement ce Dieu de justice et de miséricorde, pour
se rendre plus digne d'aller à lui par les ordres si judicieux que la
sagesse et l'expérience et la connaissance des choses et des personnes
lui firent dicter au milieu des douleurs de la mort sur tout ce qu'il
crut possible et nécessaire de régler pour l'administration de l'État
après lui, et balancer au moins avec prudence et harmonie ce qui ne
pouvait être remis en d'autres mains. Tout donné ce qui était vacant,
tout réglé ce qui était à faire après lui, il le voulut rendre public,
et le consacrer, pour ainsi dire, par le consentement des personnes les
plus proches comme les plus intéressées, et par l'approbation de tout ce
qu'il put assembler de grands et de personnes considérables de sa cour,
et de gens graves tels que sou conseil et les principaux magistrats.
Tous admirèrent tant de présence d'esprit, de sages combinaisons, de
sagacité et de prudence, tous en furent pénétrés.

La reine promit solennellement de s'y conformer, Monsieur ensuite et M.
le Prince, et tous ceux qui étaient nommés pour former le conseil. La
reine et ceux qui la gouvernaient n'en furent pas moins effrayés des
contre-poids établis à l'autorité de sa régence. Monsieur, faible,
facile, de tout temps lié avec la reine, jusque dans tous ses écarts,
pris sur-le-champ au dépourvu sans le secours de ceux qui le
conduisaient, se laissa enchanter aux flatteries de la reine, et crut
n'être que plus puissant en serrant son union avec elle par le sacrifice
de sa part de l'autorité que lui avait donnée la disposition dont on
vient de parler. Lui gagné, M. le Prince attaqué tout de suite par la
reine et par Monsieur, n'osa résister, et céda\,; à ces si principaux
exemples tout le conseil renonça tout de suite, chacun à sa voix
nécessaire, délibérative, inamissible, et une heure après la mort du roi
tout au plus, tout ce qu'il avait si sagement prévu et fait se trouva
renversé, et l'autorité entière et absolue dévolue à la reine
privativement à tous.

C'était là un grand pas fait, mais l'embarras fut que la disposition
avait été rendue publique, et lue tout haut en présence du roi et de
tous ceux qui ont été nommés, et approuvée et ratifiée de tous. Cette
publicité ne se pouvait détruire que par une autre. Le parlement, qui y
avait été mandé, y avait eu la même part par ses principaux magistrats.
On craignit les mouvements de cette compagnie, et à son appui, le
repentir de Monsieur et de M. le Prince. On voulut donc ménager et
flatter le parlement pour lever tout obstacle. Le dernier exemple
autorisait l'imitation et frayait le chemin. Dès l'après-dînée, car le
roi mourut dans la fin de la matinée, on pratiqua le parlement, on le
brigua toute la nuit, et le lendemain matin, la reine accompagnée de
Monsieur et de M. le Prince, des pairs et des officiers de la couronne,
vint de Saint-Germain droit au parlement. Ils y déclarèrent la cession
qu'ils faisaient à la reine de l'autorité qu'ils avaient reçue de la
disposition du feu roi, pour la lui laisser à elle seule tout entière\,;
que le conseil nommé par le feu roi en faisait de même\,; et la régence
fut ainsi faite et déclarée au parlement, à ces conditions, dont la
France ne s'est pas mieux trouvée, et qui se sentira peut-être encore
longues et cruelles années des pestifères maximes et de l'odieux
gouvernement du cardinal Mazarin.

Deux reines étrangères d'inclination, et de principes fort éloignées des
maximes françaises pour le gouvernement de l'État et des vues si saines
des rois leurs maris, dont elles ne regardèrent la perte que par le seul
objet de leur grandeur personnelle, dont elles étaient de longue main
tout occupées, que la dernière à la vérité n'a due au moins qu'à la
nature, Marie dominée par Conchine\footnote{Concini, plus connu sous le
  nom de maréchal d'Ancre.} et sa femme, Anne par Mazarin, Italiens de
la dernière bassesse, et qui ignoraient jusqu'à notre langue, qui ne
soupiraient qu'après le timon de l'État dont ils se saisirent tout
aussitôt, et à qui il n'importait comment ni à quel titre, il n'est pas
surprenant que méprisant ce qu'ils ignoraient, c'est-à-dire toutes les
formes, les usages, les règles, les droits, ils se soient jetés à corps
perdu avec leur reine, à ce qui leur sembla assurer davantage l'autorité
qui allait faire le fondement certain de la toute-puissance qu'ils
s'étaient bien promis de saisir, surtout avec les raisons qu'on a vues
dans la première de s'assurer du parlement, et dans l'autre de le
ménager.

M. le duc d'Orléans ne se trouvait pas en ces termes. Rien à couvrir par
les ténèbres\,; ni fils de France ni prince du sang avec qui lutter,
point d'indignes et de vils étrangers à faire régner\,; point de
faiblesse de sexe à étayer, nul usage utile à faire de l'appui du
parlement, et tout au contraire, à en craindre par les noirs artifices
du duc du Maine, et les manèges de son premier président appuyés des
dispositions du roi et de l'intérêt du parlement à s'arroger la fonction
de modérateur et de juge, de nourrir la division, de semer les occasions
de s'y faire valoir, et d'usurper cette autorité de tuteurs des rois si
destituée de tout fondement, et, tant qu'ils ont pu, si hardiment
tentée, sur laquelle on verra dans la suite jusqu'à quel point ils
osèrent la porter, faire repentir le régent de sa mollesse, et le forcer
à briser périlleusement sur leur tête le joug que peu à peu il s'était
laissé imposer Je le fis souvenir de ce que tous nos rois, jusqu'à Louis
XIV inclusivement, avaient montré de fermeté toutes les fois que le
parlement avait osé vouloir passer ses bornes du jugement des procès et
des enregistrements d'édits et d'ordonnances, et leur avaient déclaré
que la connaissance de rien de ce qui était au delà n'était de leur
compétence.

Je lui remis cette vérité, dont jusqu'à présent le parlement n'a osé
disconvenir, que s'il est arrivé quelquefois que des matières plus
hautes que les procès des particuliers, ou des enregistrements qui
avaient quelque chose de plus que l'\emph{ut notum sit} pour y conformer
les jugements, avaient été traités au parlement par la volonté ou la
permission du roi, c'était sa présence et des grands qui l'y
accompagnaient, ou, en son absence, celle des pairs qui y étaient mandés
par le roi, qui donnait toute la force, à l'ombre desquels les
magistrats du parlement y opinaient\,; chose tellement certaine, que
leur présence a toujours été nécessairement énoncée dans l'arrêt qui s'y
rendait, par ces termes consacrés\,: \emph{la cour suffisamment garnie
de pairs}, si essentielle au jugement même du parlement que toutes les
fois qu'il y a eu des troubles où le parlement s'était laissé entraîner,
comme sous la dernière régence, il ne s'était point fait de délibération
au parlement, concernant ces affaires, que le parlement lui-même
n'envoyât prier les pairs, et quelquefois même les officiers de la
couronne qui se trouvaient à Paris, d'y venir assister. Il résulte de
cette vérité que ceux qui ne peuvent connaître d'aucune matière d'État,
et de leur propre aveu, sans la présence des pairs qui leur en
communique la faculté (on parle ici de l'usage reçu, non du droit que
les magistrats auraient peine à prouver), ne sont pas nécessaires à
aucune sorte de délibération ni de sanction d'État, et que ceux-là seuls
de la présence desquels ils tirent cette faculté, qu'ils conviennent
n'avoir point en leur absence, peuvent en tout droit délibérer sans eux,
et faire toute sanction d'État.

L'unique objet qui se pourrait faire pour éblouir, mais sans aucune
solidité, c'est que les matières et les sanctions d'État s'étant souvent
trouvées mêlées de jurisprudence et de matières légales, comme les
confiscations des grands fiefs, leur réunion à la couronne par
forfaiture, comme il est arrivé des anciennes pairies possédées par les
rois d'Angleterre et par l'empereur Charles-Quint, ces matières avaient
été traitées au parlement pour en éclairer les pairs, le roi même, et
les officiers de la couronne qui l'y accompagnaient, ce qui, ayant
ouvert la bouche aux magistrats du parlement pour opiner sur ces
matières, leur en avait donné l'usage en d'autres moins mêlées des lois,
lorsque le roi y avait fait assembler les pairs pour les y traiter comme
en lieu naturellement public\,: mais cette réponse telle qu'elle puisse
être ne répond pas au principe dont le parlement convient, et ne lui
donne pas un caractère qu'il n'a pas par lui-même\,; il reste toujours
vrai qu'il n'est admis à délibérer sur ces matières que par la présence
des pairs, que leur absence l'en rend incompétent\,: donc il en est par
soi-même incapable, et les pairs seuls et les officiers de la couronne
uniquement capables et compétents par eux-mêmes, d'où il se conclut
qu'il n'est nul besoin du parlement pour faire ou déclarer une régence,
comme il n'a pas été question de cette compagnie pour aucune des
régences qui depuis tous les temps ont précédé celle de la minorité de
Louis XIII, et qu'elles ne se doivent faire et déclarer que par les
pairs nés, autres pairs, et les officiers de la couronne privativement à
qui que ce soit.

Que si les rois ont été au parlement déclarer leur majorité, ou étant
majeurs aussitôt après leur avènement à la couronne, cet ancien usage
n'a rien de commun avec ce qui vient d'être dit sur les régences. Une
longue prescription fondée sur la sagesse et le bien de l'État à
prévenir les troubles qui, dans l'étourdissement que cause toujours la
mort d'un roi, naîtraient aisément des prétentions à la régence, en a
établi le droit au plus proche du sang du roi mineur mâle ou femelle,
encore que celles-ci soient exclues de la couronne, mais cela même rend
témoignage que la régence n'est pas comme la couronne, et qu'elle était
déférée par l'avis des grands qui renfermait un jugement\,: au lieu que
la séance du roi au parlement, dès qu'il est parvenu majeur à la
couronne, ou pour y déclarer sa majorité s'il était mineur, n'a pour
objet aucun jugement à rendre ni réel, ni fictif, comme est l'objet de
faire et de déclarer une régence, parce que la faire était un jugement
réel autrefois, dont on retient l'image\,; et la déclaration, déclarer
le jugement rendu de l'adjudication de la régence.

Cette première séance du roi au parlement, soit majeur en succédant à la
couronne, soit mineur qui y vient déclarer sa majorité, n'est donc autre
chose que de venir au lieu public, et le plus solennellement destiné à
rendre à ses sujets la justice en son nom, pour y faire publiquement et
solennellement sa fonction de juge unique et suprême de tous ses sujets,
de qui émane le pouvoir de juger à tous les divers degrés de
juridictions, et de juge de son suprême fief, qui est son royaume, à
cause de sa couronne et de son caractère royal qui est unique en sa
personne. Cette séance, où assistent les pairs et où le roi est suivi
des officiers de la couronne, n'est donc en soi qu'une pure cérémonie
sans délibération sur rien par elle-même, ni matière aucune de jugement.
Le roi y reçoit les hommages de la personne qui a exercé la régence, et
qui lui remet toute l'autorité que sa minorité l'empêchait d'exercer par
lui-même, offre de lui rendre compte de l'administration qu'elle a eue
entre les mains, quand il lui plaira de le recevoir, si c'est un roi
mineur qui déclare sa majorité, puis les hommages collectifs de tous.
Que si, à cette occasion, il se met quelque matière en délibération
fictive ou effective, cela retombe dans les cas qui viennent d'être
dissertés, et ne tient que par hasard à la cérémonie.

Je fis observer à M. le duc d'Orléans la jalousie, l'attention toujours
vigilante du parlement à prétendre, à entreprendre, et à créer à son
avantage quelque chose de rien par ce qui arriva à la majorité de
Charles IX. Il ne s'y agissait pas, comme dans les autres, d'une simple
cérémonie telle qu'elle vient d'être expliquée. La loi faite par Charles
V pour la fixation de l'âge de la majorité des rois, et par les grands
qui l'approuvèrent, avait toujours été entendue et pratiquée suivant son
sens naturel de quatorze ans accomplis, quoique le terme
\emph{accomplis} n'y fût pas exprimé. Sans allonger ce récit de ce que
personne n'ignore de l'histoire de ces temps difficiles, Catherine de
Médicis, bien assurée de gouverner toujours, avait intérêt que la
minorité de Charles IX finît, et il était encore éloigné de plusieurs
mois des quatorze ans accomplis. Elle voulut donc faire interpréter la
loi de Charles V à quatorze ans commencés. La cour était en Normandie,
et les affaires ne lui permettaient pas de la quitter. Elle mena donc
Charles IX, suivi des pairs et des officiers de la couronne qui s'y
trouvèrent, au parlement de Rouen, où la loi de Charles V fut
interprétée comme elle le désirait, et Charles IX déclaré majeur, ce qui
pour l'âge a été suivi en toutes les majorités depuis. Le parlement de
Paris jeta les hauts cris, députa vers le roi et la reine, prétendit
qu'un tel acte ne pouvait être fait dans un autre parlement. On se moqua
d'eux. La reine leur répondit que la cour des pairs n'était aucun
parlement, mais le lieu tel qu'il fût où le roi se trouvait, et où il
lui plaisait d'assembler les pairs. La maxime est si vraie que, sans la
circonstance de ces temps si difficiles, où la reine avait besoin de
tout, elle n'avait que faire du parlement de Rouen pour une
interprétation de la loi de Charles V, sur laquelle ce parlement ne put
opiner que par la présence des pairs, comme il a été expliqué, lesquels
seuls la pouvaient faire avec les officiers de la couronne\,; mais comme
il fallait en même temps déclarer le roi majeur qui est la simple
cérémonie qui a été expliquée, qui ne se pouvait faire qu'au parlement
de Rouen, puisque le roi était en cette ville, ce fut un véhicule pour y
faire le tout ensemble. Le parlement de Paris se plaignit longtemps,
sans pouvoir alléguer aucune raison, et il se tut enfin, quand il fut
las de se plaindre, sans avoir reçu le moindre compliment.

Fondé sur des vérités si certaines et de si solides raisons, je proposai
à M. le duc d'Orléans d'assembler tous les pairs et les officiers de la
couronne, aussitôt que le roi serait mort, dans une des pièces de
l'appartement de Sa Majesté en rang et en séance, avec M. le Duc, le
seul des princes du sang en âge, le duc du Maine et le comte de
Toulouse. Que là tous assis et couverts seuls, dans la pièce, avec les
trois secrétaires d'État au bas bout et derrière la séance vis-à-vis de
lui, ayant une table garnie devant eux, car le chancelier était le
quatrième, Son Altesse Royale fit un court discours de louange et de
regrets du roi, de la nécessité urgente d'une administration, de son
droit à la régence qui ne pouvait être contesté, du soin qu'il aurait
d'éclairer ses bonnes intentions par leurs lumières\,; et subitement les
regarder tous en leur disant avec un air de confiance, mais
d'autorité\,: «\,Je ne soupçonne pas qu'aucun de vous s'y oppose\,;» se
lever, gracieuser un chacun, les convier de se trouver l'après-dînée au
parlement, et si le roi mourait le soir, ne faire cette assemblée que le
lendemain matin, pour ne laisser pas la nuit au duc du Maine à cabaler
le parlement, et au premier président d'y haranguer. Arrivé droit au
parlement, lui dire qu'il voulait par l'estime qu'il avait pour la
compagnie, sans rien de plus, leur venir faire part lui-même et se
condouloir avec eux de la perte que la France venait de faire, et de la
régence qui lui échéait par le droit de sa naissance, et les assurer du
soin qu'il aurait de se faire éclairer de leurs lumières dans les
besoins qu'il en aurait\,; que, pour commencer à leur témoigner le désir
qu'il en avait, il leur communiquait le plan qu'il estimait le meilleur
après M. le duc de Bourgogne, dans la cassette duquel il avait été
trouvé, et déclarer là les conseils sans nommer personne. Abréger
matière, et finir la séance.

Comme la régence était faite et déclarée avant que d'y entrer, les gens
du roi n'auraient point eu à parler, ni le parlement à opiner ni rendre
d'arrêt. Si M. du Maine s'était mis en devoir de parler, l'interrompre
et lui dire que c'était à lui moins qu'à personne à vouloir contredire
ce qui s'était fait comme dans toutes les régences précédentes à celle
des deux dernières reines, dont le cas particulier de chacune d'elles
demandait la forme qu'elles avaient prise, qu'elle était trop nouvelle
et trop différente de celle de tous les temps pour avoir la force de la
changer par ces deux seuls exemples, et qu'après toutes les choses
inouïes qu'il avait obtenues, il devait éviter avec soin de parler de ce
qui était de règle, comme de ce qui n'y était pas, et sans attendre de
réponse, lever la séance. Si le premier président avait voulu parler sur
la même chose, l'interrompre pareillement, lui dire qu'il marquerait
toujours au parlement toute l'estime et la considération qu'il méritait,
mais qu'il ne croirait jamais que l'équité et la sagesse de la compagnie
exigeât que ce fût aux dépens des droits de sa naissance et de ceux à
qui il s'était adressé, ni qu'elle pût prétendre que deux exemples
uniques et modernes prescrivissent une règle ignorée jusque-là de toute
l'antiquité, et pareillement lever la séance\,; en se levant, passer les
yeux sur tout le monde, et se faire suivre par tous les pairs,
intéressés ainsi que les officiers de la couronne à soutenir ce qui
s'était passé avec eux. Si le roi avait fait des dispositions, ajouter
qu'il aurait toujours tout le respect pour la mémoire du roi, et tous
les égards qu'il lui serait possible pour ses volontés, mais que tous
les siècles apprenaient que toute l'autorité personnelle des rois
finissait avec eux, qu'ils n'en ont aucune sur une régence dont personne
ne peut prendre prétexte par sa naissance de partager l'autorité\,; que
ce serait manquer à ce qu'il se doit à soi-même de souffrir que son
honneur, sa fidélité pour la personne du roi, son attachement au bien de
l'État demeurassent soupçonnés, et par son propre aveu, en se soumettant
à des dispositions inspirées par l'ambition de qui avait voulu profiter
de la faiblesse de l'âge et des approches de la mort\,; que les
dispositions si sages et si utiles de Charles V et de Louis XIII
n'avaient eu aucun effet\,; que celles de Louis XIV, qui était bien
éloigné des circonstances qui avaient porté ces deux grands rois à les
faire, ne pouvaient donc être plus recommandables que les leurs, ni
avoir un sort plus consistant\,; qu'en un mot, celles de ces deux
princes n'allaient qu'à maintenir le bon ordre et le repos de l'État\,;
que celles du roi n'y pourraient mettre que du trouble, dont il n'est
pas juste que l'État soit menacé ni travaillé pour l'ambition
particulière de quelques-uns, et pour exécuter aveuglément les dernières
volontés du roi en matière d'État, quand celles de pas un de ses
nombreux prédécesseurs qui en avaient laissé n'avaient jamais été
considérées un seul moment, et étaient tombées avec eux. Cela dit, lever
la séance.

Je représentai à M. le duc d'Orléans que s'il avait affaire à un duc de
Guise pour l'ambition, le duc du Maine n'avait ni le parti ni les
soutiens étrangers, ni le personnel des Guise\,; que c'était un homme
timide à qui il fallait imposer et à son premier président tout
d'abord\,; que cela seul les ferait trembler, et que dans le très peu de
gens sur lesquels ce fantôme de Guise se flattait de pouvoir compter
dans le décri où était sa personne, et l'indignation publique de tout ce
à quoi il était parvenu, il n'y en aurait aucun qui, sur un appui aussi
odieux et aussi frêle, osât lever la tête contre un régent unique en sa
naissance, dont la valeur était connue, et qui savait montrer le courage
d'esprit que je lui conseillais, et la fermeté qui serait son salut, et
qui fonderait sa gloire et son autorité entière et paisible pour tout le
cours de sa régence. Que le parlement, adroit à se prévaloir de tout,
mais n'ayant personne pour soi par l'intérêt des pairs et des officiers
de la couronne, qui se trouveraient engagés d'honneur par ce qui se
serait passé le matin avec eux sur la régence à Versailles, sentirait
promptement son impuissance et l'embarras du fond et de la forme\,: du
fond, d'ériger en loi, lui tout seul, deux exemples récents contraires à
tous ceux qui les avaient précédés, et deux exemples singuliers par
leurs circonstances et les conjonctures, et de se roidir à faire passer
en règle les dispositions de Louis XIV odieuses par elles-mêmes, contre
l'exemple constant de toutes les autres dispositions pareilles, dont pas
une n'avait eu le moindre effet, quoique si sages et si nécessaires\,;
de la forme, par leur incompétence, reconnue par eux-mêmes, de
délibérer, encore moins de statuer rien en matière d'État qu'avec les
pairs et par leur présence et concours, et mandés pour ce par le roi, ou
en minorité par le régent\,; et si dans des temps de trouble le
parlement entraîné contre la cour avait quelquefois voulu entreprendre
de se mêler d'affaires d'État ou de gouvernement, ce n'avait jamais été
qu'au moyen et à l'ombre de la présence des pairs, et quelquefois des
officiers de la couronne qu'il envoyait convier d'y venir prendre leurs
places, chose qui n'était pas à craindre en cette occasion, par
l'intérêt des pairs, et des officiers de la couronne de ne se prêter pas
au dessein de détruire leur droit autant qu'il était en eux, et leur
ouvrage, pour soumettre l'un et l'autre aux magistrats qui n'en avaient
aucun\,; que, pour quelques-uns d'eux qui, en très petit nombre, se
trouveraient nommés dans les dispositions, la jalousie du grand nombre
qui n'y aurait point de part l'empêcherait de se prêter à soutenir cette
disposition et les entreprises du parlement contre eux-mêmes, encore
moins quand la déclaration des conseils, sans nommer personne, leur
montrerait un bien plus grand nombre de places considérables à remplir,
et à y succéder par vacance, que les dispositions du roi n'en auraient
établi, dont l'espérance encore les retiendrait tous, et le choix
achèverait de les attacher à lui. Enfin que je m'attendais bien aux
plaintes du parlement, mais qu'elles seraient si semblables à celles
qu'il fit sur la majorité de Charles IX et l'interprétation de la loi de
Charles V faite au parlement de Rouen, que je comptais aussi que l'effet
et la fin en serait toute pareille, ce qui diminuerait d'autant le nom,
le crédit, l'autorité du parlement, à l'augmentation du pouvoir du
régent, et rendrait cette ardente compagnie d'autant plus retenue à
entreprendre.

J'ajoutai un détail des pairs et des officiers de la couronne qui le
devait bien rassurer, outre l'esprit qui régnait alors si peu favorable
aux bâtards, par conséquent aux dispositions que le roi ne pourrait
avoir faites qu'en leur faveur. Je fus d'avis que sur tout ce qui ne
toucherait ni l'État ni le gouvernement en aucune sorte, M. le duc
d'Orléans se fit honneur d'en faire un entier à ces mêmes dispositions
du roi, non pas comme faisant loi et par nécessité de les suivre, mais
par un respect volontaire et bienséant, par sa propre autorité à lui, et
pour s'éloigner de la bassesse de porter des coups au lion mort. Par la
même raison, je fus d'avis que M\textsuperscript{me} de Maintenon jouît
pleinement, et son Saint-Cyr, de tout ce que ces dispositions auraient
fait en leur faveur, et que s'il n'y en avait point, que toute liberté
lui fût laissée de se retirer où elle voudrait, et que rien de
pécuniaire qu'elle désirerait ne lui fût refusé. Il n'y avait plus rien
à craindre de cette fée presque octogénaire\,; sa puissante et
pernicieuse baguette était brisée, elle était redevenue la vieille
Scarron. Mais je crus aussi qu'excepté liberté et le pécuniaire
personnel, tout crédit et toute sorte de considération lui devaient être
soigneusement ôtés et refusés. Elle avait mérité bien pis de l'État et
de M. le duc d'Orléans.

Parmi ces mesures, je n'oubliai pas celles que, dispositions du roi
faites ou non, la prudence devait inspirer. C'était de s'assurer du
régiment des gardes, ce qui était fort aisé avec le duc de Guiche pour
de l'argent. Contade, qui le gouvernait et qui de plus était fort
accrédité dans le régiment, était honnête homme et bien intentionné, et
depuis longtemps je m'étais attaché à gagner Villars qui n'était qu'un
avec Contade, et qui avait son crédit personnel sur le duc de Guiche.
J'ai déjà parlé de ces deux hommes. S'assurer de Reynold, colonel du
régiment des gardes suisses, le premier et le plus accrédité de ce corps
et qui le menait, fort homme d'honneur et peu content en secret du joug
du duc du Maine\,; s'attacher Saint-Hilaire, qui pour l'artillerie était
au même point que Reynold dans les Suisses\,; et ne pas négliger
d'Argenson. Tout cela fut fait, et avec cela rien à craindre dans Paris,
ni du parlement qui se trouverait environné du régiment des gardes quand
le régent y irait. Rien à faire dans les provinces, où personne n'avait
d'autorité, qui toutes étaient indignées de la grandeur des bâtards et
qui n'oseraient branler. Pour les frontières, du Bourg, qui commandait
en Alsace, était honnête homme, sans liaisons de cour, qui voulait le
bâton de maréchal de France qu'il avait bien mérité, et qui lui
viendrait bien plus naturellement par le régent que par des troubles\,;
ainsi des vues et de la situation des autres principaux des frontières.
Il ne restait donc qu'à avoir du courage, de la suite, du sang-froid, un
air de sécurité, de bonté, mais de fermeté, et à marcher tranquille et
tête levée aussitôt que la mort du roi ouvrirait cette grande scène.

Je m'aperçus aisément que M. le duc d'Orléans était peiné de trouver
tant d'évidence aux raisons dont j'appuyais la proposition que je lui
faisais de se passer du parlement pour sa régence. Il m'interrompit
souvent dans les diverses conversations qui roulèrent là-dessus\,; il
avouait que j'avais raison, mais il ne pouvait ni contester mon avis ni
s'y rendre, quoiqu'il ne le rejetât pas. Il fallait, pour l'embrasser
utilement, plus de nerf, de résolution et de suite que la nature n'en
avait mis en lui, plus savoir payer d'autorité, de droit, d'assurance
par soi-même et sur le pré, et vis-à-vis des gens et sans secours
d'autrui, qu'il n'était en lui de le faire. Je me contentai de lui
inculquer ce que je pensais, et les raisons de se conduire comme je le
pensais, à diverses reprises, sans le presser au delà de ce qu'il en
pouvait porter. Sa défiance, qui n'avait point de bornes, m'arrêta dans
celles-ci. Je crus voir qu'elle venait au secours de sa faiblesse, et
que, pour se la cacher à lui-même, il se persuada que je voulais me
servir de lui en haine du parlement, par rapport à l'affaire du bonnet,
et revendiquer le droit des pairs par rapport à la régence sur
l'usurpation moderne du parlement. L'expérience de ce qui s'y passa sur
sa régence le fit repentir de ses soupçons, et de s'être laissé
entraîner à des gens peu fidèles que sa faiblesse favorisa, et qui le
jetèrent dans le dernier péril de se perdre avant de commencer d'être,
comme on le verra en son lieu. Ces gens étaient Maisons, Effiat, deux
scélérats dévoués au duc du Maine et au parlement\,; Canillac, gouverné
par l'encens de Maisons, devenu par là son oracle\,; peut-être Nocé, par
ignorance, ébloui du nom du parlement.

Nocé était un grand homme, qui avait été fort bien fait, qui avait assez
servi pour sa réputation, qui avait de l'esprit et quelque ornement dans
l'esprit, et de la grâce quand il voulait plaire. Il avait du bien assez
considérablement, et n'était point marié, parce qu'il estimait la
liberté par-dessus toutes choses. Il était fort connu de M. le duc
d'Orléans, parce qu'il était fils de Fontenay, qui avait été son
sous-gouverneur, et il lui avait plu par la haine de toute contrainte,
par sa philosophie tout épicurienne, par une brusquerie qui, quand elle
n'allait pas à la brutalité, ce qui arrivait assez souvent, était
quelquefois plaisante sous le masque de franchise et de liberté\,;
d'ailleurs un assez honnête mondain, pourtant fort particulier. Il était
fort éloigné de s'accommoder de tout le monde, fort paresseux, ne se
gênait pour rien, ne se refusait rien. Le climat, les saisons, les
morceaux rares qui ne se trouvaient qu'en certains temps et en certaines
provinces, les sociétés qui lui plaisaient, quelquefois une maîtresse ou
la salubrité de l'air l'attiraient ici et là, et l'y retenaient des
années et quelquefois davantage. D'ailleurs poli, voulait demeurer à sa
place, ne se souciait de rien que de quelque argent, sans être trop
avide, pour jeter librement à toutes ses fantaisies, dont il était plein
en tout genre, et à pas une desquelles il ne résista jamais. Tout cela
plaisait à M. le duc d'Orléans, et lui en avait acquis l'amitié et la
considération. C'était un de ceux qu'il voyait toutes les fois qu'il
allait à Paris, quand Nocé y était lui-même, avec lesquels tous je
n'avais ni liaison ni connaissance, parce que je ne voyais jamais M. le
duc d'Orléans à Paris, et que ces personnes-là ne venaient jamais à
Versailles. Depuis la régence, je n'eus guère plus de commerce avec eux.
Leur partage était les soupers et les amusements du régent, le mien les
affaires, sans aucun mélange avec ses plaisirs.

\hypertarget{chapitre-xi.}{%
\chapter{CHAPITRE XI.}\label{chapitre-xi.}}

1715

~

{\textsc{Survivances, brevets de retenue et charges à rembourser.}}
{\textsc{- Raisons et moyen de le faire, et multiplication de
récompenses à procurer.}} {\textsc{- Taxe proposée n'a rien de contraire
à la convocation des états généraux, qui lui est favorable.}} {\textsc{-
Autres remboursements peu à peu dans la suite.}} {\textsc{- Nulle grâce
expectative.}} {\textsc{- Remplir subitement les vacances.}} {\textsc{-
Réparation des chemins par les troupes.}} {\textsc{- Extérieur du roi à
imiter, et fort utile\,; et conduite personnelle.}}

~

J'avais depuis fort longtemps une idée dans la tête que je voulus
examiner, et voir si elle était possible, lorsque je commençai à
m'apercevoir de la diminution de la santé du roi. Je fis sur cela un
travail à la Ferté, où je m'aidai de gens plus propres que moi au
calcul, sans leur communiquer à quoi il tendait, et je connus qu'il y
avait de l'étoffe. Voici quelle elle était\,: je voulais rendre M. le
duc d'Orléans maître de toutes les principales charges de la cour, à
mesure qu'elles viendraient à vaquer, et d'autres dont je parlerai
après, et lui donner auprès du roi l'honneur de les lui faire trouver
libres à sa majorité. Il n'y en avait presque plus qui ne fussent en
survivance ou chargées de gros brevets de retenue qui tendaient au même
effet. Par ce moyen elles étaient rendues héréditaires. Qui n'en avait
point n'en pouvait espérer, le roi n'avait rien à disposer. Les fils
succédant aux pères obtenaient sûrement, ou sur-le-champ ou tôt après,
le même brevet de retenue\,; et si, par un hasard d'une fois en vingt
ans, il s'en trouvait une à disposer, c'était en payant le brevet de
retenue par le successeur, qui alors en obtenait sur-le-champ un pareil.
Cette grâce lui faisait bien trouver la somme entière du prix de la
charge, mais les arrérages de cet emprunt étaient au moins égaux aux
appointements de la charge, en sorte qu'il la faisait à ses dépens et
s'y ruinait souvent. Je voulais donc payer tous ces brevets de retenue.
C'eût été une grâce inespérée pour ceux qui en avaient que cela eût
libérés du fonds hypothéqué dessus, et leur eût laissé libre et en gain
la jouissance de leurs appointements.

Tout le gré de tant de gens considérables en eût été à M. le duc
d'Orléans, qui, dans le cours de sa régence, aurait eu le choix libre
pour remplir les vacances, et l'aurait remis au roi à sa majorité. Mais
aussi la condition essentielle était de se faire une loi immuable de ne
donner jamais ni survivances ni brevets de retenue pour quelque raison
que ce pût être. Chacun alors aurait espéré et se serait conduit de
façon à fortifier son espérance, et on aurait banni l'indécence de voir
des enfants exercer les premières charges, et de jeunes gens gorgés les
déshonorer par leur conduite, fondée sur une situation brillante qui ne
peut leur manquer, et qui ne leur laisse ni crainte de perdre ni désir
d'obtenir. Or les hommes se mènent presque tous beaucoup mieux par
l'espérance et par la dépendance que par la reconnaissance et par
d'autres égards, ce qui rendait ce remboursement beau coup plus utile
encore à un régent, qui par là acquérait l'un et l'autre.

J'en voulais faire autant et par mêmes raisons, pour les gouvernements
de province dont l'objet n'était pas fort, non plus que leurs
lieutenances générales que j'avais encore plus à cœur. Voici ma raison
d'affection particulière. Le nombre d'officiers généraux était devenu
excessif dans ces guerres continuelles, par cette détestable méthode de
faire de nombreuses promotions par l'ordre du tableau. En même temps
presque point de récompenses\,; en sorte qu'on a vu des maréchaux de
camp et force brigadiers demander, accepter avec joie, et n'obtenir pas
toujours des emplois dont, avant cette foule, les commandants de
bataillons des vieux corps se croyaient mal récompensés. Un gouvernement
de place de quinze ou seize mille livres de rente à tout tirer,
ordinairement à résidence, est tout ce qu'un bon et ancien lieutenant
général peut espérer. Les gouvernements bons et médiocres ne sont pas en
très grand nombre\,; de sorte que beaucoup de lieutenants généraux
attendent longtemps, et que plusieurs n'en ont jamais, et c'est pourtant
tout ce qu'ils peuvent espérer. Les grands-croix de Saint-Louis sont en
très petit nombre, et quelque prostitution qu'il se soit faite des
colliers de l'ordre du Saint-Esprit, ils sont rares pour ces
récompenses, et ne donnent pas de subsistance. Je voulais donc affecter
toutes les lieutenances générales des provinces à la récompense des
lieutenants généraux, et les lieutenances de roi des provinces aux
maréchaux de camp, ce qui, avec les gouvernements de places qui leur en
servent jusqu'à cette heure, fournirait à tous, en observant que le même
n'eût jamais l'un et l'autre. Rien de plus naturel, de plus convenable,
ni de plus utile au vrai service du roi et à celui des provinces que
cette sorte de récompense qui laisserait les très petits gouvernements
de places et de forts, et tous les états-majors des places, aux
brigadiers et à ce grand nombre d'officiers si dignes de récompense. Je
voulais que ces lieutenants généraux et ces lieutenants de roi des
provinces en fissent les fonctions, et remettre ainsi l'épée en lustre
et en autorité, en bridant et humiliant les intendants des provinces,
et, cette foule de trésoriers de France, d'élus\footnote{On appelait
  \emph{élus}, des magistrats qui jugeaient en première instance les
  procès relatifs à l'assiette des tailles et autres impôts. Leur nom
  venait de ce que primitivement ils avaient été élus par l'assemblée
  des états généraux, en 1357.}, de petits juges, de gens de rien,
enrichis et enorgueillis, qui sous les intendants sont les tyrans des
provinces, le marteau continuel de la noblesse, et le fléau du peuple
qu'ils dévorent.

Rien de si indécent que la manière dont ces lieutenances générales et de
roi des provinces se trouvaient remplies. Les premières étaient devenues
le patrimoine des possesseurs\,; c'étaient souvent des enfants, presque
toujours des personnes aussi ineptes. Les autres héréditaires par l'édit
assez nouveau de leur création n'étaient presque remplies que de gens
qui n'étaient pas ou bien à peine gentilshommes, et qui pour leur argent
avaient couru après ce petit titre pour se recrépir. Rembourser les uns
et les autres, c'était ôter des images la plupart ridicules, pour leur
substituer mérite, valeur, âge, maintien, usage de commander, en même
temps se dévouer tout le militaire par une telle et si nombreuse
destination de récompenses. Le moyen était par une taxe sourde aux gens
d'affaires. L'expérience doit avoir dégoûté des chambres de justice.
L'argent et la protection y sauvent tous les gros richards qui ne se
sont pas rendus absolument odieux, et de ceux-là encore il s'en tire
beaucoup d'affaire. On les vexe pour enrichir le protecteur\,; les
alliances que la misère des gens de qualité leur a fait faire avec eux
en délivrent encore un grand nombre\,; les médiocres financiers ont
aussi leurs ressources pour échapper\,; les taxes, faites pour la forme,
obtiennent des remises et des modérations\,; en un mot beaucoup de bruit
qui perd le crédit dont on a besoin tant que la finance demeure sur le
pied où elle est\,; grands frais que le roi paye\,; force grâces à
droite et à gauche aux dépens des malheureux\,; au bout nul profit pour
le roi, ou si mince qu'on est honteux de l'avouer. Au lieu d'une si
ruineuse méthode, parler à l'oreille à ces gens-là, leur dire qu'on ne
veut ni les décréditer, ni les tourmenter, ni mettre leurs affaires au
jour, mais qu'on n'est pas aveugle aussi sur leurs gains excessifs,
qu'il est raisonnable qu'ils en aident le roi, et qu'ils ne se
commettent pas à un traitement rigoureux, au lieu du gré qu'ils
acquerront à faire les choses de bonne grâce, et se prépareront les
voies à remplir une partie du vide qu'ils s'imposeront\,; les assurer
que ce qu'on leur demande demeurera secret, pour ne pas intéresser leur
crédit et leur réputation\,; leur faire à chacun des propositions
modérées et proportionnées à ce que l'on peut raisonnablement savoir de
leurs profits\,; leur répartir les brevets de retenue et les
lieutenances générales des provinces par lots, suivant ce qu'on serait
convenu avec eux, et le temps court pour apporter les démissions et les
quittances\,; et si quelques-uns d'eux faisaient les insolents, les
traiter militairement, de Turc à More, et subitement sans merci pour
donner exemple aux autres.

À l'égard de ceux qui sont revêtus de ces emplois, dont il se trouverait
quelques-uns à conserver jusqu'à vacance, leur parler civilement, mais
en leur montrant qu'on veut être obéi. Pour les lieutenances de roi, où
il y en aurait peut-être fort peu à conserver\,: mais en leur déclarant
qu'il n'y a plus d'hérédité, la plupart se trouveraient de telle espèce
qu'il n'y aurait pas grande différence entre elles et les charges
municipales créées de même, et qui ont été supprimées aux dernières
paix, et point ou très peu remboursées. Quelle comparaison entre le
mécontentement des remboursés et des supprimés de ces charges, et
l'acclamation de toutes les troupes que M. le duc d'Orléans se
dévouerait par la réalité et par l'espérance de cette multiplication de
belles récompenses, depuis le premier lieutenant général jusqu'au
dernier enseigne et cornette, parce que ce grand nombre de différentes
récompenses déboucherait bien plus aisément les tètes des corps, et
donnerait de justes espérances à la queue de monter plus tôt, et
d'arriver\,; et quelle sûreté et quelle facilité dans tout le cours de
la régence\,; et quelle considération après recueillerait ce prince de
s'être ainsi attaché toute la cour et tout le militaire de tout grade,
et de les avoir mis de plus dans sa dépendance par ces solides
espérances\,! Je dis jusqu'au dernier cornette\,: en voici la raison.

En proposant à M. le duc d'Orléans tout ce qui vient d'être expliqué
dans cet article, je lui fis considérer que toutes les récompenses
au-dessous des officiers généraux n'étaient que pour l'infanterie qui
est le nerf de l'État, et ne devaient aussi aller qu'à elle, parce que
la cavalerie n'entend point les places\,; qu'en même temps la cavalerie
était aussi trop maltraitée depuis que les extrêmes besoins avaient
engagé à retrancher les bons quartiers d'hiver et mille autres
revenants-bons qui n'étaient pas de règle, mais sur lesquels M. de
Louvois, et son fils après lui, fermaient les yeux pour un bien-être
nécessaire à entretenir de belle cavalerie, et à suppléer aux
récompenses dont les officiers sont privés en se retirant presque tous,
parce qu'elles ne consistent qu'en pensions rares et modiques, et que ce
moyen n'était pas onéreux, comme eût été d'en augmenter le pied. Ainsi
je proposai à M. le duc d'Orléans de se faire une règle inaltérable de
borner les officiers d'infanterie aux états-majors que les officiers
supérieurs ne leur embleraient plus, et à la plus modique portion qu'il
se pourrait de grâces sur l'ordre de Saint-Louis, d'en affecter toutes
les autres à la cavalerie et aux dragons, et toutes les pensions de
retraite que le roi se trouverait en état de donner, sans plus aucune à
l'infanterie, au moyen de quoi il empêcherait par cette étoffe et par
cette espérance la tête de ces régiments de quitter par ennui, par
dégoût, par craindre d'achever de se ruiner, inconvénient qui renouvelle
sans cesse ces corps, et qui les dépouille d'officiers expérimentés et
capables.

En même temps je le pressai de songer, autant que les finances le
pourraient porter, au rétablissement de la marine, d'où dépend en un
royaume flanqué des deux mers toute la sûreté et la prospérité de son
commerce et de ses colonies, qui est la source de l'abondance\,; objet
dont la nécessité et l'importance augmente à mesure que la longue paix
intérieure de l'Angleterre, paix inouïe jusqu'ici depuis la durée de
cette monarchie, l'a mise en état de couvrir toutes les mers de ses
vaisseaux, et d'y donner la loi à toutes les autres puissances, tandis
qu'il a été un temps où le roi a disputé l'empire de la mer à
l'Angleterre et à la Hollande unies contre lui, et y a eu des succès et
des victoires. Par cette même raison, augmenter l'émulation, en ne
souffrant plus à l'avenir que les vice-amiraux devenant maréchaux de
France conservassent leur vice-amirauté, puisqu'ils se trouvaient
revêtus du premier grade militaire qui commandait à tous, par quoi ce
dédoublement ferait monter tout le monde\,; et destiner aussi des
récompenses, dont la marine est presque totalement privée, en lui
affectant le gouvernement de tous les ports, et tous leurs états-majors,
ce qui éviterait de plus mille inconvénients pour le service, et des
tracasseries sans fin entre les officiers de terre et de mer.

Revenant après sur mes pas à la taxe, je dis à M. le duc d'Orléans que
cette entreprise n'avait rien de contraire à ma proposition d'assembler
les états généraux, parce que leur convocation n'était faite que pour
rendre publique la situation forcée où il trouvait les finances, et leur
donner le choix des remèdes et de l'ordre qu'ils seraient d'avis d'y
apporter. Que, quelque taxe qu'on se pût proposer par une chambre de
justice, ou par toute autre voie, elle ne pouvait remplir aucun de ces
deux objets\,; et que celle qu'il ferait ne touchait aussi ni à l'un ni
à l'autre, par quoi il serait toujours vrai de dire aux états qu'il
n'avait fait, en attendant leur assemblée et leur délibération, que
continuer la forme de l'administration qu'il avait trouvée dans les
finances, sans innover en rien, pour leur laisser toutes choses
entières. J'ajoutai que je ne voyais point d'occasion plus favorable de
faire et de presser la taxe telle que je la proposais, qu'au moment de
la première publicité de la convocation des états, pour faire peur aux
financiers d'être abandonnés à leur merci, et les assurer qu'en payant
avant leur première assemblée, ils seraient garantis de leur haine, de
leur vengeance et de tout ce qu'ils avaient tant lieu d'en appréhender,
ce qui serait le plus puissant et le plus pressant véhicule à céder et à
payer promptement. Mon projet pour les suites, dont je fis sentir
l'importance et la convenance à M. le duc d'Orléans, était de trouver
moyen de payer peu à peu tous les régiments de cavalerie, d'infanterie
et de dragons pour en ôter la vénalité à jamais, qui ferme la porte à
tout grade militaire à qui n'y peut atteindre, et en laisserait la libre
disposition au roi. La France est le seul pays du monde où les offices
de la couronne, les charges de la cour et de la guerre, et les
gouvernements soient vénaux\,; les inconvénients de cet usage aussi
pernicieux qu'il est unique sont infinis, et il n'est point immense de
l'abolir. À l'égard des autres sortes de charges, il serait chimérique
de penser sérieusement à en ôter la vénalité, tant cette mer est vaste,
mais bien important de ne perdre pas les occasions de rendre libres les
charges des premiers présidents, et des procureurs généraux des
parlements, chambres des comptes et cours des aides, pour que le roi en
pût disposer librement.

Je n'oubliai pas encore de remontrer à M. le duc d'Orléans avec combien
de raison le roi s'était rendu si difficile sur les coadjutoreries
d'évêchés et d'abbayes, qu'on n'en voyait plus depuis longtemps,
l'inconvénient de l'ambition des parents, et si souvent {[}celui{]} de
la mésintelligence qui se mettait entre les titulaires et les
coadjuteurs\,; je le fis souvenir du juste repentir qu'avait eu le roi
de la complaisance qu'il avait eue de permettre celle de Cluni, et
combien il se devait garder, et le roi, lorsqu'il serait majeur, de
prendre jamais d'engagement avec qui que ce fût pour rien qui ne fût pas
vacant, et combien il était utile tant pour les places de l'Église que
pour toutes les autres, de se former un état de ceux qu'on croit devoir
placer par étages et par classes, afin de pouvoir choisir soi-même le
successeur d'une place dont le titulaire menace une ruine prochaine, ou
dont on apprend la mort, pour n'être pas en proie aux demandeurs, à gens
quelquefois qu'on ne veut pas refuser, et pouvoir disposer sur-le-champ
de la vacance pour donner soi-même, en avoir le gré, et ne se les
laisser pas arracher avec peu ou point de reconnaissance, et encore
moins de choix. Je le fis souvenir du très juste scrupule qui avait
obligé le roi à délivrer de vénalité les charges de ses aumôniers, parce
qu'elles étaient le chemin ouvert aux bénéfices et aux prélatures, et le
soin qu'il devait se prescrire de ne l'y pas laisser rentrer\,; chose,
s'il n'y était exact, qui serait trouvée bien plus mauvaise de lui par
la licence de sa vie jusqu'alors, qui lui ferait mépriser les faubourgs
de la simonie que le roi avait si saintement anéantis.

Je lui parlai aussi de l'affreux état où on avait laissé tomber les
chemins par tout le royaume, tandis que chaque généralité payait de si
grosses sommes pour leur réparation et entretien, et que si on en
employait quelque chose, il en demeurait la moitié dans la poche des
entrepreneurs, qui faisaient encore de très mauvais ouvrages, et qui ne
duraient rien\,; que cet article était de la dernière importance pour le
commerce intérieur du royaume qu'il interceptait totalement en beaucoup
d'endroits, faute de ponts et de chaussées qui manquaient sans nombre,
et qui obligeaient à faire de longs détours, ce qui, joint au nombre
doublé et triplé de chevaux pour traîner les voitures dans les chemins
rompus où elles s'embourbaient et se cassaient continuellement, causait
une triple dépense, qui, sans compter la peine et le travail, dégoûtait
les moins malaisés, et passait les forces de tous les autres\,; que la
Flandre espagnole ou conquise, l'Alsace, la Lorraine, la Franche-Comté,
le Languedoc lui donnaient un exemple qu'il fallait suivre, et qui
méritait qu'il entrât dans la comparaison de l'aisance et du profit qu'y
trouvaient ces provinces, pour leurs commerces de toutes les sortes,
avec le dommage qu'éprouvait tout le reste du royaume. Que pour y
parvenir, il était aisé de répandre en pleine paix les troupes par le
royaume, et de se servir d'elles pour la réparation des chemins\,;
qu'elles y trouveraient un bien-être qui ne coûterait pas le demi-quart
de ce qui s'y dépenserait par tout autre moyen, que les officiers y
veilleraient à un travail assidu, continuel, et toutefois réparti de
façon à ne pas trop fatiguer les troupes\,; que les ingénieurs qu'on
emploierait à visiter ces travaux, et les officiers qui en seraient les
témoins, tiendraient de court les entrepreneurs sur la bonté de
l'ouvrage et la solidité, de même que sur les gains illicites des gens
du métier qui y seraient employés, et sur les friponneries des
secrétaires et des domestiques des intendants, et souvent des intendants
eux-mêmes, sur leurs négligences, leurs préférences, et qu'en quatre
ans, et pour fort peu de chose qui encore tournerait au profit des
troupes, les chemins se trouveraient beaux, bons et durables.

À l'égard des ponts, qu'il n'était pas difficile d'avoir un état de ceux
qui étaient à refaire ou à réparer\,; destiner ce qu'on pourrait pour le
faire peu à peu, commençant par les plus nécessaires, et choisir les
ingénieurs les plus en réputation d'honneur et d'intelligence en
ouvrages, pour se trouver présents avec autorité aux adjudications qui
en seraient faites par les intendants, et tenir de près les
entrepreneurs sur la bonté, la solidité et la diligence des ouvrages
qu'ils auraient entrepris, mais qu'à tout cela il fallait suite et
fermeté, et se résoudre à des châtiments éclatants à quiconque les
mériterait, sans qu'aucune considération les en pût garantir\,; que
c'est à l'impunité qui a porté l'audace au comble qu'il se faut prendre
des voleries immenses qui appauvrissent le roi, ruinent le peuple,
causent mille sortes de désordres partout et enrichissent ceux qui les
font, et beaucoup tête levée, assurés qu'ils sont qu'il n'en sera autre
chose par la protection qu'ils ont, et souvent pécuniaire, ou même par
leur propre considération, et de ce qu'ils sont eux-mêmes\,; et si une
fois en vingt ans il arrive quelque excès si poussé qu'il ne soit pas
possible de n'en pas faire quelque sorte de justice, jamais elle n'a été
plus loin que de déposséder le coupable de l'emploi dont il a abusé,
qui, peu après, se raccroche à un autre, au pis aller demeure oisif, et
jouit de ses larcins sans être recherché de rien de tout ce qu'il a
commis.

Cette méthode, à l'égard des chemins, ôterait de soi-même un autre abus,
qui est multiplié à l'infini, qui est que sur une somme destinée et
touchée effectivement pour tel ou tel chemin, l'homme de crédit qui s'en
trouve à quelque distance, un intendant des finances, un fermier
général, un trésorier de toute espèce, suprêmement les ministres,
détournent ce fonds en partie, quelquefois en total pour leur faire des
chemins, des pavés, des chaussées, des ponts qui ne conduisent qu'à
leurs maisons de campagne, et dans leurs terres, moyennant quoi il ne se
parle plus de la première et utile destination pour le public, et
l'intendant qui y a connivé y trouve une protection sûre, qui le fait
regarder avec distinction par les maîtres de son avancement. Je comptai
à ce propos à M. le duc d'Orléans que c'était ainsi que les puissants de
ce temps-ci, c'est-à-dire de la plume et de la robe, car il n'y en a
plus d'autres, avaient embelli leurs parcs et leurs jardins de pièces
d'eau revêtues de canaux, de conduites d'eau, de terrasses qui avaient
coûté infiniment, et dont ils n'avaient déboursé que quelques pistoles,
et que le roi parlant à M\textsuperscript{me} de La Vrillière dans son
carrosse où était M\textsuperscript{me} la duchesse de Berry et
M\textsuperscript{me} de Saint-Simon, allant à la chasse de Châteauneuf
où elle avait été de Fontainebleau, elle lui en avait vanté la terrasse,
qui est en effet d'une rare beauté sur la Loire\,: «\,Je le crois bien,
répondit sèchement le roi, c'est à mes dépens qu'elle a été faite, et
sur les fonds des ponts et chaussées de ces pays-là pendant bien des
années.\,» J'ajoutai que si l'image d'un secrétaire d'État, car cette
charge n'est pas autre chose, avait osé faire ce trait sans qu'il en ait
rien été, que n'auront pas fait tous les autres secrétaires d'État, et
gens en place considérables dans la robe, dans la plume, et en
sous-ordre, les financiers et les petits tyranneaux que j'ai nommés dans
les provinces\,? Tout cela fut fort goûté et approuvé\,; et il me parut
que M. le duc d'Orléans était résolu à cette exécution.

Je ne manquai pas de le prier de se souvenir combien de fois lui et moi,
tête à tête, nous nous étions échappés à l'envi sur les détails dont le
roi se piquait, qui le persuadaient, aidés de l'adresse, de l'intérêt,
des artifices de ses ministres, qu'il voyait, qu'il faisait, qu'il
gouvernait tout par lui-même, tandis qu'amusé par des bagatelles, il
laissait échapper le grand qui devenait la proie de ses ministres, parce
que le jour n'a que vingt-quatre heures, et que le temps qu'on emploie
au petit, on le perd pour le grand, sur lequel ils le faisaient tomber
insensiblement du côté qu'ils voulaient, chacun dans son tripot. Je lui
dis que, malgré la force de cet exemple et de son propre sentiment, il
devait être en garde continuelle avec lui-même sur l'appât des détails,
qui sont la curiosité, les découvertes, tenir les gens en bride, briller
aisément à ses propres yeux et à ceux des autres par une intelligence
qui perce tant de différentes parties, le plaisir de paraître avec peu
de peine, de sentir qu'on est maître et qu'on n'a qu'à commander, au
lieu que le grand vous commande, oblige aux réflexions, aux
combinaisons, à la recherche et à la conduite des moyens, occupe tout
l'esprit sans l'amuser, et fait sentir l'impuissance de l'autorité qui
humilie au lieu de flatter, et qui bande l'application à la recherche et
à la suite de ce qui peut amener le succès auquel on tend, et fait
sentir les fautes qu'on y a faites et l'inquiétude de les réparer, en
sorte que rien de plus satisfaisant que les détails qui sont tous sous
la main du prince, mais qui ne lui rapportent que du vent, parce qu'ils
sont le partage du subalterne sous ses ordres généraux, qui là-dessus en
sait plus que lui\,; et que rien n'est plus pénible et ne flatte moins
que le travail en grand, du succès duquel dépend la prospérité des
affaires, et la gloire et la réputation du prince qui s'y donne, parce
qu'il ne peut être le partage d'un autre, et qui y réussit. Non qu'il
faille abandonner tous les détails aux autres, mais s'y appliquer et
s'en faire rendre compte, de manière à tenir tout en ordre et en
haleine, sans pourtant s'imaginer que ce soit si parfaitement que rien
n'échappe, parce qu'il ne faut pas se proposer l'impossible, mais y
entrer de façon qu'on n'y donne que très peu d'un temps, court,
précieux, et qui s'enfuit sans cesse, qui doit de préférence être
employé au plus important, et se contenter pour le reste d'une direction
générale, surtout comprendre que ne pouvant suffire à tout, force est de
se fier à ceux qu'on a choisis pour le courant, et souvent bien
davantage\,; que cette confiance excite et pique d'honneur et
d'attachement, au contraire de la défiance qui ne sert qu'à être trompé,
à décourager, à dégoûter, et souvent à se proposer de tromper, puisque
le prince mérite de l'être par son injuste défiance.

Je le conjurai aussi de se défaire absolument de cet esprit de
tracasserie puisé d'enfance dans la cour de Monsieur, entretenu depuis
par l'habitude avec les femmes, et par la fausse idée de découvrir et de
croire être mieux servi en brouillant les uns avec les autres, parce que
pour une fois que cela réussit avec des étourdis, ou par une surprise de
colère, trompe sans cesse le prince par cela même dont il est rendu la
dupe, dès qu'il est reconnu pour user de ce bas artifice qui lui éloigne
et ferme la bouche à ses vrais serviteurs, et lui rend les autres
ennemis. Ce n'est pas qu'il n'y ait mesure à tout, singulièrement entre
l'abandon aux gens et la vigilante défiance. C'est où le sens, la
connaissance des personnes, l'expérience, la suite des choses et des
affaires conduisent l'esprit. Se fermer aux rapports, surtout aux avis
anonymes, c'est-à-dire aux fripons, tenir les yeux ouverts à tout, mais
avec tranquillité, éplucher à part soi des apparences qui se trouvent si
souvent trompeuses\,; si l'examen persuade qu'il y ait cause
d'approfondir, le faire avec précaution et délicatesse\,; être en garde
s'il n'y a rien au bout contre la honte et quelquefois le dépit de
s'être trompé\,; si au contraire il se rencontre infidélité réelle ou
incapacité dangereuse, se défaire sans délai irrémissiblement du sujet,
plus ou moins honnêtement, suivant le mérite de la chose, également pour
se délivrer du danger, et pour servir d'exemple aux autres, car j'y
reviens toujours, nous périssons en tout genre par l'impunité.
J'insistai souvent sur tout ce dernier article, par la connaissance que
j'avais du caractère de M. le duc d'Orléans.

Je lui dis aussi qu'il ne fallait pas moins se souvenir qu'après nous
être souvent licenciés sur les détails du roi dans nos conversations,
nous y étions convenus aussi d'une de ses plus grandes parties, qu'il
fallait bien inspirer à son successeur d'imiter, et à laquelle je
souhaitais passionnément que son image qu'il allait être voulût faire
l'effort de se conformer. Cette partie si utile est la dignité
constante, et la règle continuelle de son extérieur. L'une présentait en
tous les moments qu'il pouvait être vu une décence majestueuse qui
frappait de respect\,; l'autre une suite de jours et d'heures, où, en
quelque lieu qu'il fût, on n'avait qu'à savoir quel jour et quelle heure
il était, pour savoir aussi ce que le roi faisait, sans jamais
d'altération en rien, sinon d'employer les heures qu'il passait dehors,
ou à des chasses, ou à de simples promenades. Il n'est pas croyable
combien cette exactitude en apportait en son service, à l'éclat de sa
cour, à la commodité de la lui faire et de lui parler, si on n'avait que
peu à lui dire, combien de règle à chacun, de commodité au commerce des
uns avec les autres, d'agrément en ces demeures, de facilité et
d'expédition à ses affaires, et à celles de tout le monde, ni combien
son habitation constante hors de Paris faisait d'une part un triage
salutaire et commode, de l'autre un rassemblement continuel qui faisait
tout trouver à chacun sous sa main, et qui faisait plus d'affaires, et
donnait plus d'accès à tous les ministres et à tous leurs bureaux en un
jour, qu'en quinze si la cour était à Paris, par la dispersion des
demeures et la dissipation du lieu.

Outre ces raisons également essentielles et vraies, j'en avais d'autres
de craindre le séjour de la cour prochaine à Paris, par le caractère de
M. le duc d'Orléans, sa facilité d'écouter, et de se laisser en prise à
tout le monde, et à un monde éloigné par état et par habitude de la
cour, et qui ne l'irait pas chercher à Versailles, ou bien rarement et
bien incommodément, par conséquent hors de portée de recharges et de
cabales entre eux pour l'attaquer par plusieurs et par divers côtés,
gens ineptes en affaires d'État et de cour, ignorants, suffisants,
croyant devoir tout gouverner\,; à un autre monde encore aussi ignorant,
non moins avide, familiarisé avec lui par les plaisirs et les étranges
parties, d'autant plus dangereux qu'ils le connaissaient mieux, et dont
tout le soin pour le posséder et le gouverner serait de le dissiper, de
lui faire perdre tout son temps, de l'amuser par des ridicules toujours
aisés à donner, dont le périlleux effet sur ceux qu'ils attaqueraient
serait funeste aux affaires et au prince\,; enfin les indécences, les
maîtresses, un fréquent opéra où il allait de plain-pied de son
appartement, et mille inconvénients semblables, des soupers scandaleux
et des sorties nocturnes qui les ramassaient tous ensemble.

Je lui dis, en lui représentant tous ces détails fort au long, qu'il
savait que depuis très longtemps je m'abstenais de lui parler de la vie
qu'il menait, parce que j'en avais reconnu l'inutilité\,; mais que
l'extrême nécessité où son nouvel état l'allait mettre de la quitter
m'ouvrait la bouche pour le supplier de penser sérieusement, et de bonne
foi en lui-même ce qu'il trouverait et ce qu'il ne pourrait s'empêcher
de dire, s'il était particulier, d'un régent du royaume qui, à plus de
quarante ans, mènerait et se piquerait de plus de mener la vie d'un
jeune mousquetaire de dix-huit ans, avec des compagnies souvent
obscures, et telles que des gens de caractère n'oseraient voir\,; quel
poids une telle conduite pouvait donner à son autorité au dedans, à sa
considération dans les pays étrangers, à son crédit dès que le roi
commencerait à voir et à entendre, quels contre-temps aux affaires,
quelle indécence à tout, quelle prise sur sa faveur aux petits
compagnons de ses plaisirs, quelle honte, et quel embarras à lui-même
vis-à-vis des personnages français et étrangers, quelle large porte aux
discours, quel péril de mépris et du peu d'obéissance qui le suit
toujours\,! J'ajoutai que le comble de la mesure serait l'impiété, et
tout ce qui la sentirait, qui ferait ses ennemis de toute la nation
dévote, cléricale, monacale, dont le danger était extrême, et qui en
même temps lui éloignerait les honnêtes gens, et ceux qui auraient des
mœurs, de la gravité, surtout de la religion\,; que par là il
rétorquerait contre lui ce raisonnement des libertins, qu'il aimait à
répéter et à applaudir\,; que la religion est une chimère que les
habiles gens ont inventée pour contenir les hommes, les faire vivre sous
certaines lois qui maintiennent la société, pour s'en faire craindre,
respecter, obéir, et qui était nécessaire aux rois et aux républiques
pour cet usage, à tel point qu'il n'y avait point eu de peuples policés
qui n'en aient eu une que leur gouvernement avait soigneusement
maintenue, jusqu'aux différents peuples sauvages, à quoi leurs anciens
et leur conseil étaient très exacts pour eux-mêmes, et pour ceux qui
leur obéissaient. Qu'il devait donc comprendre l'intérêt qu'il avait de
respecter la religion par ses propres principes, et de ne montrer pas un
exemple d'impiété qui le rendrait odieux.

J'appuyai beaucoup sur un article si principal, et je lui dis ensuite
qu'il ne s'agissait point d'hypocrisie, qui est une autre extrémité fort
méprisable, mais de s'interdire tout propos libre sur la religion, de
traiter avec sérieux tout ce qui y a rapport, et d'en observer au moins
les dehors par une pratique bien facile, dès qu'on s'en tient à
l'écorce, et au pur indispensable de cette écorce\,; de ne souffrir en
sa présence, ni plaisanterie, ni discours indiscret là-dessus, et de
vivre au moins en honnête mondain qui respecte la religion du pays qu'il
habite, et qui ne montre rien du peu de cas qu'il en fait. Je lui fis
sentir le danger d'une maîtresse dans la place qu'il allait remplir, et
je le conjurai que, s'il avait là-dessus des faiblesses, il eût soin de
changer continuellement d'objet, pour ne se laisser pas prendre et
subjuguer par l'amour qui naîtrait de l'habitude, et de se conduire dans
cette misère avec toutes les précautions qu'y apportent certains prélats
qui veulent conserver leur réputation par le secret profond de leur
désordre.

Je lui représentai qu'il aurait désormais tant d'occupations, et si
intéressantes, qu'il lui serait aisé de ne plus dépendre de son corps,
si son esprit n'était plus corrompu que l'animal de son âge, et qu'il
avait un intérêt si pressant de se faire aimer, estimer, respecter,
considérer et obéir, que c'était bien de quoi contenir et occuper son
esprit. Qu'en toutes choses la mécanique était bien plus importante
qu'elle ne semblait l'être\,; que celle de ses journées servirait
entièrement à la règle des affaires et à sa réputation, à éviter que
tout ne tombât l'un sur l'autre, et que lui-même pensât à la débauche,
non pas même à regretter ces sortes de plaisirs. Que pour cela, il se
fallait tout d'abord établir un arrangement de journée, d'affaires, de
cour, et de quelque délassement qui se pût soutenir, et qui ne lui
laissât aucun vide, auquel il fallait être fidèle, et se regarder comme
faisaient les ministres du roi fort employés, qui disaient qu'ils
n'avaient pas le temps de se déranger d'un quart d'heure, qui disaient
vrai, et qui le pratiquaient. Ne se pas excéder d'une tâche trop forte,
dont la nouveauté plaît d'abord, que l'importance des choses fait
regarder comme nécessaire, mais dont on se lasse, et qui se change
imperceptiblement à bien moins qu'il ne faut, dont on profite aux dépens
du prince, et qui met bientôt les affaires en désordre. Se garder aussi
de perdre beaucoup de temps en audiences, surtout de femmes, qui en
demandent souvent pour fort peu de choses, qui dégénèrent en
conversations et en plaisanteries, qui ont souvent un but dont le prince
ne s'aperçait pas, et qui tirent vanité de leur longueur et, si elles le
peuvent, de leur fréquence. Les accoutumer à attendre chez Madame et
chez M\textsuperscript{me} la duchesse d'Orléans, les heures où il va
chez elles, et dans leur antichambre, parler debout à celles qui
sortiront au-devant de lui, écouter bien le nécessaire, suivre
soigneusement l'excellente pratique du feu roi qui presque jamais ne
répondait qu'un\,: «\,Je verrai\,;» couper fort poliment très court, et
hors des cas fort rares, n'en voir jamais ailleurs pour affaires, et se
mettre sur le pied qu'une fois entré dans la pièce où est Madame et
M\textsuperscript{me} la duchesse d'Orléans, aucune femme ne le tire à
part, ou s'approchant de lui, parle d'aucune affaire. Une éconduite
polie, mais sèche, aux premières quelles qu'elles puissent être, qui
voudraient tenter cette familiarité, empêchera sûrement qu'aucune s'y
hasarde. À l'égard des hommes, tout l'ordinaire du monde lui parlera en
passant comme on faisait au roi, et cela en débouche beaucoup chaque
jour.

Les personnes des conseils, ce qui en emporte un nombre considérable et
des principaux, le pourront aisément en travaillant avec lui et en
entrant au conseil, dans la pièce précédente duquel les gens d'une
considération distinguée lui parleront, avec lesquels il en usera comme
avec les dames. Ce doit être là aussi où le gros du monde n'entrera
point, où les audiences lui seront demandées en lui disant en deux mots
le pourquoi. Ce sera à lui à juger si la chose le mérite, ou se peut
expliquer là en peu de paroles. En général il doit être très sobre à
accorder des audiences qui font perdre beau coup de temps. Avec de
l'exactitude à éviter tout détail non nécessaire, à ne point écrémer les
conseils, et à être jaloux de les maintenir dans leurs fonctions, il se
trouvera que la matière des audiences sera bien rétrécie. Je n'oubliai
pas le soin de voir le roi tous les jours, souvent à des heures
différentes et rompues pour se tenir dans l'usage d'y aller à toute
heure sans nouveauté et d'en être reçu sans surprise, avec un respect
qui lui plaise, parce qu'il n'y a rien de si glorieux que les enfants,
et que ceux qui l'environneront y seront bien attentifs, et avec la
familiarité aussi qui convient à la naissance et à la place, qui ménagée
avec esprit accoutume et apprivoise les enfants. Aller quelquefois aux
heures de lui présenter le service, y être ouvert et gracieux à ses
gens, avoir pour eux l'accès facile, les écouter avec patience si
quelqu'un d'eux veut lui parler en entrant ou en sortant, mais pour les
réponses en user comme avec les autres, et toutefois être attentif à
leur faire plaisir.

À l'égard des princes et princesses du sang qui arriveront tout droit
dans son cabinet, sans que cela se puisse empêcher, les recevoir debout
tant qu'il pourra, pour les obliger par ce mésaise d'abréger, alléguer
les affaires pressées pour couper le plus court, et leur proposer de
s'épargner cette peine en lui envoyant quelqu'un de leur confiance sur
l'affaire dont il s'agit, afin de s'en mieux éclaircir, en effet pour
perdre moins de temps et être plus libre d'abréger\,; pour les ministres
étrangers qui ne chercheront toujours qu'à le pénétrer et l'engager,
force honnêtetés, force clôture, force fermeté, et les renvoyer aux
affaires étrangères. Cela lui procurera toujours le loisir d'examiner,
de délibérer, et de se tenir hors de toute prise.

Le roi n'a jamais traité avec pas un\,; il savait d'avance quelle serait
la matière de l'audience demandée, répondait courtement et sans jamais
enfoncer ni s'engager encore moins\,; si le ministre insistait, ce qu'il
n'osait guère, il lui disait honnêtement qu'il ne pouvait s'expliquer
davantage, en lui montrant Torcy, qui était toujours présent, comme
celui qui savait ses intentions, et avec qui le ministre pouvait
traiter. Il l'éconduisait ainsi, et si le ministre faisait la sourde
oreille, il le quittait avec une légère inclination de tête, et se
retirait dans un autre cabinet. Il fallait bien alors que le ministre
étranger s'en allât, à qui Torcy en montrait civilement le chemin. C'est
l'imitation que je proposai entière et ferme à M. le duc d'Orléans, avec
les suppléments de politesse que demande la différence qui est entre un
régent et un roi tel surtout que Louis XIV. J'eus toujours attention à
ne lui rien dire sur M\textsuperscript{me} la duchesse de Berry, que
j'affectai de ne nommer jamais directement ni indirectement\,;
l'aventure de Fontainebleau que j'ai racontée m'avait rendu sage\,; mais
mon silence sur un point qui se présentait si naturellement, en traitant
tous les autres, devait au moins être expressif, même éloquent. Si la
suite fait voir combien je perdis mon temps et mes peines, la vérité
veut que je ne retienne rien et que j'expose tout avec sincérité.

\hypertarget{chapitre-xii.}{%
\chapter{CHAPITRE XII.}\label{chapitre-xii.}}

1715

~

{\textsc{Ondes de la cour.}} {\textsc{- Agitation du duc de Noailles.}}
{\textsc{- Curiosité très embarrassante de M\textsuperscript{me} la
duchesse d'Orléans.}} {\textsc{- Maisons me fait une proposition énorme
et folle, et ne se rebute point de la vouloir persuader à M. le duc
d'Orléans et à moi.}} {\textsc{- Réflexions sur le but de Maisons.}}
{\textsc{- Rare impiété et fin de Maisons et de sa famille.}}

~

Plus le temps paraissait s'avancer par la décadence extérieure du roi,
dont pourtant les journées étaient toujours les mêmes, plus chacun
pensait à soi, quoique la terreur qu'on avait de ce monarque dépérissant
à vue d'œil fût telle que M. le duc d'Orléans n'en était pas moins
absolument esseulé jusque dans le salon de Marly. Mais je remarquais
bien qu'on cherchait à s'approcher de moi, et gros du monde, et gens les
plus considérables, et de ces politiques aussi dont le manège effronté
court après ceux à qui ils n'ont jamais parlé, dès qu'ils se les croient
pouvoir rendre utiles, auprès desquels leur souplesse fait effort de les
approcher. Je m'étais souvent moqué de ces prompts amis du crédit et des
places\,; je riais en moi-même de ce vil empressement pour un homme qui
n'en avait encore que l'espérance, et j'en divertissais M. le duc
d'Orléans pour le prémunir d'avance là-dessus lui-même.

Le duc de Noailles, qui ne le voyait qu'en Nicodème\footnote{En secret,
  comme Nicodème visita d'abord Jésus-Christ.}, redoublait peu à peu ses
visites. Il tâchait inutilement de s'attirer quelque confidence sur les
projets d'un prochain avenir. Il m'en faisait des plaintes amères, il se
rabattait sur la peine où le mettait de ne pouvoir rien tirer sur les
places que je lui avais dit que je désirais pour lui et pour son oncle.
Je le tenais en haleine, je lui disais que la proposition que j'en avais
faite avait bien pris, mais que je n'en pouvais savoir davantage. Tantôt
il me priait d'insister, tantôt il m'assurait que je savais bien à quoi
m'en tenir, et me conjurait de rompre mon silence. Je voyais en lui une
passion extrême de cette place des finances, dont il m'entretenait sans
cesse, mais le roi ne me paraissait pas assez proche de sa fin, même
après son testament fait, pour qu'on pût s'expliquer à personne de ce
qui le devait survivre, de sorte que je m'en tins là avec le duc de
Noailles, et M. le duc d'Orléans aussi. Mais le testament fait, j'eus
lieu de douter qu'il se tînt dans la même réserve sur ce qui regardait
Maisons avec lui, et quoique ce qui se verra de ce magistrat semble fort
contrarier ce soupçon, tout ce que je remarquai depuis le testament
surtout et dans l'un et dans l'autre, me persuada que Maisons comptait
fermement sur les sceaux et sur le premier crédit, sans toutefois que ni
l'un ni l'autre m'en aient rien laissé entendre.

M\textsuperscript{me} la duchesse d'Orléans n'était pas la moins
inquiète des limbes où on la laissait sur l'avenir. Elle sentait toute
la situation du duc du Maine. Elle ne pouvait se dissimuler ce qu'il
méritait de M. le duc d'Orléans. Cet intérêt à part, qui lui était le
plus sensible, elle était touchée de celui de M. le duc d'Orléans, et de
ce qu'il pouvait former de projets, et prendre de mesures pour après le
roi. Ses tête-à-tête avec moi, surtout depuis le testament et l'habilité
des bâtards à la couronne, roulaient pour la plupart là-dessus, rarement
la duchesse Sforce en tiers, et me mettaient à la torture. Elle ne
doutait point que M. le duc d'Orléans n'eût en moi une confiance
entière\,; elle ne voyait que moi avec qui il pût s'ouvrir, consulter,
projeter sur l'avenir. L'expérience lui avait appris qu'il se reposait
beaucoup trop sur moi des vues, des mesures, des projets qu'il n'était
pas trop bon lui-même pour faire et pour imaginer, et que, quand cela
lui arrivait, c'était à moi qu'il les confiait, et avec qui il en
délibérait. L'imminence de tout le grand qui allait tomber sur lui ne
permettait pas de croire que ni lui ni moi n'eussions rien là-dessus
dans l'esprit, et la même expérience que M\textsuperscript{me} la
duchesse d'Orléans avait de l'un et de l'autre la persuadait bien que,
s'il était possible que M. le duc d'Orléans n'eût encore rien de
débrouillé dans la tête, il s'en fallait tout que je fusse au même
point. Sa curiosité était donc extrême, et ses questions par
conséquent\,: c'était des contours adroits pour me surprendre, des gens
dont elle me demandait ce que je pensais, en un mot tout ce que l'art,
le manège, la supériorité, le raisonnement, la liberté, l'amitié, la
confiance, le plus proche intérêt, peuvent déployer sous toutes sortes
de faces, avec tout l'esprit, la justesse et l'insinuation possible, mis
sans cesse en œuvre avec une infatigable persévérance.

J'avais affaire à une personne fort supérieure, fort clairvoyante, fort
appliquée, fort réfléchie, fort de suite, et qui par tout ce que j'avais
manié de concert avec elle, et sous ses yeux, me connaissait trop pour
que je pusse me cacher de penser à l'avenir. Le plus grand intérêt et le
même intérêt d'elle comme épouse, de moi à tout ce que je leur étais,
et, depuis le raccommodement que j'avais fait de M. le duc d'Orléans
avec elle en le séparant de M\textsuperscript{me} d'Argenton, l'amitié
la plus intime et la confiance la plus entière établies entre elle et
moi, et par le désir commun de M. le duc d'Orléans et d'elle, sans la
plus légère altération jusqu'alors, devenaient en ces moments des liens
bien embarrassants pour moi. Il fallait donc ménager et maintenir cette
amitié, cette confiance, ce respect, cet air de communauté d'intérêts,
surtout ne lui pas paraître rêver, comme l'on dit, à la suisse, dans de
pareilles conjonctures, après lui en avoir montré tant de différence
dans de grandes affaires\,: telles que celle d'Espagne, celle du mariage
de M\textsuperscript{me} la duchesse de Berry, celle des noires et
affreuses imputations, et de tant d'autres importantes ou de cour, ou
d'intérieur de la famille royale. En même temps me bien garder de
laisser rien entrevoir, ni même soupçonner des secrets qui n'étaient pas
les miens, raisonner toujours et répondre à tout comme à la sœur du duc
du Maine, pour la grandeur duquel elle aurait sacrifié avec transport de
joie mari, enfants et elle-même.

Je ne trouvai donc de ressource que dans la longueur des verbiages pour
consumer le temps, l'embarras des combinaisons, le danger de penser à
rien pendant la vie du roi, l'inutilité de tous projets, si le roi
faisait des dispositions, et après qu'il les eut faites, la folie
d'imaginer les pouvoir attaquer, qui fut mon plus sûr retranchement et
le plus utile, enfin la paresse d'esprit, la légèreté, le peu de suite
qu'elle connaissait dans M. le duc d'Orléans\,; paraphraser longuement
toutes ces difficultés, les tourner de tous les sens, surtout me tenir
de fort court sur les personnes, sur lesquelles elle me promenait et me
demandait ce que j'en pensais, plus encore en garde contre mon air et
mon visage qu'elle observait toujours, pour tâcher attentivement à y
découvrir mieux que dans mes paroles. Je me rabattais encore pour
m'excuser de penser là-dessus par l'inutilité de le faire, sur la
sagesse du gouvernement du roi, sur la longue et générale habitude qu'on
s'était faite de l'admiration, de la soumission, de la crainte\,; sur le
danger de tout changement dans ces moments critiques\,; sur la
difficulté de trouver mieux ni aussi bien\,; sur la rareté des sujets,
sur les jalousies et le péril des méprises en matière d'innovation et de
choix\,; sur le fâcheux état des finances et de l'intérieur du royaume,
enfin sur le testament du roi, après qu'il fut su qu'il en avait fait
un, qui me donna beau champ sur le respect qu'un tel et si long règne
avait imprimé dans l'esprit de tout le monde pour ses volontés, dont
l'exécution serait le seul parti sage et le meilleur qu'on pût prendre
en soi, et dans un pays où la longue habitude de l'obéissance aveugle a
tellement passé en loi qu'il n'y a plus personne qui imagine qu'il soit
permis ni possible de s'y soustraire.

Tous ces propos, enflés et allongés, ne satisfaisaient point
M\textsuperscript{me} la duchesse d'Orléans. Elle avait eu trop
d'occasions de me voir des sentiments plus libres, et de regimber contre
l'éperon, pour se payer de ce que je lui répondais. Elle m'objecta le
testament de Louis XIII, et en raisonna au mieux sur les conséquences à
en tirer et à en prévoir pour celui de Louis XIV. Je sentis incontinent
toute sa défiance de mes réponses, et toute celle qu'elle avait de la
solidité de ce dernier testament, dont, à ce qui s'y était passé et qui
a été rapporté t. XI, p.~173 et suiv., elle ne se pouvait cacher que le
roi ne doutât lui-même autant, ou plus que personne. Il était très
important de la rassurer sur l'une et sur l'autre défiance.

Je me mis donc à raisonner sur la comparaison des temps, des personnes,
des conjonctures, sur la différence d'un règne plein de factions et de
guerres civiles, d'avec un autre du double de durée, d'une puissance
absolue déployée en tout genre, sans la plus légère, non pas
contradiction, mais représentation, qui non seulement avait anéanti
toute autre autorité que la sienne immédiate, mais encore tout crédit,
toute union, toute autre considération que la sienne et de ses
ministres, par conséquent tout personnage et toute autre fonction
d'emploi quelconque et de charges que des domestiques, ce qui ne
laissait personne aujourd'hui en aucun moyen de s'opposer ni de résister
à quoi que ce soit, si tant est qu'il y eût encore quelqu'un qui
s'avisât de se souvenir qu'esclave et sujet n'est pas la même chose\,;
qu'il y avait loin d'une reine de quarante et un ans, fille d'Espagne,
qui avait elle-même passé déjà par plus d'une étamine en affaires
d'État, en tous les temps jusqu'alors intimement unie à la reine sa
belle-mère et à Monsieur, qui avait des généraux et des ministres
attachés à elle, et dans les pays étrangers des créatures habiles, comme
la duchesse de Chevreuse dans le considérable, et dans le bas, mais non
moins utiles, comme Beringhen et d'autres que leurs aventures communes
avec elle y avait fait fuir pour leur sûreté, à M. le duc d'Orléans qui
n'avait que sa naissance, mais ni gouvernement, ni charge, ni troupes
sous ses ordres, et qu'elle voyait elle-même dans un abandon si
universel quoique si proche du timon du royaume\,; qu'il y avait loin
encore d'un prince faible tel que Gaston, qui ne savait jamais prendre
aucun parti par lui-même, ni soutenir aucun de ceux qu'on lui avait fait
prendre, saisi à la chaude, au dépourvu, à l'instant, sans avoir un
moment pour parler à quelqu'un, par une reine avec qui tout l'avait tenu
uni jusqu'alors dans toutes les différentes situations de sa vie, par
conséquent accoutumé à se croire un avec elle, d'ailleurs sans force par
lui-même pour résister aux cajoleries de cette reine et à une parole à
lui donner sur-le-champ, dont il fut assez simple pour se promettre plus
qu'il ne lui quittait, et de M. le Prince pris avec la même promptitude,
à qui l'exemple de Monsieur ferma la bouche, qui ne le pressait pas
moins de le suivre que faisait la reine, et dont l'union contre lui,
s'il leur résistait, lui fit tout appréhender, et dont le consentement
entraîna aussitôt celui de tout le conseil de régence, hors d'état de
leur résister seuls à tous les trois\,; qu'il y avait bien loin de la
situation si brusque de ces trois mêmes personnes et de la leur
d'ailleurs en elle-même, et de celle de M. le duc d'Orléans, d'avec la
situation des personnes en faveur de qui il est croyable que le roi a
fait des dispositions, qui sont apparemment en volonté et en moyens de
les défendre\,; qui n'ont ni les raisons de faiblesse et d'intimes
liaisons qu'eut Gaston, ni le poids, ni le péril d'un tel exemple, en
refusant de s'y conformer comme M. le Prince ne l'osa, ni la disparité
et la nudité de ceux du conseil de régence pour maintenir la part qui
leur était donnée au gouvernement, quand Monsieur et M. le Prince s'en
dépouillaient en faveur de la reine\,; que de plus les dispositions de
Louis XIII avaient été rendues publiques par la lecture que ce monarque
en avait fait faire dans sa chambre, en présence de la reine, de
Monsieur, de M. le Prince, des grands et des plus considérables de sa
cour, même des principaux magistrats qu'il y avait mandés\,; la reine,
ainsi que tout le monde, savait leur contenu, au lieu qu'à l'égard de
celles que le roi a faites, M. le duc d'Orléans est avec tout le monde
dans les plus profondes ténèbres, dont le voile ne sera levé qu'après
que le roi ne sera plus, et levé pour M. le duc d'Orléans et pour tout
le monde à la fois, en plein parlement, par l'ouverture et la lecture du
testament qui y sera faite\,; qu'ainsi la différence est entière entre
la facilité de la reine qui savait à quoi tendre et comment y tendre, et
l'épaisse obscurité de M. le duc d'Orléans qui le tient dans la plus
invincible ignorance de ce qu'il a à faire, à qui il a à faire, et même
s'il a quelque chose à faire. «\,Il n'en faut pas tant, madame,
ajoutai-je avec feu, pour servir de raison à ne rien faire, même à ne
pas penser, à un homme aussi difficile à mettre en mouvement que vous
devez connaître M. le duc d'Orléans, même dans les choses les plus
aplanies et les plus importantes, s'il vous plaît de vous souvenir du
mariage de M\textsuperscript{me} la duchesse de Berry et de beaucoup
d'autres que vous avez vues comme moi.\,»

C'est ainsi que je m'efforçais d'échapper aux filets de toutes les
sortes qui m'étaient continuellement tendus. Mais cette fausseté
indispensable me coûtait si prodigieusement, que j'étais toujours en
crainte de la trahison de mon visage, du son de ma voix, de toute ma
contenance. Il n'est pas possible d'exprimer le combat qui se passe au
fond d'une âme franche, droite, naturelle, vraie, qui, au milieu des
périls de la plus dangereuse cour du monde n'a jamais pu se masquer même
sur rien, et à qui il en a bien des fois coûté cher, sans avoir pu se
résoudre à prendre leçon de ses expériences, dont ces Mémoires sont
pleins\,; quel tourment, dis-je, elle souffre lorsqu'elle se trouve en
ce détroit unique\,: ou de perdre l'État que je comptais sauver et
réparer, perdre M. le duc d'Orléans dont j'avais seul le secret, et me
perdre moi-même\,; ou de tromper avec soin, art et industrie, une
princesse avec qui je vivais depuis des années dans la plus intime et la
plus réciproque amitié et confiance, qu'il fallait voir sans cesse sur
ce même pied, en être attaqué sans mesure aussi avec toute sorte d'art
et d'industrie, et la tromper continuellement par toutes sortes de
détours. Je revenais quelquefois de chez elle chez M. le duc d'Orléans
l'avertir promptement, pour qu'il se trouvât de la conformité dans ce
qu'il lui répondrait avec les discours que je lui avais tenus, souvent
aux larmes, et si plein de rage et de désespoir, qu'il augmentait encore
par en rire, lui à qui ce personnage n'était pas si nouveau, que je me
licenciais de colère à lui en dire plus que très librement mon avis\,;
et c'est de la sorte que s'écoula tout le temps jusqu'à la mort du roi.

On a vu que l'édit qui appelle les bâtards du roi à la couronne, etc.,
comme ayant l'honneur d'être ses fils et petits-fils, est de juillet
1714, enregistré le 2 août, même année\,; que le roi remit son testament
aux premier président et procureur général le dimanche matin 27 août,
même année\,; qu'il n'y eut que vingt-six jours entre l'édit et le
testament, et que le duc du Maine, M\textsuperscript{me} de Maintenon et
le chancelier surent bien employer le temps et n'en point perdre. Il n'y
en eut guère non plus entre le testament fait et livré et le dernier
voyage que le roi ait fait à Fontainebleau, pendant lequel le duc du
Maine commença à ourdir la noire et profonde trame de l'affaire du
bonnet, et qu'il sut conduire comme on l'a vu. Je ne sais si Maisons
était entré avec lui dans la confidence de ce chef-d'oeuvre de scélérate
politique, et qu'en ce cas il eût prévu que le fracas de la fin de cette
affaire me rendrait peu accessible à lui, et moins capable de me prêter
à ses raisonnements. Quoi qu'il en soit, il ne tarda pas à m'en venir
faire un si surprenant, aussitôt que le testament fut déposé au
parlement, qu'il est nécessaire, avant de le rapporter, de remettre
courtement ici devant les yeux ce qui se passa à cet égard.

Mesmes et d'Aguesseau, premier président et procureur général, mandés de
se trouver à l'issue du lever du roi à Versailles pour le dimanche 27
août 1714, y arrivèrent droit chez le chancelier, qui leur remit un édit
fort court et fort sec, signé et scellé, pour le faire enregistrer le
lendemain. Le roi y déclarait que «\,le paquet remis par lui aux premier
président et procureur général du parlement contenait son testament, par
lequel il avait pourvu à la garde et à la tutelle du roi mineur, et au
choix d'un conseil de régence, dont, pour de justes considérations, il
n'avait pas voulu rendre les dispositions publiques\,; qu'il voulait que
ce dépôt fût conservé au greffe du parlement pendant sa vie, et qu'au
moment qu'il plairait à Dieu de le retirer de ce monde, toutes les
chambres du parlement s'assemblassent avec tous les princes de la maison
royale, et tous les pairs de France qui s'y pourraient trouver, pour, en
leur présence, y être fait ouverture du testament, et après sa lecture,
les dispositions qu'il contenait être rendues publiques et exécutées,
sans qu'il fût permis à personne d'y contrevenir, et le duplicata dudit
testament être envoyé à tous les parlements du royaume, par les ordres
du conseil de régence, pour y être enregistré.\,»

Pas un mot dans cet édit d'honnêteté pour le parlement, ni terme
d'estime ni de confiance\,; nulle nomination, ni indication même
d'exécuteur du testament\,; enfin, ce n'est point au parlement ni à
personne qu'il est confié. L'édit ordonne seulement qu'il sera déposé au
greffe, sans parler d'aucune sorte de précaution pour l'y garder, et le
greffe est choisi simplement comme un lieu public et ordinaire de dépôt.
Ainsi le parlement n'y est chargé de rien, ni pas un de ses
magistrats\,; et le greffe ne l'est que comme de tous autres actes qui y
sont déposés. Les duplicata envoyés aux parlements du royaume par les
ordres du conseil de régence font voir une attention marquée pour
l'autorité de ce conseil, et pour omettre le nom de régent, laquelle est
bien significative, et qui relève bien aussi toute la négligence
affectée dans l'édit pour le parlement, qui était l'occasion et le lieu
de dire des choses à flatter cette compagnie, dont il résulte deux
choses\,: l'une, que le parlement n'y fut pour rien, ni en corps, ni par
aucun de ses membres\,; l'autre, que les précautions si grandes pour la
conservation du dépôt furent uniquement du cru et du fait du premier
président, pour rendre odieux le seul homme en haine duquel le testament
parût fait, comme étant capable de s'en saisir par violence, et mettre
ce dépôt ainsi que le duc du Maine, en faveur duquel il parut
visiblement fait, sous la protection de la justice, du parlement, du
peuple, de la multitude. Il est certain que le duc du Maine ne pouvait
rien ajouter à de telles précautions, ni plus complètement profiter d'un
premier président qui lui avait livré son âme.

Le premier président et le procureur général allèrent chez le roi, au
sortir de chez le chancelier. Ce voyage si concerté n'avait point de
moments convenables pour une visite du premier président à M. du Maine,
dont sûrement il avait bien auparavant reçu les ordres et les
instructions, et tout débattu et concerté avec lui. Le roi, en leur
disant ce qui a été rapporté, et sans parler d'aucune précaution, leur
donna le paquet cacheté qui renfermait son testament, et au sortir du
cabinet du roi ils s'en retournèrent à Paris. En y arrivant, ils
envoyèrent chercher des ouvriers. Ils les conduisirent dans une tour du
palais, qui est derrière la buvette de la grand'chambre et le cabinet du
premier président, laquelle répond au greffe et le joint. Ils firent
creuser un grand trou dans la muraille de cette tour, qui est fort
épaisse, y déposèrent le testament, en firent fermer l'ouverture d'une
porte de fer, d'une grille aussi de fer en seconde porte, et murailler
par-dessus. La porte et la grille eurent chacune trois différentes
serrures, mais les mêmes à la porte et à la grille, et une clef pour
chacune des trois, qui par conséquent ouvrait chacune deux serrures, une
de la grille et une de la porte. Le premier président en garda une, le
procureur général une autre, et la troisième fut confiée au greffier en
chef du parlement, sous prétexte que le dépôt était tout contre la
chambre du greffe, en effet, pour éviter occasion de jalousie entre
l'ancien des présidents à mortier et le doyen du parlement, et la
division entre les présidents et les conseillers qu'elle aurait pu faire
naître.

Le lendemain lundi 28 août, le premier président assembla les chambres
dès le matin, leur rendit compte du sujet de son voyage de la veille,
fit présenter l'édit par les gens du roi, qui fut enregistré, paraphrasa
les sages et justes précautions du roi avec force louanges, et n'oublia
pas de suppléer au silence de l'édit par tout ce qu'il put de superbes
flatteries, et de ce qu'il crut de plus propre à intéresser la compagnie
à la protection des dispositions du roi, lorsqu'il en serait temps, et à
la piquer d'honneur pour en procurer l'entière exécution.

Revenons présentement à Maisons. Ce président, comme je l'ai déjà dit,
venait presque tous les dimanches au lever du roi, et après sa messe
chez moi, où la porte était fermée à tout le monde, de règle tant qu'il
y était, et c'était toujours tête à tête. Il vint donc le premier
dimanche d'après celui où le roi avait remis son testament au premier
président et au procureur général, c'est-à-dire le huitième jour après.
Le dépôt était enfermé, et l'édit qui l'annonçait enregistré il y en
avait cinq. Il me fit un discours pathétique où il disserta fortement
l'éclat, le venin, les motifs plus que très apparents du testament, tout
ce dont M. le duc d'Orléans était menacé. Il n'oublia pas de m'exciter
par tout ce qu'il en put croire capable sur le surcroît de grandeur, et
tout le pouvoir qui en résulterait à M. du Maine et à la bâtardise, et
de fois à autre s'interrompant sur la séduction, et par des déclamations
vives contre les auteurs et les coopérateurs d'une pièce si funeste à
l'État et à la maison royale.

Quand il eut bien péroré, je lui dis qu'il ne me persuadait rien de
nouveau\,; que je voyais les mêmes vérités que lui avec la même
évidence\,; que le pis que j'y trouvais, c'est qu'il n'y avait point de
remède. «\,Point de remède\,! m'interrompit-il avec son rire en dessous,
il y en a toujours aux choses les plus extrêmes avec du courage et de
l'esprit\,; et je m'étonne qu'avec ce que vous avez de l'un et de
l'autre, de vous trouver court sur ce qui va tout mettre en
confusion\,;» et de là, à s'étendre sur ce qu'il y allait de tout pour
M. le duc d'Orléans, {[}dans le cas{]} qu'une pièce qui ne pouvait avoir
été fabriquée qu'entre M. du Maine, M\textsuperscript{me} de Maintenon
et le chancelier, et où sûrement rien n'avait été oublié en faveur du
duc du Maine et contre M. le duc d'Orléans, vît jamais le jour. Je
convins que ce serait bien le plus court\,; en même temps je lui
demandai comment supprimer un testament déclaré par un édit enregistré,
pièce par conséquent publique et solennelle encore par sa nature,
déposée de plus avec tant d'éclat, et de si solides précautions connues
de tout le monde, dans l'intérieur le plus enfoncé du palais, et le plus
sûr par la nature et par l'art qui y avait été ajouté. «\,Vous voilà
donc bien embarrassé, me répliqua Maisons\,; avoir à l'instant de la
mort du roi des troupes sûres et des officiers sages, avisés et affidés
tout prêts, avec eux des maçons et des serruriers, marcher au palais,
enfoncer les portes et la niche, enlever le testament, et qu'on ne le
voie jamais.\,»

Dans ma surprise extrême, je lui demandai quel fruit d'une si
prodigieuse violence, et de plus quelle mécanique pour en venir à bout.
J'ajoutai que, quoi qu'il y eût dans le testament, je ne voyais point de
comparaison entre la possible espérance qu'il n'eût pas plus d'exécution
qu'en avait eu celui de Louis XIII, comme le roi lui-même ne s'était pas
caché de le penser, entre essuyer même ses dispositions quelles qu'elles
fussent, et violer à main armée un dépôt public et solennel, de cette
qualité unique et si royale, dans le sein du sanctuaire de la justice,
au milieu de la capitale, soulever le peuple et les provinces, la
raison, la nature, ce que les hommes ont de plus sacré entre eux, donner
aux ennemis de M. le duc d'Orléans les armes les plus spécieuses, lui
débaucher ce qu'il peut avoir d'amis sages et raisonnables par la honte
et le péril de lui demeurer attachés, donner aux horreurs répandues
contre lui un poids que tous les artifices et toute l'autorité n'avaient
pu leur acquérir, autoriser tout ce qui se déclarerait contre lui à
tirer les plus grands usages de cette folie, et armer la juste fureur du
parlement si grandement outragé par un attentat de cette nature, et dans
le moment critique où l'usage abusif presque tourné en loi lui donnait
une autorité avec laquelle il fallait compter dès cet instant même, et
souvent encore dans le cours de la régence. Que si, dans l'exécution si
odieuse par elle-même, et que les bâtards et le parlement qu'elle
réunirait pour toujours avaient tant d'intérêt d'empêcher, il arrivait
une sédition, peut-être appuyée des Suisses, et qu'il y eût du sang
répandu, personne ne pouvait prévoir jusqu'où cette action était capable
de conduire, laquelle, quoi qu'il en succédât, comblerait M. le duc
d'Orléans d'opprobre, de la plus grande, de la plus juste, de la plus
universelle haine, et d'un mépris égal, si par l'événement le testament
échappait à l'attaque.

Tout cela fut commenté bien plus au long, sans que Maisons pût être
ébranlé le moins du monde, et toutefois sans qu'il eût rien à répondre
que l'importance de soustraire un testament qu'il était clair qu'on
n'avait fait que contre M. le duc d'Orléans et en faveur des bâtards.
Maisons, au partir de chez moi, alla faire à M. le duc d'Orléans la même
proposition avec les mêmes instances, et me gagna de la main, espérant
apparemment de le persuader s'il lui parlait avant moi. Heureusement il
n'en fut pas mieux reçu. Nous lui fîmes à peu près les mêmes objections,
parce qu'elles se présentaient d'elles-mêmes, sans lui faire changer de
sentiment\,? et nous nous le contâmes l'un à l'autre, M. le duc
d'Orléans et moi, et tous deux dans un étonnement extrême. Ce qui nous
en donna davantage, c'est qu'il persista jusqu'à sa mort, qui précéda de
très peu de jours celle du roi, à presser M. le duc d'Orléans de cette
extravagance, et moi jusqu'à la persécution.

Il ne tint pas à ses instances redoublées que je ne fisse la sottise
d'aller à la buvette de la grand'chambre reconnaître les lieux sur les
indications qu'il m'en donnait, moi qui n'en avais aucun prétexte, et
qui de plus n'allait jamais au palais que pour des réceptions de pairs,
ou des occasions où le roi les y mandait, et qui même alors n'avais
jamais approché seulement de la buvette. Ne pouvant vaincre là-dessus ce
qu'il appelait mon opiniâtreté, il me demanda au moins de m'arrêter sur
le quai de la Mégisserie, où on vend tant de ferrailles, et d'examiner
de là, la rivière entre-deux, la tour où était le testament, qu'il me
désigna et qui donnait sur le quai des Morfondus, mais en arrière des
bâtiments de ce quai. On peut juger quelle connaissance on pouvait en
tirer de ce point de vue. Je lui promis, non de m'arrêter sur ce quai
pour me faire regarder des passants, mais d'y passer, et de voir ainsi
ce que je pourrais remarquer, en ajoutant que c'était par complaisance,
et pour le satisfaire sur une chose en soi indifférente, parce que rien
au monde ne me pourrait tenter, encore moins me persuader, sur une
pareille entreprise. L'incompréhensible est comment elle avait pu entrer
dans une tête aussi sensée, et que jusqu'à la mort, quoiqu'il nous ait
trouvés inébranlables, M. le duc d'Orléans et moi, il ne se soit jamais
lassé de nous presser là-dessus, ni rebuté de l'espérance de nous y
amener.

Le plus mortel ennemi de M. le duc d'Orléans n'aurait pu imaginer rien
de plus funeste à lui persuader, et je ne sais si on aurait trouvé
plusieurs personnes assez dépourvues de sens pour y donner sérieusement.
Que penser donc d'un président à mortier, de la considération que
Maisons s'était acquise au palais, à la ville, à la cour, où il avait
toujours passé pour un homme d'esprit, sage, avisé, intelligent, capable
et mesuré\,? Était-il assez infatué de la nécessité dont il était pour
M. le duc d'Orléans de supprimer le testament, assez aveuglé de la
parole des sceaux qu'il avait enfin arrachée de ce prince, à ce que j'en
pus juger, et de toute l'autorité qu'il se promettait de tirer de cette
place, qu'il sentait bien qui serait conservée à Voysin si M. du Maine
était maître, après tout ce que cette âme damnée avait si nouvellement
fait pour lui, que la passion l'empêchât de voir les suites affreuses et
indispensables de l'entreprise qu'il proposait, que je lui mettais sans
cesse devant les yeux, et à pas une desquelles il n'avait d'autre
réponse que le danger évident des dispositions du testament pernicieuses
pour M. le duc d'Orléans, toutes pour la grandeur du duc du Maine qui
les saurait bien faire valoir, établi comme il l'était, et la nécessité
dès là indispensable de le supprimer comme que ce pût être\,?

Sa persévérance de près d'une année, qui ne put être, non pas rebutée,
mais même le moins du monde ralentie, ni par des raisons si palpables,
ni par la résistance toujours égale qu'il trouva en M. le duc d'Orléans
et en moi\,; sa réserve là-dessus pour Canillac, dont il se servait
auprès de M. le duc d'Orléans pour soi-même, pour le parlement et pour
tant d'autres choses, réserve dont il n'excepta personne, sans exception
là-dessus que M. le duc d'Orléans et moi, donneraient-elles d'autres
pensées\,? Aurait-il été assez noir pour, de concert avec le duc du
Maine, ouvrir cet abîme sous nos pas, et ne se lasser point de nous y
pousser pour nous perdre, et par la chute de M. le duc d'Orléans, unique
par son âge entre tous les princes du sang à pouvoir être revêtu de la
régence, y porter le duc du Maine, qui de là à la couronne n'aurait eu
qu'un pas à faire, et qui n'en ignorait pas les moyens\,? Un si puissant
objet pour une âme de la trempe de celle du duc du Maine, et qui avait
su se le préparer avec tant d'art et de si loin, n'est rien moins
qu'incroyable, si l'on se rapproche par quels chemins ce fils de
ténèbres était parvenu à escalader tous les degrés du trône dont la
place s'était aplanie et nettoyée devant lui, et tout ce qu'il avait mis
en œuvre pour noircir avec tant de succès le seul obstacle qui lui
restait à vaincre, d'un crime si fatal et si étranger à ce prince, crime
qui, pour le moins, n'était pas fatal au duc du Maine pour la sûreté
jusque-là plus que douteuse, jusqu'aux yeux du roi même, de tout ce
qu'il en avait obtenu jusqu'alors, et par les pas de géant qu'il fit
après vers la couronne. Ce service de Maisons valait bien le sacrifice
de Voysin qui ne pouvait plus être utile au duc du Maine, et d'éblouir
Maisons de tout ce que le savant art de ce futur maire du palais
n'aurait pas manqué de présenter à son ambition.

Qu'on se rappelle les anciennes liaisons de Maisons avec le duc du
Maine, assez fortes pour en avoir espéré la place de premier président,
refroidies par la préférence donnée à Mesmes\,; le renouement de ces
liaisons ensuite, leur secret et celui dont il couvrait toujours celles
qu'il prit tant de soin de faire et d'étreindre avec M. le duc
d'Orléans, et combien promptement et d'avance il fut toujours instruit
avant personne des pas derniers des bâtards vers le trône\,; la scène
qu'à ce propos il me donna chez lui pour m'aveugler, et par moi M. le
duc d'Orléans, car la course qu'il me fit faire à Paris pour m'y
apprendre ce qui fut le soir même public à Marly, était sans ce
\emph{retentum}\footnote{On appelait ordinairement \emph{retentum} la
  partie d'un arrêt qui n'était pas rendue publique (\emph{quod erat
  retentum in mente judicis}).} parfaitement inutile\,; le contraste de
cette scène avec ce dîner à huis clos qu'il donna mystérieusement aux
deux bâtards le jour de leur visite au parlement pour l'enregistrement
de leur habilité à la couronne\,; l'embarras extrême où il tomba quand
il m'en vit informé\,; son manège avec M. et M\textsuperscript{me} du
Maine sur l'affaire du bonnet, et sous ce prétexte ses visites si
fréquentes à Sceaux, où il ne paraissait point, mais où il passait deux
heures chaque fois enfermé seul avec M. et M\textsuperscript{me} du
Maine\,; les distinctions que seul de sa robé il recevait du roi sur ses
fins, toutes les fois qu'il se présentait devant lui, et celle qu'il eut
dans les derniers mois, encore plus unique, d'aller de Maisons à Marly
quand il voulait, comme le duc de Berwick de Saint-Germain, sous
prétexte d'un voisinage dont on ne s'était pas avisé jusque-là, et qui
avec raison avait été de tout temps pour le duc de Berwick\,; enfin la
douleur si marquée de sa mort, arrivée le jeudi au soir, 22 août de
cette année, dix jours avant celle du roi, que témoigna le duc du Maine
qui n'en était pas prodigue, et l'ardeur si empressée avec laquelle il
emporta dès le lendemain, vendredi matin, la charge de président à
mortier pour le jeune Maisons qui n'avait pas dix-sept ans, et qui était
accouru à lui de Paris dans cette confiance\,; qu'on ramasse tout cela,
je le dis avec horreur, conclura-t-on que ce soit pousser trop loin les
soupçons\,?

À mon égard, il lui fallait un homme toujours à portée de M. le duc
d'Orléans, et à portée de tout avec lui, et qui fût dans le secret de
leur liaison. Canillac ne voyait ce prince qu'à Paris où il n'avait que
des moments, et assez rarement depuis un temps\,; Maisons n'en pouvait
donc espérer le même usage, et il se flattait de me vaincre par le coin
de la bâtardise que Canillac avait bien aussi, mais peut-être moins que
moi, parce qu'il perdait moins avec eux. Maisons, de longue main en
grande société avec lui, eût peut-être été fâché de le perdre, et pour
moi c'était double gain à tous égards, pour un bâtard et pour un
président à mortier, et de s'ouvrir à d'autres n'allait pas à leur but,
et y était même directement contraire. Enfin Maisons voulait-il voir si
à la fin M. le duc d'Orléans ou moi serions assez dépourvus de sens
commun pour mordre à un si pernicieux hameçon, nous conduire au bord du
précipice, nous y laisser jeter dans l'espérance que le désordre
effroyable qui en naîtrait mettrait la dictature du royaume entre les
mains du parlement, que lui par son crédit dans la compagnie et par ses
accès\footnote{Ce mot est peu lisible dans le manuscrit. L'auteur a
  peut-être écrit \emph{amis}, ce qui ferait un sens préférable.}, il se
rendrait l'entremetteur entre les partis, et ferait longuement ainsi la
première et la plus utile figure\,; ou, nous voyant près de tenter
l'entreprise, y faire naître lui-même des difficultés, nous affubler
après de l'ignominie d'une résolution si folle et si désespérée, et se
donner auprès du duc du Maine, du parlement, du public, l'honneur de
l'avoir empêchée\,? Quoi qu'il en soit, il est incompréhensible qu'un
président à mortier sage, sensé, et de conduite toujours approuvée, avec
beaucoup d'esprit, de réputation et de connaissance du monde, fort riche
et fort compté partout, ait pu concevoir un projet d'une extravagance
aussi parfaite et aussi désespérée, le proposer, en presser, et ne se
point lasser de faire les derniers efforts pour le persuader, et
continuellement, et sans se rebuter de rien pendant toute une année, et
jusqu'à sa mort. Il n'a pas assez vécu pour donner le temps de percer
ces étranges ténèbres. Elles suffisent du moins pour consoler de sa mort
les gens sages, les gens de bien et d'honneur, et ceux qui aiment la
paix, et qui détestent les désordres. Achevons tout de suite ce qui
regarde Maisons et les siens, pour n'en pas interrompre les derniers
jours de Louis XIV.

Il n'est malheureusement que trop commun de trouver de ces prétendus
esprits forts qui se piquent de n'avoir point de religion, et qui,
séduits par leurs mœurs et par ce qu'ils croient le bel air du monde,
laissent volontiers voir ce qu'ils tâchent de se persuader là-dessus,
sans toutefois en pouvoir venir à bout avec eux-mêmes. Mais il est bien
rare d'en trouver qui n'aient point de religion, sans que, par leur état
dans le monde, ils osent s'en parer. Pour le prodige que je vais
exposer, je doute qu'il ait jamais eu d'exemple, en même temps que je
n'en puis douter par ce que mes enfants et ceux qui étaient auprès d'eux
m'en ont appris, qui dès leur première jeunesse, comme on l'a vu
ci-dessus, ont vécu avec le fils de Maisons dans la plus grande
familiarité, et dans l'amitié la plus intime qui n'a fini qu'avec la vie
de ce jeune magistrat. Son père était sans aucune religion. Veuf sans
enfants fort jeune, il épousa la sœur aînée de la maréchale de Villars,
qui se trouva n'avoir pas plus de religion que lui. Ils eurent ce fils
unique pour lequel ils mirent tous leurs soins à chercher un homme
d'esprit et de mise qui joignit la connaissance du monde à une belle
littérature, union bien rare, mais ce qui l'est encore plus, et dont le
père et la mère firent également leur capital, un précepteur qui n'eût
aucune religion, et qui, par principes, élevât avec soin leur fils à
n'en point avoir. Pour leur malheur, ils rencontrèrent ce phénix
accompli dans ces trois parties, d'agréable compagnie, qui se faisait
désirer dans la bonne, sage, mesuré, savant, de beaucoup d'esprit, très
corrompu en secret, mais d'un extérieur sans reproche et sans
pédanterie, réservé dans ses discours. Pris sur le pied et pour le
dessein d'ôter toute religion à son pupille, en gardant tous les dehors
indispensables, il s'en acquitta avec tant de succès, qu'il le rendit
sur la religion parfaitement semblable au père et à la mère, qui ne
réussirent pas moins bien à en faire un homme du grand monde comme eux,
et comme eux parfaitement décrassé des fatuités de la présidence, du
langage de la robe, des airs aussi de petit-maître qui méprise son
métier, auquel, avec du sens et beaucoup d'esprit, il s'adonna de façon
à surpasser son père en tout, s'il eut vécu. Il était unique, et le père
et la mère et lui s'aimaient passionnément. J'ai suffisamment parlé de
M. et de M\textsuperscript{me} de Maisons pour n'avoir plus que ce mot à
ajouter.

Au milieu des richesses, de la considération publique, d'amis distingués
en tout genre, touchant de la main à la plus haute fortune de son état
et la plus ardemment désirée, il est surpris d'un léger dévoiement dans
ce temps de crise où il n'avait pas le temps de s'écouter. Il prend mal
à propos deux ou trois fois de la rhubarbe, plus mal à propos le
cardinal de Bissy le vient entretenir longtemps sur la constitution, et
contraint l'effet de la rhubarbe\,; le feu se met dans les entrailles
sans qu'il veuille consentir à être malade\,; le progrès devient extrême
en peu d'heures\,; les médecins bientôt à bout n'osent l'avouer, le mal
augmente à vue d'œil\,; tout devient éperdu chez lui\,; il y meurt à
quarante-huit ans, au milieu d'une foule d'amis, de clients, de gens qui
se font de fête, sans volonté ou sans loisir, de penser un moment à ce
qui allait arriver à son âme.

Sa femme, après les premiers transports, et un long désespoir d'une si
cruelle trahison de la fortune, car son mari n'avait point de secret
pour elle, paya enfin de courage et ramassa ses forces pour conserver
les amis et les familiers de la maison, et la continuer sur le pied que
son mari l'avait mise. Mais l'âme n'y était plus. Restaient les
nouvelles, les petites intrigues, les cabales du parlement, les discours
des gens oisifs et mécontents, un reste de tribunal en peinture qui
ressemblait mieux à un café renforcé qu'elle faisait valoir tout ce
qu'elle pouvait, dans lequel elle éleva son fils sur les traces de son
père. La vie de M\textsuperscript{me} de Maisons se passa dix ou douze
ans de la sorte, en projets et en travaux dont la chimère et les vaines
espérances la flattaient, pleine d'opulence, de santé, d'autorité sur
son fils, et de celle du reste de ses charmes sur ses amis et sur tout
ce qui venait chez elle, soutenue de la considération après laquelle
elle courait, lorsque, surprise d'une apoplexie dans son jardin, elle
rassura son fils et ses amis au lieu de profiter pour penser à elle d'un
intervalle de peu de jours, au bout desquels une seconde attaque
l'emporta, sans lui laisser un moment de libre, le 5 mai 1727, dans sa
quarante-sixième année.

Son fils, longtemps fort affligé, chercha à se continuer à s'acquérir
des amis, surtout à se distinguer dans son métier. Il s'y attira en
effet de l'estime et du crédit, et de la considération dans le monde,
comme un jeune homme tourné à devenir un grand sujet. Les exemples
domestiques ne lui servirent que pour ce monde à courir après la
fortune, lorsque plein de vues, et ne se refusant rien de ce que peut
donner l'abondance, il fut surpris à Paris de la petite vérole. La
prompte déclaration de ce mal lui tourna la tête. Il se crut mort, il
pensa à ce qu'il avait méconnu toute sa vie, mais la frayeur qui le
tourna subitement à la mort ne lui laissa plus de liberté, et il mourut
de la sorte dans sa trente-troisième année, le 13 septembre 1731,
laissant un fils unique, qui, au milieu d'une troupe de femmes qui ne le
perdaient jamais de vue, tomba d'entre leurs bras, et en mourut en peu
de jours à dix-huit mois, un an après son père, dont les grands biens
allèrent à des collatéraux. Je n'ai pu refuser cette courte remarque à
une aussi rare impiété. Ces Mémoires ne sont pas un traité de morale\,;
aussi me suis-je contenté d'un récit le plus simple et le plus nu\,;
mais qu'il me soit permis d'y appliquer ces deux versets du psaume xxxvi
qui paraissent si faits exprès\,: «\,J'ai vu l'impie exalté comme les
cèdres du Liban\,: je n'ai fait que passer, il n'était déjà plus\,; je
n'en ai pas même trouvé la moindre trace.\,»

\hypertarget{chapitre-xiii.}{%
\chapter{CHAPITRE XIII.}\label{chapitre-xiii.}}

1715

~

{\textsc{Le duc de Noailles apprend enfin sa destination.}} {\textsc{-
Folles propositions qu'il me fait.}} {\textsc{- M. le duc d'Orléans ne
peut se résoudre à ne pas passer par le parlement pour sa régence, et se
dégoûte du projet d'assembler les états généraux.}} {\textsc{-
M\textsuperscript{me} la duchesse d'Orléans, en crainte des pairs pour
la première séance au parlement après le roi sur les bâtards, a recours
à moi.}} {\textsc{- Je la rassure, et pourquoi, en lui déclarant que si
les princes du sang les attaquent, en quelque temps que ce soit, les
pairs les attaqueront à l'instant.}} {\textsc{- Prise du roi avec le
procureur général sur l'enregistrement pur et simple de la
constitution.}} {\textsc{- Dernier retour de Marly.}} {\textsc{- Espèce
de journal du roi jusqu'à sa fin.}} {\textsc{- Audience de congé de
l'ambassadeur de Perse.}} {\textsc{- Détail de la santé du roi et des
causes de sa mort.}} {\textsc{- Magnifique entrée à Paris du comte de
Ribeira, ambassadeur de Portugal.}} {\textsc{- J'obtiens de M. le duc
d'Orléans qu'il continuera à Chamillart sa pension de soixante mille
livres et la permission de le lui mander.}} {\textsc{- Le duc de
Noailles, seul d'abord, puis aidé du procureur général, me propose
l'expulsion radicale des jésuites hors du royaume.}} {\textsc{- Retour
de M\textsuperscript{me} de Saint-Simon des eaux de Forges à
Versailles.}} {\textsc{- Dames familières.}} {\textsc{- Duc du Maine
chargé de voir la gendarmerie pour, au nom et avec l'autorité du roi,
qui l'avait fait venir et n'en put faire la revue.}} {\textsc{- Mon avis
là-dessus à M. le duc d'Orléans.}} {\textsc{- Je me joue de
Pontchartrain.}} {\textsc{- Je méprise Desmarets.}} {\textsc{- Le roi,
hors d'état de s'habiller, veut choisir le premier habit qu'il
prendra.}} {\textsc{- Courte réflexion.}}

~

Le roi diminua si considérablement dans la seconde moitié du voyage de
Marly, que je crus qu'il était temps de mettre fin aux angoisses du duc
de Noailles, pour être en état de lui parler ouvertement sur ce qui
regardait l'avenir par rapport aux finances, et d'en raisonner avec lui.
M. le duc d'Orléans à qui je le représentai en jugea de même. Il me
permit de lui dire sa destination, et celle de son oncle, et la lui
confirma lui-même la première fois qu'il le vit chez lui. Il est
difficile d'exprimer, et tout à la fois de contenir plus de joie\,; le
sentiment fut le premier ressort, la vanité le second. L'adresse se
plâtra de l'intérêt du cardinal de Noailles, avouant aussi combien les
finances étaient de son goût, parce qu'il s'y était, disait-il, toujours
appliqué, et en dernier lieu sous Desmarets depuis son retour, et qu'il
se flattait d'y réussir moins mal que tout autre qu'on y pourrait
mettre. Il ne m'épargna pas les protestations de la plus parfaite
amitié, de la confiance la plus entière, du concert le plus parfait avec
moi en tout, qu'il me demanda avec instance, enfin de la reconnaissance
la plus vive de tout ce que j'avais fait pour lui auprès des ducs de
Chevreuse et de Beauvilliers, si éloignés de lui et de son oncle, et
dans un temps de disgrâce profonde personnelle à tous les deux,
d'abandon et du dernier embarras à son rappel d'Espagne, et par ces ducs
auprès du Dauphin et de la Dauphine, dans leur plus éclatant apogée\,;
après, de l'avoir raccommodé avec M. {[}le duc{]} et
M\textsuperscript{me} la duchesse d'Orléans, et conduit où il se voyait
enfin aussi bien que son oncle.

La porte une fois ouverte avec lui sur le futur, nous raisonnâmes sur la
destination des autres chefs et présidents des conseils, qu'il approuva.
Il me parla de d'Antin qui depuis son duché me courtisait fort, avec
louange et surprise de ne l'entendre destiné à rien\,; nous nous
parlâmes là-dessus avec confiance\,; il ne me nia point ses défauts,
comme je lui avouai aussi ce que j'en pensais de bon. Tous deux
convînmes que ceux qui étaient destinés à la tête des conseils lui
étaient préférables par leur situation personnelle, qu'il n'y avait même
que le conseil du dedans qui lui pût convenir pour y entrer, ou pour en
être chef si la place en devenait vacante. Il applaudit surtout à la
destruction des secrétaires d'État et à la disgrâce du chancelier, sur
laquelle nous disputâmes en amitié pour les sceaux. Il les désirait pour
le procureur général, je les croyais mieux placés entre les mains du
père\,; outre que, placés là, ils influaient sur le fils, c'était un
échelon de convenance au mérite de l'un et de l'autre que la perspective
d'y pouvoir succéder. Il disserta force choses avec moi, et j'y donnais
volontiers lieu, parce qu'il y en avait d'autres dont je ne voulais pas
l'instruire, dont j'aimais à le laisser dépayser lui-même.

L'ouverture qu'il prenait de plus en plus avec moi sur les choses
futures le jeta dans des propos si forts à l'égard des bâtards que je
les laisserai dans le silence, et qui de chose en autre le conduisirent
à me proposer comme une chose fort raisonnable, et à faire, de fortifier
Paris. Je ne pus lui cacher ma surprise. «\,Paris, lui dis-je\,! et où
les matériaux\,? Où les millions\,? où les années d'en achever les
travaux\,? et quand tout se ferait d'un coup de baguette, quelle
garnison pour le défendre\,? quel approvisionnement de munitions de
guerre et de bouche pour les troupes et pour les habitants\,? quelle
artillerie\,? enfin quel fruit s'en pourrait-on proposer quand la
possibilité en serait aussi claire que l'était la démonstration de
l'impossibilité\,?» Il battit la campagne pendant quelques jours
là-dessus, et je le laissai dire, parce que je ne craignais pas
l'exécution de ce rare projet. Voyant qu'il ne me persuadait pas, il
m'en proposa un autre. Ce fut de transporter à Versailles les cours
supérieures, les écoles publiques et tout ce qui est affaires et public.
Je le regardai avec la même surprise\,; je lui demandai où, quand, et
avec quels frais il établirait tout cela à Versailles, lieu sans rivière
ni eau bonne à boire, qui n'est que sable ou boue, à qui la nature
refuse tout, jusqu'à des abreuvoirs commodes pour des chevaux, et où il
ne croît rien loin à la ronde\,; de plus, quelle utilité d'une
translation qui, quand elle serait possible, n'apporterait que du
mésaise et de la confusion à la cour, et laisserait à Paris un vide
irréparable, ruinerait plaideurs, magistrats, suppôts de justice et
d'universités\,; en un mot, rien de praticable, rien qui eût un objet.
C'était, disait-il, pour diminuer Paris, dont la consommation ruine les
provinces, et séparer les cours supérieures de l'appui de ce peuple
nombreux, dont en plusieurs occasions l'union est dangereuse. Peu à peu
il convint de l'ingratitude de la situation de Versailles, déclama
contre l'immense établissement que le roi y avait fait, vanta celle de
Saint-Germain, et finalement me proposa comme une chose facile de
démolir Versailles, d'en emporter tout à Saint-Germain, où, avec ces
matériaux et ces richesses on ferait le plus sain et le plus admirable
séjour de l'Europe.

À ce troisième \emph{sproposito} la parole me manqua. Voici un fou, me
dis-je à moi-même, qui me va peut-être sauter aux yeux\,: «\,Eh\,!
qu'ai-je fait\,? et que vont devenir les finances\,?» Tandis que je me
parlais ainsi sans remuer les lèvres, il discourait toujours, enchanté
du plus beau lieu du monde qu'allait devenir Saint-Germain des
dépouilles entières de Versailles. À la fin mon silence l'arrêta, il me
pria de le rompre. «\,Monsieur, lui dis-je, quand vous aurez les fées à
votre disposition avec leurs baguettes, je serai de votre avis pour
ceci\,; car, en effet, rien ne serait plus admirable, et je n'ai jamais
compris qu'on ait pu choisir Versailles, beaucoup moins préférer ce
cloaque à ce qu'est Saint-Germain\,; mais pour ce que vous me proposez,
il nous faut les fées\,; jusqu'à ce que vous les ayez en main, il n'y a
pas moyen d'en raisonner.\,» Il se mit à rire, et voulut soutenir que
sans fées la chose était possible, et n'était pas un objet tel qu'il
voyait bien que je le pensais. Des trois propositions, ce fut celle
qu'il appuya le moins et le moins longtemps, mais je n'en demeurai pas
moins effarouché.

Il y avait déjà du temps qu'il m'en avait fait une autre que je n'avais
pas moins rejetée, et qu'il ne cessait point de remettre toujours sur le
tapis. Je lui faisais des objections auxquelles il ne put jamais faire
la moindre réponse\,; il n'avait que l'unique ressource de Maisons sur
la sienne, qui était le danger du testament, et il n'en pouvait trouver
à exécuter ce qu'il proposait, et néanmoins, comme Maisons, il ne cessa
point de me presser là-dessus. Nous verrons bientôt, non par
conjectures, comme sur la proposition d'enlever le testament du roi,
mais par les faits, quel était l'objet de Noailles dans une proposition
si ridicule, mais si opiniâtre, et c'est alors que l'une et l'autre
seront expliquées.

Je m'aperçus sur la fin de Marly que M. le duc d'Orléans avait traité le
point de l'assemblée des états généraux avec le duc de Noailles. Il me
l'avoua comme chose trop connexe aux finances par l'objet qu'on s'en
proposait, pour la lui cacher, après lui avoir dit sa destination. Le
duc de Noailles me l'avoua de même avec quelque embarras, et il me parut
bientôt après que M. le duc d'Orléans n'était plus si déterminé à les
assembler. Je le vis aussi mollir tout à fait à l'égard du parlement
pour la régence. Cet article lui avait toujours paru dur, et le dépôt du
testament lui fut un prétexte dont il se servit pour cacher sa
faiblesse. Je la connaissais trop pour me flatter de l'emporter sur elle
pour deux articles aussi majeurs que l'étaient celui-là et celui des
états généraux. Ce dernier me sembla toujours si extrêmement important,
et à tant de grands égards, que je ne balançai pas à lui sacrifier
l'autre. J'espérai d'autant mieux de cette conduite, que ma complaisance
délivrait M. le duc d'Orléans de la dispute et de la présence d'un objet
où il fallait payer de sa personne, et que je ramassais toutes mes
forces pour maintenir l'autre qu'il avait constamment goûté et résolu
jusqu'alors, où il n'avait nul tour de force à tirer de sol, où au
contraire tout était riant pour lui, gracieux pour toute la France,
aplani partout. C'est ce que je continuai de faire, mais avec peu de
progrès jusqu'à la veille de la mort du roi qu'il me déclara nettement
qu'il n'y fallait plus penser.

Dès lors j'en vis assez pour mal augurer des affaires. Je sentis
l'intérêt du duc de Noailles, qui, dans le plan de la convocation des
états généraux, n'aurait pas été maître dans les finances, et qu'il
avait fait comprendre au régent que lui-même ne le serait pas. Je ne
dissimulerai pas que cela ne fût vrai, et même l'un des biens qui m'en
paraissait résulter. L'expérience de ce qui s'est passé depuis dans les
finances a dû montrer si j'avais eu raison. Avec le projet d'assembler
les états généraux tomba celui de la banqueroute\,: il ôtait trop les
moyens de pêcher en eau trouble. Les liquidations et la continuation des
impôts et des traités y ouvrait une large porte aux fortunes, aux
grâces, aux défaveurs dont M. le duc d'Orléans, et mieux encore le duc
de Noailles, aurait le robinet entre les mains. Par là aussi tomba le
projet des taxes, et du même coup celui des remboursements et de la
multiplication des récompenses qui ont été expliquées. Il n'est pas
temps encore de parler des tristes réflexions dont ce début m'accabla,
et des autres choses qui les fortifièrent. Les matières vont tellement
se multiplier pendant un mois ou six semaines, que ce sera beaucoup
faire de n'en rien oublier, et de les démêler pour les présenter avec
quelque netteté et quelque ordre.

Tout à la fin de Marly, le roi parut si affaibli, quoiqu'il n'eût encore
rien changé dans ses journées, que M\textsuperscript{me} la duchesse
d'Orléans me tourna sur ses frères, et qu'après quelques détours assez
empêtrés, car l'orgueil luciférien souffrait bien d'en venir là, elle me
témoigna son inquiétude de la première séance au parlement après le roi,
et qu'elle m'aurait une grande obligation si je pouvais détourner les
pairs d'y rien faire en des moments déjà si accablants pour elle. Je
n'avais pas à être embarrassé de la réponse\,: je lui dis «\,que je ne
croyais pas que les pairs songeassent {[}à autre chose{]} qu'aux
affaires indispensables d'une séance qui en serait aussi chargée, et
qu'elle pouvait se rassurer là-dessus. --- Mais, monsieur, reprit-elle,
m'en voudriez-vous bien donner vôtre parole, au moins me promettre de
faire ce qui sera en vous pour que MM. les pairs ne fassent rien ce
jour-là contre le rang de mes frères\,? --- Oui, madame, lui dis-je, du
dernier s'entend, car je ne suis pas le maître de mes égaux, comme vous
le pouvez bien penser, mais de les détourner autant qu'il me sera
possible à cet égard, et je m'y engage d'autant plus librement que je ne
vois pas qu'ils y pensent. Mais tout d'un temps, madame, puisque Votre
Altesse Royale me force à lui parler sur un article si délicat, qu'elle
prenne garde aux princes du sang\,; c'est leur affaire plus que la nôtre
depuis l'habilité à la couronne, le nom et la qualité et totalité en
tout de princes du sang donnée à MM. vos frères et à leur postérité, et
tenez-vous au moins pour avertie que si les princes du sang les
attaquent, dans l'instant même nous revendiquerons notre rang à ce qu'il
n'y ait personne dans l'intervalle entre les princes du sang et nous, et
que tous soient comme nous dans leur rang de pairie.\,»

Cette déclaration, si amère en soi pour M\textsuperscript{me} la
duchesse d'Orléans, passa le plus doucement du monde au moyen du répit
que je lui promettais, et du mépris qu'il lui plaisait faire de jeunes
princes du sang et de M\textsuperscript{me}s leurs mères. Elle me
remercia même fort honnêtement, et avec des marques d'amitié et de
confiance. Elle me craignait étrangement sur ce point de ses frères
qu'elle nomma toujours ainsi, sans oser jamais proférer en cette
occasion le nom du duc du Maine, qui en avait encore plus de peur, et
qui sûrement n'avait pas oublié la dernière visite qu'il avait reçue de
moi, en conséquence de laquelle je m'étais conduit depuis à son égard
sans mesure. Ma promptitude à répondre à M\textsuperscript{me} la
duchesse d'Orléans ne me coûta guère. Il n'y avait pas moyen d'attaquer
les bâtards et le bonnet tout à la fois, et de détourner les affaires de
l'État à des intérêts personnels à régler dans la première séance au
parlement, après la mort du roi. L'occasion du bonnet, qui ne s'y
pouvait éviter, ne laissait pas de choix entre cette affaire et celle
des bâtards\,; ainsi je ne hasardais rien à leur égard avec
M\textsuperscript{me} la duchesse d'Orléans par ma réponse.

Le vendredi 9 août, le P. Tellier répéta le roi longtemps le matin sur
l'enregistrement pur et simple de la constitution, et vit là-dessus le
premier président et le procureur général qu'il avait mandés la veille.
Le roi courut le cerf après dîner dans sa calèche qu'il mena lui-même à
l'ordinaire pour la dernière fois de sa vie, et parut très abattu au
retour. Il eut le soir grande musique chez M\textsuperscript{me} de
Maintenon. Le samedi 10 août, il se promena, avant dîner, dans ses
jardins à Marly\,; il en revint à Versailles sur les six heures du soir
pour la dernière fois de sa vie, et ne revoir jamais cet étrange ouvrage
de ses mains. Il travailla le soir chez M\textsuperscript{me} de
Maintenon avec le chancelier, et parut fort mal à tout le monde. Le
dimanche 11 août, il tint le conseil d'État, s'alla promener
l'après-dînée à Trianon pour ne plus sortir de sa vie. Il avait mandé le
procureur général avec lequel il eut une forte prise. Il en avait déjà
eu une avec lui en présence du premier président et du chancelier, le
jeudi précédent à Marly, sur l'enregistrement pur et simple de la
constitution. Il trouva le procureur général, seul, armé des mêmes
raisons et de la même fermeté. Il ne se sentait pas en état d'aller lui
même au parlement comme il l'avait annoncé. Quoiqu'il n'en eût pas perdu
l'espérance, il n'en fut que plus outré contre le procureur général,
jusqu'à sortir de son naturel, et en venir aux menaces de lui ôter sa
charge en lui tournant le dos. Ce fut ainsi que finit cette audience
dont ce magistrat ne fut pas plus ébranlé.

Le lendemain 12 août, il prit médecine à son ordinaire et vécut à son
ordinaire aussi de ces jours-là. On sut qu'il se plaignait d'une
sciatique à la jambe et à la cuisse. Il n'avait jamais eu de sciatique
ni de rhumatisme\,; jamais enrhumé, et il y avait longtemps qu'il
n'avait eu de ressentiment de goutte. Il y eut le soir petite musique
chez M\textsuperscript{me} de Maintenon, et ce fut la dernière fois de
sa vie qu'il marcha.

Le mardi 13 août, il fit son dernier effort pour donner, en revenant de
la messe, où il {[}se{]} fit porter, l'audience de congé, debout et sans
appui, à ce prétendu ambassadeur de Perse. Sa santé ne lui permit pas
les magnificences qu'il s'était proposées comme à sa première
audience\,; il se contenta de le recevoir dans la pièce du trône, et il
n'y eut rien de remarquable. Ce fut la dernière action publique du roi,
où Pontchartrain trompait si grossièrement sa vanité pour lui faire sa
cour. Il n'eut pas honte de terminer cette comédie par la signature d'un
traité dont les suites montrèrent le faux de cette ambassade. Cette
audience, qui fut assez longue, fatigua fort le roi. Il résista en
rentrant chez lui à l'envie de se coucher\,; il tint le conseil de
finance, dîna à son petit couvert ordinaire, se fit porter chez
M\textsuperscript{me} de Maintenon, où il y eut petite musique, et, en
sortant de son cabinet, s'arrêta pour la duchesse de La Rochefoucauld
qui lui présenta la duchesse de La Rocheguyon sa belle-fille, qui fut la
dernière dame qui lui ait été présentée. Elle prit le soir son tabouret
au souper du roi qui fut le dernier de sa vie au grand couvert. Il avait
travaillé seul chez lui après son dîner avec le chancelier. Il envoya le
lendemain force présents et quelques pierreries à ce bel ambassadeur
qu'on mena deux jours après chez un bourgeois à Chaillot, et à peu de
distance, au Havre-de-Grâce, où il s'embarqua. Ce fut ce même jour que
la princesse des Ursins, effrayée, comme on l'a dit, de l'état du roi,
partit de Paris pour gagner Lyon en diligence, le lendemain mercredi,
veille de l'Assomption.

Il y avait plus d'un an que la santé du roi tombait. Ses valets
intérieurs s'en aperçurent d'abord, et en remarquèrent tous les progrès,
sans que pas un osât en ouvrir la bouche. Les bâtards, ou, pour mieux
dire, M. du Maine le voyait bien aussi, qui, aidé de
M\textsuperscript{me} de Maintenon et de leur chancelier-secrétaire
d'État, hâta tout ce qui le regardait. Fagon, premier médecin, fort
tombé de corps et d'esprit, fut de tout cet intérieur le seul qui ne
s'aperçut de rien. Maréchal, premier chirurgien, lui en parla plusieurs
fois, et fut toujours durement repoussé. Pressé enfin par son devoir et
par son attachement, il se hasarda un matin vers la Pentecôte d'aller
trouver M\textsuperscript{me} de Maintenon. Il lui dit ce qu'il voyait,
et combien grossièrement Fagon se trompait. Il l'assura que le roi, à
qui il avait tâté le pouls souvent, avait depuis longtemps une petite
fièvre lente, interne\,; que son tempérament était si bon, qu'avec des
remèdes et de l'attention, tout était encore plein de ressources, mais
que, si on laissait gagner le mal, il n'y en aurait plus.
M\textsuperscript{me} de Maintenon se fâcha, et tout ce qu'il remporta
de son zèle fut de la colère\,: Elle lui dit qu'il n'y avait que les
ennemis personnels de Fagon qui trouvassent ce qu'il lui disait là de la
santé du roi, sur laquelle la capacité, l'application, l'expérience du
premier médecin ne se pouvait tromper. Le rare est que Maréchal, qui
avait autrefois taillé Fagon de la pierre, avait été mis en place de
premier chirurgien par lui, et qu'ils avaient toujours vécu depuis
jusqu'alors dans la plus parfaite intelligence. Maréchal outré, qui me
l'a conté, n'eut plus de mesures à pouvoir prendre, et commença dès lors
à déplorer la mort de son maître. Fagon, en effet, était en science et
en expérience le premier médecin de l'Europe, mais sa santé ne lui
permettait plus depuis longtemps d'entretenir son expérience, et le haut
point d'autorité où sa capacité et sa faveur l'avaient porté l'avait
enfin gâté. Il ne voulait ni raison ni réplique, et continuait de
conduire la santé du roi comme il avait fait dans un âge moins avancé,
et le tua par cette opiniâtreté.

La goutte dont il avait eu de longues attaques avait engagé Fagon à
emmaillotter le roi, pour ainsi dire, tous les soirs dans un tas
d'oreillers de plume qui le faisaient tellement suer toutes les nuits,
qu'il le fallait frotter et changer tous les matins avant que le grand
chambellan et les premiers gentilshommes de la chambre entrassent. Il ne
buvait depuis longues années, au lieu du meilleur vin de Champagne dont
il avait uniquement usé toute sa vie, que du vin de Bourgogne avec la
moitié d'eau, si vieux qu'il en était usé. Il disait quelquefois, en
riant, qu'il y avait souvent des seigneurs étrangers bien attrapés à
vouloir goûter du vin de sa bouche. Jamais il n'en avait bu de pur en
aucun temps, ni usé de nulle sorte de liqueur, non pas même de thé,
café, ni chocolat. À son lever seulement, au lieu d'un peu de pain, de
vin et d'eau, il prenait depuis fort longtemps deux tasses de sauge et
de véronique\,; souvent entre ses repas et toujours en se mettant au lit
des verres d'eau avec un peu d'eau de fleur d'orange qui tenaient
chopine, et toujours à la glace en tout temps\,; même les jours de
médecine il y buvait et toujours aussi à ses repas, entre lesquels il ne
mangea jamais quoi que ce fût, que quelques pastilles de cannelle qu'il
mettait dans sa poche à son fruit avec force biscotins pour ses chiennes
couchantes de son cabinet.

Comme il devint la dernière année de sa vie de plus en plus resserré,
Fagon lui faisait manger à l'entrée de son repas beaucoup de fruits à la
glace, c'est-à-dire des mûres, des melons et des figues, et celles-ci
pourries à force d'être mûres, et à son dessert beaucoup d'autres
fruits, qu'il finissait par une quantité de sucreries qui surprenait
toujours. Toute l'année il mangeait à souper une quantité prodigieuse de
salade. Ses potages, dont il mangeait soir et matin de plusieurs, et en
quantité de chacun sans préjudice du reste, étaient pleins de jus et
d'une extrême force, et tout ce qu'on lui servait plein d'épices, au
double au moins de ce qu'on y en met ordinairement, et très fort
d'ailleurs. Cela et les sucreries n'était pas de l'avis de Fagon, qui,
en le voyant manger, faisait quelquefois des mines fort plaisantes, sans
toutefois oser rien dire, que par-ci par-là, à Livry et à Benoist, qui
lui répondaient que c'était à eux à faire manger le roi, et à lui à le
purger. Il ne mangeait d'aucune sorte de venaison ni d'oiseaux d'eau,
mais d'ailleurs de tout, sans exception, gras et maigre, qu'il fit
toujours, excepté le carême que quelques jours seulement, depuis une
vingtaine d'années. Il redoubla ce régime de fruits et de boisson cet
été.

À la fin, ces fruits pris après son potage lui noyèrent l'estomac, en
émoussèrent les digestifs, lui ôtèrent l'appétit, qui ne lui avait
manqué encore de sa vie, sans avoir jamais eu ni faim ni besoin de
manger, quelque tard que des hasards l'eussent fait dîner quelquefois.
Mais aux premières cuillerées de potage, l'appétit s'ouvrait toujours, à
ce que je lui ai ouï dire plusieurs fois, et il mangeait si
prodigieusement et si solidement soir et matin, et si également encore,
qu'on ne s'accoutumait point à le voir. Tant d'eau et tant de fruits,
sans être corrigés par rien de spiritueux, tournèrent son sang en
gangrène, à force d'en diminuer les esprits, et de l'appauvrir par ces
sueurs forcées des nuits, et furent cause de sa mort, comme on le
reconnut à l'ouverture de son corps. Les parties s'en trouvèrent toutes
si belles et si saines qu'il y eut lieu de juger qu'il aurait passé le
siècle de sa vie. Son estomac surtout étonna, et ses boyaux par leur
volume et leur étendue au double de l'ordinaire, d'où lui vint d'être si
grand mangeur et si égal. On ne songea aux remèdes que quand il n'en fut
plus temps, parce que Fagon ne voulut jamais le croire malade, et que
l'aveuglement de M\textsuperscript{me} de Maintenon fut pareil
là-dessus, quoiqu'elle eût bien su prendre toutes les précautions
possibles pour Saint-Cyr et pour M. du Maine. Parmi tout cela, le roi
sentit son état avant eux, et le disait quelquefois à ses valets
intérieurs. Fagon le rassurait toujours sans lui rien faire. Le roi se
contentait de ce qu'il lui disait sans en être persuadé, mais son amitié
pour lui le retenait, et M\textsuperscript{me} de Maintenon encore plus.

Le mercredi, 14 août, il se fit porter à la messe pour la dernière fois,
tint conseil d'État, mangea gras, et eut grande musique chez
M\textsuperscript{me} de Maintenon. Il soupa au petit couvert dans sa
chambre, où la cour le vit comme à son dîner. Il fut peu dans son
cabinet avec sa famille, et se coucha peu après dix heures.

Le jeudi, fête de l'Assomption, il entendit la messe dans son lit. La
nuit avait été inquiète et altérée. Il dîna devant tout le monde dans
son lit, se leva à cinq heures, et se fit porter chez
M\textsuperscript{me} de Maintenon, où il eut petite musique. Entre sa
messe et son dîner il avait parlé séparément au chancelier, à Desmarets,
à Pontchartrain. Il soupa et se coucha comme la veille. Ce fut toujours
depuis de même, tant qu'il put se lever.

Le vendredi 16 août, la nuit n'avait pas été meilleure\,; beaucoup de
soif et de boisson. Il ne fit entrer qu'à dix heures. La messe et le
dîner dans son lit comme toujours depuis, donna audience dans son
cabinet à un envoyé de Wolfenbüttel, se fit porter chez
M\textsuperscript{me} de Maintenon\,; il y joua avec les dames
familières, et y eut après grande musique.

Le samedi 17 août, la nuit comme la précédente. Il tint dans son lit le
conseil de finances, vit tout le monde à son dîner, se leva aussitôt
après, donna audience dans son cabinet au général de l'ordre de
Sainte-Croix de la Bretonnerie, passa chez M\textsuperscript{me} de
Maintenon, où il travailla avec le chancelier. Le soir, Fagon coucha
pour la première fois dans sa chambre.

Le dimanche 18 août se passa comme les jours précédents. Fagon prétendit
qu'il n'avait point eu de fièvre. Il tint conseil d'État avant et après
son dîner, travailla après sur les fortifications avec Pelletier à
l'ordinaire, puis passa chez M\textsuperscript{me} de Maintenon, où il y
eut musique. Ce même jour le comte de Ribeira, ambassadeur
extraordinaire de Portugal, dont la mère, qui était morte, était sœur du
prince et du cardinal de Rohan, fit à Paris son entrée avec une
magnificence extraordinaire, et jeta au peuple beaucoup de médailles
d'argent et quelques-unes d'or. L'état du roi, qui montrait
manifestement ne pouvoir plus durer que peu de jours, et dont je savais
par Maréchal des nouvelles plus sûres que celles que Fagon se voulait
persuader à soi et aux autres, me fit penser à Chamillart, qui avait, en
sortant de place, une pension du roi de soixante mille livres. J'en
demandai la conservation et l'assurance à M. le duc d'Orléans, et je
l'obtins aussitôt avec la permission de le lui mander à Paris. Il y
était fort touché de la maladie du roi, et fort peu de toute autre
chose. Il ne laissa pas d'être agréablement surpris de ma lettre, et
d'être bien sensible à un soin de ma part qu'il n'avait pas eu pour
lui-même. Il m'envoya une lettre de remerciement que je rendis à M. le
duc d'Orléans. Je n'ai rien fait qui m'ait donné plus de plaisir. La
chose demeura secrète jusqu'à la mort du roi\,; je ne perdis pas de
temps à la faire déclarer incontinent après la régence.

Ce même jour je montai chez le duc de Noailles sur les huit heures du
soir, au bas du degré duquel je logeais. Il était enfermé dans son
cabinet, d'où il vint me trouver dans sa chambre. Après plusieurs propos
sur l'état du roi et sur l'avenir, il se mit a enfiler un assez long
discours sur les jésuites, dont la conclusion fut de me proposer de les
chasser tous de France, de remettre en leur premier état les bénéfices
qu'ils avaient fait unir à leurs maisons, et d'appliquer leurs biens aux
universités où ils se trouveraient situés. Quoique les propositions
extravagantes du duc de Noailles, dont j'ai parlé, me dussent avoir
appris qu'il en pouvait faire encore d'aussi folles, j'avoue que
celle-là me surprit autant que si elle eût été la première de ce genre.
II s'en aperçut à mon air effrayé, il se mit en raisonnements, et
cependant son cabinet s'ouvrit, d'où je vis le procureur général sortir
et venir à nous. Plusieurs du parlement étaient venus le matin savoir
des nouvelles du roi, comme en tout temps ils y venaient souvent les
dimanches, mais j'avais cru le duc de Noailles seul dans son cabinet, et
le procureur général retourné à Paris de fort bonne heure, comme ces
magistrats faisaient toujours.

À peine se fut-il tiré un siège auprès de nous, que le duc de Noailles
lui dit ce qu'il s'agitait entre lui et moi, qui pourtant n'avais pas
dit un mot encore, mais à qui un geste échappé de surprise avait mis le
duc de Noailles en plaidoyer. Il remit le peu qu'il venait de dire au
procureur général, qui l'interrompit bientôt pour me regarder
froidement, et me dire de même que c'était la meilleure et la plus utile
chose que l'on pût faire au commencement de la régence que l'expulsion
totale, radicale et sans retour des jésuites hors du royaume, et de
disposer sur-le-champ de leurs maisons et de leurs biens en faveur des
universités. Je ne puis exprimer ce que je devins à cette sentence du
procureur général\,; cette folie, assez contagieuse pour offusquer un
homme aussi sage, et dans une place qui ne lui permettait pas d'en
ignorer la mécanique et les suites, me fit peur d'en être gagné aussi.
L'étonnement où je fus me mit en doute aussi d'avoir bien entendu\,; je
le fis répéter et je demeurai stupéfait. Ils s'aperçurent bientôt à ma
contenance que j'étais plus occupé de mes pensées que de leurs
discours\,; ils me prièrent de leur dire ce que je trouvais de leur
proposition. Je leur avouai que je la trouvais tellement étrange, que
j'avais peine à croire à mes oreilles. Ils se mirent là-dessus, l'un
avec feu, l'autre avec poids et gravité, et s'interrompant l'un l'autre,
à me dire ce que chacun sait sur les jésuites, leur domination, leur
danger pour l'Église et pour l'État et pour les particuliers. À la fin
l'impatience me prit, je les interrompis à mon tour, et il me parut que
je leur faisais plaisir, dans celle où ils étaient d'entendre ce que
j'avais à leur dire.

Je leur déclarai que, pour abréger, je ne leur contesterais rien de tout
ce qu'ils voudraient alléguer contre les jésuites, et sur les avantages
que trouverait la France d'en être délivrée, encore qu'il y eût beaucoup
à dire là-dessus\,; que je me retranchais uniquement sur la cause, le
comment et sur les suites\,; sur le comment que nous n'étions pas dans
une île dont l'intérieur fût désert, comme la Sicile\,; où il n'y eût
qu'un certain nombre de maisons de jésuites dans deux villes
principales, comme Palerme et Messine, et répandues en d'autres gros
lieux sur la côte, où il avait été aisé au vice-roi Maffei de les
prendre tous au même instant d'un coup de filet, de les embarquer
sur-le-champ, de leur faire prendre le large, et de faire tout de suite
de leurs maisons et de leurs biens ce que le roi de Sicile lui avait
ordonné\,; que ce prince de plus était en droit et en raison d'en user
de la sorte avec des gens qui allumaient à visage découvert le feu de la
révolte contre lui, sur le différend qu'il avait avec la cour de Rome\,;
qui, sur des prétextes les plus frivoles d'immunité ecclésiastique qui
même n'avait pas été violée, entreprenait d'abolir le tribunal de la
monarchie accordé tel qu'il était par les papes aux premiers princes
normands qui avaient conquis la Sicile, et l'avaient bien voulu
relever\footnote{C'est-à-dire \emph{déclarer relevant des papes}, comme
  un vassal relevait de son suzerain.} des papes sans aucune nécessité
ni droit, tribunal sans l'exercice duquel les rois de Sicile se
trouveraient privés de toute autorité, pour l'abolition duquel Rome
prodiguait ses censures, et, secondée de plusieurs évoques, de
quelques-uns du clergé séculier, de presque tout le régulier, surtout
des jésuites, portait la révolte et la sédition dans tous les esprits,
et en faisait un point de conscience\,; qu'en France il ne s'était rien
passé, depuis la mort d'Henri IV jusqu'alors, sur quoi on ait pu, je ne
dis pas accuser, mais soupçonner les jésuites de brasser rien contre
l'État, ni contre Louis XIII, ni {[}contre{]} Louis XIV\,; nul délit,
par conséquent, sur lequel on pût fonder le bannissement du plus obscur
particulier\,; quelle violence donc à l'égard de toute une compagnie que
ces deux messieurs représentaient si appuyée, si puissante, si
dangereuse\,; la faire au bout de deux règnes qui l'avaient si
constamment favorisée\,; la faire à l'entrée d'une régence, qui est
toujours un temps de ménagement et de faiblesse\,; la faire enfin par un
régent accusé de n'avoir point de religion, sans parler du reste, et que
la vie publiquement débauchée et les propos peu mesurés sur la religion
rendaient infiniment moins propre à cette exécution, quand elle serait
juste et possible.

À l'égard de la manière de l'exécuter, je me trouvais l'esprit trop
borné pour en imaginer aucune sur le nombre infini de maisons de
jésuites répandues dans toutes les provinces de la domination du roi, et
le nombre immense de jésuites qui les remplissaient\,; que le tout à la
fois, comme avait fait le Maffei, était mathématiquement impossible\,;
que par parties, quels cris\,! quels troubles\,! quels mouvements dès
les premiers pas\,! Cette immensité de jésuites, leurs familles, leurs
écoliers, et les familles de ces écoliers, leurs pénitents, les
troupeaux de leurs retraites et de leurs congrégations, les sectateurs
de leurs sermons, leurs amis et ceux de leur doctrine, quel vacarme
avant qu'on en eût nettoyé la province par laquelle on aurait commencé,
et quand et comment achèverait-on dans toutes les provinces\,? Où
conduire ces exilés\,? Hors la frontière la plus prochaine,
répondra-t-on\,; mais qui les empêchera de rentrer\,? point de mer,
comme pour retourner en Sicile, ni de grande muraille comme à la Chine,
tout ouvert partout, et favorisés de ce nombre immense de tous états et
de tous lieux dont je viens de parler. C'est donc une chimère évidemment
impossible. Mais supposons-la pour un moment, non seulement faisable,
mais exécutée. Que dira la cour de Rome, dont les jésuites sont en
France les plus utiles instruments et les plus dévoués à ses prétentions
et à ses ordres\,? Que dira le roi d'Espagne, si dévot, si publiquement
jésuite, et qui est avec M. le duc d'Orléans comme chacun sait\,? Que
diront toutes les puissances catholiques, chez qui tous les jésuites ont
tant de crédit, et de qui presque toutes ils sont les confesseurs\,? Et
les peuples catholiques de toute l'Europe où par la chaire, le
confessionnal, les classes, les jésuites ont autant d'amis et de
partisans que ces mêmes moyens leur en donnent en France\,? Que diront
tous les ordres réguliers, peut-être jusqu'aux bénédictins, dominicains
et chanoines réguliers divers\footnote{Chanoines soumis à une règle
  monastique, comme les prémontrés, les antonins, les génovéfains, etc.
  Les chanoines réguliers furent institués par les conciles tenus à
  Rome, en 1059 et 1063. Ils s'établirent en France, à Saint-Victor de
  Paris, en 1119.}, les seuls peut-être d'entre les réguliers qui soient
ennemis des jésuites\,? Ne doit-on pas juger que tous frémiront d'un
coup qui peut les frapper à leur tour, si la fantaisie en prend\,;
qu'ils en craindront le menaçant exemple, et qu'ils se réuniront avec
tout ce qui se sentira, ou se croira intéressé à l'empêcher\,? et s'ils
en viennent à bout, quelle folie, quelle ignominie se sera-t-on si
gratuitement préparée, mais quel péril encore, et péril à ne plus
pouvoir espérer sûreté ni tranquillité, après s'être mis le dedans et le
dehors contre-soi avec ce qu'on appelle la religion à la tête\,! Je
conclus enfin que cette tentative, si bien concertée qu'elle pût être,
serait la perte de M. le duc d'Orléans, et un tel bouleversement que je
ne voyais pas comment ni quand on pourrait le calmer.

Mon discours fut plus étendu que je ne le rapporte, et je ne fus point
interrompu. Quand j'eus fini, je vis deux hommes étonnés et fâchés qui
ne purent répondre un seul mot à pas une des objections que je venais de
faire, et qui en même temps me déclarèrent l'un et l'autre que je ne les
avais point persuadés. Tous deux, en s'interrompant l'un l'autre,
revinrent au danger des jésuites en France pour le général de l'État et
de l'Église, et pour le particulier\,; moi à leur répéter que ce n'était
pas la question, mais la cause, les moyens et les suites, qu'ils avaient
ces trois choses à me prouver possibles et garanties. J'avais beau les
ramener, ils persistaient, le dirai-je\,? à aboyer à la lune. Leur peu
de succès avec moi, et l'heure indue pour un magistrat de regagner
Paris, nous sépara sans le moindre progrès fait de part ni d'autre. Je
sortis en même temps que le procureur général pour revenir chez moi,
noyé dans l'étonnement et la recherche de ce que le procureur général
pouvait avoir fait de son sens, de ses lumières, de sa sagesse, et
persuadé qu'ils étaient sur cette matière à délibérer ensemble quand
j'arrivai, à la manière subite dont le duc de Noailles m'en ouvrit le
propos, et dont il le remit au procureur général lorsqu'il nous vint
trouver en tiers. Je demeurai à bout sur le procureur général, qui
n'avait sûrement point de vues obliques, mais que le pouvoir du duc de
Noailles sur son esprit avait gagné, déjà ennemi personnel et
parlementaire de la société, et qui se laissa aller à la folie de son
ami, sans que des raisons aussi nettement décisives l'en pussent faire
revenir, quoiqu'il ne leur en pût opposer aucune, et c'est ce qui porta
mon étonnement jusqu'à en demeurer confondu.

Le lundi 19 août, la nuit fut également agitée, sans que Fagon voulût
trouver que le roi eût de la fièvre. Il eut envie de lui faire venir des
eaux de Bourbonne. Le roi travailla avec Pontchartrain, eut petite
musique chez M\textsuperscript{me} de Maintenon, déclara qu'il n'irait
point à Fontainebleau, et dit qu'il verrait la gendarmerie le mercredi
suivant de dessus son balcon. Il l'avait fait venir de ses quartiers
pour en faire la revue\,: ce ne fut que ce jour-là qu'il vit qu'il ne le
pourrait, et qu'il se borna à la regarder dans la grande cour de
Versailles par la fenêtre. Le mardi 20 août, la nuit fut comme les
précédentes. Il travailla le matin avec le chancelier\,; il ne voulut
voir que peu de gens distingués et les ministres étrangers à son dîner,
qui avaient, et ont encore, le mardi fixé pour aller à Versailles. Il
tint conseil de finances ensuite, et travailla après avec Desmarets
seul. Il ne put aller chez M\textsuperscript{me} de Maintenon, qu'il
envoya chercher. M\textsuperscript{me} de Dangeau et
M\textsuperscript{me} de Caylus y furent admises quelque temps après
pour aider à la conversation. Il soupa en robe de chambre dans son
fauteuil. Il ne sortit plus de son appartement, et ne s'habilla plus. La
soirée courte comme les précédentes. Fagon enfin lui proposa une
assemblée des principaux médecins de Paris et de la cour.

Ce même jour, M\textsuperscript{me} de Saint-Simon, que j'avais pressée
de revenir, arriva des eaux de Forges. Le roi entrant après souper dans
son cabinet l'aperçut. Il fit arrêter sa roulette, lui témoigna beaucoup
de bonté sur son voyage et son retour, puis continua à se faire pousser
par Bloin dans l'autre cabinet. Ce fut la dernière femme de la cour à
qui il ait parlé, parce que je ne compte pas M\textsuperscript{me}s de
Lévi, Dangeau, Caylus et d'O qui étaient les familières du jeu et des
musiques chez M\textsuperscript{me} de Maintenon, et qui vinrent chez
lui quand il ne put plus sortir. M\textsuperscript{me} de Saint-Simon me
dit le soir qu'elle n'aurait pas reconnu le roi, si elle l'avait
rencontré ailleurs que chez lui. Elle n'était partie de Marly pour
Forges que le 6 juillet.

Le mercredi 21 août, quatre médecins virent le roi, et n'eurent garde de
rien dire que les louanges de Fagon, qui lui fit prendre de la casse. Il
remit au vendredi suivant à voir la gendarmerie de ses fenêtres, tint le
conseil d'État après son dîner, travailla ensuite avec le chancelier.
M\textsuperscript{me} de Maintenon vint après, puis les dames
familières, et grande musique. Il soupa en robe de chambre dans son
fauteuil. Depuis quelques jours on commençait à s'apercevoir qu'il avait
peine à manger de la viande, et même du pain, dont toute sa vie il avait
très peu mangé, et depuis très longtemps rien que la mie, parce qu'il
n'avait plus de dents. Le potage en plus grande quantité, les hachis
fort clairs, et les œufs suppléaient, mais il mangeait fort
médiocrement.

Le jeudi 22 août, le roi fut encore plus mal. Il vit les quatre autres
médecins qui, comme les quatre premiers, ne firent qu'admirer les
savantes connaissances et l'admirable conduite de Fagon, qui lui fit
prendre sur le soir du quinquina à l'eau, et lui destina pour la nuit du
lait d'ânesse. Ne comptant plus dès la veille de pouvoir se mettre sur
un balcon pour voir la gendarmerie dans sa cour, il mit à profit pour le
duc du Maine jusqu'à sa dernière faiblesse. Il le chargea d'aller faire
la revue de ce corps d'élite en sa place, avec toute son autorité, pour
en montrer en lui les prémices aux troupes, les accoutumer de son vivant
à le considérer comme lui-même, et lui donner envers eux les grâces d'un
compte favorable et flatteur. C'est ce que ce faible échappé des Guise
et de Cromwell sut se ménager, mais comme il manquait absolument de leur
courage, la peur le saisit de ce qui pourrait lui arriver en cette
extrémité connue du roi, si M. le duc d'Orléans connaissait ses forces
naturelles, et s'avisait d'en faire usage. Il chercha donc un bouclier
qui le put mettre à couvert, et il ne lui fut pas difficile par
M\textsuperscript{me} de Maintenon de le trouver.

M\textsuperscript{me} de Ventadour, excitée par son ancien amant et ami
intime le maréchal de Villeroy, qui savait bien ce qu'il faisait, donna
envie à Mgr le Dauphin d'aller à cette revue. Il commençait à monter un
petit bidet, et il alla demander au roi la permission d'y aller. Le jeu
de cette comédie fut visible en ce que l'habit uniforme de capitaine de
gendarmerie se trouva tout fait pour M. le Dauphin, qui avait pris les
chausses depuis fort peu. Le roi trouva cette envie d'un enfant fort de
son goût, et lui permit d'y aller.

L'état du roi, qui n'était plus ignoré de personne, avait déjà changé le
désert de l'appartement de M. le duc d'Orléans en foule. Je lui proposai
d'aller à la revue, et sous prétexte d'honorer dans M. du Maine
l'autorité du roi même dont il était revêtu pour cette revue, de l'y
suivre en courtisan, comme il aurait fait le roi même, de lui répondre
sur ce ton s'il avait voulu s'en défendre, de s'attacher à lui malgré
lui, d'affecter de ne lui parler jamais que chapeau bas comme il aurait
fait au roi, et de le devancer de cinquante pas en approchant de ses
compagnies de gendarmerie pour l'y saluer à leur tête, et de le joindre
après, et le suivre chapeau bas dans leurs rangs, en même temps de
donner fréquemment le coup d'oeil à sa suite et aux troupes, de n'y
laisser pas ignorer le sarcasme par ses manières respectueusement
insultantes, et d'y montrer ce roi de carton pâmé d'effroi et
d'embarras. Outre le plaisir de lui marcher ainsi sur le ventre au
milieu de son triomphe, il y avait tout à gagner par l'impression de la
peur, et par montrer aux troupes, aux spectateurs, et par eux à la cour
et à la ville, quelle est la force de la nature sur l'usurpation, et
que, s'il ne s'opposait à rien pendant la vie du roi qui en était aux
derniers jours, il n'était pas pour laisser jouir ce bâtard des
avantages qu'il avait su se faire donner à son préjudice, et à celui du
droit et des lois. M. d'Orléans n'avait rien à craindre, le roi avait
fait tout ce qu'il avait pu par ses dispositions contre lui et pour ses
bâtards\,; personne n'en doutait, ni n'en pouvait douter, ni M. le duc
d'Orléans non plus. Rien donc à perdre dans cette conduite, dont même
l'extérieur, quelque ironique qu'il fût, n'aurait pu fournir aucune
plainte\,; et encore à qui\,? et qui eut pu faire ce Jupiter mourant\,?
et au contraire tout à gagner en intimidant le duc du Maine et les
siens, et se montrant, lui, tel qu'il devait être à toute la France. Je
voulais aussi qu'il s'y montrât nu et sans suite\,; que tout ce qui
voudrait se ramasser autour de lui, il le renvoyât avec un respect de
dérision à M. du Maine\,; que sur tout ce qui regarderait la revue, il
s'en expliquât comme le dernier particulier à qui on ferait trop
d'honneur d'en parler, et qui ne se sentirait pas en caractère d'y
répondre\,; que pour ses propres compagnies, il fit auprès du duc du
Maine le personnage d'un officier captant sa protection auprès du roi,
dans le compte qu'il lui en devait rendre, en même temps que lui-même
lui rendrait compte de ses compagnies, et lui en présenterait les
officiers en les faisant valoir comme il aurait fait au roi même, mais
avec un respect insultant et finement menaçant.

J'avoue que, s'il eût été possible, j'eusse acheté cher de pouvoir être
alors M. le duc d'Orléans pour vingt-quatre heures. Tel qu'était M. du
Maine, je ne sais s'il n'en serait pas mort de peur. La présence d'un
Dauphin de cinq ans ne devait rien déconcerter. Il n'était en âge que de
recevoir des respects, tout le reste demeurait au duc du Maine, et hors
de sa présence, même tous les respects, puisqu'il y tenait la place du
roi. Mais la faiblesse de M. le duc d'Orléans ne fut pas capable d'une
si délicieuse comédie. Il alla à la revue, il y examina ses compagnies,
il salua à leur tête Mgr le Dauphin, il s'approcha peu de M. du Maine
qui pâlit en le voyant, et dont l'embarras et l'angoisse frappa tout le
monde, qui le laissa pour accompagner toujours M. le duc d'Orléans, sans
qu'il y mît rien du sien. Tout ce qui se trouva à la revue se montra
indigné de la voir faire au duc du Maine, M. le duc d'Orléans présent\,;
qu'eut-ce été si ce prince eût eu la force de s'y conduire comme je l'en
avais pressé\,? il le sentit après, et il en fut honteux\,; je m'en
servis pour lui donner plus de courage. La gendarmerie même fut
indignée, et ne s'en cacha pas, quelque soin que le roi prît de publier
et de faire valoir, aux heures où il voyait encore le monde, aux
officiers de la gendarmerie les éloges et les merveilles du compte que
le duc du Maine lui avait rendu de ce corps.

Le public trouva cette commission fort étrange, et le duc du Maine ne
gagna rien à se l'être fait donner, quelques flatteries qu'il eût
employées envers ce corps pendant et après cette revue. Il voulut, dans
son extrême embarras, et qui fut visible à tout ce qui s'y trouva, en
faire les honneurs à M. le duc d'Orléans, qui se contenta de lui
répondre qu'il n'était venu que comme capitaine de gendarmerie, qui
n'accepta rien, et qui s'en retourna après avoir vu ses compagnies, et
avoir salué Mgr le Dauphin à leur tête. La gendarmerie fut aussitôt
après renvoyée dans ses quartiers. Ce fut là où M. le duc d'Orléans et
le duc du Maine sentirent les prémices de ce qui les attendait. Tout y
courut au premier, et laissa l'autre qui en demeura confondu\,; les
troupes mêmes furent frappées du contraste. Le public s'en expliqua
durement et librement, et trouva que cette fonction était due à M. le
duc d'Orléans, si {[}elle devait être faite{]} par un prince, ou par un
maréchal de France, ou un officier général distingué pour en rendre
simplement compte au roi.

Je me donnai en miniature de particulier le plaisir que M. le duc
d'Orléans n'avait osé prendre en prochain régent du royaume. J'allai
voir Pontchartrain chez qui je n'allais presque jamais, et j'y tombai
comme une bombe, chose toujours plus triste et plus fâcheuse pour la
bombe que pour ceux qui la reçoivent, mais qui pour cette fois ne le fut
que pour la compagnie, et me fit un double plaisir. Les ministres
étaient fort en peine de leur sort. La terreur du roi les retenait
encore, aucun d'eux n'avait osé se tourner vers M. le duc d'Orléans\,;
la vigilance du duc du Maine et la frayeur de M\textsuperscript{me} de
Maintenon les tenait de court, parce qu'il restait encore assez de vie
au roi pour les chasser, et qu'ils n'auraient pu en ce cas se flatter
d'être regardés par M. le duc d'Orléans comme ses martyrs, mais
seulement comme martyrs de leur tardive politique. Je voulus donc jouir
de l'embarras de Pontchartrain, et me donner le plaisir de me jouer à
mon tour de ce détestable cyclope.

Je le trouvai enfermé avec Besons et d'Effiat, mais ses gens, après un
instant d'incertitude, n'osèrent me refuser sa porte. J'entrai donc dans
son cabinet, où le premier coup d'œil m'offrit trois hommes assis si
proche les uns des autres, et leurs têtes ensemble, qui se réveillèrent
comme en sursaut à mon arrivée, avec un air de dépit que j'aperçus
d'abord, et qui se changea aussitôt en compliments qui tenaient du
désordre que mon importune présence leur causait. Plus je les vis
empêtrés et interrompus dans le petit conseil qu'ils tenaient, plus je
m'en divertis, et moins j'eus envie de me retirer, comme j'aurais fait
en tout autre temps. Ils l'espéraient, mais comme ils virent que je me
mis à parler de choses indifférentes, en homme qui ne songeait pas qu'il
les incommodait, Effiat fit sèchement la révérence, Besons aussitôt
après, et s'en allèrent.

Pontchartrain, qui jusqu'alors n'avait ni recueilli ni fait aucun cas de
Besons, avait réclamé leur parenté quand il sentit son besoin auprès de
M. le duc d'Orléans. Il en fit son patron, et Besons, que son
attachement à M. le duc d'Orléans avait fourré parmi ses officiers, et
qui s'était fait ami d'Effiat, l'avait mis dans les intérêts de
Pontchartrain. Dès qu'ils furent sortis, j'eus la malice de lui dire que
je croyais les avoir interrompus, et que j'aurais mieux fait de les
laisser. Pontchartrain, à travers les compliments, me l'avoua assez pour
me donner lieu à lui dire qu'il était là avec deux hommes bien en état
de le servir. L'agonie où il sentait sa fortune l'aveugla au point de ne
pas voir que je ne cherchais qu'à le faire parler pour me moquer de lui,
et d'oublier assez ses forfaits, et tout ce qui s'était passé entre lui
et moi, pour se flatter de ma visite, et me parler avec une sorte de
confiance ornée de respects à lui jusqu'alors inconnus. Je n'eus pas
même la peine de me l'attirer par des compliments vagues et des propos
de cour\,; il s'enfila de lui-même, me conta ses peines, ses
inquiétudes, son embarras, son apologie, enfin, à l'égard de M. le duc
d'Orléans, m'avoua qu'il avait eu recours à Besons, et par lui à
d'Effiat, vanta l'amitié et les bontés, car ce roi des autres se ravala
jusqu'à ce mot, qu'il recevait d'eux, et revint toujours à ses
inquiétudes, lardant par-ci par-là des demi-mots qui marquaient combien
il désirait ma protection, et combien il était embarrassé de n'oser tout
à fait me la demander.

Après m'être longtemps réjoui à l'entendre ramper de la sorte, je lui
dis que je m'étonnais qu'un homme d'esprit comme lui, qui avait tant
d'usage de la cour et du monde, pût s'inquiéter de ce qu'il deviendrait
après le roi qui en effet (le regardant bien fixement) n'en avait pas, à
ce qu'il paraissait, pour longtemps\,; qu'avec sa capacité et son
expérience dans la marine, dans laquelle il pouvait compter qu'il
n'était personne qui approchât de lui, M. le duc d'Orléans serait trop
heureux de le continuer dans une charge si nécessaire et si principale,
et dans laquelle un homme comme lui ne pouvait être succédé par personne
qui en eût la moindre notion. Il me parut que je lui rendais la vie,
mais comme il était fort prolixe, il ne laissait pas de revenir à ses
craintes, que je me plus diverses fois à appuyer à demi, à voir pâlir
mon homme, puis à le rassurer par ces mêmes discours qu'il était un
homme nécessaire dans sa place, duquel il n'était pas possible de se
passer, et qui par là, sûr de son fait, pouvait vivre en paix et n'avoir
besoin de personne. Cette savoureuse comédie que je me donnai dura bien
trois bons quarts d'heure. J'y eus grand soin de ne pas dire un seul mot
qui sentît l'offre de service, l'avis ni l'amitié passée\,; je n'eus que
la peine de lâcher de fois à autre quelques mots pour entretenir son
flux de bouche, et j'y appris que Besons et d'Effiat s'étaient rendus
ses protecteurs. J'étais journellement assuré par M. le duc d'Orléans
qu'il ne le laisserait pas en place, en déclarant le choix des membres
du conseil de marine, et je m'applaudissais ainsi de ma secrète dérision
en face\,; et de me voir si sûr et si près de lui tenir la parole dont
j'ai parlé en son temps.

Desmarets, qui ne se sentait pas mieux assuré que Pontchartrain, se
souvint alors que j'étais au monde. Louville, gendre du frère de
M\textsuperscript{me} Desmarets, me vint parler pour lui. Il était,
comme on l'a vu, de tout temps mon ami intime\,; il n'ignorait pas la
conduite que j'avais eue avec Desmarets, ni ses procédés avec moi. Il
m'étala ses respects, ses regrets, ses désirs, et les appuya de son
esprit et de son éloquence. Je ne m'ouvris point avec lui de l'expulsion
de Desmarets résolue, mais je lui dis qu'il était désormais trop tard de
se repentir à mon égard, et nettement que Desmarets était un homme dont
je m'étais bien su passer jusqu'alors, et dont je ne voulais ouïr parler
de ma vie. Cette éconduite fut suivie d'une lettre de la duchesse de
Beauvilliers, pressante au dernier point, qui parlait aussi au nom de la
duchesse de Chevreuse, et qui, pour dernier motif, voulait me toucher en
faveur de Desmarets par sa capacité pour les finances, et par les
besoins de l'État à l'égard d'une partie si principale. Je répondis tout
ce que je pus de plus respectueux, de plus dévoué, de plus soumis, pour
faire passer le refus inébranlable sur Desmarets, sans m'expliquer
d'ailleurs sur ce qu'il avait à craindre ni à espérer, tellement que la
fermeté de ces deux refus me délivra de sollicitations nouvelles, et put
augmenter les frayeurs du brutal et insolent ministre, et les regrets à
mon égard de sa folle ingratitude.

Ce même jour, jeudi 22 août, que le duc du Maine fit au lieu du roi la
revue de la gendarmerie, le roi ordonna à son coucher au duc de La
Rochefoucauld de lui faire voir le lendemain matin des habits pour
choisir celui qui lui conviendrait en quittant le deuil d'un fils de
M\textsuperscript{me} la duchesse de Lorraine, qu'on appelait le prince
François, qui avait vingt-six ans et les abbayes de Stavelo et de
Malmédy. On voit ici combien il y avait qu'il ne marchait plus, qu'il ne
s'habillait plus même les derniers jours qu'il se fit porter chez
M\textsuperscript{me} de Maintenon, qu'il ne sortait de son lit que pour
souper en robe de chambre, que les médecins couchaient dans sa chambre
et dans les pièces voisines, enfin qu'il ne pouvait plus rien avaler de
solide, et il comptait encore, comme on le voit ici, de guérir,
puisqu'il comptait de s'habiller encore, et qu'il voulut se choisir un
habit pour quand il le pourrait mettre. Aussi voit-on la même suite de
conseils, de travail, d'amusements\,; c'est que les hommes ne veulent
point mourir, et se le dissimulent tant et si loin qu'il leur est
possible.

Le vendredi 23 août se passa comme les précédents. Le roi travailla le
matin avec le P. Tellier, puis n'espérant plus pouvoir voir la
gendarmerie, il la renvoya dans ses quartiers. La singularité de ce
jour-là fut que le roi ne dîna pas dans son lit, mais debout, en robe de
chambre. Il s'amusa après avec M\textsuperscript{me} de Maintenon, puis
avec les dames familières. Pendant tous ces temps-là il faut se souvenir
que les courtisans un peu distingués entrèrent à ses repas, ceux qui
avaient les grandes ou les premières entrées à sa messe, et à la fin de
son lever, et au commencement de son coucher, M. le duc d'Orléans comme
les autres, et que le reste des journées que les conseils ou les
ministres laissaient vide, était rempli, comme quand il était debout,
par ses bâtards, bien plus M. du Maine que le comte de Toulouse, et
souvent M. du Maine y demeurait avec M\textsuperscript{me} de Maintenon
seule, et quelquefois avec les dames familières, entrant et sortant
toujours, comme à son ordinaire, par le petit degré du derrière des
cabinets, en sorte qu'on ne le voyait jamais entrer ni sortir, ni le
comte de Toulouse\,; M\textsuperscript{me} de Maintenon et les dames
familières toujours par les antichambres\,: les valets intérieurs
étaient, comme à l'ordinaire, avec le roi, quand il n'y avait que ses
bâtards ou personne, mais peu, lorsque M. du Maine était seul avec lui.

Il a fallu conduire la maladie du roi jusqu'à la veille de son
extrémité, avec ce qui s'est passé alors, sans en faire perdre de vue la
suite par un trop long récit qui y fût étranger, pour y conserver
l'ordre des choses. La même raison veut surtout que tout ce qui
appartient à son extrémité jusqu'à sa fin soit encore moins
interrompu\,: c'est ce qui m'engage à placer ici tout de suite ce qui
n'avait pu l'être en sa place précise sans déranger cette suite et la
netteté que je m'y suis proposée, pour en conserver l'ordre sans
l'altérer. Il faut maintenant retourner un peu sur ses pas, et aller
tout de suite un peu au delà du jour où nous en sommes, pour reprendre
après cette espèce de journal où nous le laissons présentement pour ne
le plus interrompre jusqu'à la mort du roi.

\hypertarget{chapitre-xiv.}{%
\chapter{CHAPITRE XIV.}\label{chapitre-xiv.}}

1715

~

{\textsc{Misère des ducs.}} {\textsc{- Duc et duchesse du Maine excitent
avec plein succès les gens de qualité et soi-disant tels contre les
ducs.}} {\textsc{- Abomination du duc de Noailles.}} {\textsc{- Il me
propose de le faire faire premier ministre.}} {\textsc{- Proposition du
duc de Noailles d'une nouveauté qu'il soutient contre toutes mes
raisons.}} {\textsc{- Le duc de Noailles m'impute la proposition que
j'avais si puissamment combattue, et soulève tout contre moi.}}
{\textsc{- Étrange embarras de Noailles avec la duchesse de
Saint-Simon.}} {\textsc{- J'apprends la scélératesse de Noailles.}}
{\textsc{- Monstrueuse ingratitude de Noailles.}} {\textsc{- Son affreux
et profond projet.}} {\textsc{- Courte réflexion.}} {\textsc{- J'éclate
sans mesure contre Noailles, qui plie les épaules et suit sa pointe
parmi la noblesse et {[}qui{]} cabale des ducs contre moi.}} {\textsc{-
Je me raccommode avec le duc de Luxembourg\,; son caractère.}}
{\textsc{- Suites de l'éclat.}} {\textsc{- Bassesse et désespoir de
Noailles.}} {\textsc{- Sa conduite à mon égard et la mienne au sien.}}
{\textsc{- Noailles n'oublie rien, mais inutilement, pour me fléchir.}}
{\textsc{- Noailles, depuis la mort de M. le duc d'Orléans, aussi
infatigable, et inutilement, à m'adoucir.}} {\textsc{- Le désir extrême
de raccommodement des Noailles fait enfin le mariage de mon fils aîné.}}
{\textsc{- Raccommodement entre Noailles et moi, et ses légères
suites.}}

~

La noire politique du duc et de la duchesse du Maine ne s'était pas
bornée à se rassurer contre les ducs par les suites de la cruelle
affaire du bonnet qu'ils avaient exprès suscitée, conduite et terminée
de la manière qui a été expliquée. Elle avait donné lieu à plusieurs
ducs de se contenir ensemble, et à veiller à ce qu'aucun ne vit le
premier président. M. d'Aumont et fort peu d'autres se démanchèrent. Le
procédé de celui-là fâcha sans étonner\,: toute sa conduite n'avait été
équivoque que pour qui n'avait pas voulu avoir des yeux, et ressemblait
trop à celle de toute sa vie pour avoir pu s'y méprendre. La vérité est
que les ducs ne paraissaient pas propres à se soutenir sur rien depuis
longtemps.

L'esprit d'intérêt particulier, de mode, de servitude, une ignorance
profonde et honteuse, incapacité de tout concert entre eux, le sot bel
air de faire les honneurs de ce qui n'appartient à nul particulier
d'entre eux, et de s'y croire montrer supérieur en en faisant sottement
litière à tout ce qui en profite en se moquant d'eux, l'habitude de leur
continuelle décadence, étaient à tout des obstacles pour eux, et des
raisons à chacun pour leur tirer des plumes. On a vu, et on l'exposera
encore mieux, quel fut toujours le roi à cet égard, en général, pour
tout ce qui ne fut ni bâtard ni ministre\,: ainsi large facilité contre
les ducs, jusque par eux-mêmes. Le nombre, sans cesse augmenté et peu
choisi, et la malapprise jeunesse de plusieurs ducs par démission de
leurs pères, augmentait l'inconsidération et la jalousie\,; et ces ducs,
qui ne se soutenaient ni ne songeaient pas seulement à être soutenus, ne
savaient que s'avilir tous les jours. Quoique les personnes sans titre
et souvent de la première qualité fissent sans cesse des alliances fort
basses, celles de cette sorte que faisaient les ducs semblaient les
mêler davantage\,? et marquer plus par la distinction de leur rang qui
irritait dans les duchesses de cette sorte les dames de qualité\,;
celles surtout qui l'étaient aussi par elles-mêmes s'en rendaient plus
libres à hasarder avec ces duchesses à ne leur rendre pas ce qui leur
était dû, et réciproquement celles-ci embarrassées et plus souples à
glisser et à supporter.

M. et M\textsuperscript{me} du Maine, qui n'ignoraient pas cette
situation, ni que l'ignorance et la sottise ne fût aussi profonde et
aussi vastement répandue parmi les gens sans titre que parmi les ducs,
s'appliquèrent à en profiter et à saisir l'occasion de l'éclat de la fin
de l'affaire du bonnet, pour encourager les gens non titrés contre les
ducs, et brouiller ceux-ci avec le même éclat, qui avait si bien réussi
à l'égard du parlement. Le duc du Maine suppléait aux vertus par les
talents les plus noirs et les plus ténébreux\,; il en avait fait de
continuelles épreuves. On a vu jusqu'à quel point il s'y était surpassé
pendant la campagne de Lille. Eh\,! plût à Dieu qu'il s'y fût borné\,!
Après ces coups de maître, son art pouvait-il trouver quelque chose de
difficile\,? Il le mit en oeuvre par le même soin et les mêmes
émissaires qui l'y avaient si bien servi, et qui, de nouveau, se
surpassèrent ainsi que lui-même et la duchesse du Maine.

D'abord on se contenta de sonder, de jeter des propos, de cultiver,
après de rassembler, mais dans les ténèbres. Il fallait d'abord infatuer
un nombre de sots glorieux et ignorants, pour s'en servir à en recruter
d'autres, attirer des personnes de cette espèce de naissance distinguée,
piquer ceux du commun de la vanité de penser comme celles-là, et de
l'honneur de s'unir à elles par un intérêt dont la communauté les
égalait à eux, faire en même temps que les gens de qualité souffrissent,
puis se prêtassent à ce difforme assemblage, par leur faire sentir la
nécessité du nombre pour réussir par le fracas et en les flattant après
le succès d'une séparation d'alliage qui ne se pourrait, disait-on,
refuser après le besoin passé, et par ces ruses, faire un groupe où
toutes sortes de gens pussent entrer, se donner le beau nom collectif de
noblesse, et, par un très grand nombre si bien dupé et masqué, causer un
si grand bruit, que les ducs ne pussent penser qu'à la défense, bien
loin de pouvoir attaquer les bâtards réunis par la première et la
seconde adresse à la robe et à la soi-disant noblesse contre eux, et en
état avec cette double multitude de faire la loi au régent\,; {[}ce{]}
qui fut la double vue du duc et de la duchesse du Maine. Ce crayon
suffira pour le présent\,; il y aura lieu bientôt de le changer en
tableau, quand l'usage de cette folle cohue sera devenu plus dangereux
pour le gouvernement. C'en est assez ici pour expliquer ce qu'en sut
faire le duc de Noailles, non moins bon ouvrier, et en même genre et
goût que le duc du Maine. On ne peut mieux exalter son infernal talent,
ni faire en même temps une comparaison plus exactement juste.

J'ai dit plus haut que le duc de Noailles m'avait fait une proposition
absurde que j'avais fort rejetée, et qu'il n'était pas temps
d'expliquer\,: c'est maintenant ce qu'il s'agit de faire. C'était qu'à
la mort du roi tout ce qui se trouverait de ducs à la cour allassent
ensemble saluer le nouveau roi à la suite de M. le duc d'Orléans et des
princes du sang. Je ne sais si dès lors il était informé du mouvement
qui se préparait parmi la noblesse\,; je ne l'étais point encore, et le
secret en était alors entier. Il revint souvent à la charge là-dessus
sans avoir pu m'ébranler ni répondre aux raisons que je lui alléguai, et
qui seront mieux plus bas en leur place. Il en parla à d'autres ducs
pour essayer de m'ébranler, et se servit pour cela des diverses petites
assemblées, qui, à mesure que le roi baissait, se faisaient chez divers
ducs sur la conduite à tenir au parlement sur le bonnet, et qui se
référaient des unes aux autres par quelqu'un de ces diverses petites
assemblées. Il s'en tenait aussi chez moi, indépendamment desquelles mon
appartement était toujours assez rempli d'amis particuliers, curieux de
tout ce qui se passait d'un moment à l'autre en des temps si vifs et si
intéressants, et bientôt je fus averti que les entours de mon
appartement étaient assiégés nuit et jour de valets de chambre et de
laquais de toutes sortes de personnes de la cour, pour voir qui y
entrait et sortait, et pénétrer ce qui s'y passait, autant que ces
dehors le pouvaient permettre.

Un soir d'assez bonne heure que je montai chez le duc de Noailles que je
trouvai seul, il se mit à raisonner avec moi pour tâcher de me déprendre
du projet de la convocation des états généraux, et à travers mille
louanges d'un si beau dessein, dont il sentait pour lui les entraves, et
combien il l'éloignerait du but qu'il s'était proposé dans sa passion
pour l'administration des finances, il tâcha d'en présenter les embarras
et les difficultés. Il s'échappa après à essayer de me faire sentir le
danger de la multitude avec un prince tel qu'était M. le duc d'Orléans,
puis l'avantage de la solitude avec lui. Il bavarda longtemps sans dire
grand'chose\,: peu à peu s'échauffant comme exprès dans son harnais,
mais possédant toute son âme, ses paroles et jusqu'à ses regards\,:
«\,Vous n'avez pas voulu, me dit-il, des finances (M. le duc d'Orléans
le lui avait dit), vous ne vouiez vous charger directement de rien\,;
vous avez raison\,: vous vous réservez pour être de tout, et vous
attacher uniquement à M. le duc d'Orléans\,: au point où vous êtes avec
lui, vous ne sauriez mieux faire\,; en nous entendant bien vous et moi,
nous en ferons tout ce que nous voudrons\,; mais pour cela, ajouta-t-il,
ce n'est pas assez des finances, il me faut les autres parties\,; il ne
faut point que nous ayons à compter avec personne.\,»

J'écoutais avec un profond étonnement une ouverture si personnelle, si
démasquée, si peu mesurée sur M. le duc d'Orléans et sur le bien de
l'État, et je pointais mes oreilles et mon entendement à pénétrer où il
voulait se conduire par de si étranges propos, lorsqu'il me mit hors du
soin de la recherche. «\,Des états généraux, poursuivit-il, c'est un
embrouillement dont vous ne sortiriez point\,; j'aime le travail, je
vous le dirai franchement\,; c'est une pensée qui m'est venue, je la
crois la meilleure\,; encore une fois, agissons de concert,
entendons-nous bien, faites-moi faire premier ministre, et nous serons
les maîtres. --- Premier ministre\,!» interrompis-je avec l'indignation
que son discours m'avait donnée, que j'avais contenue, et que cette fin
combla\,: «\,Premier ministre\,! monsieur, je veux bien que vous sachiez
que s'il y avait un premier ministre à faire, et que j'en eusse envie,
ce serait moi qui le serais, et que je pense aussi que vous ne vous
persuadez pas que vous l'emportassiez sur moi\,; mais je vous déclare
que tant que M. le duc d'Orléans m'honorera de quelque part en sa
confiance, ni moi, ni vous, ni homme qui vive ne sera jamais premier
ministre, dont je regarde la place et le pouvoir comme le fléau, la
peste, la ruine d'un État, l'opprobre et le geôlier d'un roi ou d'un
régent qui se donne ou se souffre un maître, duquel, pour tout partage,
il n'est plus que l'instrument et le bouclier.\,» J'ajoutai encore
quelques mots à cette trop véritable et naïve peinture, les yeux
toujours collés sur mon homme, sur le visage et toute la contenance
duquel l'excès de l'embarras, du dépit, du déconcertement était peint,
et néanmoins assez maître de lui-même pour soutenir une apparente
tranquillité, jusqu'à me répondre qu'il n'insistait point, d'un air le
plus détaché, le plus indifférent\,; qu'il avouait que cette pensée lui
était venue et lui avait paru bonne.

On peut juger qu'après cela la conversation languit et ne dura qu'autant
que nous pûmes nous séparer honnêtement et nous délivrer d'un
tête-à-tête devenu si pesant à tous les deux. On doit penser aussi que
mes réflexions furent profondes. Elles étaient pourtant bien éloignées
encore de ce que l'on va voir et qu'il n'est pas temps d'interrompre. M.
de Noailles me vit dès le lendemain, et toujours comme s'il n'eût pas
été question entre nous du premier ministère. Nous vécûmes quelques
jours de la sorte, qui gagnèrent les derniers jours du roi, car il en
vécut encore trois depuis ce que je vais raconter.

J'ai déjà dit que l'état désespéré et pressant du roi avait engagé les
ducs à voir entre eux, par petites assemblées particulières sans bruit,
quelle serait leur conduite sur l'affaire du bonnet qui s'allait
nécessairement présenter lorsqu'ils iraient au parlement pour la
régence, et qu'on se référait des uns aux autres ce qui se passait en
ces petites assemblées. Sur les six ou sept heures du soir, le duc de
Noailles vint dans ma chambre, où Mailly, archevêque de Reims, les ducs
de Sully, La Force, Charost, je ne sais plus qui encore, et le duc
d'Humières, quoiqu'il ne fût pas pair, traitions cette matière depuis
peu de moments qu'ils étaient arrivés. On continua avec le duc de
Noailles, qui ne dit pas grand'chose, et qui presque incontinent
interrompit l'affaire du bonnet, et proposa la salutation du roi futur
comme il me l'avait expliquée. J'en fus d'autant plus surpris qu'après
m'en avoir importuné sans cesse, il y avait plus de quinze jours qu'il
ne m'en parlait plus, et que je le croyais rendu à mes raisons,
puisqu'il avait cessé d'insister et de m'en parler. Je lui en témoignai
mon étonnement et combien j'étais éloigné de goûter une nouveauté de
cette nature.

Il faut remarquer que les mouvements de la noblesse dont j'ai parlé
éclataient fortement alors depuis quelques jours, et faisaient la
nouvelle et un sujet principal de toutes les conversations. M. de
Noailles insista, m'interrompit, prit le ton d'orateur, l'air
d'autorité, se dit appuyé de l'avis des ducs qui s'étaient vos chez le
maréchal d'Harcourt, et, à force de poumons beaucoup plus forts que les
miens, mena la parole, et toujours étouffant la mienne. De colère et
d'impatience je montai sur le gradin de mes fenêtres et m'assis sur
l'armoire, disant que c'était pour être mieux entendu, et que je voulais
aussi parler à mon tour. Je m'exprimai avec tant de feu, que ces
messieurs firent taire Noailles qui toujours voulait continuer, qui
m'interrompit d'abord une ou deux fois, et à qui j'imposai à la fin, en
lui déclarant que je voulais être entendu, et que nous n'étions pas là
pour être devant lui à plaît-il maître. Ces messieurs voulurent
m'écouter et l'obligèrent à me laisser parler.

Je leur dis que ce que le duc de Noailles proposait était une nouveauté
dont on ne trouvait pas la moindre trace, ni dans rien qui fût écrit de
l'avènement de pas un roi à la couronne, ni dans la mémoire d'aucun
homme dont pas un n'avait jamais parlé de rien de semblable à
l'avènement de Louis XIV à la couronne\,; que cette première salutation
se faisait toujours sans ordre, à mesure que chacun arrivait, plus tôt
ou plus tard, à la différence de l'hommage qui quelquefois s'était rendu
au premier lit de justice\,; mais qu'en cette première salutation on ne
voyait pas que les princes du sang même eussent jamais affecté de
l'aller faire ensemble\,; que d'entreprendre de le faire ne pouvait rien
acquérir aux ducs\,; qu'au mieux, il demeurerait qu'ils auraient salué
le roi de la sorte, ce qui ne s'étant jamais fait en cérémonie et ne s'y
faisant la même par nuls autres, ne tiendrait lieu de rien aux ducs\,;
qu'ils paraîtraient seulement les plus diligents, dont ils ne tireraient
nul avantage sur les princes étrangers, puisqu'il n'y avait jamais eu en
cette occasion de cérémonie, ni sur les gens de qualité, tant par cette
raison que par celle qu'ils n'avaient jamais été en nulle compétence
avec eux en rien, ni prétendu quoi que ce soit sur eux\,; que n'y ayant
point de cérémonie en cette première salutation, à la différence de
l'hommage quelquefois rendu au premier lit de justice, il n'y en aurait
aussi rien d'écrit, par conséquent rien qui pût faire passer cette
salutation en usage, encore moins en avantage, et qui ne pourrait en
mériter le nom, par conséquent que rien ne pouvait appuyer cette
proposition\,; qu'en même temps qu'on n'y trouvait que du vide à
acquérir, elle pouvait devenir fort nuisible dans l'effervescence qui
éclatait parmi les gens de qualité et non même de qualité à l'égard des
ducs, semée et fomentée par le duc et la duchesse du Maine, qui se
sauraient bien servir d'une nouveauté qu'ils feraient passer pour une
entreprise\,; que la noblesse prendrait aisément à cet hameçon,
s'offenserait de ce que les ducs étant allés ensemble, sans que cela se
fût jamais pratiqué, auraient voulu non seulement faire bande à part,
mais corps à part de la noblesse\,; que ceux à qui je parlais
n'ignoraient pas que l'odieux de cette idée de corps à part commençait à
y être semé, à être imputé aux ducs avec une fausseté même sans
apparence, mais avec une malignité et un art qui y suppléait\,; que le
meilleur moyen de la confirmer était d'y donner cette occasion, qui,
tout éloignée qu'elle en était, serait montrée, donnée et reçue de ce
côté-là\,; que le parlement ne demanderait pas mieux que de fasciner la
noblesse avec ces prestiges\,; que l'intérêt du parlement, le même en
cela que celui de M. et de M\textsuperscript{me} du Maine, était de la
séparer et de la brouiller avec les ducs\,; que c'était à ceux-ci à
sentir combien il était du leur d'être unis à la noblesse, leur corps et
leur ordre commun\,; qu'occupés de plus forcément à l'affaire du bonnet,
ils n'avaient pas besoin d'ennemis nouveaux et en si prodigieux
nombre\,; qu'enfin à comparer le néant de l'avantage de cette salutation
avec les inconvénients infinis et durables qu'il entraînerait et qu'il
était évident par les dispositions présentes qu'il ne pouvait manquer
d'entraîner, je ne comprenais pas qu'on pût balancer un instant.

Je donnai encore plus de force et d'étendue à ce que je rapporte ici en
raccourci. Noailles répliqua, cria, se débattit, soutint qu'il n'y avait
rien que de sûr dans ce qu'il proposait, rien que de faible dans ce qui
était objecté, et sans avoir pu articuler une seule raison, même
apparente, ce fut une impétuosité de paroles soutenue d'une force de
voix qui entraîna les autres comme d'effroi sans les persuader. Je
repris la parole à diverses reprises\,; et voyant enfin que cela
dégénérait en dispute personnelle, où l'étourdissement des autres les
empêchait de montrer grande part, je les attestai de ma résistance et du
refus net, ferme, précis de mon consentement\,; j'ajoutai que je ne me
séparerais point de mes confrères, mais que j'espérais que ceux à qui on
en parlerait seraient plus heureux que moi à leur faire faire d'utiles
et de salutaires réflexions, et je finis tout à fait hors de voix par
protester de tous les inconvénients infinis et très suivis que j'y
voyais et que je déplorais par avance.

J'avais représenté au duc de Noailles dès les premières fois qu'il
m'avait fait cette proposition tête à tète, outre les raisons qu'on
vient de voir, qu'il fallait toujours considérer un but principal que
rien ne devait faire perdre de vue, et n'y pas mettre des obstacles si
aisés à éviter\,; que ce but était de tirer la noblesse en général de
l'abaissement et du néant où la robe et la plume l'avaient réduite, et
pour cela la mettre dans toutes les places du gouvernement qu'elle
pouvait occuper par son état, au lieu des gens de robe et de plume qui
les tenaient, et peu à peu l'en rendre capable, et lui donner de
l'émulation\,; d'étendre ses emplois, et de la relever de la sorte dans
son être naturel\,; que pour cela il fallait être unis, s'entendre,
s'aider, fraterniser, et ne pas jeter de l'huile sur un feu que M. et
M\textsuperscript{me} du Maine excitaient sans cesse, car dès lors il
paraissait, parce qu'ils comprenaient que leur salut consistait à
brouiller tous les ordres entre eux, surtout celui de la noblesse avec
elle-même\,; comme le salut de la noblesse consistait en son union entre
elle, à laquelle on ne devait cesser de travailler\,; que rien n'était
si ignorant, si glorieux, si propre à tomber dans toutes sortes de
panneaux et de pièges que cette noblesse, que par noblesse j'entendais
ducs et non-ducs\,; que les ducs ne devaient songer qu'à découvrir à
ceux qui n'étaient pas ducs ces panneaux et ces pièges\,; que pour le
faire utilement, il en fallait être aimés, et puisqu'en effet il
s'agissait d'un intérêt commun, dans un moment de crise dont on pouvait
profiter pour la remettre en lustre, et qui, manqué une fois, ne
reviendrait plus, il ne fallait pas tenter leur ignorance, leur vanité,
leur sottise par une nouveauté qui, à la vérité, ne leur nuisait en
rien, puisqu'en aucune occasion la noblesse non titrée ne pouvait être
et n'avait jamais été en égalité avec la noblesse titrée, moins encore
la précéder, mais qui étant nouveauté, et dans les circonstances
présentes de l'égarement de bouche que M. et M\textsuperscript{me} du
Maine soufflaient avec tant d'art et si peu de ménagement, il était de
la prudence d'éviter toutes sortes de prétextes et d'occasions dont la
noblesse non titrée se pouvait blesser, quelque mal à propos que ce fût,
et ne songer qu'à relever les ducs et elle tout ensemble, travailler à
un rétablissement commun qui, peu à peu, rendant à chacun sa
considération, remettrait chacun en sa place, ouvrirait les yeux à tous,
et ferait sentir à la noblesse non titrée la malignité des pièges et des
panneaux qu'on lui aurait tendus, l'ignorance de son propre intérêt,
combien il en était d'être unie aux ducs\,; que tous ne pouvant être
ducs, mais le pouvant devenir, chercher à abattre les distinctions des
ducs était vouloir abattre sa propre ambition, puisque cette dignité en
était nécessairement le dernier période, et qu'en cette différence de
ceux qui avaient ou qui n'avaient pas de dignité, la France était
semblable à tous les royaumes, républiques et États de l'univers où il y
avait toujours eu des dignités et des charges\,; des gens qui n'en
avaient pas, quoique quelquefois d'aussi bonne et de meilleure maison
que ceux qui avaient des charges ou des dignités, avec toutefois grande
différence de rang et de distinction entre ceux qui en ont et ceux qui
n'en ont pas, ce qui mettait les uns au-dessus des autres sans que
personne s'en fût jamais blessé, et sans quoi le roi et ses sujets
seraient sans récompense à donner ni à recevoir, et toute émulation
éteinte, sinon médiocre et personnelle uniquement.

Tant de raisons, et {[}qui{]} à chaque fois que le duc de Noailles me
parla ne trouvèrent en lui aucune réplique, mais un enthousiasme de
sécurité et d'entêtement, auraient persuadé l'homme le moins éclairé et
le moins raisonnable, et je me flattais enfin d'y avoir réussi, parce
qu'il y avait plus de quinze jours qu'il avait tout à fait cessé de me
parler de cette folie, lorsqu'au moment que j'avais lieu de m'y attendre
le moins, il vint chez moi, en apparence sur le bonnet, en effet pour
cette scène qu'il avait préparée\,; c'est que rien ne persuade qui met
son plus cher intérêt à ne l'être ou à ne le paraître pas. On va voir
qu'il ne pensa jamais sérieusement à cette nouveauté, qu'il n'en avait
parlé à aucun autre duc que cette fois dans ma chambre, que la pièce
n'était jouée que pour moi, et l'usage pour lequel il l'avait imaginée.
Le duc de Noailles étant sorti, j'en dis encore mon avis à ceux qui
étaient dans ma chambre qui ne purent nier que je n'eusse toute la
raison possible, et qui de guerre lasse, parce que la conférence avait
été longue et infiniment vive, s'en allèrent. Plein de la chose, je
passai dans la chambre de M\textsuperscript{me} de Saint-Simon à qui je
contai ce qui venait de se passer, et avec qui je déplorai une démence
si parfaitement inutile à réussir, et dont les suites deviendraient
aussi pernicieuses.

Les ducs qui s'étaient trouvés dans ma chambre, et qui ne faisaient que
d'en sortir, n'eurent pas le temps de parler à aucun autre duc de ce qui
avait fait chez moi cette manière de scène. Dès ce moment cette belle
idée de salutation du roi se répandit en prétention, vola de bouche en
bouche. Coetquen, beau-frère de Noailles, et fort lié avec lui, quoique
fort peu avec sa soeur, courut le château, ameutant les gens de qualité
qui, comme je l'avais prévu et prédit, prirent subitement le tour et le
ton que j'avais annoncés, tellement que le soir même ce fut un grand
bruit qui se fomenta toute la nuit en allées et venues, et dont Paris
fut incontinent informé.

Outre l'affluence que l'extrémité du roi, la curiosité, les divers
intérêts, l'attente de ce qui allait suivre ce grand événement, attirait
à Versailles, ce bruit de la salutation y amena encore une infinité de
monde, et les plus petits compagnons s'empressèrent et s'honorèrent
d'augmenter le vacarme pour s'agréger aux gens de qualité, qui le
souffraient par ne s'en pouvoir défaire, et dans la fougue d'augmenter
le tumulte par le nombre. Le tout ensemble s'appela la noblesse, et
cette noblesse pénétrait partout par ses cris contre les ducs. Ceux-ci,
qui à l'exception de ceux qui s'étaient trouvés dans ma chambre
n'avaient pas ouï dire un mot de cette salutation du roi, n'entendirent
que lentement et à peine de quoi il s'agissait, qui, partie de timidité
de cet ouragan subit, partie de pique de n'avoir point été consultés, se
mirent aussi à déclamer contre leurs confrères. Mais ces confrères qu'on
ne nommait point, et contre qui l'animosité devenait si furieuse et si
générale, ne demeurèrent pas longtemps en nom collectif. Saint-Herem le
premier, plusieurs autres après, vinrent avertir M\textsuperscript{me}
de Saint-Simon que tout tombait uniquement sur moi, comme sur le seul
inventeur et auteur du projet de cette salutation, dont l'autorité
naissante avait entraîné un petit nombre de ducs malgré eux, à l'insu
des autres. Ces messieurs ajoutèrent à M\textsuperscript{me} de
Saint-Simon que je n'étais pas en sûreté dans une émotion si générale et
si furieuse, et qu'elle ferait sagement d'y prendre garde. Sa surprise
fut d'autant plus grande qu'elle n'ignorait rien de tout ce qui s'était
passé là-dessus entre Noailles et moi. Mais elle monta au comble
lorsqu'elle apprit du même Saint-Herem, et de plus de dix autres encore
et pour l'avoir ouï de leurs oreilles, que c'était Noailles qui
soufflait ce feu, qui me donnait pour l'auteur et le promoteur unique
pour cette salutation, et soi-même pour celui qui s'y était opposé de
toutes ses forces. Ce dernier avis fut donné et confirmé à la duchesse
de Saint-Simon vers le soir de la surveille de la mort du roi, laquelle
se fit bien expliquer et répéter qu'ils l'avaient eux-mêmes entendu de
la bouche de Noailles, qui allait le semant partout lui-même, et par
Coetquen et d'autres émissaires.

Le hasard fit que le lendemain matin elle rencontra le duc de Noailles
dans la galerie, qui était lors remplie à toute heure de toute la cour,
où il passait avec le chevalier depuis duc de Sully. Elle l'arrêta et le
tira dans l'embrasure d'une fenêtre. Là, elle lui demanda d'abord ce que
c'était donc que tout ce bruit contre les ducs. Noailles voulut glisser,
dit que ce n'était rien, et que cela tomberait de soi-même. Elle le
pressa, et lui ne cherchait qu'à se dépêtrer\,; mais, à la fin, après
lui avoir déduit en peu de mots l'excès de ces cris et de ces mouvements
publics, pour lui faire sentir qu'elle en était bien instruite, elle lui
témoigna sa surprise d'apprendre qu'ils tombaient tous sur moi.
Là-dessus Noailles s'embarrassa, et l'assura qu'il ne l'avait pas ouï
dire\,; mais M\textsuperscript{me} de Saint-Simon lui répondant qu'il
devait savoir mieux que personne qui était l'auteur et le promoteur de
ce projet de salutation du roi, et qui le contradicteur, par ce qui
s'était passé encore la surveille là-dessus dans ma chambre. Noailles
l'avoua, tout comme la chose a été ici racontée, et qu'il était vrai que
c'était lui qui l'avait proposé, et que je m'y étais toujours opposé, et
lui toujours persévéré. Alors M\textsuperscript{me} de Saint-Simon lui
demanda pourquoi donc il s'en excusait et me donnait pour l'auteur et le
promoteur de cette invention. Noailles, interdit et accablé, balbutia
une faible négative. Il essuya tout de suite de courts, mais de cruels
reproches de tout ce qu'il me devait, et de la noire et perfide calomnie
dont il m'en payait. Ils se séparèrent de la sorte, elle dans le froid
d'une indignation si juste, lui dans le désordre d'une faible et timide
négative, et le désespoir de la découverte de son crime, des aveux
arrachés sur tout ce qu'il me devait, et de ceux encore que la force de
la vérité avait malgré lui tirés de sa bouche sur les véritables auteur
et contradicteur de ce projet de salutation.

Une leçon si forte et si peu attendue, et en présence du frère d'un des
ducs qui s'étaient trouvés dans ma chambre à la scène du duc de Noailles
et de moi là-dessus, n'était pas pour changer un scélérat consommé dans
un crime pourpensé et amené de si loin, dont il commençait si bien à
goûter ce qu'il s'en proposait, et que ce succès animait à poursuivre
jusqu'au but qu'il s'en était promis. Il eut beau protester à
M\textsuperscript{me} de Saint-Simon qu'il dirait partout combien je
m'étais opposé à ce projet, il était bien éloigné d'une palinodie si
subite, et si destructive de ses projets particuliers. Il continua donc,
par tout ce qu'il avait mis en campagne et par lui-même, à répandre les
mêmes discours qui avaient si parfaitement réussi à son gré\,; mais
personnellement il prit mieux garde devant qui il parlait, et il fut
très attentif à m'éviter partout et M\textsuperscript{me} de Saint-Simon
aussi, même en lieux publics, autant qu'il lui fut possible.

Je ne fus informé que tard de cette exécrable perfidie, et de tout son
effet. Alors seulement les écailles me tombèrent des yeux. Je commençai
à comprendre la cause de cette étrange idée de salutation du roi, et de
cette fermeté encore plus surprenante à la soutenir, malgré mes raisons
invincibles au contraire. Je revins à ce qui s'était nouvellement passé
entre Noailles et moi sur la place de premier ministre\,; je me rappelai
son ardeur pour les finances, sa traîtreuse conduite avec Desmarets,
depuis que je savais qu'il pensait à lui succéder, et surtout depuis
qu'il en avait l'assurance. Je me rappelai aussi son éloignement doux,
mais adroit et constant, de la convocation des états généraux\,; et je
me souvins que, deux jours avant cet éclat, j'avais inutilement pressé
M. le duc d'Orléans de songer promptement, et avant tout, à donner les
ordres pour la faire, lui qui jusque-là n'avait respiré autre chose.
Enfin je vis qu'un guet-apens, de si loin et si profondément pourpensé,
si contradictoire à toute vérité, si subit, si à bout portant, et dans
une telle crise de toute espèce de choses et d'affaires, était le fruit
de la plus infernale ambition, et de l'ingratitude la plus consommée.

Sans ressource auprès du roi et de M\textsuperscript{me} de Maintenon,
aussi mal avec Mgr {[}le Duc{]} et M\textsuperscript{me} la duchesse de
Bourgogne, et par même forfaiture en abomination à la cour d'Espagne,
guère mieux à la nôtre qui l'avait mieux reconnu que moi, brouillé avec
M. {[}le Duc{]} et M\textsuperscript{me} la duchesse d'Orléans, rebuté
de tous les ministres excepté de Desmarets, son esprit me trompa. Je le
crus droit, capable, utile\,; sa faute en Espagne ne me parut qu'un
égarement d'emportement de jeunesse, de cour, et d'affaires qu'il était
vrai que M\textsuperscript{me} des Ursins perdait\,; je vainquis la
répugnance du duc de Beauvilliers à cet égard, et pour le fils et le
neveu du maréchal et du cardinal de Noailles\,; je le mis bien avec lui
à force de bras, puis par lui avec M. le duc de Bourgogne, qui apaisa
M\textsuperscript{me} la duchesse de Bourgogne\,; je le raccommodai avec
M. {[}le Duc{]} et M\textsuperscript{me} la duchesse d'Orléans, je l'y
maintins à force malgré tous ses douteux ménagements\,; enfin je forçai
ce prince à lui destiner les finances et à tirer son oncle du fond de
l'abîme pour le mettre à la tête des affaires ecclésiastiques, dernière
chose qui mettait le comble au solide du neveu, quoique ce dernier point
ne fût pas directement pour lui.

Tant de puissants coups frappés en sa faveur excitèrent sa jalousie au
lieu de reconnaissance. Il sentit qu'il faudrait compter avec moi\,; il
ne voulait compter avec personne, mais être le maître, dominer,
gouverner, en un mot être premier ministre. Je n'en puis douter
puisqu'il me proposa de lui faire donner cette épouvantable place. Ce
n'était pas que de plus loin il n'eût conçu le dessein de me perdre,
dans l'espérance de demeurer après le maître de tout. Ce fut pour cela
qu'il conçut cette idée de salutation du roi pour l'usage qu'il m'en
préparait, et qui l'empêcha si constamment de se rendre à mes raisons,
quoiqu'il ne leur en pût opposer aucune. Il voulut avant tout essayer de
me faire donner dans ce piège, pour publier avec vérité ce qu'il
répandit avec tant de calomnie, et ne se rebuta point de tâcher de m'y
faire tomber. Mais auparavant, il voulut faire un dernier essai de mon
crédit, dont il s'était si bien trouvé et si fort au-dessus de ses
espérances, pour se faire par moi premier ministre, pour s'en assurer
davantage. Désespérant de m'y faire travailler, il se garda bien d'en
montrer son dépit\,; il n'avait garde aussi de se montrer refroidi dans
un dessein qui, jusqu'à son éclat, voulait la même union pour le rendre
plus certain\,; il hâta donc son dernier effort dans ma chambre pour me
faire tomber dans ses filets, et n'y pouvant réussir, il ne tarda plus
un instant à consommer sa perfidie par la plus atroce scélératesse, et
la calomnie la plus parfaite que le démon, possédant un homme, lui
puisse faire exécuter. Les espérances les plus flatteuses s'en
présentaient à lui avec la plus parfaite confiance, que de quelque façon
que ce fût je n'en pourrais échapper. Un cri public, une noblesse
ramassée, ignorante, furieuse, répandue partout, me devait être une
source de querelles et de voies de fait au moins fréquentes, et dont les
suites mêmes, en s'en tirant avec succès, ont des recherches légales,
longues et fort embarrassantes.

Cette ressource de combats particuliers et de querelles avec tout le
monde lui parut immanquable. Si contre toute attente je sortais
heureusement d'un si dangereux labyrinthe, il se flattait que M. le duc
d'Orléans ne pourrait jamais conserver dans les affaires, dans sa
confiance publique, dans les places, un homme en butte à toute la
noblesse qui se portait publiquement contre lui. Enfin, si, contre toute
apparence, M. le duc d'Orléans ne se laissait ni vaincre ni étourdir par
ce bruit, le dépit d'essuyer de la part du public une injustice si
criante, si universelle, si continuelle, et d'un public fou en ce genre,
à l'ivresse duquel il ne me serait pas possible de faire entendre aucune
raison, moins encore de lui persuader la vérité sur ce qui le mettait en
fureur, me ferait d'indignation quitter la partie, et le délivrerait au
moins ainsi de moi.

À tout ce qu'on vient de voir qui a précédé cet éclat et qui l'a
accompagné, on ne peut soupçonner ce raisonnement d'imputation la plus
légère. Il est vrai que c'est un raisonnement de démon, duquel il a
toutes les qualités\,: profondeur, noirceur, calomnie, attentat à tout,
assassinat, ambition sans bornes, ingratitude exquise, effronterie sans
mesure, méchanceté de toute espèce la plus atroce, scélératesse la plus
raffinée, la plus consommée\,: mais il est vrai aussi que ce
raisonnement en a toute l'étendue, la réflexion, l'esprit, la finesse,
la justesse, l'adresse\,; que la conjoncture de l'exécution en couronne
toute la prudence qui s'y pouvait mettre, et que le tout ensemble est
sublimement marqué au coin du prince des démons, qui seul l'a pu
inspirer et conduire. Je bornerai là le peu de réflexions que je n'ai pu
me refuser sur une conduite de ténèbres si digne du vrai fils du père du
mensonge et du séducteur du genre humain.

Il n'était pas difficile d'imaginer à quoi m'allait porter une telle
perfidie\,; l'éclat aussi fut tel et si subit, qu'il ne fut pas
difficile d'y mettre tous les obstacles qui l'empêchèrent, d'autant que
Noailles évita avec un soin extrême toute rencontre, dont il ne se crut
pas assez en sûreté dans le château de Versailles pour s'y hasarder. Ma
ressource fut donc le témoignage que rendirent les ducs témoins de ce
qui s'était passé dans ma chambre qu'ils rendirent public, et ce que mes
amis non titrés prirent soin de répandre. J'en parlai aussi à tout ce
que je trouvai sous ma main avec une force qui n'épargna ni choses ni
termes sur le duc de Noailles, qui nomma tout par son nom, les choses
par le leur, et que je répandis à tous venants. Je m'expliquai en même
temps à M. le duc d'Orléans\,; mais la conjoncture était si chargée
d'affaires les plus importantes, et de ces pressantes bagatelles qui
prennent nécessairement alors le temps même des affaires, que cet
accablement des derniers moments, pour ainsi dire, du roi, ne permit
guère d'attention suivie à une affaire particulière.

Noailles, qui m'évita jusque chez M. le duc d'Orléans, où il craignit
mes insultes, même en sa présence, outré de tout ce qui lui revenait de
toutes parts des propos sans mesure que je tenais sur lui, s'arma de
toile cirée et de silence pour les laisser glisser, et poussa sa pointe
parmi la noblesse, sur le gros de laquelle le témoignage des ducs qui
s'étaient trouvés chez moi avec le duc de Noailles, ni ceux de mes amis
de leurs confrères sur mes sentiments à l'égard de la noblesse, ne les
put ramener. Noailles avait bien pris ses mesures pour les mettre et les
entretenir dans l'opinion et la furie qui lui convenait sur moi.

Il ne faut pas demander si M. et M\textsuperscript{me} du Maine surent
profiter d'une si favorable occasion à leurs intérêts et à leur
disposition pour moi, plus que tout quand la chose fut une fois
enfournée. L'envie et la jalousie générale de la figure que personne ne
douta que je n'allasse faire par un régent avec qui j'avais les plus
anciennes, les plus importantes, les plus uniques liaisons, qui lui
avais rendu les plus signalés services, qui étais demeuré le seul homme
dont l'attachement pour lui avait été fidèle et public sans craindre les
menaces ni les plus grands dangers, et qui étais le seul dans toute sa
confiance et vu publiquement tel. Cette gangrène du monde avait gagné
même des ducs\,; Noailles en sut profiter.

Son abattement depuis son rappel d'Espagne avait émoussé l'envie et la
jalousie sur lui\,; celle qu'on prenait de moi avait toute sa force dans
le moment naissant d'une splendeur prévue, toujours bien au-dessus de ce
qui arrive en effet. Par Canillac, ami intime de La Feuillade, il se lia
à lui. On a pu voir par divers traits qu'ils étaient tous deux assez
homogènes. Par La Feuillade {[}il se lia{]} avec les ducs de Villeroy et
de La Rochefoucauld, lequel rogue, glorieux, et aussi envieux que son
père, avec aussi peu d'esprit, n'avait pu me pardonner la préséance sur
lui, ni son beau-frère, un avec lui. Richelieu, jeune étourdi alors,
plein d'esprit, de feu, d'ambition, de légèreté, de galanterie,
apprenait à voler sous les ailes de La Feuillade, que le bel air avait
rendu son oracle, et qui, cousin germain de Noailles par sa femme, et
uni à lui par la protection ouverte de M\textsuperscript{me} de
Maintenon, se promit bien de figurer par ces messieurs, qui pour
s'autoriser d'un homme de poids firent des assemblées chez le maréchal
d'Harcourt, ami de La Rochefoucauld et de Villeroy, et qui par
M\textsuperscript{me} de Maintenon était de tout temps en mesure avec
Noailles. Harcourt ne me voulait point de mal\,; on a vu en divers
endroits qu'il s'était ouvert fort librement à moi sur les bâtards et
sur d'autres choses\,; qu'il avait tenté plus d'une fois liaison et
union avec moi, à laquelle la mienne avec M. de Beauvilliers n'avait pu
me permettre de me laisser entraîner. Comme l'autre n'avait fait que
tenter, ma retenue n'avait pu nous brouiller, mais elle avait diminué la
bienveillance, et d'ailleurs il était fort opposé en dessous à M. le duc
d'Orléans, ainsi que La Rochefoucauld, Villeroy et La Feuillade.
Néanmoins il ne fut que leur ombre. Ses diverses attaques d'apoplexie
l'avaient extrêmement abattu\,; il n'était plus que la figure extérieure
d'un homme, et sa tête ne pouvait s'appliquer, ni sa langue, embarrassée
déjà, s'expliquer bien aisément\,; mais ce groupe suppléait, et se
couvrit de son nom pour séduire autant de ducs qu'ils purent. La
Feuillade me haïssait de tout temps, sans que j'en aie jamais pu
découvrir la cause, plus encore comme l'ami de M. le duc d'Orléans, et
comme l'envie même qui surnageait à tous ses autres vices. Depuis la
disgrâce de Turin, dont il n'avait pu se relever du tout, il avait fait
le philosophe sans quitter le bel air. Il avait cherché à capter les
gens importants par leur état ou par leur réputation, surtout parmi ceux
qui étaient ou faisaient les mécontents. Il avait fait extrêmement sa
cour au marquis de Liancourt qu'il trompa par ses belles maximes, et qui
s'en sépara à la fin hautement\,; et par Liancourt, qui était plein
d'esprit, d'honneur, de savoir et de probité, qui n'était qu'un avec La
Rochefoucauld son frère, et le duc de Villeroy, il se lia étroitement
avec eux.

M. de Luxembourg, le plus intime ami de ces trois hommes, par leur
ancienne union avec feu M. le prince de Conti, fut de compagnie envahi
par La Feuillade. Luxembourg était un fort homme d'honneur, qui avait à
peine le sens commun, rectifié par le grand usage du meilleur et du plus
grand monde où son père l'avait initié. Il était plein de petitesses
dans le commerce, quoique le meilleur homme du monde\,; mais il voulait
des soins, des prévenances, qu'il rendait bien à la vérité, mais qui
étaient importunes à la continue. La bonté de son caractère, les
anciennes liaisons du temps de son père, la magnificence et la commodité
de sa maison, y avait accoutumé le monde. J'étais le seul des ducs
opposants à sa préséance qui étais demeuré brouillé avec lui. Quelques
jours avant l'éclat dont je parle, je l'avais rencontré dans la galerie
de l'aile neuve, au bout de laquelle il avait un beau logement en haut.
Je sentais l'importance de la réunion de tous les ducs. Je l'abordai et
je lui fis civilité sur les petites assemblées qui s'étaient tenues chez
moi, dont je lui dis que je voulais lui rendre compte. Il y fut sensible
au point qu'il vint chez moi, qu'il ne fut plus mention du passé, qu'il
fut, sans que je le susse qu'après, ferme à me défendre contre toutes
les attaques de ses amis et de tout le monde, qu'il me fit mille
recherches, et que nous sommes demeurés en liaison jusqu'à sa mort.

Noailles avait si bien profité de la sottise publique, et M. du Maine
aussi, qu'il me fut impossible d'y faire entendre raison et vérité\,;
mais la Providence arrêta aussi leurs cruelles espérances. Je sortis,
allai et vins tout à mon ordinaire, je ne trouvai jamais personne qui me
dît quoi que ce soit qui pût, non pas me fâcher, mais m'indisposer. Les
plus enivrés passaient leur chemin avec une salutation froide, en sorte
que je n'eus ni à courir, ni à me défendre, ni même à attaquer, et je
suis encore à le comprendre d'un nombre infini de têtes aussi
échauffées, aussi excitées, et de ce nombre d'entours du duc de Noailles
qui, quand cela se trouvait à leur portée, m'entendaient parler de lui
de la manière la plus diffamante et la plus démesurée. Je coulerai ici
cette affaire à fond pour n'avoir plus à y revenir, et pour éclaircir
par la plusieurs choses qui se sont passées depuis tout pendant la
régence et même après.

Noailles souffrit tout en coupable écrasé sous le poids de son crime.
Les insultes publiques qu'il essuya de moi sans nombre ne le rebutèrent
point. Il ne se lassa jamais de s'arrêter devant moi chez le régent, ou
en entrant et sortant du conseil de régence, avec une révérence
extrêmement marquée, ni moi de passer droit sans le saluer jamais, et
quelquefois de tourner la tête avec insulte\,; et il est très souvent
arrivé que je lui ai fait des sorties chez M. le duc d'Orléans et au
conseil de régence, dès que j'y trouvais le moindre jour, dont le ton,
les termes, les manières effrayaient l'assistance, sans qu'il répondit
jamais un mot\,; mais il rougissait, il palissait et n'osait se
commettre à une nouvelle reprise. Si rarement il répondait un mot, je le
dis avec vérité, il le faisait d'un ton et avec des paroles aussi
respectueuses que s'il eût répondu à M. le duc d'Orléans. Parmi cela,
les affaires n'en souffrirent jamais. Je m'en étais fait une loi, à
laquelle je n'ai point eu à me reprocher d'avoir jamais manqué. J'étais
de son avis quand je croyais qu'il était bon\,; il m'est arrivé
quelquefois de l'avoir appuyé contre d'autres\,; du reste, même hauteur,
mêmes propos, même conduite à son égard. Il est quelquefois sorti si
outré du Palais-Royal ou des Tuileries, de ce que je lui avais dit et
fait en face, devant le régent et tout ce qui s'y trouvait, qu'il est
allé quelquefois tout droit chez lui se jeter sur son lit comme au
désespoir, et disant qu'il ne pouvait plus soutenir les traitements
qu'il essuyait de moi, jusque-là qu'au sortir d'un conseil où je le
forçai de rapporter une affaire que je savais qu'il affectionnait, et
sur laquelle je l'entrepris sans mesure et le fis tondre, lui dictai
l'arrêt tout de suite et le lus après qu'il l'eut écrit, en lui montrant
avec hauteur et dérision ma défiance et à tout le conseil, il se leva,
jeta son tabouret à dix pas, et lui qui en place n'avait osé répondre un
seul mot que de l'affaire même avec l'air le plus embarrassé et le plus
respectueux\,: «\,Mort\ldots. dit-il en se tournant pour s'en aller\,;
il n'y a plus moyen d'y durer,\,» s'en alla chez lui, d'où ses plaintes
me revinrent, et la fièvre lui en prit. Il y avait peu de semaines qu'il
n'en essuyât de très fortes, moi toujours sans le saluer, ni lui parler
qu'en opinant, pour le bourrer dès que j'y trouvais jour, lui sans se
lasser de me faire les révérences les plus marquées, et de m'adresser
souvent la parole avec un air de respect dans les rapports qu'il
faisait, n'osant d'ailleurs s'approcher de moi, beaucoup moins me
parler.

Il ne fut pas longtemps sans chercher à m'apaiser, dans le désespoir où
il était d'avoir montré tout ce dont il était capable, sans en avoir
recueilli ce qu'il s'en était proposé, et qu'il avait compté
immanquable. Il essuyait de moi sans cesse des sorties publiques, des
hauteurs en passant devant lui dont le mépris affecté faisait regarder
tout le monde, et des propos sur lui où rien n'était ménagé. Un ennemi
qui se piquait de l'être, et de le paraître sans aucune mesure, à qui
les plus cruelles expressions étaient les plus familières, les insultes
et les sorties en toute occasion en plein conseil et au Palais-Royal, en
présence du régent, avec cette hauteur et cet air de mépris que la vertu
offensée prend sur le crime infamant, fut si pesant à ce coupable, qu'il
n'omit rien au moins pour m'émousser. Il se mit à chanter mes louanges,
à dire qu'il ignorait quelle grippe j'avais prise contre lui, que ce
n'était au plus qu'un malentendu, qu'il avait toujours été mon serviteur
et le voulait demeurer même malgré moi, et qu'il n'y avait rien qu'il ne
voulût faire pour regagner mes bonnes grâces. Sa mère, que j'avais
toujours eu lieu d'aimer, était au désespoir contre son fils, et me fit
parler.

D'une infinité d'endroits directs et indirects je fus attaqué\,;
M\textsuperscript{me} de Saint-Simon fut exhortée sur le ton de piété,
mes amis les plus particuliers furent priés de tâcher à m'adoucir. Je
répondis toujours que c'était assez d'avoir été dupe une fois pour ne
l'être pas une seconde du même homme, qu'il n'y en avait point qui eût
pu se douter, ni par conséquent, échapper à une si noire scélératesse,
si pourpensée, si profonde, si achevée\,; mais qu'il fallait croire
avoir affaire à un stupide incapable d'aucune sorte de sentiment pour
imaginer de lui faire oublier une perfidie et une calomnie de cette
espèce et de cette suite, dont le criminel auteur serait à jamais
l'objet de ma haine et de ma vengeance la plus publique et la plus
implacable, dont il pouvait compter que la mesure serait de n'en garder
aucune. Ma conduite y répondit pleinement, et la sienne à mon égard fut
aussi la même en bassesse. Ce qui le confondit et le désola le plus, au
milieu de sa prospérité, de ne pouvoir parvenir à une réconciliation
avec moi, c'était le contraste de son oncle, dont la liaison avec moi ne
souffrit pas le moins du monde, et qui était publique. Je n'en fus que
plus ardent pour le cardinal de Noailles qui venait sans cesse chez moi,
et moi chez lui, avec la plus grande confiance, et que je servis
toujours de tout ce que je pus et ouvertement.

Ce contraste tombait à plomb sur le duc de Noailles qui, à la fin, me
fit demander grâce, en propres termes, par M. le duc d'Orléans, à qui je
sus répondre de façon qu'il se garda depuis d'y revenir. Le duc de
Noailles fut accablé de ce refus. Il me fit revenir des choses que je
n'oserais écrire, parce que, quoique vraies, elles ne seraient pas
croyables\,: par exemple, que j'aurais enfin pitié de lui, si je
connaissais l'état où je le mettais, et des bassesses de toutes sortes.
Le cardinal de Noailles chercha souvent à me tourner, et enfin, me parla
de cette division à deux reprises, qui, me dit-il, le comblait de
douleur, et je ne rencontrai jamais {[}chez lui{]} le duc de Noailles,
qui avait grand soin de m'éviter. Je répondis la même chose au cardinal
toutes les deux fois. Je lui dis que, quand il lui plairait, je lui
rendrais un compte exact de ce qui l'avait causée\,; qu'il fallait, s'il
le voulait ainsi, qu'il se préparât à entendre d'étranges choses\,;
qu'après cela je ne voulais point d'autre juge que lui. Toutes les deux
fois la proposition lui ferma la bouche, et il ne m'en parla plus. Je
demeurai persuadé qu'il en savait assez pour craindre de l'entendre, et
que c'est ce qui l'arrêta tout court\,; mais il en gémissait, car il
aimait cet indigne neveu, et indigne pour lui-même comme on le verra en
son temps. Je passe d'autres tentatives très fortes du duc de Noailles
pour essayer de me rapprocher, parce qu'elles se retrouveront pendant la
régence.

Tant qu'elle dura j'en usai de la sorte avec lui, sans qu'il se soit
jamais lassé de ses révérences respectueuses, sans que je l'aie jamais
daigné saluer le moins du monde, ni payé ses façons de déférence que par
le mépris le plus marqué, ou la hauteur la plus insultante, et toujours
les sorties sur lui en face en toutes les occasions que j'en pouvais
faire naître. Douze années se passèrent de la sorte sans le moindre
adoucissement de ma part, et sans qu'en aucun temps les devoirs communs
aient cessé ni faibli entre toute sa famille et moi et la mienne. Cette
parenthèse est longue, mais il en faut voir le bout.

On verra dans la suite de la régence combien le duc de Noailles fut
infatigable, avec une persévérance sans fin, à essuyer tout de moi, et à
ne se lasser jamais de rechercher tous les moyens imaginables de se
raccommoder avec moi, pour le moins de m'adoucir. Tout fut non seulement
inutile tant qu'elle dura, mais encore après la mort de M. le duc
d'Orléans. Les occasions de nous rencontrer devinrent bien plus rares\,;
mais le maintien, quand cela arrivait, fut toujours le même des deux
parts\,; et les propos de la mienne aussi pesants, aussi fermes et aussi
sans mesure, tant qu'il s'en présentait d'occasions. C'est une chose
terrible que la poursuite intérieure du crime.

Il y avait longtemps que j'avais quitté le conseil\,; mon crédit s'était
éteint avec la vie de M. le duc d'Orléans\,; je n'avais plus de place,
et je vivais fort en particulier. M. de Noailles, au contraire, avec ses
gouvernements, et sa charge de premier capitaine des gardes du corps, se
trouvait à la tête de la famille la plus puissante en tout genre par
toutes sortes de grands établissements. Malgré cette différence totale,
ni lui ni les siens ne purent supporter cette situation avec moi. Le duc
de Guiche, maréchal de France, en 1724, où il prit le nom de maréchal de
Grammont, mort à Paris en septembre 1725, à cinquante-trois ans, avait
deux fils morts l'un après l'autre colonels du régiment des gardes après
lui, et deux filles. Il avait marié l'aînée au fils aîné de Biron, morts
tous deux, connus sous le nom de duc et de duchesse de Gontaut\,; et
l'autre au prince de Bournonville, fils du cousin germain de la
maréchale de Noailles et d'une sœur du duc de Chevreuse, tous deux
morts. Ce mariage s'était fait à la fin de mars 1719, quoique le marié,
qui n'avait guère que vingt-deux ans, eût déjà les nerfs affectés à ne
se pouvoir presque soutenir. Il devint bientôt après impotent, puis tout
à fait perclus, et menaça longuement d'une fin prochaine. La mère de sa
femme était l'aînée des sœurs du duc de Noailles, parmi lesquelles elle
avait toujours été la plus comptée. Ils songèrent tous à mon fils aîné
pour elle, dès qu'elle serait libre, comme un moyen de raccommodement.
Elle était belle, bien faite, n'était jamais sortie de dessous l'aile de
sa mère\,; et pour le bien était le plus grand parti de France alors
parmi les personnes de qualité.

Ils n'osèrent me faire rien jeter là-dessus, mais ils crurent trouver
M\textsuperscript{me} de Saint-Simon plus accessible. Ils ne se
trompèrent pas. Elle me sonda de loin avec peu de succès\,; elle ne se
rebuta point\,; elle me parla ouvertement, me prit par le monde sur
l'alliance et le bien, et par la religion comme un moyen honnête de
mettre fin à la longueur et à l'éclat toujours renaissant d'une rupture
ouverte. Je fus plus d'un an à me laisser vaincre par l'horreur du
raccommodement. Enfin, pour abréger matière, dès que j'eus consenti,
tout fut bientôt fait. Chauvelin, président à mortier, depuis garde des
sceaux, etc., était le conducteur des affaires de la maréchale de
Grammont. Il me courtisait depuis plusieurs années. Dès qu'il sut que je
m'étais enfin rendu, car jusque-là il n'avait osé m'en parler
directement, il dit que la maréchale de Grammont ne pouvait entrer en
rien pendant la vie de son gendre, mais qu'il se chargeait de tout\,; et
en effet tout fut réglé entre M\textsuperscript{me} de Saint-Simon et
lui, se faisant fort l'un et l'autre de n'être pas dédits. Dans le peu
que cela dura de la sorte, le cardinal de Noailles m'en parlait sans
cesse, et la maréchale de Grammont et sa fille ne négligeaient aucune
occasion de courtiser tout ce qui tenait intimement à nous. Le premier
article fut un raccommodement entre le duc de Noailles et moi. J'y
prescrivis qu'il ne s'y parlerait de rien, ni en aucun temps, et qu'on
n'exigerait de moi rien de plus que la bienséance commune\,; on ne
disputa sur rien.

Il arriva qu'une après-dînée j'allai par hasard à l'hôtel de Lauzun, où
je trouvai M\textsuperscript{me} de Bournonville qui jouait à l'hombre,
amenée et gardée par M\textsuperscript{me} de Beaumanoir, qui logeait
avec sa sœur la maréchale de Grammont. Un peu après on vint demander
M\textsuperscript{me} de Beaumanoir, qui sortit et rentra aussitôt,
parla bas à M\textsuperscript{me} de Lauzun, et me regarda en riant.
Elle dit après à sa nièce qu'il fallait demander permission de quitter
le jeu, et, à demi bas, aller voir M. de Bournonville qui logeait chez
la duchesse de Duras, sa sœur, depuis longtemps, et qui venait de se
trouver fort mal. Cela arrivait quelquefois, et ces sortes de longues
maladies font qu'on ne les croit jamais à leur fin. J'allai le soir à
l'archevêché\,; j'y trouvai la maréchale de Grammont et
M\textsuperscript{me} de Beaumanoir qui avait ramené et laissé sa nièce,
qui parla de M. de Bournonville comme d'un homme qui pouvait durer
longtemps. Le cardinal et elle, après une légère préface chrétienne,
laissèrent échapper leur impatience en me regardant\,; la maréchale me
regarda aussi, sourit avec eux, laissa échapper quelques mines, et se
levant tout de suite, se mit à rire tout à fait, et, m'adressant la
parole, me dit qu'il valait mieux s'en aller. Le bon cardinal me parla
après avec effusion de cœur. Chauvelin nous manda fort tard que le mal
augmentait\,; et le lendemain matin, comme j'étais chez moi avec du
monde, on me fit sortir pour un message de Chauvelin, qui me mandait que
M. de Bournonville venait de mourir.

J'envoyai dire aussitôt à M\textsuperscript{me} de Saint-Simon, qui
était à la messe aux Jacobins, tout proche du logis, que je la priais de
revenir\,; elle ne tarda pas, et me trouva avec la même compagnie,
devant qui je lui dis le fait tout bas. Il était convenu que, dès que
cela arriverait, nous ferions sur-le-champ la demande au cardinal, qui
se chargerait de tout. M\textsuperscript{me} de Saint-Simon y alla.
C'était la veille de l'Annonciation, qu'il était à table pour aller
officier aux premières vêpres à Notre-Dame. Il sortit de table et vint
au-devant d'elle les bras ouverts, dans une joie qu'il ne cacha point\,;
et, sans lui donner le temps de parler, devant tous ses gens\,: «\,Vite,
dit-il, les chevaux à mon carrosse\,!» puis à elle\,: «\,Je vois bien ce
qui vous amène\,; Dieu en a disposé, nous sommes libres\,; je m'en vais
chez la maréchale de Grammont, et vous aurez bientôt de mes
nouvelles.\,» Il la mena dans sa chambre un moment. Comme il
l'accompagnait, ses gens lui parlèrent de vêpres. «\,Mon carrosse,
répondit-il, vêpres pour aujourd'hui attendront, dépêchons.\,»
M\textsuperscript{me} de Saint-Simon revint, et nous nous mîmes à table.

Comme à peine nous en sortions, nous entendîmes un carrosse dans la
cour\,: c'était le cardinal de Noailles. Je descendis au-devant de
lui\,; il m'embrassa à plusieurs reprises, et tout aussitôt devant tout
le domestique se prit à me dire\,: «\,Où est mon neveu\,? car je veux
voir mon neveu, envoyez-le donc chercher.\,» Je répondis fort étonné
qu'il était à Marly. «\,Oh bien, envoyez-y donc tout à l'heure le
chercher, car je meurs d'envie de l'embrasser, et il faut bien qu'il
aille voir la maréchale de Grammont et sa prétendue.\,» Je ne sortais
point d'étonnement d'une telle franchise, qui apprenait tout à son
domestique et au nôtre, qui étaient là en foule. Nous montions cependant
le commencement du degré. M\textsuperscript{me} de Saint-Simon
descendait en même temps, et nous fit redescendre le peu que nous avions
monté, pour faire entrer le cardinal dans mon appartement et ne lui pas
donner la peine de monter en haut. Jamais je ne vis homme si aise. Il
nous dit que la maréchale de Grammont et sa fille étaient ravies\,; que
tout était accordé\,; qu'il avait voulu se donner la satisfaction de
nous le venir dire et de le déclarer tout haut, comme il avait fait,
parce que, au nombre de grands partis en hommes qui n'attendaient que ce
moment, de leur connaissance à tous, pour faire des démarches pour ce
mariage, il n'y avait de bon qu'à bâcler et déclarer pour leur fermer la
bouche et arrêter par là tous les manèges qui se font pour faire rompre
et se faire préférer, au lieu qu'il n'y a plus à y penser quand les
choses sont faites, déclarées et publiées par les parties mêmes\,; qu'il
aimait mieux qu'on le dit un radoteur d'avoir déclaré si vite, et que
cela fût fini. Après mille amitiés il s'en alla à ses vêpres. Il fut
convenu que le jour même M\textsuperscript{me} de Saint-Simon irait au
Bon-Pasteur, où elle trouverait la maréchale de Grammont dans sa
tribune. Mon fils arriva le soir.

Le lendemain, comme nous dînions avec assez de monde au logis,
arrivèrent tous les Grammont et plusieurs Noailles, mais non la future,
sa mère ni sa grand'mère, de manière qu'il n'y eut rien de plus public,
et la maréchale de Grammont vint au logis dès l'après-dînée. Mon fils,
qui les alla voir et la maréchale de Grammont, et que je menai chez le
cardinal, retourna le soir à Marly pour demander au roi l'agrément du
mariage, et en donner part après à ceux de nos plus proches ou de nos
plus particuliers amis qui y étaient, avant de la donner en forme. Tout
en arrivant, il trouva le duc de Chaulnes dans un des petits salons, à
qui il le dit à l'oreille. «\,Cela ne peut pas être,\,» lui répondit-il,
et ne voulut jamais le croire, quoique mon fils lui expliquât qu'il
avait vu le cardinal de Noailles, la maréchale de Grammont, etc. C'est
qu'il comptait son affaire sûre pour son fils par M\textsuperscript{me}
de Mortemart, amie intime de tout temps et de gnose de la maréchale de
Grammont, qui lui en avait fort parlé et qui l'avait laissée espérer
sans s'ouvrir, sur la raison de ne le pas pouvoir pendant la vie de M.
de Bournonville. En trois ou quatre jours tout fut signé et passa par
Chauvelin. La duchesse de Duras trouva fort bon qu'on n'eût point
attendu, et qu'on fit incessamment le mariage. Mais comme il pouvait en
arriver une grossesse prompte, tout ce qui fut consulté de part et
d'autre fut d'avis de différer de trois ou quatre mois, quoique M. de
Bournonville n'eût jamais été en état d'être avec sa femme, et qu'il n'y
logeât plus même depuis deux ou trois ans.

Tout allait bien jusque-là. Jamais tant d'empressement ni de marques de
joie, et c'en fut une toute particulière que la visite dont j'ai parlé,
parce que c'est à la famille du mari futur à aller chez l'autre famille
la première. Tout cela fait, il fut question du raccommodement. Le
président Chauvelin me fit pour le duc de Noailles les plus beaux
compliments du monde, et me pressa de sa part et de celle du cardinal,
de la maréchale de Noailles, de lui permettre de venir chez moi. La
crainte d'une visite à laquelle je ne pourrais mettre une fin aussi
prompte que je le voudrais m'empêcha d'y consentir, et je voulus si
fermement que nous nous vissions chez le cardinal de Noailles qu'il en
fallut passer par là. Ce fut où je m'en tins, sans dire si ni qui je
voulais bien qu'il s'y trouvât, et sans qu'on m'en parlât non plus. Le
duc de Noailles, qui sortait de quartier, vint donc à Paris pour le jour
marqué. Ce même jour, M\textsuperscript{me} de Saint-Simon et moi
dînions vis-à-vis du logis, chez Asfeld, depuis maréchal de France, avec
le maréchal et la maréchale de Berwick et quelques autres amis
particuliers. J'étais de fort mauvaise humeur, je prolongeais la table
tant que je pouvais, et après qu'on en fut sorti, je me fis chasser à
maintes reprises. Ils savaient le rendez-vous, qui n'en était pas un
d'amour, et ils m'exhortaient d'y bien faire et de bonne grâce. Je
retournai donc chez moi prendre haleine, et comme on dit, son escousse,
tandis que M\textsuperscript{me} de Saint-Simon s'acheminait et qu'on
attelait mon carrosse. Je partis enfin et j'arrivai à l'archevêché comme
un homme qui va au supplice.

En entrant dans la chambre ou étaient la maréchale de Grammont,
M\textsuperscript{me} de Beaumanoir, M\textsuperscript{me} de
Saint-Simon et M\textsuperscript{me} de Lauzun, le cardinal de Noailles
vint à moi dès qu'il m'aperçut, tenant le duc de Noailles par la main,
et me dit\,: «\, Monsieur, je vous présente mon neveu que je vous prie
de vouloir bien embrasser.\,» Je demeurai froid tout droit, je regardai
un moment le duc de Noailles, et je lui dis sèchement\,: «\,Monsieur, M.
le cardinal le veut,\,» et j'avançai un pas. Dans l'instant le duc de
Noailles se jeta à moi si bas que ce fut au-dessous de ma poitrine, et
m'embrassa de la sorte des deux côtés. Cela fait, je saluai le cardinal,
qui m'embrassa ainsi que ses deux nièces, et je m'assis avec eux auprès
de M\textsuperscript{me} de Saint-Simon. Tout le corps me tremblait, et
le peu que je dis dans une conversation assez empêtrée fut la parole
d'un homme qui a la fièvre. On ne parla que du mariage, de la joie et de
quelques bagatelles indifférentes. Le duc de Noailles, interdit à
l'excès, qui m'adressa deux ou trois fois la parole avec un air de
respect et d'embarras, je lui répondis courtement, mais point trop
malhonnêtement. Au bout d'un quart d'heure, je dis qu'il ne fallait pas
abuser du temps de M. le cardinal, et je me levai. Le duc de Noailles
voulut me conduire\,; les dames dirent qu'il ne fallait point
m'importuner, ni faire de façons avec moi\,; et je cours encore. Je
revins chez moi comme un homme ivre et qui se trouve mal. En effet, peu
après que j'y fus, il se fit un tel mouvement en moi, de la violence que
je m'étais faite, que je fus au moment de me faire saigner\,; la vérité
est qu'elle fut extrême. Je crus au moins en être quitte pour longtemps.

Dès le lendemain le duc de Noailles vint chez moi et me trouva. La
visite se passa tête à tête\,; c'était à la fin de la matinée. Il n'y
fut question que de noces et de choses indifférentes. Il tint le dé tant
qu'il voulut. Il parut moins embarrassé et plus à lui-même. Pour moi,
j'y étais fort peu, et souffrais fort à soutenir la conversation, qui
fut de plus de demi-heure, et qui me parut sans fin. La conduite se
passa comme à l'archevêché. J'allai le lendemain voir la maréchale de
Noailles, que je trouvai ravie. Je demandai son fils qui logeait avec
elle, et qui heureusement ne s'y trouva pas. Il chercha fort depuis à me
rapprocher, et moi à éviter. Nous nous sommes vus depuis aux occasions,
et rarement chez lui autrement, c'est-à-dire comme point, lui chez moi
tant qu'il pouvait, ou, s'il m'est permis de trancher le mot, tant qu'il
osait. Il vint à la noce. Ce fut la dernière cérémonie du cardinal de
Noailles, qui les maria dans sa grande chapelle, et qui donna un festin
superbe et exquis. J'en donnai un autre le lendemain, où le duc de
Noailles fut convié qui y vint.

Quelques années après, étant à la Ferté, la duchesse de Ruffec me dit
qu'il mourait d'envie d'y venir, et après force tours et retours
là-dessus, elle m'assura qu'il viendrait incessamment. Je demeurai fort
froid et presque muet. Quand nous nous fûmes séparés, j'appelai mon fils
qui en avait entendu le commencement\,; je lui en racontai la fin. Je
lui dis après de dire à sa femme que, par honnêteté pour elle, je
n'avais pas voulu lui parler franchement, mais qu'elle fit comme elle
voudrait avec son oncle, de la part duquel elle m'avait parlé à la fin
de son propos, mais que je ne voulais point du duc de Noailles à la
Ferté, quand même elle devrait le lui mander. Je n'avais garde de
souffrir que par ce voyage il se parât d'un renouvellement de liaison
avec moi, moins encore de m'exposer à des tête-à-tête avec lui, que les
matinées et les promenades fournissent à qui a résolu d'en profiter, et
qui ne se peuvent éviter, dont il eût pu après dire et publier tout ce
qui ne se serait ni dit ni traité entre nous, mais qu'il lui eût convenu
de répandre, ce qui m'avait fait avoir grand soin, toutes les fois qu'il
m'avait trouvé chez moi, de prier, dès qu'on l'annonçait, ce qu'il s'y
rencontrait de demeurer et de ne s'en aller qu'après lui. Il a persévéré
longtemps encore à tâcher de me rapprivoiser. À la fin le peu de succès
l'a lassé, et ma persévérance sèche, froide et précise aux simples
devoirs d'indispensable bienséance, m'a délivré, et l'a réduit au même
point avec moi. Dieu commande de pardonner, mais non de s'abandonner
soi-même, et de se livrer après une expérience aussi cruelle. Le monde a
vu et connu depuis quel homme il est, et ce qu'il a été dans la cour,
dans le conseil et à la tête des armées.

Retournons maintenant d'où nous sommes partis, qui est au jeudi 22 août,
remarquable par la revue de la gendarmerie faite au nom et avec toute
l'autorité du roi par le duc du Maine, pendant laquelle le roi s'amusa à
vouloir choisir l'habit qu'il prendrait lorsqu'il pourrait s'habiller.

\hypertarget{chapitre-xv.}{%
\chapter{CHAPITRE XV.}\label{chapitre-xv.}}

1715

~

{\textsc{Reprise du journal des derniers jours du roi.}} {\textsc{- Il
refuse de nommer aux bénéfices vacants.}} {\textsc{- Mécanique de
l'appartement du roi pendant sa dernière maladie.}} {\textsc{- Extrémité
du roi.}} {\textsc{- Le roi reçoit les derniers sacrements.}} {\textsc{-
Le roi achève son codicille\,; parle à M. le duc d'Orléans.}} {\textsc{-
Scélératesse des chefs de la constitution.}} {\textsc{- Adieux du roi.}}
{\textsc{- Le roi ordonne que son successeur aille à Vincennes et
revienne demeurer à Versailles.}} {\textsc{- Le roi brûle des papiers,
ordonne que son coeur soit porté à Paris, aux Jésuites.}} {\textsc{- Sa
présence d'esprit et ses dispositions.}} {\textsc{- Le Brun, Provençal,
malmène Fagon et donne de son élixir au roi.}} {\textsc{- Duc du
Maine.}} {\textsc{- M\textsuperscript{me} de Maintenon se retire à
Saint-Cyr.}} {\textsc{- Charost fait réparer la négligence de la
messe.}} {\textsc{- Rayon de mieux du roi.}} {\textsc{- Solitude entière
chez M. le duc d'Orléans.}} {\textsc{- Misère de M. le duc d'Orléans.}}
{\textsc{- Il change sur les états généraux et sur l'expulsion du
chancelier.}} {\textsc{- Le roi, fort mal, fait revenir
M\textsuperscript{me} de Maintenon de Saint-Cyr.}} {\textsc{- Dernières
paroles du roi.}} {\textsc{- Sa mort.}} {\textsc{- Caractère de Louis
XIV.}}

~

Le vendredi 23 août, la nuit fut à l'ordinaire, et la matinée aussi.
{[}Le roi{]} travailla avec le P. Tellier qui fit inutilement des
efforts pour faire nommer aux grands et nombreux bénéfices qui
vaquaient, c'est-à-dire pour en disposer lui-même, et ne les pas laisser
à donner par M. le duc d'Orléans. Il faut dire tout de suite que plus le
roi empira, plus le P. Tellier le pressa là-dessus, pour ne pas laisser
échapper une si riche proie, ni l'occasion de se munir de créatures
affidées avec lesquelles ses marchés étaient faits, non en argent, mais
en cabales. Il n'y put jamais réussir. Le roi lui déclara qu'il avait
assez de comptes à rendre à Dieu sans se charger encore de ceux de cette
nomination, si prêt à paraître devant lui, et lui défendit de lui en
parler davantage. Il dîna debout dans sa chambre en robe de chambre, y
vit les courtisans, ainsi qu'à son souper de même, passa chez lui
l'après-dînée avec ses deux bâtards, M. du Maine surtout,
M\textsuperscript{me} de Maintenon et les dames familières\,; la soirée
à l'ordinaire. Ce fut ce même jour qu'il apprit la mort de Maisons, et
qu'il donna sa charge à son fils, à la prière du duc du Maine.

Il ne faut pas aller plus loin sans expliquer la mécanique de
l'appartement du roi, depuis qu'il ne sortait plus. Toute la cour se
tenait tout le jour dans la galerie. Personne ne s'arrêtait dans
l'antichambre la plus proche de sa chambre, que les valets familiers, et
la pharmacie, qui y faisaient chauffer ce qui était nécessaire\,; on y
passait seulement, et vite, d'une porte à l'autre. Les entrées passaient
dans les cabinets par la porte de glace qui y donnait de la galerie qui
était toujours fermée, et qui ne s'ouvrait que lorsqu'on y grattait, et
se refermait à l'instant. Les ministres et les secrétaires d'État y
entraient aussi, et tous se tenaient dans le cabinet qui joignait la
galerie. Les princes du sang, ni les princesses filles du roi
n'entraient pas plus avant, à moins que le roi ne les demandât, ce qui
n'arrivait guère. Le maréchal de Villeroy, le chancelier, les deux
bâtards, M. le duc d'Orléans, le P. Tellier, le curé de la paroisse,
quand Maréchal, Fagon et les premiers valets de chambre n'étaient pas
dans la chambre, se tenaient dans le cabinet du conseil, qui est entre
la chambre du roi et un autre cabinet où étaient les princes et
princesses du sang, les entrées et les ministres.

Le duc de Tresmes, premier gentilhomme de la chambre en année, se tenait
sur la porte, entre les deux cabinets, qui demeurait ouverte, et
n'entrait dans la chambre du roi que pour les moments de son service
absolument nécessaire. Dans tout le jour personne n'entrait dans la
chambre du roi que par le cabinet du conseil, excepté ces valets
intérieurs ou de la pharmacie qui demeuraient dans la première
antichambre, M\textsuperscript{me} de Maintenon et les dames familières,
et pour le dîner et le souper, le service et les courtisans qu'on y
laissait entrer. M. le duc d'Orléans se mesurait fort à n'entrer dans la
chambre qu'une fois ou deux le jour au plus, un instant, lorsque le duc
de Tresmes y entrait, et se présentait un autre instant une fois le jour
sur la porte du cabinet du conseil dans la chambre, d'où le roi le
pouvait voir de son lit. Il demandait quelquefois le chancelier, le
maréchal de Villeroy, le P. Tellier, rarement quelque ministre, M. du
Maine souvent, peu le comte de Toulouse, point d'autres, ni même les
cardinaux de Rohan et de Bissy, qui étaient souvent dans le cabinet où
se tenaient les entrées. Quelquefois lorsqu'il était seul avec
M\textsuperscript{me} de Maintenon, il faisait appeler le maréchal de
Villeroy, ou le chancelier, ou tous les deux, et fort souvent le duc du
Maine. Madame ni M\textsuperscript{me} la duchesse de Berry n'allaient
point dans ces cabinets, et ne voyaient presque jamais le roi dans cette
maladie, et si elles y allaient, c'était par les antichambres, et
ressortaient à l'instant.

Le samedi 24, la nuit ne fut guère plus mauvaise qu'à l'ordinaire, car
elles l'étaient toujours. Mais sa jambe parut considérablement plus mal,
et lui fit plus de douleur. La messe à l'ordinaire, le dîner dans son
lit, où les principaux courtisans sans entrées le virent\,; conseil de
finances ensuite, puis il travailla avec le chancelier seul. Succédèrent
M\textsuperscript{me} de Maintenon et les dames familières. Il soupa
debout en robe de chambre, en présence des courtisans, pour la dernière
fois. J'y observai qu'il ne put avaler que du liquide, et qu'il avait
peine à être regardé. Il ne put achever, et dit aux courtisans qu'il les
priait de passer, c'est-à-dire de sortir. Il se fit remettre au lit\,;
on visita sa jambe, où il parut des marques noires. Il envoya chercher
le P. Tellier, et se confessa. La confusion se mit parmi la médecine. On
avait tenté le lait et le quinquina à l'eau\,; on les supprima l'un et
l'autre sans savoir que faire. Ils avouèrent qu'ils lui croyaient une
fièvre lente depuis la Pentecôte, et s'excusaient de ne lui avoir rien
fait sur ce qu'il ne voulait point de remèdes, et qu'ils ne le croyaient
pas si mal eux-mêmes. Par ce que j'ai rapporté de ce qui s'était passé
dès avant ce temps-là entre Maréchal et M\textsuperscript{me} de
Maintenon là-dessus, on voit ce qu'on en doit croire.

Le dimanche 25 août, fête de Saint-Louis, la nuit fut bien plus
mauvaise. On ne fit plus mystère du danger, et tout de suite grand et
imminent. Néanmoins, il voulut expressément qu'il ne fût rien changé à
l'ordre accoutumé de cette journée, c'est-à-dire que les tambours et les
hautbois, qui s'étaient rendus sous ses fenêtres, lui donnassent, dès
qu'il fut éveillé, leur musique ordinaire, et que les vingt-quatre
violons jouassent de même dans son antichambre pendant son dîner. Il fut
ensuite en particulier avec M\textsuperscript{me} de Maintenon, le
chancelier et un peu le duc du Maine. Il y avait eu la veille du papier
et de l'encre pendant son travail tête à tête avec le chancelier\,; il y
en eut encore ce jour-ci, M\textsuperscript{me} de Maintenon présente,
et c'est l'un des deux que le chancelier écrivit sous lui son codicille.
M\textsuperscript{me} de Maintenon et M. du Maine, qui pensait sans
cesse à soi, ne trouvèrent pas que le roi eût assez fait pour lui par
son testament\,; ils y voulurent remédier par un codicille, qui montra
également l'énorme abus qu'ils firent de la faiblesse du roi dans cette
extrémité, et jusqu'où l'excès de l'ambition peut porter un homme. Par
ce codicille le roi soumettait toute la maison civile et militaire du
roi au duc du Maine immédiatement et sans réserve, et sous ses ordres au
maréchal de Villeroy, qui, par cette disposition, devenaient les maîtres
uniques de la personne et du lieu de la demeure du roi\,; de Paris, par
les deux régiments des gardes et les deux compagnies des
mousquetaires\,; de toute la garde intérieure et extérieure\,; de tout
le service, chambre, garde-robe, chapelle, bouche, écuries\,; tellement
que le régent n'y avait plus l'ombre même de la plus légère autorité, et
se trouvait à leur merci, et en état continuel d'être arrêté, et pis,
toutes les fois qu'il aurait plu au duc du Maine.

Peu après que le chancelier fut sorti de chez le roi,
M\textsuperscript{me} de Maintenon, qui y était restée, y manda les
dames familières, et la musique y arriva à sept heures du soir.
Cependant le roi s'était endormi pendant la conversation des dames. Il
se réveilla la tâte embarrassée, ce qui les effraya et leur fit appeler
les médecins. Ils trouvèrent le pouls si mauvais qu'ils ne balancèrent
pas à proposer au roi, qui revenait cependant de son absence, de ne pas
différer à recevoir les sacrements. On envoya quérir le P. Tellier et
avertir le cardinal de Rohan, qui était chez lui en compagnie, et qui ne
songeait à rien moins, et cependant on renvoya la musique qui avait déjà
préparé ses livres et ses instruments, et les dames familières
sortirent.

Le hasard fit que je passai dans ce moment-là la galerie et les
antichambres pour aller de chez moi, dans l'aile neuve, dans l'autre
aile chez M\textsuperscript{me} la duchesse d'Orléans, et chez M. le duc
d'Orléans après. Je vis même des restes de musique dont je crus le gros
entré. Comme j'approchais de l'entrée de la salle des gardes, Pernault,
huissier de l'antichambre, vint à moi qui me demanda si je savais ce qui
se passait, et qui me l'apprit. Je trouvai M\textsuperscript{me} la
duchesse d'Orléans au lit, d'un reste de migraine, environnée de dames
qui faisaient la conversation, ne pensant à rien moins. Je m'approchai
du lit, et dis le fait à M\textsuperscript{me} la duchesse d'Orléans qui
n'en voulut rien croire, et qui m'assura qu'il y avait actuellement
musique, et que le roi était bien\,; puis, comme je lui avais parlé bas,
elle demanda tout haut aux dames si elles en avaient ouï dire quelque
chose. Pas une n'en savait un mot, et M\textsuperscript{me} la duchesse
d'Orléans demeurait rassurée. Je lui dis une seconde fois que j'étais
sûr de la chose, et qu'il me paraissait qu'elle valait bien la peine
d'envoyer au moins aux nouvelles, et en attendant de se lever. Elle me
crut, et je passai chez M. le duc d'Orléans, que j'avertis aussi, et qui
avec raison jugea à propos de demeurer chez lui, puisqu'il n'était point
mandé.

En un quart d'heure, depuis le renvoi de la musique et des dames, tout
fut fait. Le P. Tellier confessa le roi, tandis que le cardinal de Rohan
fut prendre le saint sacrement à la chapelle, et qu'il envoya chercher
le curé et les saintes huiles. Deux aumôniers du roi, mandés par le
cardinal, accoururent, et sept ou huit flambeaux portés par des garçons
bleus du château, deux laquais de Fagon, et un de M\textsuperscript{me}
de Maintenon. Ce très petit accompagnement monta chez le roi par le
petit escalier de ses cabinets, à travers desquels le cardinal arriva
dans sa chambre. Le P. Tellier, M\textsuperscript{me} de Maintenon, et
une douzaine d'entrées, maîtres ou valets, y reçurent ou y suivirent le
saint sacrement. Le cardinal dit deux mots au roi sur cette grande et
dernière action, pendant laquelle le roi parut très ferme, mais très
pénétré de ce qu'il faisait. Dès qu'il eut reçu Notre-Seigneur et les
saintes huiles, tout ce qui était dans la chambre sortit devant et après
le saint sacrement\,; il n'y demeura que M\textsuperscript{me} de
Maintenon et le chancelier. Tout aussitôt, et cet aussitôt fut un peu
étrange, on apporta sur le lit une espèce de livre ou de petite table\,;
le chancelier lui présenta le codicille, à la fin duquel il écrivit
quatre ou cinq lignes de sa main, et le rendit après au chancelier.

Le roi demanda à boire, puis appela le maréchal de Villeroy qui, avec
très peu des plus marqués, était dans la porte de la chambre au cabinet
du conseil, et lui parla seul près d'un quart d'heure. Il envoya
chercher M. le duc d'Orléans, à qui il parla seul aussi un peu plus
qu'il n'avait fait au maréchal de Villeroy. Il lui témoigna beaucoup
d'estime, d'amitié, de confiance\,; mais ce qui est terrible, avec
Jésus-Christ sur les lèvres encore qu'il venait de recevoir, il l'assura
qu'il ne trouverait rien dans son testament dont il ne dût être content,
puis lui recommanda l'État et la personne du roi futur. Entre sa
communion et l'extrême-onction et cette conversation, il n'y eut pas une
demi-heure\,; il ne pouvait avoir oublié les étranges dispositions qu'on
lui avait arrachées avec tant de peine, et il venait de retoucher dans
l'entre-deux son codicille si fraîchement fait, qui mettait le couteau
dans la gorge à M. le duc d'Orléans, dont il livrait le manche en plein
au duc du Maine. Le rare est que le bruit de ce particulier, le premier
que le roi eût encore eu avec M. le duc d'Orléans, fit courir le bruit
qu'il venait d'être déclaré régent.

Dès qu'il se fut retiré, le duc du Maine, qui était dans le cabinet, fut
appelé. Le roi lui parla plus d'un quart d'heure, puis fit appeler le
comte de Toulouse qui était aussi dans le cabinet, lequel fut un autre
quart d'heure en tiers avec le roi et le duc du Maine. Il n'y avait que
peu de valets des plus nécessaires dans la chambre avec
M\textsuperscript{me} de Maintenon. Elle ne s'approcha point tant que le
roi parla à M. le duc d'Orléans. Pendant tout ce temps-là, les trois
bâtardes du roi, les deux fils de M\textsuperscript{me} la Duchesse et
le prince de Conti avaient eu le temps d'arriver dans le cabinet. Après
que le roi eut fini avec le duc du Maine et le comte de Toulouse, il fit
appeler les princes du sang, qu'il avait aperçus sur la porte du
cabinet, dans sa chambre, et ne leur dit que peu de chose ensemble, et
point en particulier ni bas. Les médecins s'avancèrent presque en même
temps pour panser sa jambe. Les princes sortirent, il ne demeura que le
pur nécessaire et M\textsuperscript{me} de Maintenon. Tandis que tout
cela se passait, le chancelier prit à part M. le duc d'Orléans dans le
cabinet du conseil, et lui montra le codicille. Le roi pansé sut que les
princesses étaient dans le cabinet\,; il les fit appeler, leur dit deux
mots tout haut, et, prenant occasion de leurs larmes, les pria de s'en
aller, parce qu'il voulait reposer. Elles sorties avec le peu qui était
entré, le rideau du lit fut un peu tiré\,; et M\textsuperscript{me} de
Maintenon passa dans les arrière-cabinets.

Le lundi 26 août la nuit ne fut pas meilleure. Il fut pansé, puis
entendit la messe. Il y avait le pur nécessaire dans la chambre, qui
sortit après la messe. Le roi fit demeurer les cardinaux de Rohan et de
Bissy. M\textsuperscript{me} de Maintenon resta aussi comme elle
demeurait toujours, et avec elle le maréchal de Villeroy, le P. Tellier
et le chancelier. Il appela les deux cardinaux, protesta qu'il mourait
dans la foi et la soumission à l'Église, puis ajouta en les regardant
qu'il était fâché de laisser les affaires de l'Église en l'état où elles
étaient, qu'il y était parfaitement ignorant\,; qu'ils savaient, et
qu'il les en attestait, qu'il n'y avait rien fait que ce qu'ils avaient
voulu\,; qu'il y avait fait tout ce qu'ils avaient voulu\,; que c'était
donc à eux à répondre devant Dieu pour lui de tout ce qui s'y était
fait, et du trop ou du trop peu\,; qu'il protestait de nouveau qu'il les
en chargeait devant Dieu, et qu'il en avait la conscience nette, comme
un ignorant qui s'était abandonné absolument à eux dans toute la suite
de l'affaire. Quel affreux coup de tonnerre\,! mais les deux cardinaux
n'étaient pas pour s'en épouvanter, leur calme était à toute épreuve.
Leur réponse ne fut que sécurité et louanges\,; et le roi à répéter que,
dans son ignorance, il avait cru ne pouvoir mieux faire pour sa
conscience que de se laisser conduire en toute confiance par eux, par
quoi il était déchargé devant Dieu sur eux. Il ajouta que, pour le
cardinal de Noailles, Dieu lui était témoin qu'il ne le haïssait point,
et qu'il avait toujours été fâché de ce qu'il avait cru devoir faire
contre lui. À ces dernières paroles Bloin, Fagon, tout baissé et tout
courtisan qu'il était, et Maréchal qui étaient en vue, et assez près du
roi, se regardèrent et se demandèrent entre haut et bas si on laisserait
mourir le roi sans voir son archevêque, sans marquer par là
réconciliation et pardon, que c'était un scandale nécessaire à lever. Le
roi, qui les entendit, reprit la parole aussitôt, et déclara que non
seulement il ne s'y sentait point de répugnance, mais qu'il le désirait.

Ce mot interdit les deux cardinaux bien plus que la citation que le roi
venait de leur faire devant Dieu à sa décharge. M\textsuperscript{me} de
Maintenon en fut effrayée\,; le P. Tellier en trembla. Un retour de
confiance dans le roi, un autre de générosité et de vérité dans le
pasteur, les intimidèrent. Ils redoutèrent les moments où le respect et
la crainte fuient si loin devant des considérations plus
prégnantes\footnote{On a déjà vu plus haut ce mot dans le sens de
  \emph{pressantes}.}. Le silence régnait dans ce terrible embarras. Le
roi le rompit par ordonner au chancelier d'envoyer sur-le-champ chercher
le cardinal de Noailles, si ces messieurs, en regardant les cardinaux de
Rohan et de Bissy, jugeaient qu'il n'y eût point d'inconvénient. Tous
deux se regardèrent, puis s'éloignèrent jusque vers la fenêtre, avec le
P. Tellier, le chancelier et M\textsuperscript{me} de Maintenon. Tellier
cria tout bas et fut appuyé de Bissy. M\textsuperscript{me} de Maintenon
trouva la chose dangereuse\,; Rohan, plus doux ou plus politique sur le
futur, ne dit rien\,; le chancelier non plus. La résolution enfin fut de
finir la scène comme ils l'avaient commencée et conduite jusqu'alors, en
trompant le roi et se jouant de lui. Ils s'en rapprochèrent et lui
firent entendre, avec force louanges, qu'il ne fallait pas exposer la
bonne cause au triomphe de ses ennemis, et à ce qu'ils sauraient tirer
d'une démarche qui ne partait que de la bonne volonté du roi et d'un
excès de délicatesse de conscience\,; qu'ainsi ils approuvaient bien que
le cardinal de Noailles eût l'honneur de le voir, mais à condition qu'il
accepterait la constitution, et qu'il en donnerait sa parole. Le roi
encore en cela se soumit à leur avis, mais sans raisonner, et dans le
moment le chancelier écrivit conformément, et dépêcha au cardinal de
Noailles.

Dès que le roi eut consenti, les deux cardinaux le flattèrent de la
grande œuvre qu'il allait opérer (tant leur frayeur fut grande qu'il ne
revint à le vouloir voir sans condition, dont le piège était si
misérable et si aisé à découvrir), ou en ramenant le cardinal de
Noailles, ou en manifestant par son refus et son opiniâtreté invincible
à troubler l'Église, et son ingratitude consommée pour un roi à qui il
devait tout, et qui lui tendait ses bras mourants. Le dernier arriva. Le
cardinal de Noailles fut pénétré de douleur de ce dernier comble de
l'artifice. Il avait tort ou raison devant tout parti sur l'affaire de
la constitution\,; mais quoi qu'il en fût, l'événement de la mort
instante du roi n'opérait rien sur la vérité de cette matière, ni ne
pouvait opérer, par conséquent, aucun changement d'opinion. Rien de plus
touchant que la conjoncture, mais rien de plus étranger à la question,
rien aussi de plus odieux que ce piège qui, par rapport au roi, de
l'état duquel ils achevèrent d'abuser si indignement, et par rapport au
cardinal de Noailles qu'ils voulurent brider ou noircir si
grossièrement. Ce trait énorme émut tout le public contre eux, avec
d'autant plus de violence, que l'extrémité du roi rendit la liberté que
sa terreur avait si longtemps retenue captive. Mais quand on en sut le
détail, et l'apostrophe du roi aux deux cardinaux, sur le compte qu'ils
auraient à rendre pour lui de tout ce qu'il avait fait sur la
constitution, et le détail de ce qui là même s'était passé, tout de
suite sur le cardinal de Noailles, l'indignation générale rompit les
digues, et ne se contraignit plus\,; personne au contraire qui blâmât le
cardinal de Noailles, dont la réponse au chancelier fut en peu de mots
un chef-d'œuvre de religion, de douleur et de sagesse.

Ce même lundi, 26 août, après que les deux cardinaux furent sortis, le
roi dîna dans son lit en présence de ce qui avait les entrées. Il les
fit approcher comme on desservait, et leur dit ces paroles qui furent à
l'heure même recueillies\,: «\,Messieurs, je vous demande pardon du
mauvais exemple que je vous ai donné. J'ai bien à vous remercier de la
manière dont vous m'avez servi, et de l'attachement et de la fidélité
que vous m'avez toujours marqués. Je suis bien fâché de n'avoir pas fait
pour vous ce que j'aurais bien voulu faire. Les mauvais temps en sont
cause. Je vous demande pour mon petit-fils la même application et la
même fidélité que vous avez eue pour moi. C'est un enfant qui pourra
essuyer bien des traverses. Que votre exemple en soit un pour tous mes
autres sujets. Suivez les ordres que mon neveu vous donnera, il va
gouverner le royaume. J'espère qu'il le fera bien\,; j'espère aussi que
vous contribuerez tous à l'union, et que si quelqu'un s'en écartait,
vous aideriez à le ramener. Je sens que je m'attendris, et que je vous
attendris aussi, je vous en demande pardon. Adieu, messieurs, je compte
que vous vous souviendrez quelquefois de moi.\,»

Un peu après que tout le monde fut sorti, le roi demanda le maréchal de
Villeroy, et lui dit ces mêmes paroles qu'il retint bien, et qu'il a
depuis rendues\,: «\,Monsieur le maréchal, je vous donne une nouvelle
marque de mon amitié et de ma confiance en mourant. Je vous fais
gouverneur du Dauphin, qui est l'emploi le plus important que je puisse
donner. Vous saurez par ce qui est dans mon testament ce que vous aurez
à faire à l'égard du duc du Maine. Je ne doute pas que vous ne me
serviez après ma mort avec la même fidélité que vous l'avez fait pendant
ma vie. J'espère que mon neveu vivra avec vous avec la considération et
la confiance qu'il doit avoir pour un homme que j'ai toujours aimé.
Adieu, monsieur le maréchal, j'espère que vous vous souviendrez de
moi.\,»

Le roi, après quelque intervalle, fit appeler M. le Duc et M. le prince
de Conti, qui étaient dans les cabinets\,; et sans les faire trop
approcher, il leur recommanda l'union désirable entre les princes, et de
ne pas suivre les exemples domestiques sur les troubles et les guerres.
Il ne leur en dit pas davantage\,; puis entendant des femmes dans le
cabinet, il comprit bien qui elles étaient, et tout de suite leur manda
d'entrer. C'était M\textsuperscript{me} la duchesse de Berry, Madame,
M\textsuperscript{me} la duchesse d'Orléans, et les princesses du sang
qui criaient, et à qui le roi dit qu'il ne fallait point crier ainsi.

Il leur fit des amitiés courtes, distingua Madame, et finit par exhorter
M\textsuperscript{me} la duchesse d'Orléans et M\textsuperscript{me} la
Duchesse de se raccommoder. Tout cela fut court, et il les congédia.
Elles se retirèrent par les cabinets pleurant et criant fort, ce qui fit
croire au dehors, parce que les fenêtres du cabinet étaient ouvertes,
que le roi était mort, dont le bruit alla à Paris et jusque dans les
provinces.

Quelque temps après il manda à la duchesse de Ventadour de lui amener le
Dauphin. Il le fit approcher et lui dit ces paroles devant
M\textsuperscript{me} de Maintenon et le très peu des plus intimement
privilégiés ou valets nécessaires qui les recueillirent\,: «\,Mon
enfant, vous allez être un grand roi\,; ne m'imitez pas dans le goût que
j'ai eu pour les bâtiments, ni dans celui que j'ai eu pour la guerre\,;
tâchez, au contraire, d'avoir la paix avec vos voisins. Rendez à Dieu ce
que vous lui devez\,; reconnaissez les obligations que vous lui avez,
faites-le honorer par vos sujets. Suivez toujours les bons conseils,
tachez de soulager vos peuples\,; ce que je suis assez malheureux pour
n'avoir pu faire. N'oubliez point la reconnaissance que vous devez à
M\textsuperscript{me} de Ventadour. Madame, s'adressant à elle, que je
l'embrasse, et en l'embrassant lui dit\,: Mon cher enfant, je vous donne
ma bénédiction de tout mon cœur.\,» Comme on eut ôté le petit prince de
dessus le lit du roi, il le redemanda, l'embrassa de nouveau, et, levant
les mains et les yeux au ciel, le bénit encore. Ce spectacle fut
extrêmement touchant\,; la duchesse de Ventadour se hâta d'emporter le
Dauphin et de le ramener dans son appartement.

Après une courte pause, le roi fit appeler le duc du Maine et le comte
de Toulouse, fit sortir tout ce peu qui était dans sa chambre et fermer
les portes. Ce particulier dura assez longtemps. Les choses remises dans
leur ordre accoutumé, quand il eut fait avec eux, il envoya chercher M.
le duc d'Orléans qui était chez lui. Il lui parla fort peu de temps et
le rappela comme il sortait pour lui dire encore quelque chose qui fut
fort court. Ce fut là qu'il lui ordonna de faire conduire, dès qu'il
serait mort, le roi futur à Vincennes, dont l'air est bon, jusqu'à ce
que toutes les cérémonies fussent finies à Versailles et le château bien
nettoyé après, avant de le ramener à Versailles, où il destinait son
séjour. Il en avait apparemment parlé auparavant au duc du Maine et au
maréchal de Villeroy, car après que M. le duc d'Orléans fut sorti, il
donna ses ordres pour aller meubler Vincennes, et mettre ce lieu en état
de recevoir incessamment son successeur. M\textsuperscript{me} du Maine,
qui jusqu'alors n'avait pas pris la peine de bouger de Sceaux, avec ses
compagnies et ses passe-temps, était arrivée à Versailles, et fit
demander au roi la permission de le voir un moment après ces ordres
donnés. Elle était déjà dans l'antichambre\,: elle entra et sortit un
moment après.

Le mardi 27 août personne n'entra dans la chambre du roi que le P.
Tellier, M\textsuperscript{me} de Maintenon, et pour la messe seulement
le cardinal de Rohan et les deux aumôniers de quartier. Sur les deux
heures, il envoya chercher le chancelier, et, seul avec lui et
M\textsuperscript{me} de Maintenon, lui fit ouvrir deux cassettes
pleines de papiers, dont il lui fit brûler beaucoup, et lui donna ses
ordres pour ce qu'il voulut qu'il fît des autres. Sur les six heures du
soir, il manda encore le chancelier. M\textsuperscript{me} de Maintenon
ne sortit point de sa chambre de la journée, et personne n'y entra que
les valets, et dans des moments, l'apparition du service le plus
indispensable. Sur le soir il fit appeler le P. Tellier, et presque
aussitôt après qu'il lui eut parlé, il envoya chercher Pontchartrain, et
lui ordonna d'expédier aussitôt qu'il serait mort un ordre pour faire
porter son cœur dans l'église de la maison professe des jésuites à
Paris, et l'y faire placer vis-à-vis celui du roi son père, et de la
même manière.

Peu après, il se souvint que Cavoye, grand maréchal des logis de sa
maison, n'avait jamais fait les logements de la cour à Vincennes, parce
qu'il y avait cinquante ans que la cour n'y avait été\,; il indiqua une
cassette où on trouverait le plan de ce château, et ordonna de le
prendre et de le porter à Cavoye. Quelque temps après ces ordres donnés,
il dit à M\textsuperscript{me} de Maintenon qu'il avait toujours ouï
dire qu'il était difficile de se résoudre à la mort\,; que pour lui, qui
se trouvait sur le point de ce moment si redoutable aux hommes, il ne
trouvait pas que cette résolution fût si pénible à prendre. Elle lui
répondit qu'elle l'était beaucoup quand on avait de l'attachement aux
créatures, de la haine dans le cœur, des restitutions à faire. «\,Ah\,!
reprit le roi, pour des restitutions à faire, je n'en dois à personne
comme particulier\,; mais pour celles que je dois au royaume, j'espère
en la miséricorde de Dieu.\,» La nuit qui suivit fut fort agitée. On lui
voyait à tous moments joindre les mains, et on l'entendait dire les
prières qu'il avait accoutumées en santé, et se frapper la poitrine au
\emph{Confiteor}.

Le mercredi 28 août, il fit le matin une amitié à M\textsuperscript{me}
de Maintenon qui ne lui plut guère, et à laquelle elle ne répondit pas
un mot. Il lui dit que ce qui le consolait de la quitter était
l'espérance, à l'âge où elle était, qu'ils se rejoindraient bientôt. Sur
les sept heures du matin, il fit appeler le P. Tellier, et comme il lui
parlait de Dieu, il vit dans le miroir de sa cheminée deux garçons de sa
chambre assis au pied de son lit qui pleuraient. Il leur dit\,:
«\,Pourquoi pleurez-vous\,? est-ce que vous m'avez cru immortel\,? Pour
moi, je n'ai point cru l'être, et vous avez dû, à l'âge où je suis, vous
préparer à me perdre.\,»

Une espèce de manant provençal, fort grossier, apprit l'extrémité du roi
en chemin de Marseille à Paris, et vint ce matin-ci à Versailles avec un
remède qui, disait-il, guérissait la gangrène. Le roi était si mal, et
les médecins tellement à bout, qu'ils y consentirent sans difficulté en
présence de M\textsuperscript{me} de Maintenon et du duc du Maine. Fagon
voulut dire quelque chose\,; ce manant, qui se nommait Le Brun, le
malmena fort brutalement, dont Fagon, qui avait accoutumé de malmener
les autres et d'en être respecté jusqu'au tremblement, demeura tout
abasourdi. On donna donc au roi dix gouttes de cet élixir dans du vin
d'Alicante, sur les onze heures du matin. Quelque temps après il se
trouva plus fort, mais le pouls étant retombé et devenu fort mauvais, on
lui en présenta une autre prise sur les quatre heures, en lui disant que
c'était pour le rappeler à la vie. Il répondit en prenant le verre où
cela était\,: «\,A la vie ou à la mort\,! tout ce qui plaira à Dieu.\,»

M\textsuperscript{me} de Maintenon venait de sortir de chez le roi, ses
coiffes baissées, menée par le maréchal de Villeroy par-devant chez elle
sans y entrer, jusqu'au bas du grand degré où elle leva ses coiffes.
Elle embrassa le maréchal d'un oeil fort sec, en lui disant\,: «\,Adieu,
monsieur le maréchal\,!» monta dans un carrosse du roi qui la servait
toujours, dans lequel M\textsuperscript{me} de Caylus l'attendait seule,
et s'en alla à Saint-Cyr, suivie de son carrosse où étaient ses femmes.
Le soir le duc du Maine fit chez lui une gorge chaude fort plaisante de
l'aventure de Fagon avec Le Brun. On reviendra ailleurs à parler de sa
conduite, et de celle de M\textsuperscript{me} de Maintenon et du P.
Tellier en ces derniers jours de la vie du roi. Le remède de Le Brun fut
continué comme il voulut, et il le vit toujours prendre au roi. Sur un
bouillon qu'on lui proposa de prendre, il répondit qu'il ne fallait pas
lui parler comme à un autre homme\,; que ce n'était pas un bouillon
qu'il lui fallait, mais son confesseur\,; et il le fit appeler. Un jour
qu'il revenait d'une perte de connaissance, il demanda l'absolution
générale de ses péchés au P. Tellier, qui lui demanda s'il souffrait
beaucoup. «\,Eh\,! non, répondit le roi, c'est ce qui me fâche, je
voudrais souffrir davantage pour l'expiation de mes péchés.\,»

Le jeudi 29 août dont la nuit et le jour précédents avaient été si
mauvais, l'absence des tenants qui n'avaient plus à besogner au delà de
ce qu'ils avaient fait, laissa l'entrée de la chambre plus libre aux
grands officiers qui en avaient toujours été exclus. Il n'y avait point
eu de messe la veille, et on ne comptait plus qu'il y en eût. Le duc de
Charost, capitaine des gardes, qui s'était aussi glissé dans la chambre,
le trouva mauvais avec raison, et fit demander au roi par un des valets
familiers, s'il ne serait pas bien aise de l'entendre. Le roi dit qu'il
le désirait\,; sur quoi on alla quérir les gens et les choses
nécessaires, et on continua les jours suivants. Le matin de ce jeudi, il
parut plus de force, et quelque rayon de mieux qui fut incontinent
grossi, et dont le bruit courut de tous côtés. Le roi mangea même deux
petits biscuits dans un peu de vin d'Alicante avec une sorte d'appétit.
J'allai ce jour-là sur les deux heures après midi chez M. le duc
d'Orléans, dans les appartements duquel la foule était au point depuis
huit jours, et à toute heure, qu'exactement parlant, une épingle n'y
serait pas tombée à terre. Je n'y trouvai qui que ce soit. Dès qu'il me
vit, il se mit à rire et à me dire que j'étais le premier homme qu'il
eût encore vu chez lui de la journée, qui, jusqu'au soir fut entièrement
déserte chez lui. Voilà le monde.

Je pris ce temps de loisir pour lui parler de bien des choses. Ce fut où
je reconnus qu'il n'était plus le même pour la convocation des états
généraux, et qu'excepté ce que nous avions arrêté sur les conseils, qui
a été expliqué ici en son temps, il n'y avait pas pensé depuis, ni à
bien d'autres choses, dont je pris la liberté de lui dire fortement mon
avis. Je le trouvai toujours dans la même résolution de chasser
Desmarets et Pontchartrain, mais d'une mollesse sur le chancelier qui
m'engagea à le presser et à le forcer de s'expliquer. Enfin il m'avoua
avec une honte extrême que M\textsuperscript{me} la duchesse d'Orléans,
que le maréchal de Villeroy était allé trouver en secret même de lui,
l'avait pressé de le voir et de s'accommoder avec lui sur des choses
fort principales auxquelles il voulait bien se prêter sous un grand
secret, et qui l'embarrasseraient périlleusement s'il refusait d'y
entrer, s'excusant de s'en expliquer davantage sur le secret qu'elle
avait promis au maréchal, et sans lequel il ne se serait pas ouvert à
elle\,; qu'après avoir résisté à le voir, il y avait consenti\,; que le
maréchal était venu chez lui, il y avait quatre ou cinq jours, en grand
mystère, et pour prix de ce qu'il voulait bien lui apprendre et faire,
il lui avait demandé sa parole de conserver le chancelier dans toutes
ses fonctions de chancelier et de garde des sceaux, moyennant la parole
qu'il avait du chancelier, dont il demeurait garant, de donner sa
démission de la charge de secrétaire d'État, dès qu'il l'en ferait
rembourser en entier\,; qu'après une forte dispute, et la parole donnée
pour le chancelier, le maréchal lui avait dit que M. du Maine était
surintendant de l'éducation, et lui gouverneur avec toute autorité\,;
qu'il lui avait appris après le codicille et ce qu'il portait, et que ce
que le maréchal voulait bien faire était de n'en point profiter dans
toute son étendue\,; que cela avait produit une dispute fort vive sans
être convenus de rien, quant au maréchal, mais bien quant au chancelier,
qui là-dessus l'en avait remercié dans le cabinet du roi, confirmé la
parole de sa démission de secrétaire d'État aux conditions susdites, et
pour marque de reconnaissance lui avait là même montré le codicille.

J'avoue que je fus outré d'un commencement si faible et si dupe, et que
je ne le cachai pas à M. le duc d'Orléans, dont l'embarras avec moi fut
extrême. Je lui demandai ce qu'il avait fait de son discernement, lui
qui n'avait jamais mis de différence entre M. du Maine et
M\textsuperscript{me} la duchesse d'Orléans, dont il m'avait tant de
fois recommandé de me défier et de me cacher, et si souvent répété par
rapport à elle que nous étions dans un bois. S'il n'avait pas vu le jeu
joué entre M. du Maine et M\textsuperscript{me} la duchesse d'Orléans,
pour lui faire peur par le maréchal de Villeroy, découvrir ce qu'ils
auraient à faire, en découvrant comme il prendrait la proposition et la
confidence de ce qui n'allait à rien moins qu'à l'égorger, et ne
hasardant rien à tenter de conserver à si bon marché leur créature
abandonnée, et l'instrument pernicieux de tout ce qui s'était fait
contre lui, et dans une place aussi importante dans une régence dont ils
prétendaient bien ne lui laisser que l'ombre.

Cette matière se discuta longuement entre nous deux\,; mais la parole
était donnée. Il n'avait pas eu la force de résister\,; et avec tant
d'esprit, il avait été la dupe de croire faire un bon marché par une
démission, en remboursant, marché que le chancelier faisait bien
meilleur en s'assurant du remboursement entier d'une charge qu'il
sentait bien qu'il ne se pouvait jamais conserver, et qui lui valait la
sûreté de demeurer dans la plus importante place, tandis que le moindre
ordre suffisait pour lui faire rendre les sceaux, l'exiler où on aurait
voulu, et lui supprimer une charge qui, comme on l'a vu, ne lui coûtait
plus rien depuis que le roi lui en avait rendu ce qu'elle avait été
payée, lui qui sentait tout ce qu'il méritait de M. le duc d'Orléans, et
qui avec la haine et le mépris de la cour, et du militaire, qu'il
s'était si bien et si justement acquis, n'avait plus de bouclier ni de
protection après le roi, du moment que son testament serait tacitement
cassé, comme lui-même n'en doutait pas. Aux choses faites, il n'y a plus
de remède\,; mais je conjurai M. le duc d'Orléans d'apprendre de cette
funeste leçon à être en garde désormais contre les ennemis de toute
espèce, contre la duperie, la facilité, la faiblesse surtout de sentir
l'affront et le péril du codicille, s'il en souffrait l'exécution en
quoi que ce pût être.

Jamais il ne me put dire à quoi il en était là-dessus avec le maréchal
de Villeroy. Seulement était-il constant qu'il n'avait été question de
rien par rapport au duc du Maine, qui par conséquent se comptait
demeurer maître absolu et indépendant de la maison du roi civile et
militaire, ce qui subsistant, peu importait de la cascade du maréchal de
Villeroy, sinon au maréchal, mais qui faisait du duc du Maine un maire
du palais, et de M. le duc d'Orléans un fantôme de régent impuissant et
ridicule, et une victime sans cesse sous le couteau du maire du palais.
Ce prince, avec tout son génie, n'en avait pas tant vu. Je le laissai
fort pensif et fort repentant d'une si lourde faute. Il reparla si ferme
à M\textsuperscript{me} la duchesse d'Orléans qu'ils eurent peur qu'il
ne tint rien pour avoir trop promis. Le maréchal mandé par elle fila
doux, et ne songea qu'à bien serrer ce qu'il avait saisi, en faisant
entendre qu'à son égard il ne disputerait rien qui pût porter ombrage\,;
mais la mesure de la vie du roi se serrait de si près qu'il échappa
aisément à plus d'éclaircissements, et que, par ce qu'il s'était passé
dans le cabinet du roi, du chancelier et de M. le duc d'Orléans
immédiatement, la bécasse demeura bridée à son égard, si j'ose me servir
de ce misérable mot.

Le soir fort tard ne répondit pas à l'applaudissement qu'on avait voulu
donner à la journée, pendant laquelle il {[}le roi{]} avait dit au curé
de Versailles, qui avait profité de la liberté d'entrer, qu'il n'était
pas question de sa vie, sur {[}ce{]} qu'il lui disait que tout était en
prières pour la demander, mais de son salut pour lequel il fallait bien
prier. Il lui échappa ce même jour, en donnant des ordres, d'appeler le
Dauphin le jeune roi. Il vit un mouvement dans ce qui était autour de
lui. «\,Eh pourquoi\,? leur dit-il, cela ne me fait aucune peine.\,» Il
prit sur les huit heures du soir de l'élixir de cet homme de Provence.
Sa tête parut embarrassée\,; il dit lui-même qu'il se sentait fort mal.
Vers onze heures du soir sa jambe fut visitée. La gangrène se trouva
dans tout le pied, dans le genou, et la cuisse fort enflée. Il
s'évanouit pendant cet examen. Il s'était aperçu avec peine de l'absence
de M\textsuperscript{me} de Maintenon, qui ne comptait plus revenir. Il
la demanda plusieurs fois dans la journée\,; on ne lui put cacher son
départ. Il l'envoya chercher à Saint-Cyr\,; elle revint le soir.

Le vendredi 30 août, la journée fut aussi fâcheuse qu'avait été la nuit,
un grand assoupissement, et dans les intervalles la tête embarrassée. Il
prit de temps en temps un peu de gelée et de l'eau pure, ne pouvant plus
souffrir le vin. Il n'y eut dans sa chambre que les valets les plus
indispensables pour le service, et la médecine, M\textsuperscript{me} de
Maintenon et quelques rares apparitions du P. Tellier, que Bloin ou
Maréchal envoyaient chercher. Il se tenait peu même dans les cabinets,
non plus que M. du Maine. Le roi revenait aisément à la piété quand
M\textsuperscript{me} de Maintenon ou le P. Tellier trouvaient les
moments où sa tête était moins embarrassée\,; mais ils étaient rares et
courts. Sur les cinq heures du soir, M\textsuperscript{me} de Maintenon
passa chez elle, distribua ce qu'elle avait de meubles dans son
appartement à son domestique, et s'en alla à Saint-Cyr pour n'en sortir
jamais.

Le samedi 31 août la nuit et la journée furent détestables. Il n'y eut
que de rares et de courts instants de connaissance. La gangrène avait
gagné le genou et toute la cuisse. On lui donna du remède du feu abbé
Aignan, que la duchesse du Maine avait envoyé proposer, qui était un
excellent remède pour la petite vérole. Les médecins consentaient à
tout, parce qu'il n'y avait plus d'espérance. Vers onze heures du soir
on le trouva si mal qu'on lui dit les prières des agonisants. L'appareil
le rappela à lui. Il récita des prières d'une voix si forte qu'elle se
faisait entendre à travers celle du grand nombre d'ecclésiastiques et de
tout ce qui était entré. À la fin des prières, il reconnut le cardinal
de Rohan, et lui dit\,: «\,Ce sont là les dernières grâces de
l'Église.\,» Ce fut le dernier homme à qui il parla. Il répéta plusieurs
fois\,: \emph{Nunc et in hora mortis}, puis dit\,: «\,O mon Dieu, venez
à mon aide, hâtez-vous de me secourir\,!» Ce furent ses dernières
paroles. Toute la nuit fut sans connaissance, et une longue agonie, qui
finit le dimanche 1er septembre 1715, à huit heures un quart du matin,
trois jours avant qu'il eût soixante-dix-sept ans accomplis, dans la
soixante-douzième année de son règne.

Il se maria à vingt-deux ans, en signant la fameuse paix des Pyrénées en
1660. Ii en avait vingt-trois, quand la mort délivra la France du
cardinal Mazarin\footnote{Le cardinal Mazarin mourut le 9 mars 1661.}\,;
vingt-sept, lorsqu'il perdit la reine sa mère en 1666. Il devint veuf à
quarante-quatre ans en 1683, perdit Monsieur à soixante-trois ans en
1701, et survécut tous ses fils et petits-fils, excepté son successeur,
le roi d'Espagne, et les enfants de ce prince. L'Europe ne vit jamais un
si long règne, ni la France un roi si âgé.

Par l'ouverture de son corps qui fut faite par Maréchal, son premier
chirurgien, avec l'assistance et les cérémonies accoutumées, on lui
trouva toutes les parties si entières, si saines et tout si parfaitement
conformé, qu'on jugea qu'il aurait vécu plus d'un siècle sans les fautes
dont il a été parlé qui lui mirent la gangrène dans le sang. On lui
trouva aussi la capacité de l'estomac et des intestins double au moins
des hommes de sa taille\,; ce qui est fort extraordinaire, et ce qui
était cause qu'il était si grand mangeur et si égal.

Ce fut un prince à qui on ne peut refuser beaucoup de bon, même de
grand, en qui on ne peut méconnaître plus de petit et de mauvais, duquel
il n'est pas possible de discerner ce qui était de lui ou emprunté\,; et
dans l'un et dans l'autre rien de plus rare que des écrivains qui en
aient été bien informés, rien de plus difficile à rencontrer que des
gens qui l'aient connu par eux-mêmes et par expérience et capables d'en
écrire, en même temps assez maîtres d'eux-mêmes pour en parler sans
haine ou sans flatterie, de n'en rien dire que dicté par la vérité nue
en bien et en mal. Pour la première partie on peut ici compter sur
elle\,; pour l'autre on tâchera d'y atteindre en suspendant de bonne foi
toute passion.

\hypertarget{chapitre-xvi.}{%
\chapter{CHAPITRE XVI.}\label{chapitre-xvi.}}

~

{\textsc{Caractère de Louis XIV.}} {\textsc{- M\textsuperscript{me} de
La Vallière\,; son caractère.}} {\textsc{- Le roi hait les sujets, est
petit, dupe, gouverné en se piquant de tout le contraire.}} {\textsc{-
L'Espagne cède la préséance.}} {\textsc{- Satisfaction de l'affaire des
Corses.}} {\textsc{- Guerre de Hollande.}} {\textsc{- Paix
d'Aix-la-Chapelle.}} {\textsc{- Siècle florissant.}} {\textsc{-
Conquêtes en Hollande et de la Franche-Comté.}} {\textsc{- Honte
d'Heurtebise.}} {\textsc{- Le roi prend Cambrai.}} {\textsc{- Monsieur
bat le prince d'Orange à Cassel, prend Saint-Omer, et n'a pas depuis
commandé d'armée.}} {\textsc{- Siège de Gand.}} {\textsc{- Expéditions
maritimes.}} {\textsc{- Paix de Nimègue.}} {\textsc{- Luxembourg pris.}}
{\textsc{- Gênes bombardée\,; son doge à Paris.}} {\textsc{- Fin du
premier âge de ce règne.}} {\textsc{- Guerre de 1688 et sa rare
origine.}} {\textsc{- Honte de la dernière campagne du roi.}} {\textsc{-
Paix de Turin, puis de Ryswick.}} {\textsc{- Fin du second âge de ce
règne.}} {\textsc{- Vertus de Louis XIV.}} {\textsc{- Sa misérable
éducation\,; sa profonde ignorance.}} {\textsc{- Il hait la naissance et
les dignités, séduit par ses ministres.}} {\textsc{- Superbe du roi, qui
forme le colosse de ses ministres sur la ruine de la noblesse.}}
{\textsc{- Goût de Louis XIV pour les détails.}} {\textsc{- Avantages de
ses ministres, qui abattent tout sous eux, et lui persuadant que leur
puissance et leur grandeur n'est que la sienne, se font plus que
seigneurs et tout-puissants.}} {\textsc{- Raison secrète de la
préférence des gens de rien pour le ministère.}} {\textsc{- Nul vrai
accès à Louis XIV enfermé par ses ministres.}} {\textsc{- Rareté et
utilité d'obtenir audience du roi.}} {\textsc{- Importance des grandes
entrées.}} {\textsc{- Ministres.}} {\textsc{- Causes de la superbe du
roi.}}

~

Il ne faut point parler ici des premières années {[}de Louis XIV{]}. Roi
presque en naissant, étouffé par la politique d'une mère qui voulait
gouverner, plus encore par le vif intérêt d'un pernicieux ministre, qui
hasarda mille fois l'État pour son unique grandeur, et asservi sous ce
joug tant que vécut son premier ministre, c'est autant de retranché sur
le règne de ce monarque. Toutefois il pointait sous ce joug. Il sentit
l'amour, il comprenait l'oisiveté comme l'ennemie de la gloire\,; il
avait essayé de faibles parties de main vers l'un et vers l'autre\,; il
eut assez de sentiment pour se croire délivré à la mort de Mazarin, s'il
n'eut pas assez de force pour se délivrer plus tôt. C'est même un des
beaux endroits de sa vie, et dont le fruit a été du moins de prendre
cette maxime, que rien n'a pu ébranler depuis, d'abhorrer tout premier
ministre, et non moins tout ecclésiastique dans son conseil. Il en prit
dès lois une autre, mais qu'il ne put soutenir avec la même fermeté,
parce qu'il ne s'aperçut presque pas dans l'effet qu'elle lui échappât
sans cesse, ce fut de gouverner par lui-même, qui fut la chose dont il
se piqua le plus, dont on le loua et le flatta davantage, et qu'il
exécuta le moins.

Né avec un esprit au-dessous du médiocre, mais un esprit capable de se
former, de se limer, de se raffiner, d'emprunter d'autrui sans imitation
et sans gêne, il profita infiniment d'avoir toute sa vie vécu avec les
personnes du monde qui toutes en avaient le plus, et des plus
différentes sortes, en hommes et en femmes de tout âge, de tout genre et
de tous personnages.

S'il faut parler ainsi d'un roi de vingt-trois ans, sa première entrée
dans le monde fut heureuse en esprits distingués de toute espèce. Ses
ministres au dedans et au dehors étaient alois les plus forts de
l'Europe, ses généraux les plus grands, leurs seconds les meilleurs, et
qui sont devenus des capitaines en leur école, et leurs noms aux uns et
aux autre sort passé comme tels à la postérité d'un consentement
unanime. Les mouvements dont l'État avait été si furieusement agité au
dedans et au dehors, depuis la mort de Louis XIII, avaient formé
quantité d'hommes qui composaient une cour d'habiles et d'illustres
personnages et de courtisans raffinés.

La maison de la comtesse de Soissons, qui, comme surintendante de la
maison de la reine, logeait à Paris aux Tuileries, où était la cour, qui
y régnait par un reste de la splendeur du feu cardinal Mazarin, son
oncle, et plus encore par son esprit et son adresse, en était devenue le
centre, mais fort choisi. C'était où se rendait tous les jours ce qu'il
y avait de plus distingué en hommes et en femmes, qui rendait cette
maison le centre de la galanterie de la cour, et des intrigues et des
menées de l'ambition, parmi lesquelles la parenté influait beaucoup,
autant comptée, prisée et respectée lors qu'elle est maintenant oubliée.
Ce fut dans cet important et brillant tourbillon où le roi se jeta
d'abord, et où il prit cet air de politesse et de galanterie qu'il a
toujours su conserver toute sa vie, qu'il a si bien su allier avec la
décence et la majesté. On peut dire qu'il était fait pour elle, et qu'au
milieu de tous les autres hommes, sa taille, son port, les grâces, la
beauté, et la grande mine qui succéda à la beauté, jusqu'au son de sa
voix et à l'adresse et la grâce naturelle et majestueuse de toute sa
personne, le faisaient distinguer jusqu'à sa mort comme le roi des
abeilles, et que, s'il ne fût né que particulier, il aurait eu également
le talent des fêtes, des plaisirs, de la galanterie, et de faire les
plus grands désordres d'amour. Heureux s'il n'eût eu que des maîtresses
semblables à M\textsuperscript{me} de La Vallière, arrachée à elle-même
par ses propres yeux, honteuse de l'être, encore plus des fruits de son
amour reconnus et élevés malgré elle, modeste, désintéressée, douce,
bonne au dernier point, combattant sans cesse contre elle-même,
victorieuse enfin de son désordre par les plus cruels effets de l'amour
et de la jalousie, qui furent tout à la fois son tourment et sa
ressource, qu'elle sut embrasser assez au milieu de ses douleurs pour
s'arracher enfin, et se consacrer à la plus dure et la plus sainte
pénitence\,! Il faut donc avouer que le roi fut plus à plaindre que
blâmable de se livrer à l'amour, et qu'il mérite louange d'avoir su s'en
arracher par intervalles en faveur de la gloire.

Les intrigues et les aventures que, tout roi qu'il était, il essuya dans
ce tourbillon de la comtesse de Soissons, lui firent des impressions qui
devinrent funestes, pour avoir été plus fortes que lui. L'esprit, la
noblesse de sentiments, se sentir, se respecter, avoir le cœur haut,
être instruit, tout cela lui devint suspect et bientôt haïssable. Plus
il avança en âge, plus il se confirma dans cette aversion. Il la poussa
jusque dans ses généraux et dans ses ministres, laquelle dans eux ne fut
contrebalancée que par le besoin, comme on le verra dans la suite. Il
voulait régner par lui-même. Sa jalousie là-dessus alla sans cesse
jusqu'à la faiblesse. Il régna en effet dans le petit\,; dans le grand
il ne put y atteindre\,; et jusque dans le petit il fut souvent
gouverné. Son premier saisissement des rênes de l'empire fut marqué au
coin d'une extrême dureté, et d'une extrême duperie. Fouquet\footnote{Voy.
  notes à la fin du volume.} fut le malheureux sur qui éclata la
première\,; Colbert fut le ministre de l'autre en saisissant seul toute
l'autorité des finances, et lui faisant accroire qu'elle passait toute
entre ses mains, par les signatures dont il l'accabla à la place de
celles que faisait le surintendant, dont Colbert supprima la charge à
laquelle il ne pouvait aspirer.

La préséance solennellement cédée par l'Espagne, et la satisfaction
entière qu'elle fit de l'insulte faite à cette occasion par le baron de
Vatteville, au comte depuis maréchal d'Estrades, ambassadeur des deux
couronnes à Londres, et l'éclatante raison tirée de l'insulte faite au
duc de Créqui, ambassadeur de France, par le gouvernement de Rome, par
les parents du pape et par les Corses de sa garde, furent les prémices
de ce règne par soi-même.

Bientôt après, la mort du roi d'Espagne fit saisir à ce jeune prince
avide de gloire une occasion de guerre, dont les renonciations si
récentes, et si soigneusement stipulées dans le contrat de mariage de la
reine, ne purent le détourner. Il marcha en Flandre\,; ses conquêtes y
furent rapides\,; le passage du Rhin fut signalé\,; la triple alliance
de l'Angleterre, la Suède et la Hollande, ne fit que l'animer. Il alla
prendre en plein hiver toute la Franche-Comté, qui lui servit à la paix
d'Aix-la-Chapelle à conserver des conquêtes de Flandre, en rendant la
Franche-Comté.

Tout était florissant dans l'État, tout y était riche. Colbert avait mis
les finances, la marine, le commerce, les manufactures, les lettres
même, au plus haut point\,; et ce siècle, semblable à celui d'Auguste,
produisait à l'envi des hommes illustres en tout genre, jusqu'à ceux
même qui ne sont bons que pour les plaisirs.

Le Tellier et Louvois son fils, qui avaient le département de la guerre,
frémissaient des succès et du crédit de Colbert, et n'eurent pas de
peine à mettre en tête au roi une guerre nouvelle, dont les succès
causèrent une telle frayeur à l'Europe que la France ne s'en a pu
remettre, et qu'après y avoir pensé succomber longtemps depuis, elle en
sentira longtemps le poids et les malheurs. Telle fut la véritable cause
de cette fameuse guerre de Hollande à laquelle le roi se laissa pousser,
et que son amour pour M\textsuperscript{me} de Montespan rendit si
funeste à son État et à sa gloire. Tout conquis, tout pris, et Amsterdam
prête à lui envoyer ses clefs, le roi cède à son impatience, quitte
l'armée, vole à Versailles, et détruit en un instant tout le succès de
ses armes. Il répara cette flétrissure par une seconde conquête de la
Franche-Comté, en personne, qui pour cette fois est demeurée à la
France.

En 1676, le roi retourna en Flandre, prit Condé\,; et Monsieur,
Bouchain. Les armées du roi et du prince d'Orange s'approchèrent si près
et si subitement qu'elles se trouvèrent en présence, et sans séparation,
auprès de la cense d'Heurtebise\footnote{Heurtebise ou Hurtubise, près
  de Valenciennes.}. Il fut donc question de décider si on donnerait
bataille, et de prendre son parti sur-le-champ. Monsieur n'avait pas
encore joint de Bouchain, mais le roi était sans cela supérieur à
l'armée ennemie. Les maréchaux de Schomberg, Humières, Ma Feuillade,
Lorges, etc., s'assemblèrent à cheval autour du roi, avec quelques-uns
des plus distingués d'entre les officiers généraux et des principaux
courtisans, pour tenir une espèce de conseil de guerre. Toute l'armée
criait au combat, et tous ces messieurs voyaient bien ce qu'il y avait à
faire, mais la personne du roi les embarrassait, et bien plus Louvois,
qui connaissait son maître, et qui cabalait depuis deux heures que l'on
commençait d'apercevoir ou les choses en pourraient venir. Louvois, pour
intimider la compagnie, parla le premier en rapporteur pour dissuader la
bataille. Le maréchal d'Humières, son ami intime et avec grande
dépendance, et le maréchal de Schomberg, qui le ménageait fort, furent
de son avis. Le maréchal de La Feuillade, hors de mesure avec Louvois,
mais favori qui ne connaissait pas moins bien de quel avis il fallait
être, après quelques propos douteux, conclut comme eux. M. de Lorges,
inflexible pour la vérité, touché de la gloire du roi, sensible au bien
de l'État, mal avec Louvois comme le neveu favori de M. de Turenne tué
l'année précédente, et qui venait d'être fait maréchal de France, malgré
ce ministre, et capitaine des gardes du corps, opina de toutes ses
forces pour la bataille, et il en déduisit tellement les raisons, que
Louvois même et les maréchaux demeurèrent sans repartie. Le peu de ceux
de moindre grade qui parlèrent après osèrent encore moins déplaire à
Louvois\,; mais, ne pouvant affaiblir les raisons de M. le maréchal de
Lorges, ils ne firent que balbutier. Le roi, qui écoutait tout, prit
encore les avis, ou plutôt simplement les voix, sans faire répéter ce
qui avait été dit par chacun, puis, avec un petit mot de regret de se
voir retenu par de si bonnes raisons, et du sacrifice qu'il faisait de
ses désirs à ce qui était de l'avantage de l'État, tourna bride, et il
ne fut plus question de bataille.

Le lendemain, et c'est de M. le maréchal de Lorges que je le tiens, qui
était la vérité même, et à qui je l'ai ouï raconter plus d'une fois et
jamais sans dépit, le lendemain, dis-je, il eut occasion d'envoyer un
trompette aux ennemis qui se retiraient. Ils le gardèrent un jour ou
deux en leur armée. Le prince d'Orange le voulut voir, et le questionna
tort sur ce qui avait empêché le roi de l'attaquer, se trouvant le plus
fort, les deux armées en vue si fort l'une de l'autre, et en rase
campagne, sans quoi que ce soit entre-deux. Après l'avoir fait causer
devant tout le monde, il lui dit avec un sourire malin, pour montrer
qu'il était tôt averti, et pour faire dépit au roi, qu'il ne manquât pas
de dire au maréchal de Lorges qu'il avait grande raison d'avoir voulu,
et si opiniâtrement soutenu la bataille\,; que jamais lui ne l'avait
manqué si belle, ni été si aise que de s'être vu hors de portée de la
recevoir\,; qu'il était battu sans ressource et sans le pouvoir éviter
s'il avait été attaqué, dont il se mit en peu de mots à déduire les
raisons. Le trompette, tout glorieux d'avoir eu avec le prince d'Orange
un si long et si curieux entretien, le débita non seulement à M. le
maréchal de Lorges, mais au roi, qui à la chaude le voulut voir, et de
là aux maréchaux, aux généraux et à qui le voulut entendre, et augmenta
ainsi le dépit de l'armée et en fit un grand à Louvois. Cette faute, et
ce genre de faute, ne fit que trop d'impression sur les troupes et
partout, excita de cruelles railleries parmi le monde et dans les cours
étrangères\footnote{Ce fut dans la nuit du 9 au 10 mai 1676 que le
  prince d'Orange passa l'Escaut et se trouva en présence de l'armée
  ennemie. Il n'avait que trente-cinq mille hommes, et Louis XIV en
  avait au moins quarante-huit mille (voy. \emph{Oeuvres de Louis XIV},
  t. IV, p.~26). Pellisson, qui accompagnait le roi et dont les
  \emph{Lettres historiques} sont un panégyrique perpétuel du prince,
  s'exprime ainsi (\emph{Lettres}, t. III, p.~52 et suiv.)\,: «\,Le jour
  parut à peine qu'on vit l'armée des ennemis se ranger en bataille sur
  une hauteur dans cet espace plus étroit, qui est entre la contrescarpe
  de Valenciennes et les bois de Vicogne et d'Aubri, qui font partie de
  ceux de Saint-Amand. Ils descendirent même une fois de cette hauteur,
  comme pour s'avancer, mais après ils s'y retirèrent comme pour ne
  point perdre cet avantage. Au commencement ce n'était, comme on l'a su
  depuis, que la garnison de Valenciennes, qui parut pour donner lieu
  aux troupes de se poster à mesure qu'elles arrivaient. Le roi en jugea
  très sainement, et bien qu'il n'eût encore que huit ou dix escadrons
  avec lui, il proposa d'aller charger cette armée, comme elle arrivait
  encore en désordre, persuade qu'on la déferait aisément. Mais M. le
  maréchal de Schomberg, M. de La Feuillade et \emph{enfin tout ce qu'il
  y avait d'officiers généraux auprès de lui}, prirent la liberté de lui
  représenter quel inconvénient il y avait de hasarder la personne de Sa
  Majesté, sans en savoir davantage. Le roi dit à ces messieurs
  \emph{qu'ils avaient plus d'expérience que lui et qu'il leur cédait\,!
  mais à regret}.\,» Louis XIV, dans une lettre au maréchal de Villeroy
  (\emph{Oeuvres de Louis XIV}, t. IV, p 83), dit que \emph{l'affaire
  était faite, si les ennemis eussent voulu}, attribuant ainsi
  l'occasion manquée à une retraite précipitée du prince d'Orange.}. Le
roi ne demeura guère à l'armée depuis, quoiqu'on ne fût qu'au mois de
mai. Il s'en revint trouver sa maîtresse.

L'année suivante il retourna en Flandre, il prit Cambrai\,; et Monsieur
fit cependant le siège de Saint-Omer. Il fut au-devant du prince
d'Orange qui venait secourir la place, lui donna bataille près de Cassel
et remporta une victoire complète, prit tout de suite Saint-Omer, puis
alla rejoindre le roi. Ce contraste fut si sensible au monarque que
jamais depuis il ne donna d'armée à commander à Monsieur. Tout
l'extérieur fut parfaitement gardé, mais dès ce moment la résolution fut
prise, et toujours depuis bien tenue.

L'année d'après le roi fit en personne le siège de Gand, dont le projet
et l'exécution fut le chef-d'oeuvre de Louvois. La paix de Nimègue mit
fin cette année à la guerre avec la Hollande, l'Espagne, etc.\,; et au
commencement de l'année suivante, avec l'empereur et l'empire.
L'Amérique, l'Afrique, l'Archipel, la Sicile ressentirent vivement la
puissance de la France\,; et en 1684 Luxembourg fut le prix des
retardements des Espagnols à satisfaire à toutes les conditions de la
paix. Gênes bombardée se vit forcée à venir demander la paix par son
doge en personne accompagné de quatre sénateurs, au commencement de
l'année suivante. Depuis, jusqu'en 1688, le temps se passa dans le
cabinet moins en fêtes qu'en dévotion et en contrainte. Ici finit
l'apogée de ce règne, et ce comble de gloire et de prospérité. Les
grands capitaines, les grands ministres au dedans et au dehors n'étaient
plus, mais il en restait les élèves. Nous en allons voir le second âge
qui ne répondra guère au premier, mais qui en tout fut encore plus
différent du dernier.

La guerre de 1688 eut une étrange origine, dont l'anecdote, également
certaine et curieuse, est si propre à caractériser le roi et Louvois son
ministre qu'elle doit tenir place ici. Louvois, à la mort de Colbert,
avait eu sa surintendance des bâtiments. Le petit Trianon de porcelaine,
fait autrefois pour M\textsuperscript{me} de Montespan, ennuyait le roi,
qui voulait partout des palais. Il s'amusait fort à ses bâtiments. Il
avait aussi le compas dans l'oeil pour la justesse, les proportions, la
symétrie, mais le goût n'y répondait pas, comme on le verra ailleurs. Ce
château ne faisait presque que sortir de terre, lorsque le roi s'aperçut
d'un défaut à une croisée qui s'achevait de former, dans la longueur du
rez-de-chaussée. Louvois, qui naturellement était brutal, et de plus
gâté jusqu'à souffrir difficilement d'être repris par son maître,
disputa fort et ferme, et maintint que la croisée était bien. Le roi
tourna le dos, et s'alla promener ailleurs dans le bâtiment.

Le lendemain il trouve Le Nôtre, bon architecte, mais fameux par le goût
des jardins qu'il a commencé à introduire en France, et dont il a porté
la perfection au plus haut point. Le roi lui demanda s'il avait été à
Trianon. Il répondit que non. Le roi lui expliqua ce qui l'avait choqué,
et lui dit d'y aller. Le lendemain même question, même réponse\,; le
jour d'après autant. Le roi vit bien qu'il n'osait s'exposer à trouver
qu'il eût tort, ou à blâmer Louvois. Il se fâcha, et lui ordonna de se
trouver le lendemain à Trianon lorsqu'il y irait, et où il ferait
trouver Louvois aussi. Il n'y eut plus moyen de reculer.

Le roi les trouva le lendemain tous deux à Trianon. Il y fut d'abord
question de la fenêtre. Louvois disputa, Le Nôtre ne disait mot. Enfin
le roi lui ordonna d'aligner, de mesurer, et de dire après ce qu'il
aurait trouvé. Tandis qu'il y travaillait, Louvois, en furie de cette
vérification, grondait tout haut, et soutenait avec aigreur que cette
fenêtre était en tout pareille aux autres. Le roi se taisait et
attendait, mais il souffrait. Quand tout fut bien examiné, il demanda au
Notre ce qui en était\,; et Le Notre à balbutier. Le roi se mit en
colère, et lui commanda de parler net. Alors Le Nôtre avoua que le roi
avait raison, et dit ce qu'il avait trouvé de défaut. Il n'eut pas
plutôt achevé que le roi, se tournant à Louvois, lui dit qu'on ne
pouvait tenir à ses opiniâtretés, que sans la sienne à lui, on aurait
bâti de travers, et qu'il aurait fallu tout abattre aussitôt que le
bâtiment aurait été achevé. En un mot, il lui lava fortement la tête.

Louvois, outré de la sortie, et de ce que courtisans, ouvriers et valets
en avaient été témoins, arrive chez lui furieux. Il y trouva
Saint-Pouange, Villacerf, le chevalier de Nogent, les deux Tilladet,
quelques autres féaux intimes, qui furent bien alarmés de le voir en cet
état. «\,C'en est fait, leur dit-il, je suis perdu avec le roi, à la
façon dont il vient de me traiter pour une fenêtre. Je n'ai de ressource
qu'une guerre qui le détourne de ses bâtiments et qui me rende
nécessaire, et par\ldots. il l'aura.\,» En effet, peu de mois après il
tint parole, et malgré le roi et les autres puissances il la rendit
générale. Elle ruina la France au dedans, ne l'étendit point au dehors,
malgré la prospérité de ses armes, et produisit au contraire des
événements honteux.

Celui de tous qui porta le plus à plomb sur le roi fut sa dernière
campagne qui ne dura pas un mois. Il avait en Flandre deux armées
formidables, supérieures du double au moins à celle de l'ennemi, qui
n'en avait qu'une. Le prince d'Orange était campé à l'abbaye de Parc, le
roi n'en était qu'à une lieue, et M. de Luxembourg avec l'autre armée à
une demi-lieue de celle du roi, et rien entre les trois armées. Le
prince d'Orange se trouvait tellement enfermé qu'il s'estimait sans
ressource dans les retranchements, qu'il fit relever à la hâte autour de
son camp, et si perdu qu'il le manda à Vaudemont, son ami intime, à
Bruxelles, par quatre ou cinq fois, et qu'il ne voyait nulle sorte
d'espérance de pouvoir échapper, ni sauver son armée. Rien ne la
séparait de celle du roi que ces mauvais retranchements, et rien de plus
aisé ni de plus sûr que de le forcer avec l'une des deux armées, et de
poursuivre la victoire avec l'autre toute fraîche, et qui toutes deux
étaient complètes, indépendamment l'une de l'autre, en équipages de
vivres et d'artillerie à profusion.

On était aux premiers jours de juin\,; et que ne promettait pas une
telle victoire au commencement d'une campagne\,! Aussi l'étonnement
fut-il extrême et général dans toutes les trois armées, lorsqu'on y
apprit que le roi se retirait\footnote{Saint-Simon a raconté ce fait
  avec plus de détails, t. Ier, p.~86-89 de la présente édition (in-8).
  --- On fera bien de rapprocher de ces deux passages une note de M.
  Théophile Lavallée (\emph{Lettres historiques et édifiantes de
  M\textsuperscript{me} de Maintenon}, t. Ier, p.~302 et suiv.).}, et
faisait deux gros détachements de presque toute l'armée qu'il commandait
en personne\,: un pour l'Italie, l'autre pour l'Allemagne sous
Monseigneur. M. de Luxembourg, qu'il manda le matin de la veille de son
départ pour lui apprendre ces nouvelles dispositions, se jeta à genoux,
et tint les siens longtemps embrassés pour l'en détourner, et pour lui
remontrer la facilité, la certitude et la grandeur du succès, en
attaquant le prince d'Orange. Il ne réussit qu'à importuner, d'autant
plus sensiblement, qu'il n'y eut pas un mot à lui opposer. Ce fut une
consternation dans les deux armées qui ne se peut représenter. On a vu
que j'y étais. Jusqu'aux courtisans, si aises d'ordinaire de retourner
chez eux, ne purent contenir leur douleur. Elle éclata partout aussi
librement que la surprise, et à l'une et l'autre succédèrent de fâcheux
raisonnements.

Le roi partit le lendemain pour aller rejoindre M\textsuperscript{me} de
Maintenon et les dames, et retourna avec elles à Versailles, pour ne
plus revoir la frontière ni d'armées que pour le plaisir en temps de
paix.

La victoire de Neerwinden, que M. de Luxembourg remporta six semaines
après sur le prince d'Orange, que la nature, prodigieusement aidée de
l'art en une seule nuit avait furieusement retranché, renouvela d'autant
plus les douleurs et les discours, qu'il s'en fallait tout que le poste
de l'abbaye de Parc ressemblât à celui de Neerwinden\,; presque tout que
nous eussions les mêmes forces, et plus que tout que, faute de vivres et
d'équipages suffisants d'artillerie, cette victoire pût être poursuivie.

Pour achever ceci tout à la fois, on sut que le prince d'Orange, averti
du départ du roi, avait mandé à Vaudemont qu'il en avait l'avis d'une
main toujours bien avertie, et qui ne lui en avait jamais donné de faux,
mais que pour celui-là il ne pouvait y ajouter foi, ni se livrer à
l'espérance\,: et par un second courrier, que l'avis était vrai, que le
roi partait, que c'était à son esprit de vertige et d'aveuglement qu'il
devait uniquement une si inespérée délivrance. Le rare est que
Vaudemont, établi longtemps depuis en notre cour, l'a souvent conté à
ses amis, même à ses compagnies, et jusque dans le salon de Marly.

La paix qui suivit cette guerre, et après laquelle le roi et l'État aux
abois soupiraient depuis longtemps, fut honteuse. Il fallut en passer
par où M. de Savoie voulut, pour le détacher de ses alliés, et
reconnaître enfin le prince d'Orange pour roi d'Angleterre, après une si
longue suite d'efforts, de haine et de mépris personnels, et recevoir
encore Portland, son ambassadeur, comme une espèce de divinité. Notre
précipitation nous coûta Luxembourg\,; et l'ignorance militaire de nos
plénipotentiaires, qui ne fut point éclairée du cabinet, donna aux
ennemis de grands avantages pour former leur frontière. Telle fut la
paix de Ryswick, conclue en septembre 1697.

Le repos des armes ne fut guère que de trois ans\,; et on sentit
cependant toute la douleur des restitutions de pays et de places que
nous avions conquis, avec le poids de tout ce que la guerre avait coûté.
Ici se termine le second âge de ce règne.

Le troisième s'ouvrit par un comble de gloire et de prospérité inouïe.
Le temps en fut momentané. Il enivra et prépara d'étranges malheurs,
dont l'issue a été une espèce de miracle. D'autres sortes de malheurs
accompagnèrent et conduisirent le roi au tombeau, heureux s'il n'eût
survécu que de peu de mois à l'avènement de son petit-fils à la totalité
de la monarchie d'Espagne, dont il fut d'abord en possession sans coup
férir. Cette dernière époque est encore si proche de ce temps qu'il n'y
a pas lieu de s'y étendre. Mais ce peu qui a été retracé du règne du feu
roi était nécessaire pour mieux faire entendre ce qu'on va dire de sa
personne, en se souvenant toutefois de ce qui s'en trouve épars dans ces
Mémoires, et ne se dégoûtant pas s'il s'y en trouve quelques redites,
nécessaires pour mieux rassembler et former un tout.

Il faut encore le dire. L'esprit du roi était au-dessous du médiocre,
mais très capable de se former. Il aima la gloire, il voulut l'ordre et
la règle\,; il était né sage, modéré, secret, maître de ses mouvements
et de sa langue\,; le croira-t-on, il était né bon et juste, et Dieu lui
en avait donné assez pour être un bon roi, et peut-être même un assez
grand roi. Tout le mal lui vint d'ailleurs. Sa première éducation fut
tellement abandonnée, que personne n'osait approcher de son appartement.
On lui a souvent ouï parler de ces temps avec amertume, jusque-là qu'il
racontait qu'on le trouva un soir tombé dans le bassin du jardin du
Palais-Royal à Paris, où la cour demeurait alors.

Dans la suite, sa dépendance fut extrême. À peine lui apprit-on à lire
et à écrire, et il demeura tellement ignorant, que les choses les plus
connues d'histoire, d'événements, de fortunes, de conduites, de
naissance, de lois, il n'en sut jamais un mot. Il tomba, par ce défaut,
et quelquefois en public, dans les absurdités les plus grossières.

M. de La Feuillade plaignant exprès devant lui le marquis de Resnel, qui
fut tué depuis lieutenant général et mestre de camp général de la
cavalerie, de n'avoir pas été chevalier de l'ordre en 1661, le roi
passa, puis dit avec mécontentement qu'il fallait aussi se rendre
justice. Resnel était Clermont-Gallerande ou d'Amboise, et le roi, qui
depuis n'a été rien moins que délicat là-dessus, le croyait un homme de
fortune. De cette même maison était Monglat, maître de sa garde-robe,
qu'il traitait bien et qu'il fit chevalier de l'ordre en 1661, qui a
laissé de très bons Mémoires. Monglat avait épousé la fille du fils du
chancelier de Cheverny. Leur fils unique porta toute sa vie le nom de
Cheverny, dont il avait la terre. Il passa sa vie à la cour, et j'en ai
parlé quelquefois, ou dans les emplois étrangers. Ce nom de Cheverny
trompa le roi, il le crut peu de chose\,; il n'avait point de charge, et
ne put être chevalier de l'ordre. Le hasard détrompa le roi à la fin de
sa vie. Saint-Herem avait passé la sienne grand louvetier, puis
gouverneur et capitaine de Fontainebleau, il ne put être chevalier de
l'ordre. Le roi, qui le savait beau-frère de Courtin, conseiller d'État,
qu'il connaissait, le crut par là fort peu de chose. Il était Montmorin,
et le roi ne le sut que fort tard par M. de La Rochefoucauld. Encore lui
fallut-il expliquer quelles étaient ces maisons, que leur nom ne lui
apprenait pas.

Il semblerait à cela que le roi aurait aimé la grande noblesse, et ne
lui en voulait pas égaler d'autre\,; rien moins. L'éloignement qu'il
avait pris de celle des sentiments, et sa faiblesse pour ses ministres,
qui haïssaient et l'abaissaient, pour s'élever, tout ce qu'ils n'étaient
pas et ne pouvaient pas être, lui avait donné le même éloignement pour
la naissance distinguée. Il la craignait autant que l'esprit\,; et si
ces deux qualités se trouvaient unies dans un même sujet, et qu'elles
lui fussent connues, c'en était fait.

Ses ministres, ses généraux, ses maîtresses, ses courtisans
s'aperçurent, bientôt après qu'il fut le maître, de son faible plutôt
que de son goût pour la gloire. Ils le louèrent à l'envi et le gâtèrent.
Les louanges, disons mieux, la flatterie lui plaisait à tel point, que
les plus grossières étaient bien reçues, les plus basses encore mieux
savourées. Ce n'était que par là qu'on s'approchait de lui, et ceux
qu'il aima n'en furent redevables qu'a heureusement rencontrer, et à ne
se jamais lasser en ce genre. C'est ce qui donna tant d'autorité à ses
ministres, par les occasions continuelles qu'ils avaient de l'encenser,
surtout de lui attribuer toutes choses, et de les avoir apprises de lui.
La souplesse, la bassesse, l'air admirant, dépendant, rampant, plus que
tout l'air de néant sinon par lui, étaient les uniques voies de lui
plaire. Pour peu qu'on s'en écartât, on n'y revenait plus, et c'est ce
qui acheva la ruine de Louvois.

Ce poison ne fit que s'étendre. Il parvint jusqu'à un comble incroyable
dans un prince qui n'était pas dépourvu d'esprit et qui avait de
l'expérience. Lui-même, sans avoir ni voix ni musique, chantait dans ses
particuliers les endroits les plus à sa louange des prologues des
opéras. On l'y voyait baigné, et jusqu'à ses soupers publics au grand
couvert, où il y avait quelquefois des violons, il chantonnait entre ses
dents les mêmes louanges quand on jouait les airs qui étaient faits
dessus.

De là ce désir de gloire qui l'arrachait par intervalles à l'amour\,; de
là cette facilité à Louvois de l'engager en de grandes guerres, tantôt
pour culbuter Colbert, tantôt pour se maintenir ou s'accroître, et de
lui persuader en même temps qu'il était plus grand capitaine qu'aucun de
ses généraux, et pour les projets et pour les exécutions, en quoi les
généraux l'aidaient eux-mêmes pour plaire au roi. Je dis les Condé, les
Turenne, et à plus forte raison tous ceux qui leur ont succédé. Il
s'appropriait tout avec une facilité et une complaisance admirable en
lui-même, et se croyait tel qu'ils le dépeignaient en lui parlant. De là
ce goût de revues, qu'il poussa si loin, que ses ennemis l'appelaient
«\,le roi des revues,\,» ce goût de sièges pour y montrer sa bravoure à
bon marché, s'y faire retenir à force, étaler sa capacité, sa
prévoyance, sa vigilance, ses fatigues, auxquelles son corps robuste et
admirablement conformé était merveilleusement propre, sans souffrir de
la faim, de la soif, du froid, du chaud, de la pluie, ni d'aucun mauvais
temps. Il était sensible aussi à entendre admirer, le long des camps,
son grand air et sa grande mine, son adresse à cheval et tous ses
travaux. C'était de ses campagnes et de ses troupes qu'il entretenait le
plus ses maîtresses, quelquefois ses courtisans. Il parlait bien, en
bons termes, avec justesse\,; il faisait un conte mieux qu'homme du
monde, et aussi bien un récit. Ses discours les plus communs n'étaient
jamais dépourvus d'une naturelle et sensible majesté.

Son esprit, naturellement porté au petit, se plut en toutes sortes de
détails. Il entra sans cesse dans les derniers sur les troupes\,:
habillements, armements, évolutions, exercices, discipline, en un mot,
toutes sortes de bas détails. Il ne s'en occupait pas moins sur ses
bâtiments, sa maison civile, ses extraordinaires de bouche\,; il croyait
toujours apprendre quelque chose à ceux qui en ces genres-là en savaient
le plus, qui de leur part recevaient en novices des leçons qu'ils
savaient par cœur il y avait longtemps. Ces pertes de temps, qui
paraissaient au roi avec tout le mérite d'une application continuelle,
étaient le triomphe de ses ministres, qui, avec un peu d'art et
d'expérience à le tourner, faisaient venir comme de lui ce qu'ils
voulaient eux-mêmes et qui conduisaient le grand selon leurs vues, et
trop souvent selon leur intérêt, tandis qu'ils s'applaudissaient de le
voir se noyer dans ces détails. La vanité et l'orgueil, qui vont
toujours croissant, qu'on nourrissait et qu'on augmentait en lui sans
cesse, sans même qu'il s'en aperçût, et jusque dans les chaires par les
prédicateurs en sa présence, devinrent la base de l'exaltation de ses
ministres par-dessus toute autre grandeur. Il se persuadait par leur
adresse que la leur n'était que la sienne, qui, au comble en lui, ne se
pouvait plus mesurer, tandis qu'en eux elle l'augmentait d'une manière
sensible, puisqu'ils n'étaient rien par eux-mêmes, et utile en rendant
plus respectables les organes de ses commandements, qui les faisaient
mieux obéir. De là les secrétaires d'État et les ministres
successivement à quitter le manteau, puis le rabat, après l'habit noir,
ensuite l'uni, le simple, le modeste, enfin à s'habiller comme les gens
de qualité\,; de là à en prendre les manières, puis les avantages, et
par échelons admis à manger avec le roi\,; et leurs femmes, d'abord sous
des prétextes personnels, comme M\textsuperscript{me} Colbert longtemps
avant M\textsuperscript{me} de Louvois, enfin, des années après elle,
toutes à titre de droit des places de leur mari, manger et entrer dans
les carrosses, et n'être en rien différentes des femmes de la première
qualité.

De ce degré, Louvois, sous divers prétextes, ôta les honneurs civils et
militaires dans les places et dans les provinces à ceux à qui on ne les
avait jamais disputés, et {[}en vint{]} à cesser d'écrire
\emph{monseigneur} aux mêmes, comme il avait toujours été pratiqué. Le
hasard m'a conservé trois {[}lettres{]} de M. Colbert, lors contrôleur
général, ministre d'État et secrétaire d'État, à mon père à Blaye, dont
la suscription et le dedans le traitent de \emph{monseigneur}, et que
Mgr le duc de Bourgogne, à qui je les montrai, vit avec grand plaisir.
M. de Turenne, dans l'éclat où il était alors, sauva le rang de prince
de l'écriture, c'est-à-dire sa maison qui l'avait eu par le cardinal
Mazarin, et conséquemment les maisons de Lorraine et de Savoie, car les
Rohan ne l'ont jamais pu obtenir, et c'est peut-être la seule chose où
ait échoué la beauté de M\textsuperscript{me} de Soubise. Ils ont été
plus heureux depuis. M. de Turenne sauva aussi les maréchaux de France
pour les honneurs militaires\,; ainsi pour sa personne il conserva les
deux. Incontinent après, Louvois s'attribua ce qu'il venait d'ôter à
bien plus grand que lui, et le communiqua aux autres secrétaires d'État.
Il usurpa les honneurs militaires, que ni les troupes, ni qui que ce
soit, n'osa refuser à sa puissance d'élever et de perdre qui bon lui
semblait\,; et il prétendit que tout ce qui n'était point duc ni
officier de la couronne, ou ce qui n'avait point le rang de prince
étranger ni de tabouret de grâce, lui écrivît \emph{monseigneur}, et lui
leur répondre dans la souscription\,: très humble et très affectionné
serviteur, tandis que le dernier maître des requêtes, ou conseiller au
parlement, lui écrivait \emph{monsieur}, sans qu'il ait jamais prétendu
changer cet usage.

Ce fut d'abord un grand bruit\,: les gens de la première qualité, les
chevaliers de l'ordre, les gouverneurs et les lieutenants généraux des
provinces, et, à leur suite, les gens de moindre qualité, et les
lieutenants généraux des armées se trouvèrent infiniment offensés d'une
nouveauté si surprenante et si étrange. Les ministres avaient su
persuader au roi l'abaissement de tout ce qui était élevé, et que leur
refuser ce traitement, c'était mépriser son autorité et son service,
dont ils étaient les organes, parce que d'ailleurs, et par eux-mêmes,
ils n'étaient rien. Le roi, séduit par ce reflet prétendu de grandeur
sur lui-même, s'expliqua si durement à cet égard, qu'il ne fut plus
question que de ployer sous ce nouveau style\,; ou de quitter le
service, et tomber en même temps, ceux qui quittaient, et ceux qui ne
servaient pas même, dans la disgrâce marquée du roi, et sous la
persécution des ministres, dont les occasions se rencontraient à tous
moments.

Plusieurs gens distingués qui ne servaient point, et plusieurs gens de
guerre du premier mérite et des premiers grades, aimèrent mieux renoncer
à tout et perdre leur fortune, et la perdirent en effet, et la plupart
pis encore\,; et dans la suite, assez prompte, peu à peu personne ne fit
plus aucune difficulté là-dessus.

De là l'autorité personnelle et particulière des ministres montée au
comble, jusqu'en ce qui ne regardait ni les ordres ni le service du roi,
sous l'ombre que c'était la sienne\,; de là ce degré de puissance qu'ils
usurpèrent\,; de là leurs richesses immenses, et les alliances qu'ils
firent tous à leur choix.

Quelque ennemis qu'ils fussent les uns des autres, l'intérêt commun les
ralliait chaudement sur ces matières, et cette splendeur usurpée sur
tout le reste de l'État dura autant que dura le règne de Louis XIV. Il
en tirait vanité\,; il n'en était pas moins jaloux qu'eux\,; il ne
voulait de grandeur que par émanation de la sienne. Toute autre lui
était devenue odieuse. Il avait sur cela des contrariétés qui ne se
comprenaient pas, comme si les dignités, les charges, les emplois avec
leurs fonctions, leurs distinctions, leurs prérogatives n'émanaient pas
de lui comme les places de ministre et les charges de secrétaire d'État
qu'il comptait seules de lui, lesquelles pour cela il portait au faite,
et abattait tout le reste sous leurs pieds.

Une autre vanité personnelle l'entraîna encore dans cette conduite. Il
sentait bien qu'il pouvait accabler un seigneur sous le poids de sa
disgrâce, mais non pas l'anéantir, ni les siens, au lieu qu'en
précipitant un secrétaire d'État de sa place, ou un autre ministre de la
même espèce, il le replongeait lui et tous les siens dans la profondeur
du néant d'où cette place l'avait tiré, sans que les richesses qui lui
pourraient rester le pussent relever de ce non-être. C'est là ce qui le
faisait se complaire à faire régner ses ministres sur les plus élevés de
ses sujets, sur les princes de son sang en autorité comme sur les
autres, et sur tout ce qui n'avait ni rang ni office de la couronne, en
grandeur comme en autorité au-dessus d'eux. C'est aussi ce qui éloigna
toujours du ministère tout homme qui pouvait y ajouter du sien ce que le
roi ne pouvait ni détruire ni lui conserver, ce qui lui aurait rendu un
ministre de cette sorte en quelque façon redoutable et continuellement à
charge, dont l'exemple du duc de Beauvilliers fut l'exception unique
dans tout le cours de son règne, comme il a été remarqué en parlant de
ce duc, le seul homme noble qui ait été admis dans son conseil depuis la
mort du cardinal Mazarin jusqu'à la sienne, c'est-à-dire pendant
cinquante-quatre ans\,; car, outre ce qu'il y aurait à dire sur le
maréchal de Villeroy, le peu de mois qu'il y a été depuis la mort du duc
de Beauvilliers jusqu'à celle du roi ne peut pas être compté, et son
père n'a jamais entré dans le conseil d'État.

De là encore la jalousie si précautionnée des ministres, qui rendit le
roi si difficile à écouter tout autre qu'eux, tandis qu'il
s'applaudissait d'un accès facile, et qu'il croyait qu'il y allait de sa
grandeur, de la vénération et de la crainte dont il se complaisait
d'accabler les plus grands, de se laisser approcher autrement qu'en
passant. Ainsi le grand seigneur, comme le plus subalterne de tous
états, parlait librement au roi en allant ou revenant de la messe, en
passant d'un appartement à un autre, ou allant monter en carrosse\,; les
plus distingués, même quelques autres, à la porte de son cabinet, mais
sans oser l'y suivre. C'est à quoi se bornait la facilité de son accès.
Ainsi on ne pouvait s'expliquer qu'en deux mots, d'une manière fort
incommode, et toujours entendu de plusieurs qui environnaient le roi,
ou, si on était plus connu de lui, dans sa perruque, ce qui n'était
guère plus avantageux. La réponse sûre était un \emph{je verrai}, utile
à la vérité pour s'en donner le temps, mais souvent bien peu
satisfaisante, moyennant quoi tout passait nécessairement par les
ministres, sans qu'il pût y avoir jamais d'éclaircissement, ce qui les
rendait les maîtres de tout, et le roi le voulait bien, ou ne s'en
apercevait pas.

D'audiences à en espérer dans son cabinet, rien n'était plus rare, même
pour les affaires du roi dont on avait été chargé. Jamais, par exemple,
à ceux qu'on envoyait ou qui revenaient d'emplois étrangers, jamais à
pas un officier général, si on en excepte certains cas très singuliers,
et encore, mais très rarement, quelqu'un de ceux qui étaient chargés de
ces détails de troupes où le roi se plaisait tant\,; de courtes aux
généraux d'armée qui partaient, et en présence du secrétaire d'État de
la guerre, de plus courtes à leur retour\,; quelquefois ni en partant,
ni en revenant. Jamais de lettres d'eux qui allassent directement au roi
sans passer auparavant par le ministre, si on en excepte quelques
occasions infiniment rares et momentanées, et le seul M. de Turenne sur
la fin, qui, ouvertement brouillé avec Louvois, et brillant de gloire et
de la plus haute considération, adressait ses dépêches au cardinal de
Bouillon, qui les remettait directement au roi, qui n'en étaient pas
moins vues après par le ministre, avec lequel les ordres et les réponses
étaient concertés.

La vérité est pourtant, que, quelque gâté que fût le roi sur sa grandeur
et sur son autorité qui avaient étouffé toute autre considération en
lui, il y avait à gagner dans ses audiences, quand on pouvait tant faire
que de les obtenir, et qu'on savait s'y conduire avec tout le respect
qui était dû à la royauté et à l'habitude. Outre ce que j'en ai su
d'ailleurs, j'en puis parler par expérience. On a vu en leur temps ici
que j'ai obtenu, et même usurpé {[}des audiences{]}, et forcé le roi
fort en colère contre moi, et toujours sorti, lui persuadé et content de
moi, et le marquer après et à moi et à d'autres. Je puis donc aussi
parler de ces audiences qu'on en avait quelquefois, par ma propre
expérience.

Là, quelque prévenu qu'il fût, quelque mécontentement qu'il crût avoir
lieu de sentir, il écoutait avec patience, avec bonté, avec envie de
s'éclaircir et de s'instruire\,; il n'interrompait que pour y parvenir.
On y découvrait un esprit d'équité et de désir de connaître la vérité,
et cela quoique en colère quelquefois, et cela jusqu'à la fin de sa vie.
Là, tout se pouvait dire, pourvu encore une fois que ce fût avec cet air
de respect, de soumission, de dépendance, sans lequel on se serait
encore plus perdu que devant, mais avec lequel aussi, en disant vrai, on
interrompait le roi à son tour, on lui niait crûment des faits qu'il
rapportait, on élevait le ton au-dessus du sien en lui parlant, et tout
cela non seulement sans qu'il le trouvât mauvais, mais se louant après
de l'audience qu'il avait donnée, et de celui qui l'avait eue, se
défaisant des préjugés qu'il avait pris, ou des faussetés qu'on lui
avait imposées, et le marquant après par ses traitements. Aussi les
ministres avaient-ils grand soin d'inspirer au roi l'éloignement d'en
donner, à quoi ils réussirent comme dans tout le reste.

C'est ce qui rendait les charges qui approchaient de la personne du roi
si considérables, et ceux qui les possédaient si considérés, et des
ministres mêmes, par la facilité qu'ils avaient tous les jours de parler
au roi, seuls, sans l'effaroucher d'une audience qui était toujours sue,
et de l'obtenir sûrement, et sans qu'on s'en aperçût, quand ils en
avaient besoin. Surtout les grandes entrées par cette même raison
étaient le comble des grâces, encore plus que de la distinction, et
c'est ce qui, dans les grandes récompenses des maréchaux de Boufflers et
de Villars, les fit mettre de niveau à la pairie et à la survivance de
leurs gouvernements à leurs enfants tous jeunes, dans le temps que le
roi n'en donnait plus à personne.

C'est donc avec grande raison qu'on doit déplorer avec larmes l'horreur
d'une éducation uniquement dressée pour étouffer l'esprit et le cœur de
ce prince, le poison abominable de la flatterie la plus insigne qui le
déifia dans le sein même du christianisme, et la cruelle politique de
ses ministres qui l'enferma, et qui pour leur grandeur, leur puissance
et leur fortune l'enivrèrent de son autorité, de sa grandeur, de sa
gloire jusqu'à le corrompre, et à étouffer en lui, sinon toute la bonté,
l'équité, le désir de connaître la vérité que Dieu lui avait donné, au
moins l'émoussèrent presque entièrement, et empêchèrent sans cesse qu'il
fît aucun usage de ces vertus, dont son royaume et lui-même furent les
victimes.

De ces sources étrangères et pestilentielles lui vint cet orgueil
{[}tel{]} que ce n'est point trop de dire que, sans la crainte du diable
que Dieu lui laissa jusque dans ses plus grands désordres, il se serait
fait adorer et aurait trouvé des adorateurs\,; témoin entre autres ces
monuments si outrés, pour en parler même sobrement\,: sa statue de la
place des Victoires, et sa païenne dédicace où j'étais, où il prit un
plaisir si exquis\,; et de cet orgueil\footnote{Cette phrase, que les
  précédents éditeurs ont corrigée, s'entend facilement en ajoutant le
  verbe {[} \emph{vint} {]} qui se trouve plus haut\,: \emph{Et de cet
  orgueil} {[} \emph{vint} {]} \emph{tout le reste}, etc.} tout le reste
qui le perdit, dont on vient de voir tant d'effets funestes, et dont
d'autres plus funestes encore se vont retrouver.

\hypertarget{chapitre-xvii.}{%
\chapter{CHAPITRE XVII.}\label{chapitre-xvii.}}

~

{\textsc{Jalousie et ambition de Louvois font toutes les guerres et la
ruine du royaume, et {[}ainsi que{]} la haine implacable du roi pour le
prince d'Orange.}} {\textsc{- Terrible conduite de Louvois pour
embarquer la guerre générale de 1688.}} {\textsc{- Catastrophe de
Louvois par deux belles actions après beaucoup d'étranges.}} {\textsc{-
Grande action de Chamlay\,; son état\,; son caractère.}} {\textsc{- Mort
et disgrâce de Louvois, et de son médecin cinq mois après celle de
Louvois.}}

~

Ce même orgueil, que Louvois sut si bien manier, épuisa le royaume par
des guerres et par des fortifications innombrables. La guerre des
Pays-Bas, à l'occasion de la mort de Philippe IV et des droits de la
reine sa fille, forma la triple alliance. La guerre de Hollande, en
1670\footnote{Nous avons reproduit exactement le manuscrit\,; mais il y
  a ici erreur évidente. La guerre de Hollande ne commença qu'au mois
  d'avril 1672.}, effraya toute l'Europe pour toujours par le succès que
le roi y eut, et qu'il abandonna pour l'amour. Elle fit revivre le parti
du prince d'Orange, perdit le parti républicain, donna aux
Provinces-Unies le chef le plus dangereux par sa capacité, ses vues, sa
suite, ses alliances, qui, par le superbe refus qu'il fit de l'aînée et
de la moins honteuse des bâtardes du roi, le piqua au plus vif, jusqu'à
n'avoir jamais pu se l'adoucir dans la suite par la longue continuité de
ses respects, de ses désirs, de ses démarches, qui, par le désespoir de
ce mépris, devint son plus personnel et son plus redoutable ennemi, et
qui sut en tirer de si prodigieux avantages, quoique toujours malheureux
à la guerre contre lui.

Son coup d'essai fut la fameuse ligue d'Augsbourg, qu'il sut former de
la terreur de la puissance de la France, qui nourrissait chez elle un
plus cruel ennemi. C'était Louvois\footnote{Voy. le portrait de Louvois,
  par Saint-Simon, dans le \emph{Journal de Dangeau} (édit. Didot, t.
  Ier, p.~361 et suiv.). Les traits dispersés dans les Mémoires sont
  réunis dans ce remarquable passage, et fortement accuses\,: «\,C'était
  un homme altier, brutal\,; grossier dans toutes ses manières\,; comme
  sa figure le montrait bien\ldots., homme terrible et absolu, et qui
  voulait et se piquait de l'être.\,» Voy., dans les notes à la fin du
  volume, le portrait de Louvois par le maréchal de camp
  Saint-Hilaire\,; on y retrouve la confirmation de tout ce que dit
  Saint-Simon.}, l'auteur et l'âme de toutes ces guerres, parce qu'il en
avait le département, et parce que, jaloux de Colbert, il le voulait
perdre en épuisant les finances, et le mettant à bout. Colbert, trop
faible pour pouvoir détourner la guerre, ne voulut pas succomber\,;
ainsi à bout d'une administration sage, mais forcée, et de toutes les
ressources qu'il avait pu imaginer, il renversa enfin ses anciennes et
vénérables barrières, dont la ruine devint nécessairement celle de
l'État, et l'a peu à peu réduit aux malheurs qui ont tant de fois épuisé
les particuliers, après avoir ruiné le royaume. C'est ce qu'opérèrent
ces places et ces troupes sans nombre qui accablèrent d'abord les
ennemis, mais qui leur apprirent enfin à avoir des armées aussi
nombreuses que les nôtres, et que l'Allemagne et le nord étaient
inépuisables d'hommes, tandis que la France s'en dépeupla.

Ce fut la même jalousie qui écrasa la marine dans un royaume flanqué des
deux mers, parce qu'elle était florissante sous Colbert et son fils, et
qui empêcha l'exécution du sage projet d'un port à la Hogue
\textsuperscript{{[}Voy. notes à la fin du volume.{]}}{[}Voy. notes à la
fin du volume.{]}, pour s'assurer d'une retraite dans la Manche, faute
énorme qui bien des années après coûta à la France, au même lieu de la
Hogue, la perte d'une nombreuse flotte qu'elle avait enfin remise en mer
avec tant de dépense, qui anéantit la marine, et ne lui laissa pas le
temps, après avoir été si chèrement relevée, de rétablir son commerce
éteint dès la première fois par Louvois, qui est la source des richesses
et pour ainsi dire l'âme d'un État dans une si heureuse position entre
les deux mers.

Cette même jalousie de Louvois contre Colbert dégoûta le roi des
négociations dont le cardinal de Richelieu estimait l'entretien
continuel si nécessaire, aussi bien que la marine et le commerce, parce
que tous les trois étaient entre les mains de Colbert et de Croissy, son
frère, à qui Louvois ne destinait pas la dépouille du sage et de
l'habile Pomponne, quand il se réunit à Colbert pour le faire chasser.

Ce fut donc dans cette triste situation intérieure que la fenêtre de
Trianon fit la guerre de 1688\,; que Louvois détourna d'abord le roi de
rien croire des avis de d'Avaux, ambassadeur en Hollande, et de bien
d'autres qui mandaient de la Haye positivement, et de bien d'autres
endroits, le projet et les préparatifs de la révolution d'Angleterre, et
nos armes de dessus les Provinces-Unies par la Flandre qui en auraient
arrêté l'exécution pour les porter sur le Rhin, et par là embarquer
sûrement la guerre. Louvois frappa ainsi deux coups à la fois pour ses
vues personnelles\,: il s'assura par cette expresse négligence d'une
longue et forte guerre avec la Hollande et l'Angleterre, où il était
bien assuré que la haine invétérée du roi pour la personne du prince
d'Orange ne souffrirait jamais sa grandeur et son établissement sur les
ruines de la religion catholique et de Jacques II son ami personnel,
tant qu'il pourrait espérer de renverser l'un et de rétablir l'autre\,;
et en même temps il profitait de la mort de l'électeur de Cologne, qui
ouvrait la dispute de l'élection en sa place, entre le prince Clément de
Bavière son neveu et le cardinal de Furstemberg son coadjuteur, portés
ouvertement chacun par l'empereur et par la France, et sous ce prétexte
persuade au roi d'attaquer l'empereur et l'Empire par le siège de
Philippsbourg, etc.\,; et pour rendre cette guerre plus animée et plus
durable, fait brûler Worms, Spire, et tout le Palatinat jusqu'aux portes
de Mayence dont il fait emparer les troupes du roi. Après ce subit
début, et certain par là de la plus vive guerre avec l'empereur,
l'empire, l'Angleterre et la Hollande, l'intérêt particulier de la faire
durer lui fit changer le plan de son théâtre.

Pousser sa pointe en Allemagne dénuée de places et pleine de princes
dont les médiocres États dépourvus n'auraient pu la soutenir, le
menaçait de ce côté d'une paix trop prompte, malgré la fureur qu'il y
avait allumée par ses cruels incendies. La Flandre, au contraire, était
hérissée de places, où, après une déclaration de guerre, il n'était pas
aisé de pénétrer. Ce fut donc de la Flandre dont il persuada au roi de
faire le vrai théâtre de la guerre, et rien en Allemagne qu'une guerre
d'observation et de subsistance. Il le flatta de conquérir des places en
personne, et de châtier une autre fois les Hollandais qui venaient de
mettre le prince d'Orange sur le trône du roi Jacques, réfugié en France
avec sa famille, et engagea ainsi une guerre à ne point finir\,; tandis
qu'elle eût été courte au moins avec l'empereur et l'empire, en portant
brusquement la guerre dans le milieu de l'Allemagne, et demeurant sur la
défensive en Flandre, où les Hollandais, contents de leurs succès
d'Angleterre, n'auraient pas songé à faire des progrès parmi tant de
places.

Mais ce ne fut pas tout. Louvois voulut être exact à sa parole\,: la
guerre qu'il venait d'allumer ne lui suffit pas\,: il la veut contre
toute l'Europe. L'Espagne inséparable de l'empereur, et même des
Hollandais, à cause de la Flandre espagnole, s'était déclarée\,: ce fut
un prétexte pour des projets sur la Lombardie, et ces projets en
servirent d'un autre pour faire déclarer le duc de Savoie. Ce prince ne
désirait que la neutralité, et comme le plus faible, de laisser passer à
petites troupes limitées, avec ordre et mesure, ce qu'on aurait voulu
par son pays en payant. Cela était bien difficile à refuser\,; aussi
Catinat, déjà sur la frontière avec les troupes destinées à ce passage,
eut-il ordre d'entrer en négociation. Mais, à mesure qu'elle avançait,
Louvois demandait davantage et envoyait d'un courrier à l'autre des
ordres si contradictoires que M. de Savoie ni Catinat même n'y
comprenaient rien. M. de Savoie prit le parti d'écrire au roi pour lui
demander ses volontés à lui-même et s'y conformer.

Ce n'était pas le compte de Louvois qui voulait forcer ce prince à la
guerre. Il osa supprimer la lettre au roi, et faire à son insu des
demandes si exorbitantes, que les accorder et livrer tous ses États à la
discrétion de la France était la même chose. Le duc de Savoie se récria,
et offensé déjà du mépris de ne recevoir point de réponse du roi, à lui
directe, il se plaignit fort haut. Louvois en prit occasion de le
traiter avec insolence, de le forcer par mille affronts à plus que de
simples plaintes, et là-dessus fit agir Catinat hostilement, qui ne
pouvait comprendre le procédé du ministre, qui, sans guerre avec la
Savoie, obtenait au delà de ce qu'il se pouvait proposer.

Pendant cette étrange manière de négocier, l'empereur, le prince
d'Orange et les Hollandais qui regardaient avec raison la jonction du
duc de Savoie avec eux comme une chose capitale, surent en profiter. Ce
prince se ligua donc avec eux par force et de dépit, et devint par sa
situation l'ennemi de la France le plus coûteux et le plus redoutable,
et c'est ce que Louvois voulait, et qu'il sut opérer.

Tel fut l'aveuglement du roi, telle fut l'adresse, la hardiesse, la
formidable autorité d'un ministre le plus éminent pour les projets et
pour les exécutions, mais le plus funeste pour diriger en
premier\footnote{L'historien Vittorio Siri a dit dans le même sens que
  «\,Louvois \emph{était le plus grand et le plus brutal des hommes.
  }\,» Voici le texte\,: dans le portrait de Louvois, que contiennent
  les notes de Saint-Simon sur le \emph{Journal de Dangeau} (t. Ier, p
  361 et suiv., édit. Didot), on lit\,: «\,M. de Louvois \emph{n'était
  bon qu'à être premier ministre en plein}, et il est fort douteux que
  son esprit tout tourné aux détails et aux entreprises, eût eu ce vaste
  général et cette combinaison immense qui est si nécessaire à un
  premier ministre pour tout embrasser.\,» Il y a une contradiction
  évidente entre les deux parties de cette phrase. Il faudrait lire très
  probablement\,: \emph{M. de Louvois n'était pas bon à être premier
  ministre en plein}. C'est la vraie pensée de Saint-Simon, comme le
  prouvent ses Mémoires et la suite même de la phrase dans la note sur
  Dangeau. Saint-Simon dit un peu plus bas dans le passage cite
  (\emph{Journal de Dangeau}, t. Ier, p.~362)\,: «\,A quoi il aurait été
  le plus excellent, \emph{c'eût été d'être sous un premier ministre},
  ou sous un roi capable de s'en bien servir.\,»}\,; qui, sans être
premier ministre abattit tous les autres, sut mener le roi où et comme
il voulut, et devint en effet le maître. Il eut la joie de survivre à
Colbert et à Seignelay, ses ennemis et longtemps ses rivaux. Elle fut de
courte durée.

L'épisode de la disgrâce et de la fin d'un si célèbre ministre est trop
curieuse pour devoir être oubliée, et ne peut être mieux placée qu'ici.
Quoique je ne fisse que poindre lorsqu'elle arriva, et poindre encore
dans le domestique, j'en ai été si bien informé depuis que je ne
craindrai pas de raconter ici ce que j'en ai appris des sources, et dans
la plus exacte vérité, parce qu'elles n'y étaient en rien intéressées.

La fenêtre de Trianon a montré un échantillon de l'humeur de Louvois\,;
à cette humeur qu'il ne pouvait contraindre se joignait un ardent désir
de la grandeur et de la prospérité du roi et de sa gloire, qui était le
fondement et la plus assurée protection de sa propre fortune, et de son
énorme autorité. Il avait gagné la confiance du roi à tel point qu'il
eut la confidence de l'étrange résolution d'épouser
M\textsuperscript{me} de Maintenon, et d'être l'un des deux témoins de
la célébration de cet affreux mariage. Il eut aussi le courage de s'en
montrer digne en représentant au roi quelle serait l'ignominie de le
déclarer jamais, et de tirer de lui sa parole royale qu'il ne le
déclarerait en aucun temps de sa vie, et de faire donner en sa présence
la même parole à Harlay, archevêque de Paris, qui, pour suppléer aux
bans et aux formes ordinaires, devait aussi comme diocésain être présent
à la célébration.

Plusieurs années après, Louvois qui était toujours bien informé de
l'intérieur le plus intime, et qui n'épargnait rien pour l'être
fidèlement et promptement, sut les manèges de M\textsuperscript{me} de
Maintenon pour se faire déclarer\,; que le roi avait eu la faiblesse de
le lui promettre, et que la chose allait éclater. Il mande à Versailles
l'archevêque de Paris, et, au sortir de dîner, prend des papiers, et
s'en va chez le roi, et, comme il faisait toujours, entre droit dans les
cabinets. Le roi, qui allait se promener, sortait de sa chaise percée,
et raccommodait encore ses chausses. Voyant Louvois à heure qu'il ne
l'attendait pas, il lui demande ce qui l'amène. «\,Quelque chose de
pressé et d'important, » lui répond Louvois d'un air triste qui étonna
le roi, et qui l'engagea à commander à ce qui était toujours là de
valets intérieurs de sortir. Ils sortirent en effet\,; mais ils
laissèrent les portes ouvertes, de manière qu'ils entendirent tout, et
virent aussi tout par les glaces\,: c'était là le grand danger des
cabinets.

Eux sortis, Louvois ne feignit point de dire au roi ce qui l'amenait. Ce
monarque était souvent faux\,; mais il n'était pas au-dessus du
mensonge. Surpris d'être découvert, il s'entortilla de faibles et
transparents détours, et, pressé par son ministre, se mit à marcher pour
gagner l'autre cabinet, où étaient les valets, et se délivrer de la
sorte\,; mais Louvois, qui l'aperçait, se jette à ses genoux et
l'arrête, tire de son côté une petite épée de rien qu'il portait, en
présente la garde au roi, et le prie de le tuer sur-le-champ s'il veut
persister à déclarer son mariage, lui manquer de parole ou plutôt à
soi-même, et se couvrir aux yeux de toute l'Europe d'une infamie qu'il
ne veut pas voir. Le roi trépigne, pétille, dit à Louvois de le laisser.
Louvois le serre de plus en plus par les jambes, de peur qu'il ne lui
échappe\,; lui représente l'horrible contraste de sa couronne, et de la
gloire personnelle qu'il y a jointe, avec la honte de ce qu'il veut
faire, dont il mourra après de regret et de confusion, en un mot fait
tant qu'il tire une seconde fois parole du roi qu'il ne déclarera jamais
ce mariage.

L'archevêque de Paris arrive le soir\,; Louvois lui conte ce qu'il a
fait. Le prélat courtisan n'en aurait pas été capable, et en effet ce
fut une action qui se peut dire sublime, de quelque côté qu'elle puisse
être considérée, surtout dans un ministre tout-puissant, qui tenait si
fort à son autorité et à sa place, et, par cela même qu'il faisait,
sentait tout le poids de celle de M\textsuperscript{me} de Maintenon,
conséquemment tout celui de sa haine, s'il était découvert, comme il
avait trop de connaissances pour se flatter que son action lui demeurât
cachée. L'archevêque, qui n'eut qu'à confirmer le roi dans sa parole
commune à Louvois et à lui, et qui venait d'être réitérée à ce ministre,
n'osa lui refuser une démarche si honorable et sans danger. Il parla
donc le lendemain matin au roi, et il en tira aisément le renouvellement
de cette parole

Celle du roi à M\textsuperscript{me} de Maintenon n'avait point mis de
délai\,; elle s'attendait à tous moments d'être déclarée. Au bout de
quelques jours, inquiète de ce que le roi ne lui parlait de rien
là-dessus, elle se hasarda de lui en toucher quelque chose. L'embarras
où elle mit le roi la troubla fort. Elle voulut faire effort\,; le roi
coupa court sur les réflexions qu'il avait faites, les assaisonna comme
il put, mais il finit par la prier de ne plus penser à être déclarée et
à ne lui en parler jamais. Après le premier bouleversement que lui causa
la perte d'une telle espérance, et si près d'être mise à effet, son
premier soin fut de rechercher à qui elle en était redevable. Elle
n'était pas de son côté moins bien avertie que Louvois. Elle apprit
enfin ce qui s'était passé, et quel jour, entre le roi et son ministre.

On ne sera pas surpris après cela si elle jura sa perte et si elle ne
cessa de la préparer, jusqu'à ce qu'elle en vint à bout\,; mais le temps
n'y était pas propre. Il fallait laisser vieillir l'affaire avec un roi
soupçonneux, et se donner le loisir des conjonctures pour miner peu à
peu son ennemi, qui avait toute la confiance de son maître, et que la
guerre lui rendait si nécessaire.

Le personnage qu'avait fait l'archevêque de Paris ne lui échappa pas non
plus, quelque léger qu'il eût été, et même après coup\,; et c'est, pour
le dire en passant, ce qui creusa peu à peu la disgrâce qui s'augmenta
toujours, dont les dégoûts continuels, qui succédèrent à une faveur si
déclarée et si longue, abrégèrent peut-être ses jours, qui néanmoins
surpassèrent de trois ans ceux de Louvois.

À l'égard de ce ministre, dont la sultane manquée avait plus de hâte de
se délivrer, elle ne manqua aucune occasion d'y préparer les voies.
Celle de ces incendies du Palatinat lui fut d'un merveilleux usage. Elle
ne manqua pas d'en peindre au roi toute la cruauté\,; elle n'oublia pas
de lui en faire naître les plus grands scrupules, car le roi en était
lors plus susceptible qu'il ne l'a été depuis\,; Elle s'aida aussi de la
haine qui en retombait à plomb sur lui, non sur son ministre, et des
dangereux effets qu'elle pouvait produire. Enfin elle vint à bout
d'aliéner fort le roi et de le mettre de mauvaise humeur contre Louvois.

Celui-ci, non content des terribles exécutions du Palatinat, voulut
encore brûler Trèves. Il le proposa au roi comme plus nécessaire encore
que ce qui avait été fait à Worms et à Spire, dont les ennemis auraient
fait leurs places d'armes, et qui en feraient une à Trêves, dans une
position à notre égard bien plus dangereuse. La dispute s'échauffa sans
que le roi pût ou voulût être persuadé. On peut juger que
M\textsuperscript{me} de Maintenon après n'adoucit pas les choses.

À quelques jours de là, Louvois, qui avait le défaut de l'opiniâtreté,
et en qui l'expérience avait ajouté de ne douter pas d'emporter toujours
ce qu'il voulait, vint à son ordinaire travailler avec le roi chez
M\textsuperscript{me} de Maintenon. À la fin du travail, il lui dit
qu'il avait bien senti que le scrupule était la seule raison qui l'eut
retenu de consentir à une chose aussi nécessaire à son service que
l'était le brûlement de Trêves\,; qu'il croyait lui en rendre un
essentiel de l'en délivrer en s'en chargeant lui-même\,; et que, pour
cela, sans lui en avoir voulu reparler, il avait dépêché un courrier
avec l'ordre de brûler Trêves à son arrivée.

Le roi fut à l'instant, et contre son naturel, si transporté de colère,
qu'il se jeta sur les pincettes de la cheminée, et en allait charger
Louvois sans M\textsuperscript{me} de Maintenon, qui se jeta aussitôt
entre-deux, en s'écriant\,: «\,Ah\,! sire, qu'allez-vous faire\,?» et
lui ôta les pincettes des mains. Louvois cependant gagnait la porte. Le
roi cria après lui pour le rappeler, et lui dit, les yeux étincelants\,:
«\,Dépêchez un courrier tout à cette heure avec un contre-ordre, et
qu'il arrive à temps, et sachez que votre tête en répond, si on brûle
une seule maison.\,» Louvois, plus mort que vif, s'en alla sur-le-champ.

Ce n'était pas dans l'impatience de dépêcher le contre-ordre\,; il
s'était bien gardé de laisser partir le premier courrier. Il lui avait
donné ses dépêches portant l'ordre de l'incendie\,; mais il lui avait
ordonné de l'attendre tout botté au retour de son travail. Il n'avait
osé hasarder cet ordre après la répugnance et le refus du roi d'y
consentir, et il crut par cette ruse que le roi pourrait être fâché,
mais que ce serait tout. Si la chose se fût passée ainsi par ce piège,
il faisait partir le courrier en revenant chez lui. Il fut assez sage
pour ne se pas commettre à le dépêcher auparavant, et bien lui en prit.
Il n'eut que la peine de reprendre ses dépêches et de faire débotter le
courrier. Il passa toujours auprès du roi pour parti, et le second pour
être arrivé assez à temps pour empêcher l'exécution.

Après une aussi étrange aventure, et aussi nouvelle au roi,
M\textsuperscript{me} de Maintenon eut beau jeu contre le ministre. Une
seconde action, louable encore, acheva sa perte. Il fit, dans l'hiver de
1690 à 1691, le projet de prendre Mons à l'entrée du printemps, et même
auparavant. Comme tout ne se mesure que par comparaison, les finances,
abondantes alors eu égard à ce qu'elles ont été depuis, mais fort
courtes par l'habitude précédente d'y nager, engagèrent Louvois de
proposer au roi de faire le voyage de Mons sans y mener les dames.
Chamlay, qui était de tous les secrets militaires, même avec le roi,
avertit Louvois de prendre garde à une proposition qui offenserait
M\textsuperscript{me} de Maintenon, qui déjà ne l'aimait pas, et qui
avait assez de crédit pour le perdre. Louvois trouva tant de dépense et
tant d'embarras au voyage des dames, qu'il préféra le bien de l'État et
la gloire du roi à son propre danger, et le siège se fit par le roi, qui
prit la place, et les dames demeurèrent à Versailles, où le roi les
revint trouver aussitôt qu'il eut pris Mons. Mais comme c'est la
dernière goutte d'eau qui fait répandre le verre, un rien arrivé à ce
siège consomma la perte de Louvois.

Le roi, qui se piquait de savoir mieux que personne jusqu'aux moindres
choses militaires, se promenant autour de son camp, trouva une garde
ordinaire de cavalerie mal placée, et lui-même la replaça autrement. Se
promenant encore le même jour l'après-dînée, le hasard fit qu'il repassa
devant cette même garde, qu'il trouva placée ailleurs. Il en fut surpris
et choqué. Il demanda au capitaine qui l'avait mis où il le voyait, qui
répondit que c'était Louvois qui avait passé par là. «\,Mais, reprit le
roi, ne lui avez-vous pas dit que c'était moi qui vous avais placé\,?
--- Oui, sire,\,» répondit le capitaine. Le roi piqué se tourne vers sa
suite, et dit\,: «\,N'est-ce pas là le métier de Louvois\,? Il se croit
un grand homme de guerre et savoir tout\,;» et tout de suite replaça le
capitaine avec sa garde où il l'avait mis le matin. C'était en effet
sottise et insolence de Louvois, et le roi avait dit vrai sur son
compte. Mais il en fut si blessé qu'il ne put le lui pardonner, et
qu'après sa mort, ayant rappelé Pomponne dans son conseil d'État, il lui
conta cette aventure, piqué encore de la présomption de Louvois, et je
la tiens de l'abbé de Pomponne.

De retour de Mons, l'éloignement du roi pour lui ne fit qu'augmenter, et
à tel point que ce ministre si présomptueux, et qui au milieu de la plus
grande guerre se comptait si indispensablement nécessaire, commença à
tout appréhender. La maréchale de Rochefort, qui était demeurée son amie
intime, étant allée avec M\textsuperscript{me} de Blansac, sa fille,
dîner avec lui à Meudon, qui me l'ont conté toutes les deux, il les mena
à la promenade. Ils n'étaient qu'eux trois dans une petite calèche
légère qu'il menait. Elles l'entendirent se parler à lui-même, rêvant
profondément, et se dire à diverses reprises\,: «\,Le ferait-il\,? Le
lui fera-t-on faire\,? non\,; mais cependant\ldots. non il n'oserait. »
Pendant ce monologue il allait toujours, et la mère et la fille se
taisaient, et se poussaient quand tout à coup la maréchale vit les
chevaux sur le dernier rebord d'une pièce d'eau, et n'eut que le temps
de se jeter en avant sur les mains de Louvois pour arrêter les rênes,
criant qu'il les menait noyer. À ce cri et ce mouvement, Louvois se
réveilla comme d'un profond sommeil, recula quelques pas, et tourna,
disant qu'en effet il rêvait et ne pensait pas à la voiture.

Dans cette perplexité, il se mit à prendre des eaux les matins à
Trianon. Le 16 juillet j'étais à Versailles pour une affaire assez
sauvage, dont le roi avait voulu donner tout l'avantage à mon père, qui
était à Blaye avec ma mère, contre Sourdis, qui commandait en chef en
Guyenne, et que Louvois avait inutilement soutenu. Ce nonobstant, je fus
conseillé de l'aller remercier, et j'en reçus autant de compliments et
de politesses que s'il avait bien servi mon père. Ainsi va la cour. Je
ne lui avais jamais parlé. Sortant le même jour du dîner du roi, je le
rencontrai au fond d'une très petite pièce qui est entre la grande salle
des gardes et ce grand salon qui donne sur la petite cour des princes,
M. de Marsan lui parlait, et il allait travailler chez
M\textsuperscript{me} de Maintenon avec le roi, qui devait se promener
après dans les jardins de Versailles à pied, où les gens de la cour
avaient la liberté de le suivre. Sur les quatre heures après midi du
même jour, j'allai chez M\textsuperscript{me} de Châteauneuf, où
j'appris qu'il s'était trouvé un peu mal chez M\textsuperscript{me} de
Maintenon, que le roi l'avait forcé de s'en aller, qu'il était retourné
à pied chez lui, où le mal avait subitement augmenté\,; qu'on s'était
hâté de lui donner un lavement qu'il avait rendu aussitôt, et qu'il
était mort en le rendant, et demandant son fils Barbezieux, qu'il n'eut
pas le temps de voir, quoiqu'il accourût de sa chambre.

On peut juger de la surprise de toute la cour. Quoique je n'eusse guère
que quinze ans, je voulus voir la contenance du roi à un événement de
cette qualité. J'allai l'attendre, et le suivis toute sa promenade. Il
me parut avec sa majesté accoutumée, mais avec je ne sais quoi de leste
et de délivré, qui me surprit assez pour en parler après, d'autant plus
que j'ignorais alors, et longtemps depuis, les choses que je viens
d'écrire. Je remarquai encore qu'au lieu d'aller voir ses fontaines et
de diversifier sa promenade, comme il faisait toujours, dans ces
jardins, il ne fit jamais qu'aller et venir le long de la balustrade de
l'orangerie, et d'où il voyait, en revenant vers le château, le logement
de la surintendance où Louvois venait de mourir, qui terminait
l'ancienne aile du château sur le flanc de l'orangerie, et vers lequel
il regarda sans cesse toutes les fois qu'il revenait vers le château.

Jamais le nom de Louvois ne fut prononce, ni pas un mot de cette mort si
surprenante et si soudaine, qu'à l'arrivée d'un officier que le roi
d'Angleterre envoya de Saint-Germain, qui vint trouver le roi sur cette
terrasse, et qui lui fit de sa part un compliment sur la perte qu'il
venait de faire. «\,Monsieur, lui répondit le roi, d'un air et d'un ton
plus que dégagés, faites mes compliments et mes remerciements au roi et
à la reine d'Angleterre, et dites-leur de ma part que mes affaires et
les leurs n'en iront pas moins bien.\,» L'officier fit une révérence, et
se retira, l'étonnement peint sur le visage et dans tout son maintien.
J'observai curieusement tout cela, et que les principaux de ce qui était
à sa promenade s'interrogeaient des yeux sans proférer une parole.

Barbezieux avait eu la survivance de secrétaire d'État dès 1685, qu'il
n'avait pas encore dix-huit ans, lorsque son père la fit ôter à
Courtenvaux son aîné, qu'il en jugea incapable. Ainsi Barbezieux, à la
mort de Louvois, l'avait faite sous lui en apprenti commis près de six
ans, et en avait vingt-quatre à sa mort, et cette mort arriva bien juste
pour sauver un grand éclat. Louvois était, quand il mourut, tellement
perdu qu'il devait être arrêté le lendemain et conduit à la Bastille.
Quelles en eussent été les suites\,? C'est ce que sa mort a scellé dans
les ténèbres, mais le fait de cette résolution prise et arrêtée par le
roi est certain, je l'ai su depuis par des gens bien informés\,; mais ce
qui demeure sans réplique, c'est que le roi même l'a dit à Chamillart,
lequel me l'a conté. Or voilà ce qui explique, je pense, ce désinvolte
du roi le jour de la mort de ce ministre, qui se trouvait soulagé de
l'exécution résolue pour le lendemain, et de toutes ses importunes
suites.

Le roi, en rentrant de la promenade chez lui, envoya chercher Chamlay,
et lui voulut donner la charge de secrétaire d'État de Louvois, à
laquelle est attaché le département de la guerre. Chamlay remercia, et
refusa avec persévérance. Il dit au roi qu'il avait trop d'obligation à
Louvois, à son amitié, à sa confiance, pour se revêtir de ses dépouilles
au préjudice de son fils, qui en avait la survivance. Il parla de toute
sa force en faveur de Barbezieux, s'offrit de travailler sous lui à tout
ce à quoi on voudrait l'employer, et à lui communiquer tout ce que
l'expérience lui aurait appris, et conclut par déclarer que, si
Barbezieux avait le malheur de n'être pas conservé dans sa charge, il
aimait mieux la voir en quelques mains que ce fût qu'entre les siennes,
et qu'il n'accepterait jamais celle de Louvois et de son fils.

Chamlay était un fort gros homme, blond et court, l'air grossier et
paysan, même rustre, et l'était de naissance, avec de l'esprit, de la
politesse, un grand et respectueux savoir-vivre avec tout le monde, bon,
doux, affable, obligeant, désintéressé, avec un grand sens et un talent
unique à connaître les pays, et n'oublier jamais la position des
moindres lieux, ni le cours et la nature du plus petit ruisseau. Il
avait longtemps servi de maréchal des logis des armées, où il fut
toujours estimé des généraux et fort aimé de tout le monde. Un grand
éloge pour lui est que M. de Turenne ne put et ne voulut jamais s'en
passer jusqu'à sa mort, et que, malgré tout l'attachement qu'il conserva
pour sa mémoire, M. de Louvois le mit dans toute sa confiance. M. de
Turenne, qui l'avait fort vanté au roi, l'en avait fait connaître. Il
était déjà entré dans les secrets militaires\,; M. de Louvois ne lui
cacha rien, et y trouva un grand soulagement pour les dispositions et
les marches des troupes qu'il destinait secrètement aux projets qu'il
voulait exécuter. Cette capacité, jointe à sa probité et à la facilité
de son travail, de ses expédients, de ses ressources, le mirent de tout
avec le roi, qui l'employa même en des négociations secrètes et en des
voyages inconnus. Il lui fit du bien et lui donna la grand'croix de
Saint-Louis. Sa modestie ne se démentit jamais, jusque-là qu'il fut
surpris et honteux de l'applaudissement que reçut la belle action qu'il
venait de faire, que le roi ne cacha pas, et que Barbezieux, à qui elle
valut sa charge, prit plaisir de publier.

On sera moins surpris dans la suite, quand le roi et
M\textsuperscript{me} de Maintenon seront plus développés, de leur voir
confier à un homme de vingt-quatre ans une charge si importante, au
milieu d'une guerre générale avec toute l'Europe\,; et au fils de ce
ministre qu'ils allaient envoyer à la Bastille lorsque sa mort les
prévint. Je joins ici le roi et M\textsuperscript{me} de Maintenon
ensemble, parce que ce fut elle qui perdit le père, elle qui fit donner
la charge au fils. Le roi, à son ordinaire, passa chez elle après la
conversation de Chamlay, et ce fut ce soir-là même que la résolution fut
prise en faveur de Barbezieux.

La soudaineté du mal et de la mort de Louvois fit tenir bien des
discours, bien plus encore quand on sut par l'ouverture de son corps
qu'il avait été empoisonné\footnote{Voy. notes à la fin du volume.}. Il
était grand buveur d'eau, et en avait toujours un pot sur la cheminée de
son cabinet, à même duquel il buvait. On sut qu'il en avait bu ainsi en
sortant pour aller travailler avec le roi, et qu'entre sa sortie de
dîner avec bien du monde, et son entrée dans son cabinet pour prendre
les papiers qu'il voulait porter à son travail avec le roi, un frotteur
du logis était entré dans ce cabinet, et y était resté quelques moments
seul. Il fut arrêté et mis en prison. Mais à peine y eut-il demeuré
quatre jours, et la procédure commencée, qu'il fut élargi par ordre du
roi, ce qui avait déjà été fait jeté au feu, et défense de faire aucune
recherche. Il devint même dangereux de parler là-dessus, et la famille
de Louvois étouffa tous ces bruits, d'une manière à ne laisser aucun
doute que l'ordre très précis n'en eût été donné.

Ce fut avec le même soin que l'histoire du médecin, qui éclata peu de
mois après, fut aussi étouffée, mais dont le premier cri ne se put
effacer. Le hasard me l'a très sincèrement apprise\,; elle est trop
singulière pour s'en tenir à ce mot, et pour ne pas finir par elle tout
le curieux et l'intéressant qui vient d'être raconté sur un ministre
aussi principal que l'a été M. de Louvois.

Mon père avait depuis plusieurs années un écuyer qui était un
gentilhomme de Périgord, de bon lieu, de bonne mine, fort apparenté et
fort homme d'honneur, qui s'appelait Clérand. Il crut faire quelque
fortune chez M. de Louvois\,; il en parla à mon père qui lui voulait du
bien, et qui trouva bon qu'il le quittât pour être écuyer de
M\textsuperscript{me} de Louvois, deux ou trois ans avant la mort de ce
ministre. Clérand conserva toujours son premier attachement, et nous
notre amitié pour lui, et il venait au logis le plus souvent qu'il
pouvait. Il m'a conté, étant toujours à M\textsuperscript{me} de Louvois
depuis la mort de son mari, que Séron, médecin domestique de ce
ministre, et qui l'était demeuré de M. de Barbezieux, logé dans sa même
chambre au château de Versailles, dans la surintendance que Barbezieux
avait conservée quoiqu'il n'eut pas succédé aux bâtiments, s'était
barricadé dans cette chambre, seul, quatre ou cinq mois après la mort de
Louvois\,; qu'aux cris qu'il y fit on était accouru à sa porte, qu'il ne
voulut jamais ouvrir\,; que ces cris durèrent presque toute la journée,
sans qu'il voulût ouïr parler d'aucun secours temporel, ni spirituel, ni
qu'on pût venir à bout d'entrer dans sa chambre\,; que sur la fin on
l'entendit s'écrier qu'il n'avait que ce qu'il méritait, que ce qu'il
avait fait à son maître\,; qu'il était un misérable indigne de tout
secours\,; et qu'il mourut de la sorte en désespéré au bout de huit ou
dix heures, sans avoir jamais parlé de personne, ni prononcé un seul
nom.

À cet événement les discours se réveillèrent à l'oreille\,; il n'était
pas sûr d'en parler. Qui a fait faire le coup\,? c'est ce qui est
demeuré dans les plus épaisses ténèbres. Les amis de Louvois ont cru
l'honorer en soupçonnant des puissances étrangères\,; mais elles
auraient attendu bien tard à s'en défaire, si quelqu'une avait conçu ce
détestable dessein. Ce qui est certain, c'est que le roi en était
entièrement incapable, et qu'il n'est entré dans l'esprit de qui que ce
soit de l'en soupçonner. Revenons maintenant à lui.

\hypertarget{chapitre-xviii.}{%
\chapter{CHAPITRE XVIII.}\label{chapitre-xviii.}}

~

{\textsc{Fautes de la guerre de 1688 et du camp de Compiègne.}}
{\textsc{- Gens d'esprit et de mérite pesants au roi, cause de ses
mauvais choix.}} {\textsc{- Fautes insignes de la guerre de la
succession d'Espagne.}} {\textsc{- Extrémité de la France, qui s'en tire
par la merveille de la paix d'Angleterre, qui fait celle d'Utrecht.}}
{\textsc{- Bonheur du roi en tout genre.}} {\textsc{- Autorité du roi
sans bornes.}} {\textsc{- Sa science de régner.}} {\textsc{- Sa
politique sur le service, où il asservit tout et rend tout peuple.}}
{\textsc{- Louvois éteint les capitaines, et en tarit le germe pour
toujours par l'invention de l'ordre du tableau.}} {\textsc{- Pernicieuse
adresse de Louvois et de son ordre du tableau.}} {\textsc{- Promotions
funestement introduites.}} {\textsc{- Invention des inspecteurs.}}
{\textsc{- Invention du grade de brigadier.}}

~

La paix de Ryswick semblait enfin devoir laisser respirer la France\,;
si chèrement achetée, si nécessairement désirée après de si grands et de
si longs efforts. Le roi avait soixante ans, et il avait, à son avis,
acquis toute sorte de gloire. Ses grands ministres étaient morts et ils
n'avaient point laissé d'élèves. Les grands capitaines non seulement
l'étaient aussi, mais ceux qu'ils avaient formé avaient passé de même,
ou n'étaient plus en âge ni en santé d'être comptés pour une nouvelle
guerre\,; et Louvois, qui avait gémi avec rage sous le poids de ces
anciens chefs, avait mis bon ordre à ce qu'il ne s'en formât plus à
l'avenir dont le mérite pût lui porter ombrage. Il n'en laissa s'élever
que de tels qu'ils eussent toujours besoin de lui pour se soutenir. Il
n'en put recueillir le fruit\,; mais l'État en porta toute la peine, et
de main en main la porte encore aujourd'hui.

À peine était-on en paix, sans avoir eu encore le temps de la goûter,
que l'orgueil du roi voulut étonner l'Europe par la montre de sa
puissance qu'elle croyait abattue, et l'étonna en effet. Telle fut la
cause de ce fameux camp de Compiègne où, sous prétexte de montrer aux
princes ses petits-fils l'image de la guerre, il étala une magnificence
et dans sa cour et dans toutes ses nombreuses troupes inconnue aux plus
célèbres tournois, et aux entrevues des rois les plus fameuses. Ce fut
un nouvel épuisement au sortir d'une si longue et rude guerre. Tous les
corps s'en sentirent longues années, et il se trouva vingt ans après des
régiments qui en étaient encore obérés\,; on ne touche ici qu'en passant
ce camp trop célèbre. On s'y est étendu en son temps. On ne tarda pas
d'avoir lieu de regretter une prodigalité si immense et si déplacée, et
encore plus la guerre de 1688 qui venait de finir, au lieu d'avoir
laissé le royaume se repeupler, et se refaire par un long soulagement,
remplir cependant les coffres du roi avec lenteur, et les magasins de
toute espèce, réparer la marine et le commerce, laisser par les années
refroidir les haines et les frayeurs, séparer peu à peu des alliés si
unis, et si formidables étant ensemble, et donner lieu avec prudence, en
profitant des divers événements entre eux, à la dissolution radicale
d'une ligue qui avait été si fatale, et qui pouvait devenir funeste.
L'état de la santé de deux princes y conviait déjà puissamment\,: dont
l'un par la profondeur de sa sagesse, de sa politique, de sa conduite,
s'était acquis assez d'autorité et de confiance en Europe pour y donner
le branle à tout\,; et l'autre souverain de la plus vaste monarchie, qui
n'avait ni oncles, ni tantes, ni frères, ni sœurs, ni postérité. En
effet, moins de quatre ans après la paix de Ryswick, le roi d'Espagne
mourut, et le roi Guillaume n'en pouvait presque plus, et ne le survécut
guère.

Ce fut alors que la vanité du roi mit à deux doigts de sa perte ce grand
et beau royaume, dans les suites de ce grand événement qui fit reprendre
les armes à toute l'Europe. C'est ce qu'il faut reprendre de plus loin.

On a dit que le roi craignait l'esprit, les talents, l'élévation des
sentiments, jusque dans ses généraux et dans ses ministres. C'est ce qui
ajouta à l'autorité de Louvois un moyen si aisé d'écarter des élévations
militaires tout mérite qui lui pût être suspect, et d'empêcher, avec
l'adresse qu'on expliquera plus bas, qu'il se formât des sujets pour
remplacer les généraux.

À considérer ceux qui depuis que le roi se fut rendu suspect l'esprit et
le mérite au temps et à l'occasion qui ont été rapportés, on ne trouvera
qu'un bien petit nombre de courtisans en qui l'esprit n'ait pas été un
obstacle à la faveur, si on en excepte ceux qui, personnages ou simples
courtisans, l'avaient dompté par l'âge, et par l'habitude dans les
premiers temps qui suivirent la mort du cardinal Mazarin, et qu'il
n'avait pas choisis ni approchés de lui-même. M. de Vivonne, avec
infiniment d'esprit, l'amusait sans se pouvoir faire craindre. Le roi en
faisait volontiers encore cent contes plaisants. D'ailleurs il était
frère de M\textsuperscript{me} de Montespan, et c'était un grand titre,
quelque opposé que le frère parut à la conduite de la sœur, et de plus
le roi l'avait trouvé premier gentilhomme de sa chambre. Il trouva de
même M. de Créqui dans la même charge, qui le soutint, et dont la vie
tout occupée de plaisir, de bonne chère, du plus gros jeu, rassurait le
roi, dans l'habitude de familiarité qu'il avait prise avec lui de
jeunesse. Le duc du Lude, aussi premier gentilhomme de la chambre dès
ces premiers temps, tenait par les modes, le bel air, la galanterie, la
chasse\,; et au fond, pas un des trois n'avait rien qui pût se faire
craindre par le genre de leur esprit, quoiqu'ils en eussent beaucoup,
qui ne passa jamais celui de bons courtisans. La catastrophe de M. de
Lauzun, dont l'esprit était d'une autre trempe, vengea le roi de
l'exception\,; et la brillante singularité de son retour ne lui
réconcilia jamais qu'en apparence, comme on l'a vu par ce que le roi en
dit, lors de son mariage, à M. le maréchal de Lorges. Des ducs de
Chevreuse et de Beauvilliers, on en a parlé en leur lieu. Pour tous les
autres, ils lui pesèrent tellement à la fin chacun, qu'il le fit sentir
à la plupart, et qu'il se réjouit de leur mort comme d'une délivrance.
Il ne put s'empêcher de s'en expliquer sur M. de La Feuillade, et sur M.
de Paris, Harlay, et tout retenu et mesuré qu'il était, il lui échappa
de parler à Marly à table, et tout haut, où entre autres dames étaient
les duchesses de Chevreuse et de Beauvilliers, de la mort de Seignelay,
leur frère, et de celle de Louvois, comme d'un des grands soulagements
qu'il eût reçus de sa vie.

Depuis ceux-là, il n'en eut que deux d'un esprit supérieur\,: le
chancelier de Pontchartrain, qui longtemps avant sa retraite n'en était
supporté qu'avec peine, et dont au fond, quoi qu'il en voulût montrer,
il était aisé de voir qu'il fut ravi d'en être défait\,; et Barbezieux,
dont la mort si prompte, à la fleur de l'âge et de la fortune, fit pitié
à tout le monde. On a vu en son lieu que dès le soir même le roi n'en
put contenir sa joie, à son souper public à Marly\footnote{Voy. les
  notes de la fin du volume sur la conduite de Louis XIV à l'égard de
  Barbezieux.}.

Il avait été fatigué de la supériorité d'esprit et de mérite de ses
anciens ministres, de ses anciens généraux, de ce peu d'espèces de
favoris qui en avaient beaucoup. Il voulait primer par l'esprit, par la
conduite dans le cabinet et dans la guerre, comme il dominait partout
ailleurs. Il sentait qu'il ne l'avait pu avec ceux dont on vient de
parler\,; c'en fut assez pour sentir tout le soulagement de ne les avoir
plus, et pour se bien garder d'en choisir en leur place qui pussent lui
donner la même jalousie. C'est ce qui le rendit si facile sur les
survivances de secrétaire d'État, tandis qu'il s'était fait une loi de
n'en accorder de pas une autre charge, et qu'on a vu des novices et des
enfants même, exercer, et quelquefois en chef, ces importantes
fonctions, tandis que pour celles des moindres emplois, ou pour ceux-là
même qui n'avaient que le titre, il n'y avait point d'espérance. C'est
ce qui fit que, lorsque les emplois de secrétaires d'État et ceux de
ministres étaient à remplir, il ne consulta que son goût, et qu'il
affecta de choisir des gens fort médiocres. Il s'en applaudissait même,
jusque-là qu'il lui échappait souvent de dire qu'il les prenait pour les
former, et qu'il se piquait en effet de le faire.

Ces nouveaux venus lui plaisaient même à titre d'ignorance, et
s'insinuaient d'autant plus auprès de lui qu'ils la lui avouaient plus
souvent, qu'ils affectaient de s'instruire de lui jusque des plus
petites choses. Ce fut par là que Chamillart entra si avant dans son
cœur qu'il fallut tous les malheurs de l'État et la réunion des plus
redoutables cabales pour forcer le roi à s'en priver, toutefois sans
cesser de l'aimer toujours, et de lui en donner des marques en toute
occasion le reste de sa vie. Il fut sur le choix de ses généraux comme
sur celui de ses ministres. Il s'applaudissait de les conduire de son
cabinet\,; il voulait qu'on crût que, de son cabinet, il commandait
toutes ses armées. Il se garda bien d'en perdre la jalouse habitude, que
Louvois lui avait inspirée, comme on le verra bientôt, et pourquoi, dont
il ne put que pour des moments bien rares se résoudre d'en sacrifier la
vanité aux inconvénients continuels qui sautaient aux yeux de tout le
monde.

Tels étaient la plupart des ministres et tous les généraux à l'ouverture
de la succession d'Espagne. L'âge du roi, son expérience, cette
supériorité, non d'esprit ni de capacité ou de lumières, mais de poids,
et de poids immense, sur des conseillers et des exécuteurs de cette
sorte, l'habitude et le poison du plus mortel encens, confondit dès
l'entrée tous les miracles de la fortune. La monarchie entière d'Espagne
tomba sans coup férir entre les mains de son petit-fils\,; et Puységur,
si lard devenu maréchal de France en 1735, eut la gloire du projet et de
l'exécution de l'occupation de toutes les places espagnoles des
Pays-Bas, toutes au même instant, toutes sans brûler une amorce, toutes
en se saisissant et désarmant les troupes hollandaises, qui en formaient
presque toutes les garnisons.

Le roi, dans l'ivresse d'une prospérité si surprenante, se souvint mal à
propos du reproche que lui avait attiré l'injustice de ses guerres\,; et
que, de la frayeur qu'il avait causée à l'Europe s'étaient formées ces
grandes unions sous lesquelles il avait pensé succomber. Il voulut
éviter ces inconvénients\,; et au lieu de profiter de l'étourdissement
où ce grand événement avait jeté toutes les puissances, priver les
Hollandais de tant de troupes de ces nombreuses garnisons, les retenir
prisonniers, forcer les armes à la main toutes ces puissances désarmées,
et non encore unies, à reconnaître par des traités formels le duc
d'Anjou pour l'héritier légitime de tous les États que possédait le feu
roi d'Espagne, et dont dès lors le nouveau roi se trouvait entièrement
nanti, il se piqua de la folle générosité de laisser aller ces troupes
hollandaises, et se reput de l'espérance insensée que les traités, sans
les armes, feraient le même effet. Il se laissa amuser tant qu'il
convint à ses ennemis de le faire, pour se donner le temps d'armer et de
s'unir étroitement, après quoi il ne fut plus question que de guerre\,;
et le roi, bien surpris, se vit réduit à la soutenir partout, après
s'être si grossièrement mécompté.

Il l'entama par une autre lourdise où un enfant ne serait pas tombé. Il
la dut à Chamillart, au maréchal de Villeroy et à la puissante intrigue
des deux filles de M\textsuperscript{me} de Lislebonne. Ce fut l'entière
confiance en Vaudemont, leur oncle, l'ennemi personnel du roi, autant
que la distance le pouvait permettre, de l'insolence duquel, en Espagne
et en Italie, le roi n'avait pas dédaigné autrefois de se montrer très
offensé, et jusqu'à l'en faire sortir, l'ami confident du roi Guillaume,
le plus ardent et le plus personnel de tous les ennemis que le roi
s'était faits, et gouverneur du Milanais par ce même roi Guillaume et
par la plus pressante sollicitation de l'empereur Léopold auprès du roi
d'Espagne Charles II, enfin père d'un fils unique, qui se trouva, dès la
première hostilité en Italie, la seconde personne de l'armée de
l'empereur, et qui y est mort.

Il n'y avait celui qui ne vit clairement qu'il était averti de tout par
son père. La trahison dura même après que ce fils fut mort, et tant
qu'elle fut utile à Vaudemont, même avec grossièreté. Jamais le roi, son
ministre, ni Villeroy, son général, n'en soupçonnèrent la moindre
chose\,; jamais la faveur, la confiance, les préférences pour Vaudemont
ne diminuèrent\,; jamais personne assez hardi pour oser ouvrir les yeux
là-dessus au roi, ni à son ministre. Catinat, trahi par Vaudemont et par
M. de Savoie, y flétrit ses lauriers, et le maréchal de Villeroy, envoyé
en héros pour réparer ses fautes, tomba lourdement dans leurs filets. Le
duc de Vendôme, arrivé comme le réparateur, n'épargna pas M. de Savoie,
mais il avait de trop fortes raisons de ne toucher pas à Vaudemont\,;
volonté ou duperie, peut-être tous les deux, de franc dessein de ne rien
apercevoir.

La faiblesse du roi pour plaire à Chamillart sur La Feuillade, son
gendre, duquel il avait été si éloigné, et dont il avait voulu empêcher
le mariage, le fit tout d'un coup général d'armée, et lui confia le
siège de Turin, c'est-à-dire la plus importante affaire de l'État.
Tallard, si fait pour la cour, et si peu pour tout ce qui passe la
petite intrigue, fut défait à Hochstedt, sans presque aucune perte que
de ceux qui voulurent bien se rendre. Du fond de l'empire une armée
entière, et les trois quarts de l'autre fut rechassée au deçà du Rhin,
où tout de suite elles virent prendre Landau. Ce malheur avait été
précédé de la délivrance du maréchal de Villeroy, que le roi se piqua de
remettre en honneur. Il se fit battre à Ramillies, où, sans perte à
peine de deux mille hommes, il fut rechassé du fond des Pays-Bas dans le
milieu des nôtres, sans que rien le pût arrêter.

Restait l'espérance de l'Italie, où M. le duc d'Orléans fut enfin
relever Vendôme, mandé pour sauver les débris de la Flandre. Mais le
neveu du roi fut muni d'un tuteur, sans l'avis duquel il ne pouvait rien
faire, et ce tuteur était une linotte qui lui-même aurait eu grand
besoin d'en avoir un. Il n'eut jamais devant les yeux que la crainte de
La Feuillade et de son beau-père. On a vu dans son lieu à quels excès
ces ménagements le portèrent, les malheurs prévus et disputés par le
jeune prince, dépité à la fin jusqu'à ne vouloir plus se mêler de rien,
et la catastrophe qui suivit de si près.

Ainsi, après de prodigieux succès de toutes les sortes, l'infatigable
faveur de Villeroy, celle de Tallard, la constante confiance en
Vaudemont, les folles et ignorantes opiniâtretés de La Feuillade, le
tremblant respect de Marsin pour lui jusqu'au bout, coûtèrent
l'Allemagne, les Pays-Bas, l'Italie en trois batailles, qui, toutes les
trois ensemble, ne coûtèrent pas elles-mêmes quatre mille morts.

L'engouement pour Vendôme et ses perverses vue s'acheva de tout perdre
en Flandre.

En 1706, Tessé, par la le vue du siège de Barcelone dans la même année
que les défaites de Ramillies et de Turin, avait réduit le roi d'Espagne
à traverser du Roussillon en Navarre par la France, et à voir l'archiduc
proclamé dans Madrid en personne. Le duc de Berwick y rétablit les
affaires, M. le duc d'Orléans ensuite. Elles s'y perdirent de nouveau
par la perte de la bataille de Saragosse, qui ébranla une autre fois le
trône de Philippe V, tandis qu'on nous enlevait les places en Flandre,
et que la frontière s'y réduisait à rien. Qu'il y avait loin des portes
d'Amsterdam et des conquêtes des Pays-Bas espagnols et hollandais à
cette situation terrible\,!

Comme un malade qui change de médecins, le roi avait changé ses
ministres, donné les finances à Desmarets, enfin la guerre à Voysin.
Comme les malades aussi, il ne s'en trouvait pas mieux. La situation des
affaires était alors si extrême, que le roi ne pouvait plus soutenir la
guerre, ni parvenir à être reçu à faire la paix. Il consentait à tout\,:
abandonner l'Espagne, céder sur ses frontières tout ce qu'on voudrait
exiger\footnote{Voy. les Pièces (Note de Saint-Simon).}. Ses ennemis se
jouaient de sa ruine, et ne négociaient que pour se moquer. Enfin on a
vu en son lieu le roi aux larmes dans son conseil, et Torcy très
légèrement parti pour aller voir par lui-même à la Haye, si, et de quoi
on pouvait se flatter. On a vu aussi les tristes et les honteux succès
de cette tentative, et l'ignominie des conférences de Gertruydemberg qui
suivirent, où sans parler des plus que très étranges restitutions, on
n'exigeait pas moins du roi que de donner passage aux armées ennemies au
travers de la France pour aller chasser son petit-fils d'Espagne, avec
encore quatre places de sûreté en France entre leurs mains, dont
Cambrai, Metz, la Rochelle, et je crois Bayonne, si le roi n'aimait
mieux le détrôner lui-même à force ouverte, et encore dans un temps
limité. Voilà où conduisit l'aveuglement des choix, l'orgueil de tout
faire, la jalousie des anciens ministres et capitaines, la vanité d'en
choisir de tels qu'on ne pût leur rien attribuer, pour ne partager la
réputation de grand avec personne, la clôture exacte qui, fermant tout
accès, jeta dans les affreux panneaux de Vaudemont, puis de Vendôme,
enfin toute cette déplorable façon de gouverner qui précipita dans le
plus évident péril d'une perte entière, et qui jeta dans le dernier
désespoir ce maître de la paix et de la guerre, ce distributeur des
couronnes, ce châtieur des nations, ce conquérant, ce grand par
excellence, cet homme immortel pour qui on épuisait le marbre et le
bronze, pour qui tout était à bout d'encens.

Conduit ainsi jusqu'au dernier bord du précipice avec l'horrible loisir
d'en reconnaître toute la profondeur, la toute-puissante main qui n'a
posé que quelques grains de sable pour bornes aux plus furieux orages de
la mer, arrêta tout d'un coup la dernière ruine de ce roi si
présomptueux et si superbe, après lui avoir fait goûter à longs traits
sa faiblesse, sa misère, son néant. Des grains de sable d'un autre
genre, mais grains de sable par leur ténuité, opérèrent ce chef-d'œuvre.
Une querelle de femme chez la reine d'Angleterre pour des riens\,; de là
une intrigue, puis un désir vague et informe en faveur de son sang,
détachèrent l'Angleterre de la grande alliance. L'excès du mépris du
prince Eugène pour nos généraux donna lieu à ce qui se peut appeler pour
la France la délivrance de Denain, et ce combat si peu meurtrier eut de
telles suites qu'on eut enfin la paix, et une paix si différente de
celle qu'on aurait ardemment embrassée, si les ennemis avaient daigné y
entendre avant cet événement\,; événement dans lequel on ne put
méconnaître la main de Dieu, qui élève, qui abat, qui délivre, comme et
quand il lui plaît.

Mais toutefois cette paix qui coûta bien cher à la France, et à
l'Espagne la moitié de sa monarchie, ce fut le fruit de ce qui a été
exposé, et depuis encore, de n'avoir jamais voulu se faire justice à
soi-même dans les commencements de la décadence de nos affaires, avoir
toujours compté les rétablir, et n'avoir jamais voulu alors, comme je
l'ai rapporté en son lieu, céder un seul moulin de toute la monarchie
d'Espagne\,; autre folie dont on ne tarda guère à se bien repentir, et
de gémir sous un poids qui se fait encore sentir, et se sentira encore
longtemps par ses suites.

Ce peu d'historique, eu égard à un règne si long et si rempli, est si
lié au personnel du roi qu'il ne se pouvait omettre pour bien
représenter ce monarque tel qu'il a véritablement été. On l'a vu, grand,
riche, conquérant, arbitre de l'Europe, redouté, admiré tant qu'ont duré
les ministres et les capitaines qui ont véritablement mérité ce nom. À
leur fin, la machine a roulé quelque temps encore, d'impulsion, et sur
leur compte. Mais tôt après, le tuf s'est montré, les fautes, les
erreurs se sont multipliées, la décadence est arrivée à grands pas, sans
toutefois ouvrir les yeux à ce maître despotique si jaloux de tout faire
et de tout diriger par lui-même, et qui semblait se dédommager des
mépris du dehors par le tremblement que sa terreur redoublait au dedans.
Prince heureux s'il en fut jamais, en figure unique, en force
corporelle, en santé égale et ferme, et presque jamais interrompue, en
siècle si fécond et si libéral pour lui en tous genres qu'il a pu en ce
sens être comparé au siècle d'Auguste\,; en sujets adorateurs prodiguant
leurs biens, leur sang, leurs talents, la plupart jusqu'à leur
réputation, quelques-uns même leur honneur, et beaucoup trop leur
conscience et leur religion pour le servir, souvent même seulement pour
lui plaire. Heureux surtout en famille s'il n'en avait eu que de
légitime\,; en mère contente des respects et d'un certain crédit\,; en
frère dont la vie anéantie par de déplorables goûts, et d'ailleurs
futile par elle-même, se noyait dans la bagatelle, se contentait
d'argent, se retenait par sa propre crainte et par celle de ses favoris,
et n'était guère moins bas courtisan que ceux qui voulaient faire leur
fortune\,; une épouse vertueuse, amoureuse de lui, infatigablement
patiente, devenue véritablement Française, d'ailleurs absolument
incapable\,; un fils unique toute sa vie à la lisière, qui à cinquante
ans ne savait encore que gémir sous le poids de la contrainte et du
discrédit, qui, environné et éclairé de toutes parts, n'osait que ce qui
lui était permis, et qui absorbé dans la matière ne pouvait causer la
plus légère inquiétude\,; en petits-fils dont l'âge et l'exemple du
père, les brassières dans lesquelles ils étaient scellés, rassuraient
contre les grands talents de l'aîné, sur la grandeur du second qui de
son trône reçut toujours la loi de son aïeul dans une soumission
parfaite, et sur les fougues de l'enfance du troisième qui ne tinrent
rien de ce dont elles avaient inquiété\,; un neveu qui, avec des pointes
de débauches, tremblait devant lui, en qui son esprit, ses talents, ses
velléités légères et les fous propos de quelques débordés qu'il
ramassait, disparaissaient au moindre mot, souvent au moindre regard.
Descendant plus bas, des princes du sang de même trempe, à commencer par
le grand Condé, devenu la frayeur et la bassesse même, jusque devant les
ministres, depuis son retour à la paix des Pyrénées\footnote{Cet
  abaissement du grand Condé, après son retour en France (1661), est
  confirmé par le témoignage des contemporains. On lit dans le journal
  inédit d'Olivier d'Ormesson, à la date du mardi 2 août 1667, une
  anecdote relative au triste rôle auquel le vainqueur de Rocroy était
  réduit. L'historien Vittorio Siri, qui était alors à Paris, dit, en
  présence d'Olivier d'Ormesson et de l'abbé Le Tellier, fils du
  ministre, «\,que M. le Prince était obligé de faire sa cour aux
  ministres et à leurs commis, et de faire mille bassesses indignes d'un
  grand seigneur. M. l'abbé Le Tellier l'ayant prié de s'expliquer, il
  dit que M. le Prince avait été obligé de venir exprès de Saint-Germain
  pour entendre son sermon (le sermon de l'abbé Le Tellier) dans
  l'église des jésuites, et de revenir le lendemain pour lui en faire
  ses compliments. L'abbé Le Tellier demeura surpris de ce discours et
  le tourna en raillerie. L'abbé Siri ajouta que, M. le Prince avait été
  à l'acte (à la thèse) du fils de M. Colbert tout le premier pour se
  montrer.\,»}\,; M. le Prince son fils, le plus vil et le plus
prostitué de tous les courtisans, M. le Duc avec un courage plus élevé,
mais farouche, féroce, par cela même le plus hors de mesure de pouvoir
se faire craindre, et avec ce caractère, aussi timide que pas un des
siens, à l'égard du roi et du gouvernement\,; des deux princes de Conti
si aimables, l'aîné mort sitôt, l'autre avec tout son esprit, sa valeur,
ses grâces, son savoir, le cri public en sa faveur jusqu'au milieu de la
cour, mourant de peur de tout, accablé sous la haine du roi, dont les
dégoûts lui coûtèrent enfin la vie.

Les plus grands seigneurs lassés et ruinés des longs troubles, et
assujettis par nécessité. Leurs successeurs séparés, désunis, livrés à
l'ignorance, au frivole, aux plaisirs, aux folles dépenses, et pour ceux
qui pensaient le moins mal, à la fortune, et dès lors à la servitude et
à l'unique ambition de la cour. Des parlements subjugués à coups
redoublés, appauvris, peu à peu l'ancienne magistrature éteinte avec la
doctrine et la sévérité des mœurs, farcis en la place d'enfants de gens
d'affaires, de sots du bel air, ou d'ignorants pédants, avares,
usuriers, aimant le sac, souvent vendeurs de la justice, et de quelques
chefs glorieux jusqu'à l'insolence, d'ailleurs vides de tout. Nul corps
ensemble, et par laps de temps, presque personne qui osât même à part
soi avoir aucun dessein, beaucoup moins s'en ouvrir à qui que ce soit.
Enfin jusqu'à la division des familles les plus proches parmi les
considérables, l'entière méconnaissance des parents et des parentes, si
ce n'est à porter les deuils les plus éloignés, peu à peu tous les
devoirs absorbés par un seul que la nécessité fit, qui fut de craindre
et de tâcher à plaire. De là cette intérieure tranquillité jamais
troublée que par la folie momentanée du chevalier de Rohan, frère du
père de M. de Soubise, qui la paya incontinent de sa tête, et par ce
mouvement des fanatiques des Cévennes qui inquiéta plus qu'il ne valut,
dura peu et fut sans aucune suite, quoique arrivée en pleine et fâcheuse
guerre contre toute l'Europe.

De là cette autorité sans bornes qui put tout ce qu'elle voulut, et qui
trop souvent voulut tout ce qu'elle put, et qui ne trouva jamais la plus
légère résistance, si on excepte des apparences plutôt que des réalités
sur des matières de Rome, et en dernier lieu sur la constitution. C'est
là ce qui s'appelle vivre et régner\,; mais il faut convenir en même
temps qu'en glissant sur la conduite du cabinet et des armées jamais
prince ne posséda l'art de régner à un si haut point. L'ancienne cour de
la reine sa mère, qui excellait à la savoir tenir, lui avait imprimé une
politesse distinguée, une gravité jusque dans l'air de galanterie, une
dignité, une majesté partout qu'il sut maintenir toute sa vie, et lors
même que vers sa fin il abandonna la cour à ses propres débris.

Mais cette dignité, il ne la voulait que pour lui, et que par rapport à
lui\,; et celle-là même relative, il la sapa presque toute pour mieux
achever de ruiner toute autre, et de la mettre peu à peu, comme il fit,
à l'unisson, en retranchant tant qu'il put toutes les cérémonies et les
distinctions, dont il ne retint que l'ombre, et certaines trop marquées
pour les détruire, en semant même dans celles-là des zizanies qui les
rendaient en partie à charge et en partie ridicules. Cette conduite lui
servit encore à séparer, à diviser, à affermir la dépendance en la
multipliant par des occasions sans nombre, et très intéressantes, qui,
sans cette adresse, seraient demeurées dans les règles, et sans produire
de disputes, et de recours à lui. Sa maxime encore n'était que de les
prévenir, hors des choses bien marquées, et de ne les point juger\,; il
s'en savait bien garder pour ne pas diminuer ces occasions qu'il se
croyait si utiles. Il en usait de même à cet égard pour les provinces\,;
tout y devint sous lui litigieux et en usurpations, et par là il en tira
les mêmes avantages.

Peu à peu il réduisit tout le monde à servir et à grossir sa cour,
ceux-là même dont il faisait le moins de cas. Qui était d'âge a servir
n'osait différer d'entrer dans le service. Ce fut encore une autre
adresse pour ruiner les seigneurs, et les accoutumer à l'égalité, et à
rouler pêle-mêle avec tout le monde.

Cette invention fut due à lui et à Louvois, qui voulait régner aussi sur
toute seigneurie, et la rendre dépendante de lui, en sorte que les gens
nés pour commander aux autres demeurèrent dans les idées et ne se
trouvèrent plus dans aucune réalité.

Sous prétexte que tout service militaire est honorable, et qu'il est
raisonnable d'apprendre à obéir avant que de commander, il assujettit
tout, sans autre exception que des seuls princes du sang, à débuter par
être cadets dans ses gardes du corps, et à faire tout le même service
des simples gardes du corps, dans les salles des gardes, et dehors,
hiver et été, et à l'armée. Il changea depuis cette prétendue école en
celle des mousquetaires, quand la fantaisie de ce corps lui prit, école
qui n'était pas plus réelle que l'autre, et où, comme dans la première,
il n'y avait dans la vérité rien du tout à apprendre qu'à se gâter, et à
perdre du temps\,; mais aussi on s'y ployait par force à y être confondu
avec toute sorte de gens et de toutes les espèces, et c'était là tout ce
que le roi prétendait en effet de ce noviciat, où il fallait demeurer
une année entière dans la plus exacte régularité de tout cet inutile et
pédantesque service, après laquelle il fallait essuyer encore une
seconde école, laquelle au moins en pouvait être une. C'était une
compagnie de cavalerie pour ceux qui voulaient servir dans la cavalerie,
et pour ceux qui se destinaient à l'infanterie, une lieutenance dans le
régiment du roi, duquel le roi se mêlait immédiatement, comme un
colonel, et qu'il avait exprès fort distingué de tous les autres.

C'était une autre station subalterne où le roi retenait plus ou moins
longtemps avant d'accorder l'agrément d'acheter un régiment qui lui
donnait, et à son ministre, plus ou moins lieu d'exercer grâce ou
rigueur, selon qu'il voulait traiter les jeunes gens sur les témoignages
qu'il en recevait, et plus sous main qu'autrement, ou leurs parents
encore, desquels la façon d'être avec lui, ou avec son ministre,
influait entièrement là-dessus. Outre l'ennui et le dépit de cet état
subalterne, et la naturelle jalousie les uns des autres à en sortir le
plus tôt, c'est qu'il était peu compté pour obtenir un régiment, et non
limité, et pour rien du tout en soi-même, parce qu'il fut établi que la
première date d'où l'avancement dans les grades militaires serait compté
était celle de la commission de mestre de camp ou de colonel.

Au moyen de cette règle, excepté des occasions rares et singulières,
comme d'action distinguée, de porter une grande nouvelle de guerre,
etc., il fut établi que quel qu'on pût être, tout ce qui servait
demeurait, quant au service et aux grades, dans une égalité entière.

Cela rendit l'avancement ou le retardement d'avoir un régiment bien plus
sensible, parce que de là dépendait tout le reste des autres
avancements, qui ne se firent plus que par promotions suivant
l'ancienneté, qu'on appela \emph{l'ordre du tableau\,;} de là tous les
seigneurs dans la foule de tous les officiers de toute espèce\,; de là
cette confusion que le roi désirait\,; de là peu à peu cet oubli de
tous, et, dans tous, de toute différence personnelle et d'origine, pour
ne plus exister que dans cet état de service militaire devenu
populaire\,; tout entier sous la main du roi, beaucoup plus sous celle
de son ministre, et même de ses commis, lequel ministre avait des
occasions continuelles de préférer et de mortifier qui il voulait, dans
le courant, et qui ne manquait pas d'en préparer avec adresse les moyens
d'avancer ses protégés, malgré l'ordre du tableau, et d'en reculer de
même ceux que bon lui semblait.

Si d'ennui, de dépit, ou par quelque dégoût on quittait le service, la
disgrâce était certaine\,; c'était merveille si après des années
redoublées de rebuts on parvenait à revenir sur l'eau. À l'égard de ce
qui n'était point de la cour, et même du commun, outre que le roi y
tenait l'œil lui-même, le ministre de la guerre en faisait son étude
particulière, et de ceux-là, qui quittait, était assuré lui et sa
famille d'essuyer dans sa province ou dans sa ville toutes les
mortifications, et souvent les persécutions dont on pouvait s'aviser,
dont on rendait les intendants des provinces responsables, et qui très
ordinairement influaient sur les terres et sur les biens.

Grands et petits, connus et obscurs, furent donc forcés d'entrer et de
persévérer dans le service, d'y être un vil peuple en toute égalité, et
dans la plus soumise dépendance du ministre de la guerre, et même de ses
commis.

J'ai vu Le Guerchois, mort conseiller d'État, lors intendant d'Alençon,
me montrer, à la Ferté, un ordre de faire recherche des gentilshommes de
sa généralité qui avaient des enfants en âge de servir, et qui n'étaient
pas dans le service, de les presser de les y mettre, de les menacer
même, et de doubler et tripler à la capitation ceux qui n'obéiraient
pas, et de leur faire toutes les sortes de vexations dont ils seraient
susceptibles. Ce fut à l'occasion d'un gentilhomme qui était dans le
cas, et pour qui j'avais de l'amitié, et que j'envoyai chercher, en
effet, pour le résoudre. Le Guerchois fut depuis intendant à Besançon,
et il fut fait conseiller d'État dans les commencements de la régence.

Avant de finir ce qui regarde cette politique militaire, il faut voir à
quel point Louvois abusa de cette misérable jalousie du roi de tout
faire et de tout mettre dans sa dépendance immédiate, pour ranger tout
lui-même sous sa propre autorité, et comment sa pernicieuse ambition a
tari la source des capitaines en tout genre, et a réduit la France en ce
point à n'en trouver plus chez elle, et à n'en pouvoir plus espérer,
parce que des écoliers ne peuvent apprendre que sous des maîtres, et
qu'il faut que cette succession se suive et se continue de main en main,
attendu que la capacité ne se crée point par les hommes.

On a déjà vu les funestes obligations de la France à ce pernicieux
ministre. Des guerres sans mesure et sans fin pour se rendre nécessaire,
pour sa grandeur, pour son autorité, pour sa toute-puissance. Des
troupes innombrables, qui ont appris à nos ennemis à en avoir autant,
qui, chez eux, sont inépuisables, et qui ont dépeuplé le royaume\,;
enfin la ruine des négociations et de la marine, de notre commerce, de
nos manufactures, de nos colonies, par sa jalousie de Colbert, de son
frère et de son fils, entre les mains desquels était le département de
ces choses, et le dessein trop bien exécuté de ruiner la France riche et
florissante pour culbuter Colbert. Reste à voir comment il a, pour être
pleinement maître, arraché les dernières racines des capitaines en
France, et l'a mise radicalement hors de moyen d'en plus porter.

Louvois, désespéré du joug de M. le Prince et de M. de Turenne, non
moins impatient du poids de leurs élèves, résolut de se garantir de
celui de leurs successeurs, et d'énerver ces élèves mêmes. Il persuada
au roi le danger de ne tenir pas par les cordons les généraux de ses
armées, qui, ignorant les secrets du cabinet, et préférant leur
réputation à toutes choses, pouvaient ne s'en pas tenir au plan convenu
avec eux avant leur départ, profiter des occasions, faire des
entreprises dont le bon succès troublerait les négociations secrètes, et
les mauvais feraient un plus triste effet\,; que c'était à l'expérience
et à la capacité du roi de régler non seulement les plans des campagnes
de toutes ses armées, mais d'en conduire le cours de son cabinet, et de
ne pas abandonner le sort de ses affaires à la fantaisie de ses
généraux, dont aucun n'avait la capacité, l'acquis ni la réputation de
M. le Prince et de M. de Turenne, leurs maîtres.

Louvois surprit ainsi l'orgueil du roi, et, sous prétexte de le
soulager, fit les plans des diverses campagnes, qui devinrent les lois
des généraux d'armée, et qui peu à peu ne furent plus reçus à en
contredire aucun. Par même adresse il les tint tous en brassière pendant
le cours des campagnes jusqu'à n'oser profiter d'aucune occasion, sans
en avoir envoyé demander la permission, qui s'échappait presque toujours
avant d'en avoir reçu la réponse. Par là Louvois devint le maître de
porter ou non le fort de la guerre où il voulut, et de lâcher ou retenir
la bride aux généraux d'armée à sa volonté, par conséquent de les faire
valoir ou les dépriser à son gré.

Cette gêne, qui justement dépita lés généraux d'armée, causa la perte
des plus importantes occasions, et souvent des plus sûres, et une
négligence qui en fit manquer beaucoup d'autres.

Ce grand pas fait, Louvois inspira au roi cet ordre funeste du tableau,
et ces promotions nombreuses par l'ancienneté, qui flatta cette superbe
du roi de rendre toute condition simple peuple, mais qui fit aussi à la
longue que toute émulation se perdit, parce que, dès qu'il fut établi
qu'on ne montait plus qu'à son rang à moins d'événements presque uniques
auxquels encore il fallait que la faveur fût jointe, personne ne se
soucia plus de se fatiguer et de s'instruire, également sûr de n'avancer
point hors de son rang, et d'avancer aussi par sa date, sans une
disgrâce qu'on se contentait à bon marché de ne pas encourir.

Cet ordre du tableau, établi comme on l'a vu, et par les raisons qui ont
été expliquées, n'en demeura pas là. Sous prétexte que dans une armée
les officiers généraux prennent jour à leur tour, M. de Louvois, qui
voulait s'emparer de tout, et barrer toute autre voie que la sienne de
pouvoir s'avancer, fit retomber cet ordre du tableau sur les généraux
des armées. Jusqu'alors ils étaient en liberté et en usage de donner à
qui bon leur semblait les détachements gros ou petits de leurs armées.
C'était à eux, suivant la force et la destination du détachement, de
choisir qui ils voulaient pour le commander, et nul officier général ni
particulier n'était en droit d'y prétendre. Si le détachement était
important, le général prenait ce qu'il croyait de meilleur parmi ses
officiers généraux pour le commander\,; s'il était moindre, il
choisissait un officier de moindre grade. Parmi ces derniers, les
généraux d'armée avaient coutume d'essayer de jeunes gens qu'ils
savaient appliqués et amoureux de s'instruire. Ils voyaient comment ils
s'y prenaient à mener ces détachements, et les leur donnaient plus ou
moins gros, et une besogne plus ou moins facile, suivant qu'ils avaient
déjà montré plus ou moins de capacité. C'est ce qui faisait dire à M. de
Turenne qu'il n'en estimait pas moins ceux qui avaient été battus\,;
qu'au contraire on n'apprenait bien que par là à prendre son parti une
autre fois, et qu'il fallait l'avoir été deux ou trois fois pour pouvoir
devenir quelque chose. Si les généraux d'armée reconnaissaient par ces
expériences un sujet peu capable, ils le laissaient doucement\,; s'ils y
trouvaient du talent et de la ressource, ils le poussaient. Par là ils
étaient toujours bien servis. Les officiers généraux et particuliers
sentaient que leur réputation et leur fortune dépendait de leur
application, de leur conduite, de leurs actions\,; que la distinction
journelle y était attachée par la préférence ou par le délaissement\,;
tout contribuait donc en eux à l'émulation de s'appliquer, d'apprendre,
de s'instruire\,; et c'était parmi les jeunes à faire leur cour à ceux
qui étaient les plus employés pour être reçus par eux à s'instruire, et
à s'en laisser accompagner dans les détachements pour les voir faire et
apprendre sous eux. Telle fut l'école qui de plus en plus gros
détachements, qui de plus en plus de besogne importante, conduisit au
grand les élèves de ces écoles, et qui, suivant la capacité, forma cette
foule d'excellents officiers généraux, et ce petit nombre de grands
capitaines.

Les généraux d'armée qui rendaient compte d'eux à mesure par leurs
dépêches, en rendaient un plus étendu à leur retour. Tous sentaient le
besoin qu'ils avaient de ces témoignages pour leur réputation et pour
leur fortune\,; tous s'empressaient donc de les mériter, et de plaire,
c'est-à-dire de se présenter à tout, et de soulager et d'aider, chacun
selon sa portée, le général d'armée sous qui ils servaient, ou
l'officier général dans le corps duquel ils se trouvaient détachés. Cela
opérait une volonté, une application, une vigilance, dont le total
servait infiniment au général et au succès de la campagne.

Ceux qui se distinguaient le plus cheminaient aussi à proportion\,; ils
devenaient promptement lieutenants généraux, et presque tous ceux qui
sont parvenus au bâton de maréchal de France, avant que Louvois le
procurât, y étaient parvenus avant quarante ans. L'expérience a appris
qu'ils en étaient bien meilleurs, et suivant le cours de nature, ils
avaient vingt-cinq ou trente ans à employer leurs talents à la tête des
armées. Des guerriers de ce mérite ne ployaient pas volontiers sous
Louvois\,; aussi les détruisit-il, et avec eux leur pépinière\,; ce fut
par ce fatal ordre du tableau.

Il avait déjà réduit les généraux d'armée à recevoir de sa main les
projets de campagne comme venant du roi. Il les avait exclus d'y
travailler sans lui, et de s'expliquer de rien avec le roi, ni le roi
avec eux qu'en sa présence, tant en partant qu'en revenant\,; enfin il
les avait mis à la lisière peu à peu, de plus en plus resserrée, à
n'oser faire un pas, ni presque jamais oser profiter de l'occasion la
plus glissante de la main, sans ordre ou permission, et les avait
réduits sous les courriers du cabinet. Il alla plus loin.

Il fit entendre au roi que l'emploi de commander une armée était de
soi-même assez grand pour ne devoir pas chercher à le rendre plus
puissant par la facilité de s'attacher des créatures, et même les
familles de ces créatures dont ils pouvaient s'appuyer beaucoup\,; que
ce choix de faire marcher qui ils voulaient à l'armée était nécessaire
avant ce sage établissement de l'ordre du tableau qui mettait tout en la
main de Sa Majesté\,; mais que désormais, l'ayant établi, il devait
s'étendre à tout, et ne plus laisser de choix aux généraux d'armée qui
devenait même injurieux aux officiers généraux et particuliers, puisque
c'était montrer une préférence qui ne pouvait que marquer plus de
confiance, par conséquent plus d'estime pour l'un que pour l'autre, qui
n'était souvent que d'éloignement ou de caprice contre l'un, de
fantaisie, d'amitié, ou de raison personnelle pour l'autre\,; qu'il
fallait donc que les officiers généraux et particuliers qui prenaient
jour, ou qui étaient de piquet, en pareil grade les uns après les
autres, suivant leur ancienneté, marchassent de même pour les
détachements, sans en intervertir l'ordre à la volonté du général, et
ôter par cet unisson tout lieu aux jalousies, et aux généraux de pousser
et de reculer qui bon leur semblait.

Le goût du roi, fort d'accord avec les vues de son ministre qu'il
n'aperçut pas, embrassa aisément sa proposition. Il en fit une règle qui
a toujours depuis été observée. De manière que si un général d'armée a
un détachement délicat à faire, il est forcé de le donner au balourd qui
est à marcher, et s'il s'en trouve plusieurs de suite, comme cela
n'arrive que trop souvent, il faut qu'il en essuie le hasard ou qu'il
fatigue ses troupes d'autant de détachements inutiles qu'il y a de
balourds à marcher, jusqu'à celui qu'il veut charger du détachement
important\,; et si encore cela se trouvait un peu réitéré, ce seraient
des plaintes et des cris à l'honneur et à l'injustice, dès que cela
serait aperçu. On voit assez combien cet inconvénient est important pour
une armée, mais l'essentiel est que cette règle est devenue la perte de
l'école de la guerre, de toute instruction, de toute émulation. Il n'y a
plus où, ni de quoi apprendre, plus d'intérêt de plaire aux généraux, ni
de leur être d'aucune utilité par son application et sa vigilance. Tout
est également sous la loi de l'ancienneté ou de l'ordre du tableau. On
se dit qu'il n'y a qu'à dormir et faire ric à rac son service, et
regarder la liste des dates\,; puisque rien n'avance que la date seule
qu'il n'y a qu'à attendre en patience et en tranquillité, sans devoir
rien à personne, ni à soi-même. Voilà l'obligation qu'a la France à
Louvois qui a sapé toute formation de capitaines pour n'avoir plus à
compter avec le mérite, et que l'incapacité eût un continuel besoin de
sa protection\,: voilà ce que le royaume doit à l'aveugle superbe de
Louis XIV.

Les promotions introduites achevèrent de tout défigurer par achever de
tout confondre\,: mérite, actions, naissance, contradictoire de tout
cela moyennant le tour de l'ancienneté, et les rares exceptions que
Louvois y sut bien faire dès en les établissant, pour ceux qu'il voulut
avancer, comme aussi pour ceux qu'il voulut reculer et dégoûter. Le
prodigieux nombre de troupes que le roi mettait en campagne servit à
grossir et à multiplier les promotions\,; et ces promotions, devenues
bien plus fréquentes et bien plus nombreuses depuis, ont accablé les
armées d'un nombre sans mesure de tous les grades. Un autre inconvénient
en est résulté\,: c'est qu'à force d'officiers généraux et de
brigadiers, c'est merveille s'ils marchent chacun trois ou quatre fois
dans toute une campagne, et ce n'en est pas une s'ils ne marchent qu'une
fois ou deux. Or, sans leçon, sans école, quel moyen reste-t-il
d'apprendre et de se former que de se trouver souvent en besogne pour
s'instruire, si l'on peut, par la besogne même, à force de voir et de
faire\,? et ils n'y sont jamais, et ils n'y peuvent être.

Une autre chose a mis le comble à ce désordre et à l'ignorance de la
guerre\,: ce sont les troupes d'élite. J'appelle ainsi dans l'infanterie
les régiments des gardes françaises et suisses, et le régiment du roi\,;
dans la cavalerie, la maison du roi et la gendarmerie. Le roi, pour les
distinguer, y a confondu tous les grades, et y a fait presque dans
chaque promotion une fourmilière d'officiers généraux. Les officiers de
ces corps ne peuvent même apprendre le peu que font les autres, parce
que, tout avancés qu'ils sont, ils ne font jamais que le service de
lieutenant ou de capitaine d'infanterie et de cavalerie, qui est celui
de l'intérieur de leurs corps. Si on les fait servir d'officiers
généraux, ils sautent immédiatement à ce service sans en avoir vu ni
appris quoi que ce soit, ni du service encore des gardes qui sont
entre-deux. On laisse à penser de celui qu'ils peuvent rendre, et de
l'embarras que cette multiplication, qui se peut dire foule, cause dans
une armée par eux-mêmes et par leurs équipages.

Et après tout cela on est surpris d'avoir tant de maréchaux de France,
et si peu à s'en servir, et dans une immensité d'officiers généraux un
nombre si court qui sache quelque chose, et de n'en pouvoir discerner
aucun à mettre en chef, ou le bâton de maréchal de France à la main,
qu'à titre de son ancienneté. De là le malheur des armées, et la honte
d'avoir recours à des étrangers fort nouveaux pour les commander, et
sans espérance d'y pouvoir former personne. Les maîtres ne sont plus,
les écoles sont éteintes, les écoliers disparus, et avec eux tout moyen
d'en élever d'autres. Mais le pouvoir sans bornes des secrétaires d'État
de la guerre, qui tous ont bien soutenu là-dessus les errements de
Louvois, est un dédommagement que qui y pourrait chercher du remède
trouve apparemment suffisant. Le roi a craint les seigneurs et a voulu
des garçons de boutique\,; quel est le seigneur qui eût pu porter un
coup si mortel à la France pour son intérêt et sa grandeur\,?

Après tant de montagnes devenues vallées sous le poids de Louvois, il
trouva encore des collines à abattre\,; un souffle de sa bouche en vint
à bout. Les régiments étaient sous la disposition de leurs colonels dans
l'infanterie, la cavalerie, les dragons. Leur fortune dépendait de les
tenir complets, bons, exacts dans le service, et leur honneur de les
avoir vaillants et bien composés\,; leur estime d'y vivre avec justice
et désintéressement, en bons pères de famille\,; et l'intérêt des
officiers, de leur plaire et d'acquérir leur estime, puisque leur
avancement et tout détail intérieur dépendait d'eux. Aussi était-ce aux
colonels à répondre de leurs régiments en toutes choses, et ils étaient
punis de leurs négligences et de leurs injustices, s'il s'en trouvait
dans leur conduite. Cette autorité, quoique si nécessaire pour le bien
du service, si peu étendue, on peut ajouter encore si subalterne, déplut
à Louvois. Il voulut l'ôter aux colonels et l'usurper\footnote{Il
  n'était pas inutile de surveiller les jeunes nobles, charges
  d'organiser les compagnies et les régiments\,; témoin ce passage des
  lettres de M\textsuperscript{me} de Sévigné (lettre du 4 février
  1689)\,: «\,M. de Louvois dit l'autre jour tout haut à M. de
  Nogaret\,: «\, Monsieur, votre compagnie est en fort mauvais état. ---
  Monsieur, dit-il, je ne le savais pas. --- Il faut le savoir dit M. de
  Louvois\,; l'avez-vous vue\,? --- Non, Monsieur, dit Nogaret. --- Il
  faudrait l'avoir vue, monsieur. --- Monsieur, j'y donnerai ordre. ---
  Il faudrait l'avoir donné\,: il faut prendre parti, monsieur\,: ou se
  déclarer courtisan, ou s'acquitter de son devoir, quand on est
  officier. »}.

Il se servit pour y réussir de ce faible du roi pour tous les petits
détails. Il l'entretint de ceux des troupes, des inconvénients qu'il lui
forgea de les laisser à la discrétion des colonels, trop nombreux pour
pouvoir tenir un œil sur chacun d'eux aussi ouvert et aussi vigilant
qu'il serait nécessaire\,; enfin il lui proposa d'établir des
inspecteurs choisis parmi les colonels les plus appliqués et les plus
entendus au détail des troupes, qui les passeraient en revue dans les
districts qui leur seraient distribués, qui examineraient la conduite
des colonels et des officiers, qui recevraient leurs plaintes, et celles
même des soldats cavaliers et dragons, qui entreraient dans les détails
pécuniaires avec autorité, dans celui du mérite, du démérite, du service
de chacun, qui examineraient et régleraient provisoirement les disputes,
et ce qui regarderait l'habillement et l'armement sur tout le complet\,;
les chevaux et leurs équipages, qui rendraient un compte exact de toutes
ces choses deux ou trois fois l'année au roi, c'est-à-dire à lui-même,
sur lequel on réglerait toutes choses avec connaissance de cause dans
les régiments, et on connaîtrait exactement le service, la conduite et
le mérite, l'esprit même des corps des officiers qui les composaient et
des colonels, pour décider avec lumière de leur avancement, de leurs
punitions et de leurs récompenses.

Le roi, charmé de ces nouveaux détails et de la connaissance qu'il
allait acquérir si facilement de cette immensité d'officiers
particuliers qui composaient toutes ses troupes, donna dans le piège, et
en rendit par la Louvois le maître immédiat et despotique. Il sut
choisir les inspecteurs qui lui convenaient\,; c'étaient des grâces de
plus qu'il se donnait à répandre. Dans le peu qu'il laissa ces
inspecteurs rendre compte au roi pour l'en amuser, et les autoriser dans
les commencements, il eut grand soin de voir tout auparavant avec eux,
et de leur faire leur leçon, qu'ils étaient d'autant plus obliges de
suivre à la lettre, qu'il était toujours présent au compte qu'ils
rendaient au roi.

En même temps il usa d'une autre adresse pour empêcher que les
inspecteurs ne pussent lui échapper. Sous prétexte de l'étendue des
frontières et des provinces où les troupes étaient répandues l'hiver, et
de l'éloignement des différentes armées, l'été, les unes des autres, il
établit un changement continuel des mêmes inspecteurs, qui ne voyaient
jamais plusieurs fois de suite les mêmes troupes, de peur qu'ils n'y
prissent trop d'autorité, tellement qu'ils ne furent utiles qu'à ôter
toute autorité aux colonels, et inutiles pour toute autre chose, même
pour l'exécution de ce qu'ils avaient ordonné ou réformé, puisqu'ils ne
pouvaient le voir ni le suivre, et que c'était à un autre inspecteur à
s'en informer, qui le plus souvent y était trompé, ne pouvait deviner et
ordonnait tout différemment.

Ce fut un cri général dans les troupes. Les colonels généraux et les
mestres de camp généraux de la cavalerie et des dragons, surtout le
commissaire général de la cavalerie, qui en était l'inspecteur général
né, perdirent le peu d'autorité qu'ils avaient pu sauver des mains de
Louvois qui l'avait presque tout anéantie, et qui par ce dernier coup en
fit de purs fantômes. Les colonels ne demeurèrent guère autre chose\,;
les officiers sensés se dégoûtèrent de dépendre désormais de ces espèces
de passe-volants\footnote{On donnait ordinairement ce nom à des soldats
  de parade que les capitaines faisaient figurer dans les montres ou
  revues pour que leur compagnie parût au complet.} qui ne pouvaient les
connaître\,; d'autres par diverses raisons furent bien aises de ne plus
dépendre de leurs colonels.

On n'osa rien dans cette primeur où Louvois, les yeux ouverts et le
fouet à la main, châtiait rudement le moindre air de murmure, plus
encore de dépit. Mais après lui on commença à sentir dans les troupes
tout le faux d'un établissement qui ne fit que s'accroître en nombre, et
diminuer en considération. On crut y remédier en faisant des officiers
généraux directeurs de cavalerie et d'infanterie, avec les inspecteurs
sous eux. Ce ne fut que plus de confusion dans les ordres et les
détails, plus de cabales dans les régiments, plus de négligence dans le
service. Les colonels, devenus incapables de faire ni bien ni mal,
furent peu comptés dans leurs régiments, peu en état, par conséquent,
d'y bien faire faire le service, et les plus considérables peu en
volonté de se donner une peine désagréable et infructueuse. Sous
prétexte de l'avis des inspecteurs, le bureau, c'est-à-dire le ministre
de la guerre, et bien plus ses principaux commis, disposèrent peu à peu
des emplois des régiments, sans nul égard pour ceux que les colonels
proposaient, tellement que le dégoût, la confusion, le dérèglement, le
désordre, se glissèrent dans les troupes, où ce ne fut plus que brigues,
souplesses, souvent querelles et divisions, toujours mécontentements et
dégoûts.

C'est ce qui a comblé les désastres de nos dernières guerres\,; mais à
quoi l'autorité et l'intérêt du bureau empêchera toujours d'apporter le
remède unique, qui serait de remettre les choses à cet égard comme elles
étaient avant cette destructive invention. Mais elle fit passer toute
l'autorité particulière, et pour ainsi dire domestique, entre les mains
de Louvois. Il en savait trop pour n'en avoir pas senti les funestes
conséquences, mais il ne songeait qu'à lui, et ne souffrit pas longtemps
que les inspecteurs rendissent compte au roi\,; il se chargea bientôt de
le faire seul pour eux\,; et ses successeurs ont bien su se maintenir
dans cette possession, excepté des occasions fort rares, momentanées, et
toujours en leur présence.

Louvois imagina une autre nouveauté pour se rendre encore plus puissant
et plus l'arbitre des fortunes militaires\,: ce fut le grade de
brigadier\footnote{Les brigadiers, ou commandants de brigade, furent
  institués en 1665 pour la cavalerie, et en 1668 pour l'infanterie. Ils
  avaient rang au-dessus des colonels et mestres de camp.}, inconnu
jusqu'à lui dans nos troupes, et avec qui on aurait pu se passer
utilement de faire connaissance. Les autres troupes de l'Europe n'en ont
eu que depuis fort peu de temps. L'ancien des colonels de chaque brigade
la commandait\,; et dans les détachements, les plus anciens colonels qui
s'y trouvaient commandés y faisaient le service qui a depuis été
attribué à ce grade. Il est donc inutile et superflu, mais il servit à
retarder l'avancement de ce premier grade au-dessus des colonels, par
conséquent à Louvois à en avoir un de plus à avancer ou à reculer qui
bon lui semblerait, et dans la totalité des grades, à rendre le chemin
plus difficile et plus long, à arriver plus tard à celui de lieutenant
général, et à retarder le bâton à l'âge plus que sexagénaire, où alors
on n'avait ni l'acquis ni la force de lutter avec le secrétaire d'État,
ni de lui faire le plus léger ombrage.

On n'en a vu depuis d'exception que le dernier maréchal d'Estrées, pour
la marine, par un hasard heureux d'avoir eu de bonne heure la place de
vice-amiral de son père\,; et par terre, le duc de Berwick, que son
mérite seul n'eût jamais avancé sans la transcendance de sa qualité de
bâtard. On a senti et on sentira longtemps encore ce que valent ces
généraux sexagénaires, et des troupes abandonnées à elles-mêmes sous le
nom des inspecteurs et sous la férule du bureau, c'est-à-dire sous
l'ignorant et l'intéressé despotisme du secrétaire d'État de la guerre,
et sous celui d'un roi trop véritablement muselé. Venons maintenant à un
autre genre de politique de Louis XIV.

\hypertarget{chapitre-xix.}{%
\chapter{CHAPITRE XIX.}\label{chapitre-xix.}}

~

{\textsc{La cour pour toujours à la campagne\,; raisons de cette
politique.}} {\textsc{- Origine de Versailles.}} {\textsc{- Le roi veut
une grosse cour.}} {\textsc{- Ses adresses pour la rendre et la
maintenir telle.}} {\textsc{- Application du roi à être informé de
tout.}} {\textsc{- Police\,; délations.}} {\textsc{- Secret des
postes.}} {\textsc{- Le roi se pique de tenir parole, est fort secret,
se plaît aux confiances.}} {\textsc{- Singulière histoire là-dessus.}}
{\textsc{- Art personnel du roi à rendre tout précieux.}} {\textsc{- Sa
retenue\,; sa politesse mesurée.}} {\textsc{- Patience du roi, et
précision et commodité de son service et de sa cour.}} {\textsc{- Crédit
et familiarité des valets.}} {\textsc{- Jalousie du roi pour le respect
rendu à ceux qu'il envoyait.}} {\textsc{- Récit bien singulier sur le
duc de Montbazon.}} {\textsc{- Grâces naturelles du roi en tout.}}
{\textsc{- Son adresse\,; son air galant, grand, imposant.}} {\textsc{-
Politique du plus grand luxe.}} {\textsc{- Son mauvais goût.}}
{\textsc{- Le roi ne fait rien à Paris, abandonne Saint-Germain,
s'établit à Versailles, veut forcer la nature.}} {\textsc{- Ouvrages de
Maintenon.}} {\textsc{- Marly.}}

~

La cour fut un autre manège de la politique du despotisme. On vient de
voir celle qui divisa, qui humilia, qui confondit les plus grands, celle
qui éleva les ministres au-dessus de tous, en autorité et en puissance
par-dessus les princes du sang, en grandeur même par-dessus les gens de
la première qualité, après avoir totalement changé leur état. Il faut
montrer les progrès en tous genres de la même conduite dressée sur le
même point de vue.

Plusieurs choses contribuèrent à tirer pour toujours la cour hors de
Paris, et à la tenir sans interruption à la campagne. Les troubles de la
minorité, dont cette ville fut le grand théâtre, en avaient imprimé au
roi l'aversion, et la persuasion encore que son séjour y était
dangereux, et que la résidence de la cour ailleurs rendrait à Paris les
cabales moins aisées par la distance des lieux, quelque peu éloignés
qu'ils fussent, et en même temps plus difficiles à cacher par les
absences si aisées à remarquer. Il ne pouvait pardonner à Paris sa
sortie fugitive de cette ville la veille des Rois (1649), ni de l'avoir
rendue, malgré lui, témoin de ses larmes, à la première retraite de
M\textsuperscript{me} de La Vallière. L'embarras des maîtresses, et le
danger de pousser de grands scandales au milieu d'une capitale si
peuplée, et si remplie de tant de différents esprits, n'eut pas peu de
part à l'en éloigner. Il s'y trouvait importuné de la foule du peuple à
chaque fois qu'il sortait, qu'il rentrait, qu'il paraissait dans les
rues\,; il ne l'était pas moins d'une autre sorte de foule de gens de la
ville, et qui n'était pas pour l'aller chercher assidûment plus loin.
Des inquiétudes aussi, qui ne furent pas plutôt aperçues que les plus
familiers de ceux qui étaient commis à sa garde, le vieux Noailles, M.
de Lauzun, et quelques subalternes, firent leur cour de leur vigilance,
et furent accusés de multiplier exprès de faux avis qu'ils se faisaient
donner, pour avoir occasion de se faire valoir et d'avoir plus souvent
des particuliers avec le roi\,; le goût de la promenade et de la chasse,
bien plus commodes à la campagne qu'à Paris, éloigné des forêts et
stérile en lieux de promenades\,; celui des bâtiments qui vint après, et
peu à peu toujours croissant, ne lui en permettait pas l'amusement dans
une ville où il n'aurait pu éviter d'y être continuellement en
spectacle\,; enfin l'idée de se rendre plus vénérable en se dérobant aux
yeux de la multitude, et à l'habitude d'en être vu tous les jours,
toutes ces considérations fixèrent le roi à Saint-Germain bientôt après
la mort de la reine sa mère.

Ce fut là où il commença à attirer le monde par les fêtes et les
galanteries, et à faire sentir qu'il voulait être vu souvent.

L'amour de M\textsuperscript{me} de La Vallière, qui fut d'abord un
mystère, donna lieu à de fréquentes promenades à Versailles, petit
château de cartes alors, bâti par Louis XIII ennuyé, et sa suite encore
plus, d'y avoir souvent couché dans un méchant cabaret à rouliers et
dans un moulin à vent, excédés de ses longues chasses dans la forêt de
Saint-Léger et plus loin encore, loin alors de ces temps réservés à son
fils où les routes, la vitesse des chiens et le nombre gagé des piqueurs
et des chasseurs à cheval a rendu les chasses si aisées et si courtes.
Ce monarque ne couchait jamais ou bien rarement à Versailles qu'une
nuit, et par nécessité\,; le roi son fils pour être plus en particulier
avec sa maîtresse, plaisirs inconnus au juste, au héros, digne fils de
saint Louis, qui bâtit ce petit Versailles.

Les petites parties de Louis XIV y firent naître peu à peu ces bâtiments
immenses qu'il y a faits\,; et leur commodité pour une nombreuse cour,
si différente des logements de Saint-Germain, y transporta tout à fait
sa demeure peu de temps avant la mort de la reine. Il y fit des
logements infinis, qu'on lui faisait sa cour de lui demander, au lieu
qu'à Saint-Germain, presque tout le monde avait l'incommodité d'être à
la ville, et le peu qui était logé au château y était étrangement à
l'étroit.

Les fêtes fréquentes, les promenades particulières à Versailles, les
voyages furent des moyens que le roi saisit pour distinguer et pour
mortifier en nommant les personnes qui à chaque fois en devaient être,
et pour tenir chacun assidu et attentif à lui plaire. Il sentait qu'il
n'avait pas à beaucoup près assez de grâces à répandre pour faire un
effet continuel. Il en substitua donc aux véritables d'idéales, par la
jalousie, les petites préférences qui se trouvaient tous les jours, et
pour ainsi dire, à tous moments, par son art. Les espérances que ces
petites préférences et ces distinctions faisaient naître, et la
considération qui s'en tirait, personne ne fut plus ingénieux que lui à
inventer sans cesse ces sortes de choses. Marly, dans la suite, lui fut
en cela d'un plus grand usage, et Trianon où tout le monde, à la vérité,
pouvait lui aller faire sa cour, mais où les dames avaient l'honneur de
manger avec lui, et où à chaque repas elles étaient choisies\,; le
bougeoir qu'il faisait tenir tous les soirs à son coucher par un
courtisan qu'il voulait distinguer, et toujours entre les plus qualifiés
de ceux qui s'y trouvaient, qu'il nommait tout haut au sortir de sa
prière. Les justaucorps à brevet fut une autre de ces
inventions\footnote{On a conservé le brevet par lequel Louis XIV
  autorisait le grand Condé à porter un de ces justaucorps\,:
  «\,Aujourd'hui, 4 du mois de février 1665, le roi étant à Paris,
  ayant, par son ordonnance du 17 janvier dernier, ordonne que personne
  ne pourrait faire appliquer sur les justaucorps des passements de
  dentelles ou broderies d'or et d'argent, sans avoir la permission
  expresse de Sa Majesté par brevet particulier, Sa Majesté désirant
  gratifier M. le prince de Condé et lui donner des marques
  particulières de sa bienveillance qui le distinguent des autres auprès
  de sa personne et dans sa cour, elle lui a permis et permet de porter
  un justaucorps de couleur bleue garni de galons, passements,
  dentelles, ou broderies d'or et d'argent, en la forme et manière qui
  lui sera prescrite par Sa Majesté sans que, pour raison de ce, il lui
  pusse être imputé d'avoir contrevenu à la susdite ordonnance, de la
  rigueur de laquelle Sa Majesté l'a relevé et dispensé, relève et
  dispense par le présent brevet, lequel, pour témoignage de sa volonté
  elle a signé de sa main et fait contresigner par moi son conseiller
  secrétaire d'État et de ses commandements et finances.\,»Bussy-Rabutin
  se félicite, dans ses Mémoires, à l'année 1662, d'avoir obtenu un
  justaucorps à brevet. «\,Le roi, dit-il, me parut si gracieux en me
  parlant que cela m'obligea de lui demander permission de faire faire
  une casaque bleue\,; ce qu'il m'accorda. Mais pour entendre ce que
  c'était, il faut savoir que Sa Majesté avait fait choix, au
  commencement de cette année, de soixante personnes qui le pourraient
  suivre à tous ses petits voyages de plaisir, sans lui en demander
  permission, et leur avait ordonné de faire faire chacun une casaque de
  moire bleue en broderie d'or et d'argent pareille à la sienne.\,»La
  mode si capricieuse, surtout en France, fit bientôt abandonner le
  justaucorps à brevet. Il devint même ridicule, comme tout ce qui est
  suranné, et lorsque Vardes, qu'on avait jadis admiré comme le modèle
  des courtisans, revint d'exil en 1682, après une absence de près de
  vingt ans, et se présenta devant Louis XIV avec son justaucorps à
  brevet, on se moqua de lui\,: «\,Sire, lui dit Vardes, quand on est
  assez misérable pour être éloigné de vous, non seulement on est
  malheureux, mais on est ridicule.\,» (Lettre de M\textsuperscript{me}
  de Sévigné, en date du 26 mai 1682.)}. Il était bleu doublé de rouge
avec les parements et la veste rouge, brodé d'un dessin magnifique or et
un peu d'argent, particulier à ces habits. Il n'y en avait qu'un nombre,
dont le roi, sa famille, et les princes du sang étaient\,; mais ceux-ci,
comme le reste des courtisans, n'en avaient qu'à mesure qu'il en
vaquait. Les plus distingués de la cour par eux-mêmes ou par la faveur
les demandaient au roi, et c'était une grâce que d'en obtenir. Le
secrétaire d'État ayant la maison du roi en son département en expédiait
un brevet, et nul d'eux n'était à portée d'en avoir. Ils furent imaginés
pour ceux, en très petit nombre, qui avaient la liberté de suivre le roi
aux promenades de Saint-Germain à Versailles sans être nommés, et depuis
que cela cessa, ces habits ont cessé aussi de donner aucun privilège,
excepté celui d'être portés quoiqu'on fût en deuil de cour ou de
famille, pourvu que le deuil ne fût pas grand ou qu'il fût sur ses fins,
et dans les temps encore où il était défendu de porter de l'or et de
l'argent. Je ne l'ai jamais vu porter au roi, à Monseigneur ni à
Monsieur, mais très souvent aux trois fils de Monseigneur et à tous les
autres princes\,; et jusqu'à la mort du roi, dès qu'il en vaquait un,
c'était à qui l'aurait entre les gens de la cour les plus considérables,
et si un jeune seigneur l'obtenait c'était une grande distinction. Les
différentes adresses de cette nature qui se succédèrent les unes aux
autres, à mesure que le roi avança en âge, et que les fêtes changeaient
ou diminuaient, et les attentions qu'il marquait pour avoir toujours une
cour nombreuse, on ne finirait point à les expliquer.

Non seulement il était sensible à la présence continuelle de ce qu'il y
avait de distingué, mais il l'était aussi aux étages inférieurs. Il
regardait à droite et à gauche à son lever, à son coucher, à ses repas,
en passant dans les appartements, dans ses jardins de Versailles, où
seulement les courtisans avaient la liberté de le suivre\,; il voyait et
remarquait tout le monde, aucun ne lui échappait, jusqu'à ceux qui
n'espéraient pas même être vus. Il distinguait très bien en lui-même les
absences de ceux qui étaient toujours à la cour, celles des passagers
qui y venaient plus ou moins souvent\,; les causes générales ou
particulières de ces absences, il les combinait, et ne perdait pas la
plus légère occasion d'agir à leur égard en conséquence. C'était un
démérite aux uns, et à tout ce qu'il y avait de distingué, de ne faire
pas de la cour son séjour ordinaire, aux autres d'y venir rarement, et
une disgrâce sûre pour qui n'y venait jamais, ou comme jamais. Quand il
s'agissait de quelque chose pour eux\,: «\,Je ne le connais point,\,»
répondait-il fièrement. Sur ceux qui se présentaient rarement\,:
«\,C'est un homme que je ne vois jamais\,; » et ces arrêts-là étaient
irrévocables. C'était un autre crime de n'aller point à Fontainebleau,
qu'il regardait comme Versailles, et pour certaines gens de ne demander
pas pour Marly, les uns toujours, les autres souvent, quoique sans
dessein de les y mener, les uns toujours ni les autres souvent\,; mais
si on était sur le pied d'y aller toujours, il fallait une excuse
valable pour s'en dispenser, hommes et femmes de même. Surtout il ne
pouvait souffrir les gens qui se plaisaient à Paris. Il supportait assez
aisément ceux qui aimaient leur campagne, encore y fallait-il être
mesuré ou avoir pris ses précautions avant d'y aller passer un temps un
peu long.

Cela ne se bornait pas aux personnes en charges, ou familières, ou bien
traitées, ni à celles que leur âge ou leur représentation marquait plus
que les autres. La destination seule suffisait dans les gens habitués à
la cour. On a vu sur cela, en son lieu, l'attention qu'eut le roi à un
voyage que je fis à Rouen pour un procès, tout jeune que j'étais, et à
m'y faire écrire de sa part par Pontchartrain pour en savoir la raison.

Louis XIV s'étudiait avec grand soin à être bien informé de ce qui se
passait partout, dans les lieux publics, dans les maisons particulières,
dans le commerce du monde, dans le secret des familles et des maisons.
Les espions et les rapporteurs étaient infinis. Il en avait de toute
espèce\,: plusieurs qui ignoraient que leurs délations allassent jusqu'à
lui, d'autres qui le savaient, quelques-uns qui lui écrivaient
directement en faisant rendre leurs lettres par les voies qu'il leur
avait prescrites, et ces lettres-là n'étaient vues que de lui, et
toujours avant toutes autres choses, quelques autres enfin qui lui
parlaient quelquefois secrètement dans ces cabinets, par les derrières.
Ces voies inconnues rompirent le cou à une infinité de gens de tous
états, sans qu'ils en aient jamais pu découvrir la cause, souvent très
injustement, et le roi une fois prévenu ne revenait jamais, ou si
rarement que rien ne l'était davantage.

Il avait encore un défaut bien dangereux pour les autres, et souvent
pour lui-même par la privation de bons sujets. C'est qu'encore qu'il eut
la mémoire excellente et pour reconnaître un homme du commun qu'il avait
vu une fois, au bout de vingt ans, et pour les choses qu'il avait sues,
et qu'il ne confondait point, il n'était pourtant pas possible qu'il se
souvint de tout, au nombre infini de ce qui chaque jour venait à sa
connaissance. S'il lui était revenu quelque chose de quelqu'un qu'il eût
oublié de la sorte, il lui restait imprimé qu'il y avait quelque chose
contre lui, et c'en était assez pour l'exclure. Il ne cédait point aux
représentations d'un ministre, d'un général, de son confesseur même,
suivant l'espèce de chose ou de gens dont il s'agissait. Il répondait
qu'il ne savait plus ce qui lui en était revenu, mais qu'il était plus
sûr d'en prendre un autre dont il ne lui fût rien revenu du tout.

Ce fut à sa curiosité que les dangereuses fonctions du lieutenant de
police furent redevables de leur établissement. Elles allèrent depuis
toujours croissant. Ces officiers ont tous été sous lui plus craints,
plus ménagés, aussi considérés que les ministres, jusque par les
ministres mêmes, et il n'y avait personne en France, sans en excepter
les princes du sang, qui n'eût intérêt de les ménager, et qui ne le fît.
Outre les rapports sérieux qui lui revenaient par eux, il se
divertissait d'en apprendre toutes les galanteries et toutes les
sottises de Paris. Pontchartrain, qui avait Paris et la cour dans son
département, lui faisait tellement sa cour par cette vole indigne, dont
son père était outré, qu'elle le soutint souvent auprès du roi, et de
l'aveu du roi même, contre de rudes atteintes auxquelles sans cela il
aurait succombé, et on l'a su plus d'une fois par M\textsuperscript{me}
de Maintenon, par M\textsuperscript{me} la duchesse de Bourgogne, par M.
le comte de Toulouse, par les valets intérieurs.

Mais la plus cruelle de toutes les voies par laquelle le roi fut
instruit bien des années, avant qu'on s'en fût aperçu, et par laquelle
l'ignorance et l'imprudence de beaucoup de gens continua toujours encore
de l'instruire, fut celle de l'ouverture des lettres. C'est ce qui donna
tant de crédit aux Pajot et aux Roullier qui en avaient la ferme, qu'on
ne put jamais ôter, ni les faire guère augmenter par cette raison si
longtemps inconnue, et qui s'y enrichirent si énormément tous, aux
dépens du public et du roi même.

On ne saurait comprendre la promptitude et la dextérité de cette
exécution. Le roi voyait l'extrait de toutes les lettres où il y avait
des articles que les chefs de la poste, puis le ministre qui la
gouvernait, jugeaient devoir aller jusqu'à lui, et les lettres entières
quand elles en valaient la peine par leur tissu, ou par la considération
de ceux qui étaient en commerce. Par là les gens principaux de la poste,
maîtres et commis, furent en état de supposer tout ce qu'il leur plut,
et à qui il leur plut\,; et comme peu de chose perdait sans ressource,
ils n'avaient pas besoin de forger ni de suivre une intrigue. Un mot de
mépris sur le roi ou sur le gouvernement, une raillerie, en un mot un
article de lettre spécieux et détaché, noyait sans ressource, sans
perquisition aucune, et ce moyen était continuellement entre leurs
mains. Aussi à vrai et à faux est-il incroyable combien de gens de
toutes les sortes en furent plus ou moins perdus. Le secret était
impénétrable, et jamais rien ne coûta moins au roi que de se taire
profondément, et de dissimuler de même.

Ce dernier talent, il le poussa souvent jusqu'à la fausseté, mais avec
cela jamais de mensonge, et il se piquait de tenir parole. Aussi ne la
donnait-il presque jamais. Pour le secret d'autrui, il le gardait aussi
religieusement que le sien. Il était même flatté de certaines
confessions et de certaines confidences et même confiance\,; et il n'y
avait maîtresse, ministre ni favori qui pût y donner atteinte, quand le
secret les aurait même regardés.

On a su, entre beaucoup d'autres, l'aventure fameuse d'une femme de nom,
lequel a toujours été pleinement ignoré et jusqu'au soupçon même, qui
séparée de lieu depuis un an d'avec son mari, se trouvant grosse et sur
le point de le voir arriver de l'armée, à bout enfin de tous moyens, fit
demander en grâce au roi une audience secrète, dont qui que ce soit ne
put s'apercevoir, pour l'affaire du monde la plus importante. Elle
l'obtint. Elle se confia au roi dans cet extrême besoin, et lui dit que
c'était comme au plus honnête homme de son royaume. Le roi lui conseilla
de profiter d'une si grande détresse pour vivre plus sagement à
l'avenir, et lui promit de retenir sur-le-champ son mari sur la
frontière, sous prétexte de son service, tant et si longtemps qu'il ne
pût avoir aucun soupçon, et de ne le laisser revenir sous aucun
prétexte. En effet, il en donna l'ordre le jour même à Louvois, et lui
défendit non seulement tout congé, mais de souffrir qu'il s'absentât un
seul jour du poste qu'il lui assignait pour y commander tout l'hiver.
L'officier, qui était distingué, et qui n'avait rien moins que souhaité,
encore moins demandé, d'être employé l'hiver sur la frontière, et
Louvois qui y avait aussi peu pensé, furent également surpris et fâchés.
Il n'en fallut pas moins obéir à la lettre et sans demander pourquoi, et
le roi n'en a fait l'histoire que bien des années après et que lorsqu'il
fut bien sûr que les gens que cela regardait ne se pouvaient plus
démêler, comme en effet ils n'ont jamais pu l'être, pas même du soupçon
le plus vague ni le plus incertain.

Jamais personne ne donna de meilleure grâce, et n'augmenta tant par là
le prix de ses bienfaits. Jamais personne ne vendit mieux ses paroles,
son souris même, jusqu'à ses regards. Il rendit tout précieux par le
choix et la majesté, à qui la rareté et la brèveté\footnote{Il y a dans
  le manuscrit de Saint-Simon \emph{brèveté} et non \emph{brièveté},
  comme l'ont imprimé, les précédents éditeurs.} de ses paroles ajoutait
beaucoup. S'il les adressait à quelqu'un, ou de question, ou de choses
indifférentes, toute l'assistance le regardait\,; c'était une
distinction dont on s'entretenait et qui rendit toujours une sorte de
considération. Il en était de même de toutes les attentions et les
distinctions, et des préférences, qu'il donnait dans leurs proportions.
Jamais il ne lui échappa de dire rien de désobligeant à personne\,; et
s'il avait à reprendre, à réprimander ou à corriger, ce qui était fort
rare, c'était toujours avec un air plus ou moins de bonté, presque
jamais avec sécheresse, jamais avec colère, si on excepte l'unique
aventure de Courtenvaux, qui a été racontée en son lieu, quoiqu'il ne
fût pas exempt de colère\,; quelquefois avec un air de sévérité.

Jamais homme si naturellement poli, ni d'une politesse si fort mesurée,
si fort par degrés, ni qui distinguât mieux l'âge, le mérite, le rang,
et dans ses réponses quand elles passaient le «\,Je verrai,\,» et dans
ses manières. Ces étages divers se marquaient exactement dans sa manière
de saluer et de recevoir les révérences, lorsqu'on partait ou qu'on
arrivait. Il était admirable à recevoir différemment les saluts à la
tête des lignes à l'armée ou aux revues. Mais surtout pour les femmes
rien n'était pareil. Jamais il n'a passé devant la moindre coiffe sans
soulever son chapeau, je dis aux femmes de chambre, et qu'il connaissait
pour telles, comme cela arrivait souvent à Marly. Aux dames, il ôtait
son chapeau tout à fait, mais de plus ou moins loin\,; aux gens titrés,
à demi, et le tenait en l'air ou à son oreille quelques instants plus ou
moins marqués. Aux seigneurs, mais qui l'étaient, il se contentait de
mettre la main au chapeau. Il l'ôtait comme aux dames pour les princes
du sang. S'il abordait des dames, il ne se couvrait qu'après les avoir
quittées. Tout cela n'était que dehors, car dans la maison il n'était
jamais couvert. Ses révérences, plus ou moins marquées, mais toujours
légères, avaient une grâce et une majesté incomparables, jusqu'à sa
manière de se soulever à demi à son souper pour chaque dame assise qui
arrivait, non pour aucune autre, ni pour les princes du sang\,; mais sur
les fins cela le fatiguait, quoiqu'il ne l'ait jamais cessé, et les
dames assises évitaient d'entrer à son souper quand il était commencé.
C'était encore avec la même distinction qu'il recevait le service de
Monsieur, et de M. le duc d'Orléans, des princes du sang\,; à ces
derniers, il ne faisait que marquer, à Monseigneur de même, et à Mgrs
ses fils par familiarité\,; des grands officiers, avec un air de bonté
et d'attention.

Si on lui faisait attendre quelque chose à son habiller, c'était
toujours avec patience. Exact aux heures qu'il donnait pour toute sa
journée\,; une précision nette et courte dans ses ordres. Si dans les
vilains temps d'hiver qu'il ne pouvait aller dehors, qu'il passait chez
M\textsuperscript{me} de Maintenon un quart d'heure plus tôt qu'il n'en
avait donné l'ordre, ce qui ne lui arrivait guère, et que le capitaine
des gardes en quartier ne s'y trouvât pas, il ne manquait point de lui
dire après que c'était sa faute à lui d'avoir prévenu l'heure, non celle
des capitaines des gardes de l'avoir manquée. Aussi, avec cette règle,
qui ne manquait jamais, était-il servi avec la dernière exactitude, et
elle était d'une commodité infinie pour les courtisans

Il traitait bien ses valets, surtout les intérieurs. C'était parmi eux
qu'il se sentait le plus à son aise, et qu'il se communiquait le plus
familièrement, surtout aux principaux. Leur amitié et leur aversion a
souvent eu de grands effets. Ils étaient sans cesse à portée de rendre
de bons et de mauvais offices\,; aussi faisaient-ils souvenir de ces
puissants affranchis des empereurs romains, à qui le sénat et les grands
de l'empire faisaient leur cour, et ployaient sous eux avec bassesse.
Ceux-ci, dans tout ce règne, ne furent ni moins comptés ni moins
courtisés. Les ministres même les plus puissants les ménageaient
ouvertement\,; et les princes du sang, jusqu'aux bâtards, sans parler de
tout ce qui est inférieur, en usaient de même. Les charges des premiers
gentilshommes de la chambre furent plus qu'obscurcies par les premiers
valets de chambre, et les grandes charges ne se soutinrent que dans la
mesure que les valets de leur dépendance ou les petits officiers très
subalternes approchaient nécessairement plus ou moins du roi.
L'insolence aussi était grande dans la plupart d'eux, et telle qu'il
fallait savoir l'éviter, ou la supporter avec patience.

Le roi les soutenait tous, et il racontait quelquefois avec complaisance
qu'ayant dans sa jeunesse envoyé, pour je ne sais quoi, une lettre au
duc de Montbazon, gouverneur de Paris, qui était en une de ses maisons
de campagne près de cette ville, par un de ses valets de pied, il y
arriva comme M. de Montbazon allait se mettre à table, qu'il avait forcé
ce valet de pied de s'y mettre avec lui, et le conduisit, lorsqu'il le
renvoya, jusque dans la cour, parce qu'il était venu de la part du
roi\footnote{Cette anecdote se trouve déjà plus haut.}.

Il ne manquait guère aussi de demander à ses gentilshommes ordinaires,
quand ils revenaient de sa part de faire des compliments de
conjouissance ou de condoléances aux gens titrés, hommes et femmes, mais
à nuls autres, comment ils avaient été reçus\,; et il aurait trouvé bien
mauvais qu'on ne les eût pas fait asseoir, et conduits fort loin, les
hommes au carrosse.

Rien n'était pareil à lui aux revues, aux fêtes, et partout où un air de
galanterie pouvait avoir lieu par la présence des dames. On l'a déjà
dit, il l'avait puisée à la cour de la reine sa mère, et chez la
comtesse de Soissons\,; la compagnie de ses maîtresses l'y avait
accoutumé de plus en plus\,; mais toujours majestueuse, quoique
quelquefois avec de la gaieté, et jamais devant le monde rien de déplacé
ni de hasardé\,; mais jusqu'au moindre geste, son marcher, son port,
toute sa contenance, tout mesuré, tout décent, noble, grand, majestueux,
et toutefois très naturel, à quoi l'habitude et l'avantage incomparable
et unique de toute sa figure donnait une grande facilité. Aussi, dans
les choses sérieuses, les audiences d'ambassadeurs, les cérémonies,
jamais homme n'a tant imposé\,; et il fallait commencer par s'accoutumer
à le voir, si en le haranguant on ne voulait s'exposer à demeurer court.
Ses réponses en ces occasions étaient toujours courtes, justes, pleines
et très rarement sans quelque chose d'obligeant, quelquefois même de
flatteur, quand le discours le méritait. Le respect aussi qu'apportait
sa présence en quelque lieu qu'il fût imposait un silence et jusqu'à une
sorte de frayeur.

Il aimait fort l'air et les exercices, tant qu'il en put faire. Il avait
excellé à la danse, au mail, à la paume. Il était encore admirable à
cheval à son âge. Il aimait à voir faire toutes ces choses avec grâce et
adresse. S'en bien ou mal acquitter devant lui était mérite ou démérite.
Il disait que de ces choses qui n'étaient point nécessaires, il ne s'en
fallait pas mêler, si on ne les faisait pas bien. Il aimait fort à
tirer, et il n'y avait point de si bon tireur que lui, ni avec tant de
grâces. Il voulait des chiennes couchantes excellentes\,; il en avait
toujours sept ou huit dans ses cabinets, et se plaisait à leur donner
lui-même à manger pour s'en faire connaître. Il aimait fort aussi à
courre le cerf, mais en calèche, depuis qu'il s'était cassé le bras en
courant à Fontainebleau, aussitôt après la mort de la reine. Il était
seul dans une manière de soufflet, tiré par quatre petits chevaux, à
cinq ou six relais, et il menait lui-même à toute bride, avec une
adresse et une justesse que n'avaient pas les meilleurs cochers, et
toujours la même grâce à tout ce qu'il faisait. Ses postillons étaient
des enfants depuis neuf ou dix ans jusqu'à quinze, et il les dirigeait.

Il aima en tout la splendeur, la magnificence, la profusion. Ce goût il
le tourna en maxime par politique, et l'inspira en tout à sa cour.
C'était lui plaire que de s'y jeter en tables, en habits, en équipages,
en bâtiments, en jeu. C'étaient des occasions pour qu'il parlât aux
gens. Le fond était qu'il tendait et parvint par là à épuiser tout le
monde en mettant le luxe en honneur, et pour certaines parties en
nécessité, et réduisit ainsi peu à peu tout le monde à dépendre
entièrement de ses bienfaits pour subsister. Il y trouvait encore la
satisfaction de son orgueil par une cour superbe en tout, et par une
plus grande confusion qui anéantissait de plus en plus les distinctions
naturelles.

C'est une plaie qui, une fois introduite, est devenue le cancer
intérieur qui ronge tous les particuliers, parce que de la cour il s'est
promptement communiqué à Paris et dans les provinces et les armées, où
les gens en quelque place ne sont comptés qu'à proportion de leur table
et de leur magnificence, depuis cette malheureuse introduction qui ronge
tous les particuliers, qui force ceux d'un état à pouvoir voler, à ne
s'y pas épargner pour la plupart, dans la nécessité de soutenir leur
dépense, et par la confusion des États, que l'orgueil, que jusqu'à la
bienséance entretiennent, qui par la folie du gros va toujours en
augmentant, dont les suites sont infinies, et ne vont à rien moins qu'à
la ruine et au renversement général.

Rien, jusqu'à lui, n'a jamais approché du nombre et de la magnificence
de ses équipages de chasse et de toutes ses autres sortes d'équipages.
Ses bâtiments, qui les pourrait nombrer\,? En même temps, qui n'en
déplorera pas l'orgueil, le caprice, le mauvais goût\,? Il abandonna
Saint-Germain, et ne fit jamais à Paris ni ornement ni commodité, que le
pont Royal, par pure nécessité, en quoi, avec son incomparable étendue,
elle est si inférieure à tant de villes dans toutes les parties de
l'Europe.

Lorsqu'on fit la place de Vendôme, elle était carrée. M. de Louvois en
vit les quatre parements bâtis. Son dessein était d'y placer la
bibliothèque du roi, les médailles, le balancier, toutes les académies,
et le grand conseil qui tient ses séances encore dans une maison qu'il
loue. Le premier soin du roi, le jour de la mort de Louvois, fut
d'arrêter ce travail, et de donner ses ordres pour faire couper à pans
les angles de la place, en la diminuant d'autant, de n'y placer rien de
ce qui y était destiné, et de n'y faire que des maisons, ainsi qu'on la
voit.

Saint-Germain, lieu unique pour rassembler les merveilles de la vue,
l'immense plain-pied d'une forêt toute joignante, unique encore par la
beauté de ses arbres, de son terrain, de sa situation, l'avantage et la
facilité des eaux de source sur cette élévation, les agréments
admirables des jardins, des hauteurs et des terrasses, qui les unes sur
les autres se pouvaient si aisément conduire dans toute l'étendue qu'on
aurait voulu, les charmes et les commodités de la Seine, enfin, une
ville toute faite et que sa position entretenait par elle-même, il
l'abandonna pour Versailles, le plus triste et le plus ingrat de tous
les lieux, sans vue, sans bois, sans eau, sans terre, parce que tout y
est sable mouvant ou marécage, sans air par conséquent qui n'y peut être
bon.

Il se plut à tyranniser la nature, à la dompter à force d'art et de
trésors. Il y bâtit tout l'un après l'autre, sans dessin général\,; le
beau et le vilain furent cousus ensemble, le vaste et l'étranglé. Son
appartement et celui de la reine y ont les dernières incommodités, avec
les vues de cabinets et de tout ce qui est derrière les plus obscures,
les plus enfermées, les plus puantes. Les jardins dont la magnificence
étonne\,; mais dont le plus léger usage rebute, sont d'aussi mauvais
goût. On n'y est conduit dans la fraîcheur de l'ombre que par une vaste
zone torride, au bout de laquelle il n'y a plus, où que ce soit, qu'à
monter et à descendre\,; et avec la colline, qui est fort courte, se
terminent les jardins. La recoupe y brûle les pieds, mais sans cette
recoupe on y enfoncerait ici dans les sables, et là dans la plus noire
fange. La violence qui y a été faite partout à la nature repousse et
dégoûte, malgré soi. L'abondance des eaux forcées et ramassées de toutes
parts les rend vertes, épaisses, bourbeuses\,; elles répandent une
humidité malsaine et sensible, une odeur qui l'est encore plus. Leurs
effets, qu'il faut pourtant beaucoup ménager, sont incomparables\,; mais
de ce tout, il résulte qu'on admire et qu'on fuit. Du côté de la cour,
l'étranglé suffoque, et ces vastes ailes s'enfuient sans tenir à rien.
Du côté des jardins, on jouit de la beauté du tout ensemble, mais on
croit voir un palais qui a été brûlé, où le dernier étage et les toits
manquent encore. La chapelle qui l'écrase, parce que Mansart voulait
engager le roi à élever le tout d'un étage, a de partout la triste
représentation d'un immense catafalque. La main-d'œuvre y est exquise en
tous genres, l'ordonnance nulle\,; tout y été fait pour la tribune,
parce que le roi n'allait guère en bas, et celles des côtés sont
inaccessibles, par l'unique défilé qui conduit à chacune. On ne finirait
point sur les défauts monstrueux d'un palais si immense, et si
immensément cher\footnote{Voy. notes à la fin du volume.}, avec ses
accompagnements qui le sont encore davantage.

Orangerie, potagers, chenils, grande et petite écuries pareilles, commun
prodigieux\,; enfin une ville entière où il n'y avait qu'un très
misérable cabaret, un moulin à vent, et ce petit château de cartes que
Louis XIII y avait fait pour n'y plus coucher sur la paille, qui n'était
que la contenance étroite et basse autour de la cour de martre, qui en
faisait la cour, et dont le bâtiment du fond n'avait que deux courtes et
petites ailes. Mon père l'a vu et y a couché maintes fois. Encore ce
Versailles de Louis XIV, ce chef-d'oeuvre si ruineux et de si mauvais
goût, et où les changements entiers des bassins et de bosquets ont
enterré tant d'or qui ne peut paraître, n'a-t-il pu être achevé.

Parmi tant de salons entassés l'un sur l'autre, il n'y a ni salle de
comédie, ni salle de banquet, ni de bal\,; et devant et derrière il
reste beaucoup à faire. Les parcs et les avenues, tous en plants, ne
peuvent venir. En gibier, il faut y en jeter sans cesse\,; en rigoles de
quatre et cinq lieues de cours, elles sont sans nombre\,; en murailles
enfin qui, par leur immense contour, enferment comme une petite province
du plus triste et du plus vilain pays du monde.

Trianon, dans ce même parc, et à la porte de Versailles, d'abord maison
de porcelaine à aller faire des collations, agrandie après pour y
pouvoir coucher, enfin palais de marbre, de jaspe et de porphyre avec
des jardins délicieux\,; la ménagerie vis-à-vis, de l'autre côté de la
croisée du canal de Versailles, toute de riens exquis, et garnie de
toutes sortes d'espèces de bêtes à deux et à quatre pieds les plus
rares\,; enfin Clagny, bâti pour M\textsuperscript{me} de Montespan en
son propre, passé au duc du Maine, au bout de Versailles, château
superbe avec ses eaux, ses jardins, son parc\,; des aqueducs dignes des
Romains de tous les côtés, l'Asie ni l'antiquité n'offrent rien de si
vaste, de si multiplié, de si travaillé, de si superbe, de si rempli de
monuments les plus rares de tous les siècles, en marbres les plus exquis
de toutes les sortes, en bronzes, en peintures, en sculptures, ni de si
achevé des derniers.

Mais l'eau manquait quoi qu'on pût faire, et ces merveilles de l'art en
fontaines tarissaient, comme elles font encore à tous moments, malgré la
prévoyance de ces mers de réservoirs qui avaient coûté tant de millions
à établir et à conduire sur le sable mouvant et sur la fange. Qui
l'aurait cru\,? ce défaut devint la ruine de l'infanterie.
M\textsuperscript{me} de Maintenon régnait, on parlera d'elle à son
tour. M. de Louvois alors était bien avec elle, on jouissait de la paix.
Il imagina de détourner la rivière d'Eure, entre Chartres et Maintenon,
et de la faire venir, tout entière à Versailles. Qui pourra dire l'or et
les hommes que la tentative obstinée en coûta pendant plusieurs années,
jusque-là qu'il fut défendu, sous les plus grandes peines, dans le camp
qu'on y avait établi et qu'on y tint très longtemps, d'y parler des
malades, surtout des morts, que le rude travail et plus encore
l'exhalaison de tant de terres remuées tuaient\,? combien d'autres
furent des années à se rétablir de cette contagion\,! combien n'en ont
pu reprendre leur santé pendant le reste de leur vie\,! Et toutefois non
seulement les officiers particuliers, mais les colonels, les brigadiers,
et ce qu'on y employa d'officiers généraux, n'avaient pas, quels qu'ils
fussent, la liberté de s'en absenter un quart d'heure, ni de manquer
eux-mêmes un quart d'heure de service sur les travaux. La guerre enfin
les interrompit en 1688, sans qu'ils aient été repris depuis\,; il n'en
est resté que d'informes monuments qui éterniseront cette cruelle folie.

À la fin, le roi, lassé du beau et de la foule, se persuada qu'il
voulait quelquefois du petit et de la solitude. Il chercha autour de
Versailles de quoi satisfaire ce nouveau goût. Il visita plusieurs
endroits, il parcourut les coteaux qui découvrent Saint-Germain et cette
vaste plaine qui est au bas, où la Seine serpente et arrose tant de gros
lieux et de richesses en quittant Paris. On le pressa de s'arrêter à
Lucienne, où Cavoye eut depuis une maison dont la vue est enchantée,
mais il répondit que cette heureuse situation le ruinerait, et que,
comme il voulait un rien, il voulait aussi une situation qui ne lui
permit pas de songer à y rien faire.

Il trouva derrière Lucienne un vallon étroit, profond, à bords escarpés,
inaccessible par ses marécages, sans aucune vue, enfermé de collines de
toutes parts, extrêmement à l'étroit, avec un méchant village sur le
penchant d'une de ces collines qui s'appelait Marly. Cette clôture sans
vue, ni moyen d'en avoir, fit tout son mérite. L'étroit du vallon où on
ne se pouvait étendre y en ajouta beaucoup. Il crut choisir un ministre,
un favori, un général d'armée. Ce fut un grand travail que dessécher ce
cloaque de tous les environs qui y jetaient toutes leurs voiries, et d'y
apporter des terres. L'ermitage fut fait. Ce n'était que pour y coucher
trois nuits, du mercredi au samedi, deux ou trois fois l'année, avec une
douzaine au plus de courtisans en charges les plus indispensables.

Peu à peu l'ermitage fut augmenté\,; d'accroissement en accroissement
les collines taillées pour faire place et y bâtir, et celle du bout
largement emportée pour donner au moins une échappée de vue fort
imparfaite. Enfin, en bâtiments, en jardins, en eaux, en aqueducs, en ce
qui est si connu et si curieux sous le nom de machine de Marly, en parc,
en forêt ornée et renfermée, en statues, en meubles précieux, Marly est
devenu ce qu'on le voit encore\,; tout dépouillé qu'il est depuis la
mort du roi. En forêts toutes venues, et touffues qu'on y a apportées en
grands arbres de Compiègne, et de bien plus loin sans cesse, dont plus
des trois quarts mouraient, et qu'on remplaçait aussitôt\,; en vastes
espaces de bois épais et d'allées obscures, subitement changées en
immenses pièces d'eau où on se promenait en gondoles, puis remises en
forêts à n'y pas voir le jour dès le moment qu'on les plantait, je parle
de ce que j'ai vu en six semaines\,; en bassins changés cent fois\,; en
cascades de même à figures successives et toutes différentes\,; en
séjours de carpes, ornés de dorures et de peintures les plus exquises, à
peine achevées, rechangées et rétablies autrement par les mêmes maîtres,
et cela une infinité de fois\,; cette prodigieuse machine, dont on vient
de parler, avec ses immenses aqueducs, ses conduites et ses réservoirs
monstrueux, uniquement consacrée à Marly sans plus porter d'eau à
Versailles\,; c'est peu de dire que Versailles tel qu'on l'a vu n'a pas
coûté Marly\footnote{On trouvera dans les notes à la fin du volume le
  chiffre exact des dépenses auxquelles s'élevèrent les constructions de
  Versailles et de Marly jusqu'en 1690.}.

Que si on y ajoute les dépenses de ces continuels voyages, qui devinrent
enfin au moins égaux aux séjours de Versailles, souvent presque aussi
nombreux, et tout à la fin de la vie du roi le séjour le plus ordinaire,
on ne dira point trop sur Marly seul en comptant par milliards.

Telle fut la fortune d'un repaire de serpents et de charognes, de
crapauds et de grenouilles, uniquement choisi pour n'y pouvoir dépenser.
Tel fut le mauvais goût du roi en toutes choses, et ce plaisir superbe
de forcer la nature, que ni la guerre la plus pesante, ni la dévotion ne
put émousser.

\hypertarget{note-i.-le-cardinal-de-bouillon.}{%
\chapter{NOTE I. LE CARDINAL DE
BOUILLON.}\label{note-i.-le-cardinal-de-bouillon.}}

Il ne sera pas inutile de rapprocher de ce passage, où Saint-Simon
résume (p. 22 et suiv. de ce volume) la vie du cardinal de Bouillon, un
ouvrage intitulé\,: \emph{Apologie de M. le cardinal de Bouillon, écrite
par lui-même}. «\,C'était l'abbé d'Anfreville, dit le président
Hénault\footnote{\emph{Mémoires}, p.~17 et 18.}, qui en était l'auteur.
La lecture en est curieuse et infiniment agréable.\,»

On trouve aussi, dans les \emph{Nouveaux portraits des personnes qui
composent la cour de France} (1706), un morceau où le cardinal de
Bouillon est présenté sous les traits les plus favorables. Je
transcrirai ici ce portrait, qui est assez court\,: «\,Ce prélat était
né pour être plus heureux, ayant autant de mérite que de naissance. Il a
brillé dans le monde par ses belles qualités, et encore plus par ses
disgrâces. Son dernier malheur fait sa gloire\,: bon parent, il s'est
sacrifié pour sa famille\,; excellent sujet, il a paru aussi sensible à
la colère de son prince qu'il a de zèle pour son service\,; et, tout
utile, tout important qu'il était à l'État, il a mieux aimé s'exiler que
de désobéir à son roi. Il s'est fait estimer des grands et s'est fait
adorer des petits. Il doit une partie de son élévation à un
oncle\footnote{Le maréchal de Turenne.} , dont le nom et le mérite
seront immortels\,; mais il ne doit qu'à soi-même sa réputation. Tous
ceux qui connaissent Son Éminence la plaignent, et il n'y a pas un
honnête homme qui ne lui souhaite autant de bonheur qu'il y a de travers
et de bizarrerie dans son sort.\,»

\hypertarget{note-ii.-le-garde-des-sceaux-chauvelin.}{%
\chapter{NOTE II. LE GARDE DES SCEAUX
CHAUVELIN.}\label{note-ii.-le-garde-des-sceaux-chauvelin.}}

Saint-Simon parle plusieurs fois, et notamment p.~84 de ce volume, de
Chauvelin, qui devint garde des sceaux et joua un grand rôle sous Louis
XIV. J'ai réuni dans cette note quelques renseignements sur ce
personnage.

Germain-Louis Chauvelin, né le 26 mars 1685 conseiller au grand conseil
le Ier décembre 1706, maître des requêtes le 31 mai 1711, avocat général
au parlement de Paris en 1715, président à mortier le 5 décembre 1718,
garde des sceaux de France le 17 août 1727, puis ministre et secrétaire
d'État des affaires étrangères, fut disgracié le 20 février 1737, et
exilé à Bourges. Il était seigneur de Grisenoy ou Grisenois en Brie. Il
est probable que ce nom se prononçait alors \emph{Grisenoire}, car
Saint-Simon (p.~85) et l'auteur que nous allons citer appellent tous
deux Chauvelin \emph{Grisenoire}.

Le marquis d'Argenson, qui avait connu le garde des sceaux Chauvelin, en
a tracé le portrait suivant dans ses Mémoires autographes et inédits\,:

~

{\textsc{«\,Novembre 1730.}} {\textsc{- Pour définir le garde des sceaux
Chauvelin, vous saurez qu'il n'y a jamais eu au monde plus habile homme
pour ses propres affaires, pour les travailler en grand, pour faire une
grande et belle fortune, pour y aller par des moyens plus sûrs\,; mais
il est en toutes choses le centre de son cercle, sa fin dernière,
l'objet final de toutes ses méditations. S'il lui restait un peu de
volonté pour ses charges, cela irait\,; mais jamais cela ne sera. Il
faut que son esprit soit fort juste, mais peu élevé, vu la fin à
laquelle il s'est borné pour faire usage de tant de moyens. Il me prend
envie de parler plus à fond de lui et de sa fortune.}}

~

«\,Il haïssait beaucoup son frère aîné, qui avait un mérite si brillant
qu'on en était ébloui\footnote{Voy. ci-dessus, t. XI, p.~18.}. Partie de
cette haine, partie d'une saine politique, il embrassa le parti
contraire aux jésuites\footnote{Saint-Simon dit, en effet, que le frère
  aîné était favorable aux jésuites.}, pour se trouver sur ses pieds si
malheur arrivait aux intrigues de son frère. Ce frère mourut\,; notre
cadet se rendit grand travailleur, quitta les belles-lettres, les bons
airs, les chevaux\,; car il prétendait à bonnes fortunes et dansait
bien\,; il était le beau Grisenoire. Il est vrai que ses voluptés se
concentraient dans la fortune\,: il eut les bonnes grâces de
M\textsuperscript{me} de Bercy, parce qu'elle était fille de M.
Desmarets\footnote{Contrôleur général des finances dont Saint-Simon
  parle souvent.}.

«\,Il fit la charge d'avocat général plus en homme qui veut cheminer
qu'en homme qui veut passer pour un grand avocat général, et ainsi
a-t-il suivi dans ses charges. Corneille dit\,:

Mais quand le potentat se laisse gouverner,

Et que de son pouvoir les grands dépositaires

N'ont pour raison d'État que leurs propres affaires\footnote{Corneille,
  \emph{Othon}, act. I, sc I\,; édit. Lahure t. III, page 397. --- C'est
  Othon qui parle\,: Quand le monarque agit par sa propre conduite, /
  Mes pareils sans péril se rangent à sa suite. / Le mérite et le sang
  nous y font discerner\,; / Mais quand le potentat se laisse gouverner,
  / Et que de son pouvoir les grands depositaires / N'ont pour raison
  d'État que leurs propres affaires, / Ces lâches ennemis de tous les
  gens de coeur / Cherchent à nous pousser avec toute rigueur, etc.},\,»
etc.

«\,Il épousa une héritière. M\textsuperscript{lle} des Montées, grande
et bien faite, avait eu des affaires\,; elle s'était entêtée d'un
officier sans bien, {[}et{]} voulait l'épouser légitimement\,; elle
était galante. Son père était négociant d'Orléans\,; il y faisait bon.
Ce Grisenoire intrigue obscurément, l'épouse. Il l'a tendue
extérieurement si exemplaire qu'elle est aimée et admirée de la cour. Il
s'est appliqué à la réformer, y a mis tout son temps\,; il ne la
quittait pas d'un pas, étant chargé d'affaires, la veillait dans la
maison où elle soupait. Il la suit encore, lui a ôté le rouge, en sorte
qu'elle n'en a plus qu'au bout du nez. Il change ses femmes, ses valets
de chambre, se fait rendre compte un beau matin des hardes de madame par
sa femme de chambre\,; elle est chassée avant le réveil de madame\,; il
interrompt une affaire d'État pour cela\,; c'est merveille.

«\,Or le garde des sceaux doit être, de cela, ou un très médiocre génie,
ou un très grand, mais qui embrasse des choses bien petites, puisque
cela le jette dans de telles pauvretés\,; et ne doutez pas qu'il ne soit
petit dans cette détermination\,; car n'en vouloir qu'à son propre bien
si grossièrement, c'est aller contre son propre bien. Il ne se fera
grand qu'à la financière.

«\,Il devint président à mortier\footnote{Le 5 décembre 1718.} par la
plus belle intrigue de blanchisseuse et du Pont-aux-Choux qu'on ait
jamais suivie. M. de Bailleul s'ennuyait autant de sa charge qu'elle
s'ennuyait de lui\,; il fallait pourtant le déterminer à vendre. Ce
président-là était tombé dans une telle crapule et obscurité qu'il ne
vivait que parmi des blanchisseuses et joueurs de boule. M. Chauvelin
gagna ces puissances et eut la charge à bon marché. Tirons le rideau,
faute de le savoir, sur le moyen dont se servent ces messieurs du
parlement pour se rendre si utiles à la cour\,; celui qui s'y rend le
plus agréable ne peut éviter de vendre sa compagnie, de l'espionner,
etc. Il est sûr que notre héros tira grand parti de sa charge pour avoir
bien du crédit à la cour. Il fit les affaires des grands seigneurs\,; il
s'adonna à MM. de Beringhen\footnote{Saint-Simon parle très souvent de
  cette famille. À cette époque, Jacques-Louis de Beringhen était
  premier écuyer de Louis XV et avait plusieurs frères, dont l'un était
  chevalier de Malte.}, dont il était un peu parent. {[}A la mort{]} de
M. le duc d'Aumont, son parent par le même endroit, qui est par les
Louvois, il fut tuteur du petit duc d'Aumont. Il rangea à merveille ses
affaires délabrées\,; il est habile économe. Par les Beringhen, il eut
le maréchal d'Huxelles, qui aimait les beaux garçons\footnote{Voy.
  ci-dessus, t. IV, p. 92, 93.}.

«\,Il voulait parvenir sous le régent. Ce prince disait que tout lui
parlait \emph{Chauvelin\,;} les pierres mêmes lui répétaient ce nom
ennuyeux pour lui. Il apportait tous ces grands seigneurs et leurs
créatures pour lui en dire du bien et le demander pour ministre. Il lui
fallait encore plus de bien qu'il n'en avait\,: il agiota\,; son garçon
agioteur fut des Bonnelles\footnote{On trouve en 1720 un maître des
  requêtes, nommé André Bonnel, qui est probablement le même que celui
  dont il s'agit ici.}, maître des requêtes, et depuis à la Bastille\,;
et il a renié ce pauvre fripon dès qu'il a pu le servir. Il a paru dans
ses places crasseux et honorable, plaçant assez bien sa dépense pour
être comme tout le monde, et faisant passer pour modération ce que la
lésine lui fait se refuser. Il affecte un air de bon et ancien magistrat
de race, surtout en ne découchant jamais d'avec sa femme\,; il trouve
que cela sied bien. Il se vante sans doute beaucoup à ses maîtres de
n'avoir pas de maîtresse, quoique toujours beaucoup plus vigoureux qu'un
autre\,; car personne n'est plus adroit que lui à tout exercice, à faire
des armes, à la chasse, à monter à cheval, à jouer à l'hombre, à
chanter, à plaire aux dames et à les servir. C'est un Candale\footnote{Le
  duc de Candale, fils du duc d'Épernon, avait été un des seigneurs les
  plus célèbres par son élégance et ses succès auprès des femmes.} et un
Soyecourt\footnote{Le marquis de Soyecourt était renommé par un genre de
  mérite que nous font connaître les chansons du recueil de Maurepas. Il
  suffira d'y renvoyer les amateurs de couplets cyniques.}\,; et, à son
dire, tant de talents il les enfouit pour ne servir que l'État et
reconnaître les bienfaits de son maître, M. le cardinal de Fleury.

«\,Le régent mourut sans qu'il y eût rien à faire pour lui. Le temps de
M. le Duc lui parut un feu de paille. Il s'attacha à M. de Fréjus,
depuis cardinal de Fleury\,; et voyant qu'il fallait s'y attacher, il ne
s'y adonna pas médiocrement. Ce cardinal, vieux et rempli de l'esprit
des femmes, est jaloux au scrupule des attachements qu'on lui marque\,;
une bagatelle peut faire tout échouer. Le maréchal d'Huxelles le
produisit beaucoup, et le produit fit le reste à merveille. Ce vieux
b\ldots\ldots{} de maréchal obtint permission du cardinal d'en faire un
homme d'État et un personnage\,; il lui apprit les secrets de l'État, le
mit au fait de la situation présente des affaires étrangères\footnote{Le
  maréchal d'Huxelles avait été président du conseil des affaires
  étrangères sous la régence.}.

«\,M. Chauvelin avait un avantagé dont on tira en cette occasion un
parti extrême. Peu M. de Harlay\footnote{Saint-Simon a longuement parle
  de ce magistrat, t. Ier, p.~142, 143.} lui avait légué ses nombreux et
précieux manuscrits sur le droit public. Le Chauvelin en fit des tables
en les mettant en ordre\,: cela s'arrangeait sur de petites cartes de la
plus jolie façon. Il y employait tous ses amis\,; l'abbé de Laubruyère y
travailla beaucoup, et en a eu l'évêché de Soissons. M. Chauvelin est
effectivement grand travailleur par goût, et d'une assiduité
surprenante\,; il travaillait autant avant d'être en place que depuis
qu'il y est. Dès qu'il avait dîné, il regagnait le cabinet, et y restait
jusqu'à ce qu'on l'avertît qu'on eût servi le dessert chez sa femme\,;
et il ne soupe pas depuis longtemps\,; ce qui est encore une petite
chose qui suit le grand homme. Remarquez ce que c'est que de ressembler
aux grands hommes par les petites choses.

«\,Il résulta de toutes ces cartes écrites au dos un gros livre de table
universelle du droit public. On publiait que le président Chauvelin ne
travaillait qu'au droit public\,; il n'était pas à sa chaise percée
qu'on ne dit d'abord qu'il travaillait à ce droit\,; cela faisait frémir
sur l'engagement de se faire président à mortier, d'être de ces gens qui
veillent tant pour nous tandis que nous dormons. On fit accroire au
vieux cardinal que M. Chauvelin avait tout appris dans ses cartes\,; et
en effet il avait appris dans ce bureau typographique \emph{summa rerum
capita}, et assez pour ne paraître pas neuf à un ignorant, cachant avec
adresse ce qu'il ignorait. Le cardinal conçut une forte résolution de
mettre un tel homme en place, et de signaler son ministère en donnant au
roi un bras droit si nerveux. On arrangea cette affaire-là\,; on
déposséda les Fleuriau\footnote{Fleuriau d'Armenonville et Fleuriau de
  Morville (le père et le fils), furent successivement secrétaires
  d'État pendant la régence et les premières années qui la suivirent.}
dans le temps que le ministère de M. de Morville commençait à aller un
peu passablement\,; on fit revenir M. le chancelier\footnote{Le
  chancelier d'Aguesseau\,; en le rappelant d'exil on ne lui rendit pas
  les sceaux.} pour lui faire l'affront de morceler sa charge. Ce n'est
pas là le cas où \emph{volenti non fit injuria}.

«\,Le prétexte de forme pour ôter la charge de garde des sceaux à M.
d'Armenonville sans sa démission, quoique cette charge fût créée par
édit, ce prétexte fut beau et heureux pour l'autorité royale\,; il
consista en ce que cette charge n'avait été créée pour mon père que par
édit enregistré à un lit de justice, au lieu que la nouvelle charge
qu'on créa pour le Chauvelin le fut par édit enregistré volontairement
au parlement. Le parlement fit une députation au chancelier pour savoir
s'il permettait cette création\,; il dit que \emph{oui}, le bon homme\,;
et le parlement surpris enregistra. M. Chauvelin y avait conservé du
crédit\,; il leur fit accroire qu'il leur rendrait de grands services
près du trône, et on a pu juger s'il a tenu sa parole.

«\,En place, il ne s'est mêlé de rien en apparence, et de tout au monde
en réalité. Il s'est fait haïr des étrangers et du public. Il a fait le
misérable traité de Séville\footnote{Ce traite d'alliance défensive
  entre la France, l'Espagne et l'Angleterre, fut signé le 9 novembre
  1729.}, misérable parce que nous ne voulions pas l'exécuter, et que
c'est un embarquement violent pour ne faire que cacades, paroles de
pistolet et actions de neige. Il a rejeté toutes les actions de
couardise sur la bénignité du cardinal, et qu'on n'a jamais pu {[}lui{]}
persuader la moindre action virile\,; cela est incroyable. Il a fait
l'inouïe action de trahir le marquis de Brancas, en montrant à
l'ambassadeur d'Espagne les lettres que lui écrirait ce notre
ambassadeur. On n'a rien vu de bien de lui, pas même dans la conduite de
la librairie.

«\,Il se vante d'écrire tout de sa main, et il se rompt l'estomac assis
à son bureau\,; petitesse de génie, étendue d'avidité. Ce qu'il a fait
de bien a été de s'enrichir magnifiquement. Il y a un secret d'État qui
est que les Anglais donnent gros à nos ministres.

«\,Il a acheté Grosbois de M. Bernard, et quand ç'a été à payer, il a
montré des billets de Bernard fils qu'il ne pouvait avoir acquis que sur
la place\,; car il agiote mieux que jamais, ce grand magistrat. Il s'est
vanté à moi d'avoir donné de cette terre un prix extravagant, pour
satisfaire un certain amour local qu'il avait conçu pour cette maison, y
allant de jeunesse chez M. d'Harlay. Il faut bien souffrir de telles
insultes qu'on vous fait en face par ces mensonges, dont on sait
évidemment le contraire\,; mais on est appelé à cette vocation-là.

«\,C'est lui qui soutient sous main les avocats dans leur
rébellion\footnote{Voy. \emph{le Journal de l'avocat Barbier}, à l'année
  1730 (mois d'avril, octobre, novembre, décembre)\,; on y trouve
  beaucoup de détails sur l'opposition des avocats.} et les
jansénistes\,; car il embarque des entreprises pour les voir échouer, et
par là les partis qu'on voudrait abattre se fortifient étrangement. Il
se moque de son allié le cardinal de Bissy, et quoique des sots aient
cru qu'il cheminât par lui, jusques ici il n'a pas fait une faute contre
sa fortune, et j'attends le dénouement d'une si monstrueuse habileté
comme d'une pièce difficile à terminer. Il chemine sous terre comme
taupe\,; il paraît séparé de toute la cour, et il a des souteneurs tout
prêts à le porter au pinacle des que le cardinal sera retiré. On dit que
c'est la maison de Condé qu'il s'est attachée, et que les actions de la
compagnie des Indes en sont l'instrument. Pauvre royaume, qu'as-tu fait
à Dieu pour être ainsi foulé aux pieds\,?»

Ce passage avait été écrit au mois de novembre 1730. Dans la suite, le
marquis d'Argenson ajouta en marge\,: «\,Depuis ceci, j'ai connu
davantage M. le garde des sceaux, et j'ai trouvé qu'une partie de tout
ceci était faux, et qu'il méritait de vrais éloges sur son génie, sa
vertu et son amour pour le bien de l'État.\,» On voit en effet, par des
passages subséquents des Mémoires du marquis d'Argenson, que son opinion
sur Chauvelin s'était modifiée. Il écrivait en janvier 1737\,: «\,M. le
garde des sceaux est un homme plus franc qu'on ne s'imagine\,; et voilà
comme les hommes jugent mal par leurs préventions et de légères
apparences. Je dirai même avec certitude, et après une forte épreuve,
que c'est un des hommes les plus francs que j'aie jamais connus, le seul
de cette sorte qui ait peut-être jamais paru à la cour\,; car il ne
dissimule point ses haines, et voilà ce qui conduit à lui prêter des
défauts qu'il n'a pas. Or, il n'aime pas tout le monde\,; il méprise
quantité de gens, et ne cache pas son dessein de les écarter des
affaires\,; et cela prenant ces gens-là dans le retranchement de leur
amour-propre, les pousse à plus de vengeance et de fureur que si M.
Chauvelin s'en tenait à la simple haine ordinaire et personnelle, avec
désir de vengeance\,; mais comme il s'éloigne de ces gens-là par pure
mésestime, il n'est pas vindicatif, quelque injure personnelle qu'il ait
reçue, et se contente d'éloigner des affaires et des places ceux qu'il
méprise, ainsi que ceux qui l'ont offensé. Certainement je parle là
d'une grande vertu ajoutée à une autre\,: savoir être franc, et de
n'être pas vindicatif.

«\,Mais voici son grand défaut. C'est un cadet de nobles\,: il a fallu
percer à la fortune par quelques manèges nécessaires. Ces manèges n'ont
été odieux en rien, mais on y a pris cependant quelques habitudes de
finesse et de ce qu'on appelle \emph{air de brigandage}, entre autres
celui de caresses, n'étant pas caressant ni tendre naturellement\,; car
c'est un homme bilieux, un sage, un philosophe, un homme vertueux
naturellement, aimant la patrie et les honnêtes gens, un législateur
digne de l'ancienne Grèce\,; voilà ce qu'il est naturellement, faisant
du bien aux autres par rectitude d'esprit, et non par attendrissement de
coeur\,; et étant pétri ainsi par la nature, il a cru devoir se replier
aux caresses pour avoir des femmes dans sa jeunesse, et pour s'attirer
des amis à qui il fût utile et qui le fussent à son avancement\,; d'où
est arrivé que, par des caresses forcées, spirituelles et bilieuses, il
a toujours passé le but dans les sentiments témoignés et dans les
promesses faites depuis qu'il est en place, et quand il a voulu obtenir
quelque chose de quelqu'un, il a également donné dans cet excès de
promettre plus de beurre que de pain\,; ce qui lui a attiré des ennemis
de ceux même à qui il faisait du bien, et \emph{a fortiori} des autres,
à qui il n'en faisait pas.

«\,Et voilà la vraie source du peu de justice à lui rendue sur la
franchise, puisqu'on l'a fait passer au contraire pour un homme qui
fourbait du matin au soir, tandis que je soutiendrai que c'est l'homme
le plus franc que j'aie jamais connu, et je n'aime les hommes et les
femmes que tels.\,»

\hypertarget{note-iii.-anne-dautriche-et-mazarin.}{%
\chapter{NOTE III. ANNE D'AUTRICHE ET
MAZARIN.}\label{note-iii.-anne-dautriche-et-mazarin.}}

Saint-Simon a caractérisé brièvement, mais avec force et vérité, les
relations d'Anne d'Autriche et de Mazarin. On l'a vu, dit-il, en parlant
de ce ministre, maltraiter la reine mère en la dominant toujours (p.~170
du présent volume). La correspondance de Mazarin prouve, en effet, qu'il
avait pris un grand ascendant sur Anne d'Autriche, et qu'il la traitait
parfois avec une certaine rudesse. Comment expliquer cette domination du
ministre sur sa souveraine\,? La question est délicate, et l'histoire ne
doit pas la résoudre avec la légèreté des pamphlétaires de la Fronde. Ce
que l'on peut du moins affirmer et prouver par des pièces d'un caractère
authentique, c'est que la correspondance d'Anne d'Autriche et de Mazarin
n'a nullement le caractère d'une correspondance officielle. J'ai réuni
ici un certain nombre de lettres écrites\footnote{Quelques-unes de ces
  lettres ont déjà été publiées par M. Cousin, dans son étude historique
  sur \emph{M\textsuperscript{me} de Hautefort}. Elles sont autographes
  et se trouvent à la Bibl. Imp., ms., \emph{Boîtes du Saint-Esprit}, n°
  177.} par Anne d'Autriche à Mazarin pendant les années 1653, 1654 et
1655. On trouvera aussi, à la suite, deux lettres de Mazarin, où perce
un sentiment de jalousie et de mécontentement qui justifie parfaitement
la phrase de Saint-Simon. Ces documents mettront sur la voie d'une
solution probable\,; mais, avant de se prononcer, il sera utile d'en
rassembler encore beaucoup d'autres enfouis dans les archives et les
bibliothèques.

Mazarin, qui était rentré en France à la fin de l'année 1652, eut la
coquetterie de se faire attendre quelque temps. Il alla rejoindre le
maréchal de Turenne, qui prit plusieurs places sur la frontière
septentrionale de la France. Cette campagne se prolongea pendant le mois
de janvier 1653. Anne d'Autriche écrivit à cette époque plusieurs
lettres au cardinal\footnote{L'orthographe, qui est tout à fait
  irrégulière, a été modifiée.}.

«\,9 janvier 1653.

«\,Votre lettre que j'ai reçue du 24 {[}décembre 1652{]} m'a mise bien
en peine, puisque 15\footnote{Ce chiffre désigne probablement la reine
  elle-même.} a fait une chose que vous ne souhaitiez pas\,; mais vous
pouvez être assuré que ce n'a pas été à intention de vous
déplaire\ldots. 15 n'a ni n'est capable d'avoir d'autres desseins que
ceux de plaire à 16\footnote{Mazarin.}\,; et 15 (la reine) ne sera point
en repos qu'il ne sache que 16 (Mazarin) n'a pas trouvé mauvais ce qu'il
a fait, puisque non seulement il ne voudrait pas lui déplaire en effet,
mais seulement de la pensée qui n'est employée guère qu'à songer à la
chose du monde qui est la plus chère à qui est\footnote{Ces signes et
  d'autres que l'on trouve très souvent dans les lettres de la reine à
  Mazarin ont été considérés comme des symboles d'amour. On pourrait
  dans cette lettre interpréter la dernière phrase ainsi\,: \emph{la
  chose du monde qui est la plus chère à la reine qui est Mazarin}. Mais
  toutes ces explications sont fort incertaines.}.\,»

Comme l'absence de Mazarin se prolongeait, la reine lui écrit la lettre
suivante où perce autre chose que de l'impatience\,:

«\,26 janvier 1653.

«\,Je ne sais plus quand je dois attendre votre retour, puisqu'il se
présente tous les jours des obstacles pour l'empêcher. Tout ce que je
vous puis dire est que je m'en ennuie fort et que je supporte ce
retardement avec beaucoup d'impatience, et si 16 (Mazarin) savait tout
ce que je souffre sur ce sujet, je suis assuré qu'il en serait touché.
Je le suis si fort {[}touchée{]} en ce moment que je n'ai pas la force
d'écrire longtemps ni ne sais pas trop bien ce que je dis. J'ai reçu vos
lettres tous les jours, et sans cela je ne sais ce qui arriverait.
Continuez à m'en écrire aussi souvent, puisque vous me donnez du
soulagement en l'état où je suis. jusques au dernier soupir. Adieu je
n'en puis plus lui sait bien de quoi.\,»

«\,29 janvier 1653.

«\,Je viens de recevoir de vos lettres du 21 {[}janvier{]}, en quoi vous
me donnez espérance de vous revoir, mais jusqu'à ce que je sache le jour
positivement je n'en croirai rien, car j'ai été trompée bien des fois.
Je le souhaite fort, et je vous assure que vous ne le serez jamais
{[}trompé{]} de , puisque c'est la même chose que .\,»

Les deux lettres suivantes, adressées à Mazarin, qui était à l'armée
avec Louis XIV, doivent être de 1654 (probablement août).

«\,Puisque c'est par raison et non par volonté que vous ne revenez pas,
je ne trouve rien à redire. Je veux grand mal aux destinées de vous
obliger à demeurer plus longtemps que je ne voudrais, et vous croirez
aisément que je ne suis point jalouse, quand je vois le
\emph{confident\footnote{Par le \emph{confident}, la reine désigne le
  jeune Louis XIV.}} et ce qu'il aime ici.\,»

«\,Ce dimanche au soir.

«\,Ce porteur m'ayant assuré qu'il irait fort souvent, je me suis
résolue de vous envoyer ces papiers, et vous dire que, pour ce retour
que vous me remettez\footnote{C'est-à-dire que vous remettez à ma
  discrétion.}, je n'ai garde de vous en rien demander, puisque vous
savez bien que le service du roi m'est bien plus cher que ma
satisfaction. \emph{Mais je ne puis qu'empêcher de vous dire que je
crois que, quand l'on a de l'amitié, la vue de ceux que l'on aime n'est
pas désagréable, quand ce ne serait que pour quelques heures. J'ai bien
peur que l'amitié de l'armée soit plus grande que toutes les autres}.
Tout cela ne m'empêchera pas de vous prier d'embrasser de ma part notre
ancien ami\footnote{Louis XIV.}, et de croire que je serai toujours
celle que je dois, quoi qu'il arrive .\,»

En 1655, pendant une nouvelle absence de Mazarin, Anne d'Autriche lui
écrivit les lettres suivantes\,:

«\,A la Fère, ce 12 août 1655.

«\,Votre lettre du 8 août a été reçue plus tôt que celle du 9, puisque
l'une le fut hier et l'autre aujourd'hui. J'en étais en peine\,; car,
comme je suis assurée que vous m'écrivez tous les jours, cela me
manquait. Elle est arrivée, et il n'y en a eu pas une de perdue.
J'attends Gourville qui n'est pas encore arrivé, et vous croirez bien
que ce n'est pas sans quelque impatience, puisque je dois savoir vos
résolutions par lui. J'ai vu un gentilhomme que M. de Marillac envoie au
roi, et comme il y a tant de difficulté à l'aller trouver où il est, je
lui ai dit de s'en retourner à Paris trouver son maître, et aussi que je
me chargeais d'envoyer sa lettre et que lui renverrais la réponse, afin
que je la lui fasse tenir. J'ai vu que les lettres vont si sûrement par
le soin que Bridieu en prend que je me suis résolue d'envoyer la
présente au \emph{confident}, croyant qu'il ne sera pas fâché de
l'avoir, et que au pis aller ils ne gagneront rien ni la curiosité ne
sera pas trop satisfaite, puisqu'il me semble qu'ils ne comprendront pas
pour qui elle est.

«\,Je vous envoie un billet en chiffre qui vient de Paris. Il est venu
fort vite\,; car je reçus l'original dès hier au soir. Vous ne serez pas
fâché à mon avis de voir ce qui est dedans. Pour moi, je ne l'ai pas
été, et cela me fait résoudre à la patience, en cas qu'il fût nécessaire
de l'avoir, puisque le lieu où est le \emph{confident} ne plaît
nullement et donne de la crainte qu'il ne passe plus avant. Pour moi, je
le souhaite de tout mon cœur, et n'en doute pas puisqu'il suivra vos
sentiments, que je suis assurée être comme il faut. Les miens seront
toujours d'être\footnote{Ce signe est un de ceux que M. Walchnaër a
  interprétés comme désignant l'amour d'Anne d Autriche pour Mazarin.}.
C'est tout ce que j'ai à vous dire pour cette fois, et que vous
embrassiez le \emph{confident} pour moi, puisque je ne le puis pas faire
encore. Siron fera tout ce que vous lui mandez le plus tôt qu'il se
pourra.\,»

«\,A la Fère, 13 août 1655.

«\,Vous m'avez donné une grande joie par votre lettre du 10 {[}août{]}
de l'espérance de vous revoir dans cinq ou six jours. Je ne vous en
dirai pas davantage sur ce sujet\,; car vous n'en douterez pas. Nous
attendons toujours Gourville. Je crois que, si vous l'avez dépêché quand
vous me mandez, il sera ici aujourd'hui. Vous me faites bien du plaisir
de me dire que le \emph{confident} est satisfait des soins que je prends
pour lui. J'en recevrai beaucoup pour moi toutes les fois que je
trouverai moyen de l'obliger. La boîte de corail a été donnée et l'on a
été fort aise de l'avoir. Je n'ai rien à ajouter à la lettre d'hier, par
où il me semble que je mande bien des choses. Nogent est ici depuis deux
jours\,; je ne vous en dis rien\,; car lui écrit tout ce qui se peut
écrire au monde. Embrassez le confident et croyez-moi de tout mon cœur .
»

«\,A la Fère, ce 13 août 1655.

«\,Enfin Gourville est arrivé cette après-dînée, et m'a rendu vos
lettres du 11 et du 12, et dit ce que vous lui aviez donné charge de me
dire. Il m'a tiré d'une grande peine en me le disant, et vous m'en avez
sauvé une furieuse en faisant, par raison, consentir le \emph{confident}
à demeurer au Quesnoy, pendant que l'armée se promènera. Je prie Dieu
que sa promenade soit telle que je la lui souhaite J'ai été ravie
d'avoir vu dans une de vos lettres que mes sentiments aient été pareils
aux vôtres pendant la visite que le \emph{confident} me voulut faire,
puisque j'aime mieux ce qui est de sa gloire et de son service que mon
contentement particulier. Je m'assure que vous n'en doutez pas.
J'attendrai donc avec patience que ses affaires lui permettent de venir
et remets à vous d'en juger, quand il sera temps\,; car il me semble que
vous jugez assez bien de toutes choses et que le mal de tête que vous
avez eu ne vous en a pas empêché. Je suis bien aise que vous ne l'ayez
plus, et si vous avez autant de santé que je vous désire, vous serez
longtemps sans avoir aucun mal.

«\,Je ne sais si, à la fin, la quantité de mes lettres ne vous
importunerait point. Voici la deuxième d'aujourd'hui\,; mais, si vous
êtes aussi aise d'en recevoir que {[}moi de{]} vous, je suis bien
assurée qu'elles ne le feront point. Je suis bien aise que les cavaliers
de Guise s'acquittent si bien de leurs voyages\ldots. Ceux qui portent
cette lettre sont venus avec Gourville qu'il a amenés exprès, afin que
vous sachiez son arrivée et sa diligence. Pour des nouvelles d'ici,
après toutes celles que Nogent a mandées, il serait difficile d'en dire
aucune. C'est pourquoi je m'en remets entièrement à ce qu'il en a écrit.
Dites au \emph{confident} que je suis bien aise qu'il se souvienne de ce
que je lui dis en partant et qu'il s'en acquitte, puisque, de lui, lui
viendra tout son bonheur, et que lui en souhaitant beaucoup comme je
fais, je suis fort aise qu'il fasse ce qu'il faut pour cela. Je ne lui
écris point puisque aussi bien il faut que vous soyez l'interprète de ma
lettre, qui sera pour tous deux\,; mais je la finis en vous priant
toujours d'une même chose, qui est de l'embrasser bien pour moi et de
croire que je serai tant que Je vivrai .\,»

Les lettres de Mazarin à la reine sont d'un tout autre ton. Malgré la
phraséologie sentimentale, il y perce un peu d'aigreur et de jalousie.
Telle est, du moins, l'impression qui me paraît résulter des lettres
suivantes écrites en 1659\footnote{Les minutes de ces lettres se
  trouvent à la Bibl. Mazarine, ms. 1719 (H).}\,:

«\,A Saint-Jean de Luz, 1er novembre.

«\,Je viens de recevoir votre lettre du 28 du passé, et je suis au
désespoir de vous avoir donné sujet de me faire un si grand
éclaircissement, lequel, au lieu de me consoler, me donne encore plus de
peine, voyant que l'affection que vous avez pour la personne ne vous
permet pas de croire qu'elle soit capable de faire jamais aucune faute.
Je vous supplie d'avoir la bonté de me pardonner, si j'ai pris la
hardiesse de vous en parler, vous promettant de ne le faire de ma vie et
de souffrir avec patience l'enfer que cette personne me fait éprouver.
Je vous dois encore davantage que cela, et, quand je devrais mourir
mille fois, je ne manquerai pas aux obligations infinies que je vous ai,
et, quand je serais assez méchant et ingrat pour le vouloir, l'amitié
que j'ai pour vous, qui ne finira pas même dans le tombeau, m'en
empêcherait.

«\,Je souhaiterais vous pouvoir encore dire davantage, et, s'il m'était
permis de vous envoyer mon cœur, assurément vous y verriez des choses
qui ne vous déplairaient pas et plus dans cet instant que je vous écris
qu'il n'a jamais été, quoique je voie, par la lettre que vous m'avez
écrite, que vous avez oublié ce qu'il vous plut me dire avec tant de
bonté à Paris, lorsque nous parlâmes si à fond sur le sujet de la même
personne, laquelle a toujours été la seule cause de mes plaintes et du
déplaisir que vous en avez témoigné en diverses rencontres.

«\,Mais il ne faut pas vous importuner davantage, et je dois me
contenter des assurances que vous me donnez de votre amitié, sans
prétendre vous gêner à n'en avoir pas pour cette personne, puisqu'il
vous plaît de nous conserver tous deux à votre service. Je vous conjure
de nouveau à genoux de me pardonner, si je vous donne du chagrin en vous
ouvrant mon cœur qui ne vous cachera jamais rien, et je vous confirme
que, si je devais vivre cent ans, je ne vous en dirai jamais un seul mot
et que je serai toujours le même à votre égard avec certitude que vous
n'aurez pas en aucun temps le moindre sujet de douter de ma passion
extrême pour votre service, ni de mon amitié qui n'aura jamais de
semblable si les \emph{anges}\footnote{Ce mot désigne Anne d'Autriche.}
me veulent rendre justice le croyant ainsi, et je vous supplie de me
rendre de bons offices auprès d'eux, vous protestant, comme si j'étais
devant Dieu, que je les mérite.\,»

Mazarin revient sur le même sujet dans une lettre du 20 novembre 1659\,:

«\,Je reconnais bien qu'à moins que les \emph{anges} vous eussent
inspiré de m'écrire une lettre si obligeante que celle que je viens de
recevoir du 7 du courant, il vous était impossible de la former avec des
termes si tendres et si avantageux pour moi qui ne désire aucune chose
avec plus de passion que d'être toujours assuré de l'honneur de votre
amitié. Je vous déclare encore une fois que rien n'est capable de m'en
faire douter, quelque chose qui puisse arriver. Mais je vous avoue à
même temps que vous me combleriez d'obligations, si vous aviez la bonté
un jour de vouloir apporter quelque remède à ce que vous savez qui me
fait de la peine et qui me la fera toute ma vie. Je vous conjure de vous
souvenir de ce qu'il vous a plu de me faire espérer sur ce sujet, et
qu'assurément la passion et la fidélité que j'ai pour vous et pour la
moindre de vos satisfactions mérite bien que vous songiez un petit à
guérir la maladie, qui, sans votre assistance, sera incurable.

«\,Vous en avez eu depuis peu de jours une belle occasion, ayant vu
plusieurs lettres de la cour qui portaient que la personne dont est
question vous avait bien fâchée par des emportements qui étaient fort
contre le respect que tout le monde vous doit, et pour une affaire dont
il n'y a qui que ce soit qui ne la condamne, outre que l'ouverture de la
cassette sera de grand préjudice, puisqu'il fera public ce que du
Bosc\footnote{Loret donne quelques détails sur ce du Bosc (\emph{Muse
  historique}, 25 octobre 1659)\,: Lettre XLIe. «\,Monsieur Dubosq,
  digne d'estime, Jadis mon amy très intime, Et de mesme climat que moy,
  L'un des interprètes du roy, Et gentilhomme chez la reine, Dont l'âme
  estoit de vertus pleine\,; Enfin, ce Dubosq que je dis, Que je tiens
  estre en paradis, D'autant qu'il estoit bon et sage Est décédé dans le
  voyage, Depuis trois semaines, ou plus, À Baïonne ou Saint-Jean de
  Lus. «\,Il sçavoit faire des harangues, Estoit docte en diverses
  langues, Comme il a montré plusieurs fois, Servant avec esprit nos
  rois. Son âme estoit noble et loyale, Estant pour la cause royale
  Toujours ferme comme une tour\,; Et quoyqu'il fût homme de cour, Sa
  probité fut infinie\,; Il vivoit sans cérémonie. L'Éminence et Leurs
  Majestés Faisoient cas de ses qualités. Bref, chacun le chérissoit
  comme Un fort prude et fort honnête homme. «\,O cher Dubosq, esprit
  charmant, Qui, comme moy, fus bas Normand, Courtisan, à présent,
  céleste, Qui, par un sort trop tôt funeste, Es mort travaillant pour
  la paix\,; Estant avec Dieu, désormais, Sans plus redouter les
  tempestes Que le ciel suspend sur nos testes, Ce Loret qui t'estimoit
  tant Et qui, dans ce monde inconstant, Ne te peut survivre de gueres,
  Se recommande à tes prières.\,»} y avait laissé pour servir le
\emph{confident} en ce que vous savez. Je vous réplique\footnote{\emph{Répète}.}
que tout le monde témoigne d'être scandalisé du procédé de ladite
personne, et chacun sachant qu'elle ne m'aime pas, et voyant que vous
avez la bonté de souffrir la hauteur avec laquelle elle se conduit avec
sa propre maîtresse, tous tirent une conséquence qu'elle a tout pouvoir
avec vous.

«\,Je vous demande pardon de ce que je prends la liberté de vous écrire
sur cette matière, puisque cela ne procède que de l'amitié et de la
confiance que j'ai aux \emph{anges}, qui seront toujours les maîtres
d'en user en cela et en tout ce qui me regardera, comme ils voudront,
sans que je change jusqu'à la mort d'être ce que je vous dois. En quoi
vous ne m'avez pas beaucoup d'obligation, puisque, quand même je le
voudrais, il me serait impossible de l'exécuter. Mais j'ai grande joie
de savoir que je ne le pourrai et je ne le voudrai jamais.\,»

Quelle personne Mazarin désigne-t-il dans ces lettres\,? Les écrivains
du temps ne donnent à cet égard aucune explication\,; mais on voit par
des lettres inédites de Bartet à Mazarin\footnote{Voy sur Bartet, t. VI,
  p.~449 et surtout 455.} qu'il s'agit de M\textsuperscript{me} de
Beauvais, femme de chambre de la reine, que Mazarin avait déjà fait
éloigner de la cour\footnote{Voy. t. VI, p.~459, note 6.}, mais qui y
était revenue avec plus de crédit qu'auparavant. Peut-être aussi Mazarin
a-t-il en vue l'écuyer de la reine, Beaumont\,; ce qui expliquerait
mieux certaines expressions de ses lettres. Voici le passage de la
lettre de Bartet qui peut jeter quelque lumière sur l'intérieur de cette
cour. Il écrivait à Mazarin le 16 octobre 1659\,:

«\,Je ne sais si la reine vous écrira qu'elle a ici une affaire sur les
bras avec M\textsuperscript{me} de Beauvais, qu'il est juste que vous
sachiez que j'ai vue commencer et non pas finir, puisqu'elle dure
encore. Elle vous divertira assurément. Votre Éminence la trouvera
\emph{ridiculosa}.

«\,Sa Majesté apprit à Nérac la mort du pauvre M. du Bosc. D'abord M. de
Beaumont, son écuyer, lui dit qu'il y avait dans sa cassette, qui était
avec celles de la reine, trois cents louis d'or en espèces qui lui
appartenaient, et montra pour cela le billet de M. du Bosc, et la
supplia de trouver bon qu'il les prît. La reine répliqua qu'elle n'en
doutait pas\,; que, s'ils y étaient, elle lui répondait qu'il n'y
perdrait rien\,; mais qu'elle ne voulait point ouvrir la cassette qu'en
présence de Votre Éminence ou au moins que vous ne fussiez de retour.

«\,M. de Beaumont ne parut point satisfait de ce retardement-là, et
demanda à la reine qu'il plût à Sa Majesté de lui assurer son argent sur
la charge de M du Bosc. La reine le refusa et lui dit qu'elle ne se
mêlait point de sa charge. M\textsuperscript{me} de Beauvais se mêla en
tout cela pour favoriser M. de Beaumont avec beaucoup de bruit et à
diverses fois. La reine s'en défendit toujours et lui résista, en sorte
qu'on partit de Nérac et que la cassette ne fut point ouverte.

«\,Hier au soir la reine jouant à la bête, Monsieur\footnote{Philippe
  d'Orléans, frère de Louis XIV.}, ayant découvert que la cassette était
perdue, ou bien ceux qui l'avaient détournée l'ayant laissé pénétrer à
Monsieur, afin que cela allât à la reine pour en voir l'effet d'un peu
loin, dit tout haut parlant à Sa Majesté que l'âme du pauvre du Bosc
avait emporté sa cassette, et que Beaumont avait perdu ses trois cent
cinquante pistoles.

«\,La reine s'émut assez et il lui entra d'abord dans l'esprit, comme il
a paru depuis, que c'était Beaumont qui l'avait fait prendre pour ravoir
son argent, et que M\textsuperscript{me} de Beauvais, ayant parlé pour
lui à Nérac avec impétuosité et avec cette véhémence qu'elle a pour les
choses où elle se passionne, pouvait bien lui avoir donné ce conseil
hardi de faire prendre la cassette.

«\,Cette vraisemblance si violente et ces sortes d'apparences si
ajustées par le tempérament des personnes firent que la reine ne se
contraignit point de dire en particulier, quand le jeu fut fini, qu'elle
croyait que c'était Beaumont et ses amis qui le lui avaient conseillé,
et en témoigna un ressentiment considérable avec des paroles assez
aigres\,; mais, comme je dis, en un particulier si resserré qu'il n'y
avait, ce me semble, que Monsieur, M. de Cominges et moi.

«\,M\textsuperscript{me} de Beauvais, à qui Monsieur le dit, se trouva
engagée à en faire un éclaircissement et pour cela elle prit fort mal à
propos son champ de bataille dans le moment même que la reine se mettait
à table avec le roi et avec Monsieur, et véritablement vint faire une
salve sur le tort que la reine avait d'ôter ainsi l'honneur à un
gentilhomme qui était à elle, il y avait douze ans, qu'elle poussa si
loin et avec si peu de retenue que la reine prit sur elle le ton que Son
Éminence connaît, et lui dit que la pensée qu'elle avait eue de Beaumont
là-dessus n'avait été rendue publique que par elle, et qu'elle avait eu
la bonté de n'en point parler que devant deux ou trois personnes, et
encore sans beaucoup d'aigreur\,; mais, puisqu'elle le prenait comme
cela, elle était bien aise de lui dire aussi devant tout le monde qu'il
était vrai qu'elle avait pensé et dit que c'était Beaumont, lequel, par
le conseil de ses amis, ou de ses amies, avait fait prendre sa cassette
pour ravoir son argent.

«\,M\textsuperscript{me} de Beauvais, qui assurément s'était persuadée
que la reine s'adoucirait en public et que l'éclat qu'elle faisait
l'obligerait à en parler favorablement, voyant un effet tout contraire,
s'engagea dans une autre conduite encore plus fâcheuse que la
première\,; car imprudemment elle fut prendre Beaumont par la main et le
produisit à la table devant cinq cents personnes, et dit à la reine que
c'était là le gentilhomme qu'elle déshonorait. Sa Majesté piquée lui dit
qu'elle était bien aise de lui dire à son nez, et devant tout le monde,
qu'il était vrai, et que c'était lui qui avait eu la hardiesse de faire
prendre la cassette de du Bosc parmi ses cassettes.

«\,Beaumont lui repartit le plus respectueusement du monde, et, il faut
dire la vérité, le plus bel esprit et le plus honnête homme de la cour
n'eût jamais pu sortir mieux qu'il fit d'une si méchante affaire, dans
laquelle M\textsuperscript{me} de Beauvais l'entraîna, croyant que la
reine relâcherait face à face. Il se retira donc fort mortifié et avec
beaucoup de contusion, laissant la reine touchée, et tout ce qu'il y
avait là du monde, de la modestie et de la soumission avec laquelle il
fut affligé du jugement que la reine avait fait de lui.

«\,Après le souper, nous suivîmes, cinq ou six, la reine dans son
cabinet, M\textsuperscript{me} de Beauvais y vint plus allumée encore
qu'à la table. La reine s'y roidit encore plus qu'elle n'avait fait, et
je vis que l'affaire ne demeurait plus à voir si l'on avait raison et si
on disait vrai, mais à maintenir par autorité ce qu'on avait dit et ce
qu'on avait pensé\,; et, comme la conduite de Beaumont avait réussi à la
table et que la reine avait fermé la bouche à M\textsuperscript{me} de
Beauvais, ce moment me parut propre pour présenter Beaumont
heureusement. Je sortis donc du cabinet et le fus quérir dans la chambre
de la reine qui est de plein pied, où je l'avais laissé, et d'où il
entendit toutes choses. Il entra donc insensiblement après moi et
derrière moi, et en un moment je m'écartai un peu pour que la reine le
pût voir, à laquelle il demanda pardon, avec larmes, d'avoir donné
occasion à tant de paroles et à tant de choses.

«\,La reine reçut cela avec la dernière bonté et lui parla si
obligeamment qu'elle parut chercher des paroles pour lui témoigner
qu'elle n'avait pas cru qu'il l'eût fait par friponnerie, jusque-là
qu'elle lui dit qu'elle oublierait tout, s'il voulait lui dire qui lui
avait conseillé de le faire\,; mais il n'entra point là dedans, et
continuant à pleurer, la reine lui dit bonnement de se retirer et à nous
de n'en plus parler.

«\,Sa Majesté se mit à sa toilette, où, dans le temps qu'on la
décoiffait, M\textsuperscript{me} de Beauvais entra avec un grand éclat
de joie et de grands cris\,; et s'approchant d'elle cria que la cassette
était retrouvée, et que Loranval, qui a en garde toutes les cassettes de
la reine et qui avait eu aussi toujours celle-là, était là, qu'elle
avait envoyé réveiller à minuit.

«\,Il se présente à la reine\,; il dit qu'il avait la cassette. Sa
Majesté lui dit qu'elle n'était pas plus satisfaite pour cela, et qu'on
l'avait ouverte et vu peut-être les papiers qui étaient dedans. Il jure
que non. Sa Majesté lui dit qu'il était un menteur\,; elle fait venir
Visé qui a été son exempt, qui avait dit à la reine deux heures devant
qu'il avait rencontré un homme à midi dans la place vêtu de serge grise,
qui portait sous son bras la cassette de du Bosc couverte à demi de son
manteau\,; qu'il l'avait regardée à deux fois, et, comme elle était
singulière avec des chiffres de la reine et du cuivre doré, lui maintint
en présence qu'il l'avait vue. L'autre fut confondu. Dans cet embarras,
la reine se tournant vers M\textsuperscript{me} de Beauvais, lui dit que
ceux qui lui avaient fait le bec ne le lui avaient pas assez bien fait,
et qu'elle n'en croyait ni plus ni moins pour tout ce qu'il venait de
dire. Le roi et Monsieur étaient présents à tout cela qui fut dit aux
mêmes mots que je le fais.

«\,Aujourd'hui on a représenté la cassette. La reine l'a fait ouvrir\,;
on y a trouvé les trois cent cinquante pistoles. Sa Majesté les a fait
rendre à Beaumont, lequel n'est point demeuré satisfait et a été
demander permission au roi de faire informer là-dessus contre Loranval
qui garde les cassettes, afin qu'il fût obligé de dire la vérité\,; car
le témoignage de Visé est cru véritable de tout le monde.

«\,Après tout cela il se trouvera que la reine a très bien jugé et n'a
point voulu être prise pour dupe, et que la friponnerie qui pourra s'y
découvrir ne tombera point sur Beaumont\,; car assurément nous savons
tous qu'il est innocent, et il fera tout ce qu'il pourra pour qu'on
n'étouffe point l'affaire.\,»

~

~

Une lettre d'Anne d'Autriche à Mazarin, en juin 1660 \footnote{Cette
  lettre a été publiée par M. Walckenaër, t. III des \emph{Mémoires de
  M\textsuperscript{me} de Sévigné}. - Supplément, p.~471.} , montre que
les jalousies et les aigreurs entre la reine et Mazarin duraient encore.
Cette correspondance présente une suite de brouilleries et de
réconciliations, de plaintes et d'expressions d'amour qui ne paraissent
guère convenir aux relations entre une reine et son ministre.

«\,Saintes, ce 30 juin 1660.

«\,Votre lettre m'a donné une grande joie\,; je ne sais si je serai
assez heureuse pour que vous le croyiez, et que, si j'eusse cru qu'une
de mes lettres vous eût autant plu, j'en aurais écrit de bon coeur, et
il est vrai que d'en voir tant et des transports avec {[}lesquels{]}
l'on les reçut et je les voyais lire, me faisait fort souvenir d'un
autre temps, dont je me souviens presque à tous moments, quoique vous en
puissiez croire et douter. Je vous assure que tous ceux de ma vie seront
employés à vous témoigner que jamais il n'y a eu d'amitié plus véritable
que la mienne, et, si vous ne le croyez pas, j'espère de la justice que
j'ai, que vous vous repentirez quelque jour d'en avoir jamais douté, et
si je vous pouvais aussi bien faire voir mon cœur que ce que je vous dis
sur ce papier, je suis assurée que vous seriez content, ou vous seriez
le plus ingrat homme du monde, et je ne crois pas que cela soit. La
reine\footnote{La jeune reine Marie-Thérèse.}, qui écrit ici sur ma
table, me dit de vous dire que ce que vous mandez du \emph{confident} ne
lui déplaît pas, et que je lui assure de son affection\,; mon fils vous
remercie aussi, et 22\footnote{C'est Anne d'Autriche elle-même, selon M.
  Walckenaër.} me prie de vous dire que jusques au dernier soupir quoi
que vous en croyiez .\,»

\hypertarget{note-iv.-des-conseils.}{%
\chapter{NOTE IV. DES CONSEILS.}\label{note-iv.-des-conseils.}}

Les conseils, dont Saint-Simon parle souvent dans ses Mémoires, et
notamment (p.~175 de ce volume), présentaient des avantages et des
inconvénients que le marquis d'Argenson a bien
caractérisés\footnote{Mémoires publies en 1825, p. 174-176.}\,:
«\,Lorsque l'on eut senti l'abus des conseils établis par M. le duc
d'Orléans, et que l'on s'aperçut enfin qu'il y fallait
renoncer\footnote{En octobre 1718.}, on leur donna une sorte
d'extrême-onction en chargeant l'abbé de Saint-Pierre, qui les avait
d'abord approuvés, d'en faire l'apologie. Il s'en acquitta en composant
un ouvrage qu'il intitula \emph{la Polysynodie}, ou l'avantage de la
pluralité des conseils, avec cette épigraphe tirée des proverbes de
Salomon\,: \emph{Ubi multa consilia, salus}\footnote{Saint-Simon parle
  dans ses Mémoires, à l'année 1718, de cet ouvrage de l'abbé de
  Saint-Pierre.}**. Il avait raison, à un certain point\,; mais il est
obligé lui-même de convenir qu'autant les conseils peuvent être utiles
quand ils sont dirigés, que les questions qui leur sont soumises ont été
d'avance préparées par l'autorité, et que celle-ci décide souverainement
après les avoir consultés, autant sont-ils dangereux, lorsqu'au lieu de
leur laisser le soin d'éclairer le pouvoir, on le leur abandonne tout
entier. Alors ils dégénèrent en vraie \emph{pétaudière\,;} on tracasse,
on dispute, personne ne s'entend, et il n'en résulte que désordre et
anarchie.

«\,Pourtant de la suppression absolue des conseils, ou du moins de
l'oisiveté dans laquelle on laisse languir ceux qui grossissent encore
nos almanachs, on doit conclure que l'on ignore en France le parti que
l'on en pourrait tirer. Je ne parle point de ces petites affaires
particulières dont on amuse actuellement le tapis dans les conseils
royaux des finances et des dépêches\footnote{On donnait ce nom au
  conseil chargé du gouvernement intérieur de la France.}, lorsqu'on les
assemble, mais des ordonnances, des règlements généraux, de tout ce qui
fait loi et établit des principes fixes en administration.

«\,Les ministres ne sentent pas assez combien il leur serait important
d'obtenir des garants pour de semblables règlements. En les prenant sur
eux, ils s'exposent à répondre des difficultés qu'ils éprouvent à
l'enregistrement et à l'exécution. Ils en sont souvent les victimes et
fournissent contre eux-mêmes des occasions de déplacement. Ces
règlements leur serviraient de boucliers contre les demandes injustes\,;
et combien n'est-il pas important qu'ils s'en défendent\,? Pour une
grâce contre règle et raison que le ministre accorde à ses protégés
personnels et véritables, il est obligé d'en accorder vingt aux protégés
de ses propres protecteurs, à des personnes auxquelles il n'a rien à
refuser\,; alors, quand on le presse, il ne sait que répondre. S'il
refuse aux uns ce qu'il accorde aux autres, il se fait des tracasseries
abominables. Un homme sage, en entrant en place, doit s'arranger bien
plus pour pouvoir refuser sans se faire beaucoup de tort que pour
pouvoir tout accorder à sa fantaisie. Car il est bien sûr qu'il n'en
viendra jamais à bout\,; mais il faut refuser sans humeur, et recevoir,
même avec douceur, les demandes les plus déraisonnables, surtout ne pas
compromettre ce que l'on n'est pas sûr de pouvoir tenir. \emph{Hoc opus,
hic labor est}.\,»

Le marquis d'Argenson va plus loin dans ses Mémoires manuscrits. Il y
écrit à la date du 21 mai 1734\,: «\,Qu'on ne me parle plus des conseils
comme devant gouverner ce royaume-ci. Nous ne sommes pas faits pour
cela\,: sous Henri IV et sous Louis XIV, sous Charlemagne, sous Charles
V et sous Louis XI, les conseils ont-ils gouverné\,? Les conseils ont
l'esprit si petit, quoique composés de grands hommes, {[}que{]}, s'ils
apportent quelque sagesse dans les affaires, c'est de cette sagesse qui
vient de médiocrité\,; ce qui n'est point sagesse par le grand sens et
par la prévoyance, mais parce qu'elle exempte de folie.\,»

\hypertarget{note-v.-uxe9tats-guxe9nuxe9raux-mode-de-nomination-des-duxe9putuxe9s-aux-uxe9tats-guxe9nuxe9raux}{%
\chapter{NOTE V. ÉTATS GÉNÉRAUX\,; MODE DE NOMINATION DES DÉPUTÉS AUX
ÉTATS
GÉNÉRAUX}\label{note-v.-uxe9tats-guxe9nuxe9raux-mode-de-nomination-des-duxe9putuxe9s-aux-uxe9tats-guxe9nuxe9raux}}

Saint-Simon indique brièvement le mode de nomination des députés aux
états généraux\,; il parle d'une double élection\,: 1° pour les
assemblées des bailliages, 2° pour les députés des états généraux. Comme
on connaît peu aujourd'hui les anciennes institutions de la France, il
ne sera pas inutile d'insister sur ces usages qui, pour Saint-Simon et
ses contemporains, n'avaient pas besoin de commentaire.

Le roi seul ou le régent avait le droit de convoquer les états généraux.
Il adressait à cet effet des lettres patentes aux gouverneurs des
provinces ainsi qu'aux baillis et sénéchaux qui, sous leur autorité,
ôtaient chargés de l'administration provinciale. Elles indiquaient
l'époque et le lieu où devaient se réunir les députés. En vertu des
ordres du roi, les ecclésiastiques et les nobles étaient nominativement
convoqués pour l'élection de leurs députés. Les gouverneurs et baillis
envoyaient copie des lettres patentes aux maires et échevins des villes
ainsi qu'aux juges et curés des villages. Les bourgeois et vilains
étaient avertis au prône, à son de trompe, par affiches apposées au
pilori et à la porte des églises.

Le s nobles et les ecclésiastiques nommaient directement les députés qui
devaient les représenter aux états généraux. Il n'en était pas de même
pour les bourgeois et les paysans\,: réunis dans les villes et dans les
villages, sous la présidence des baillis, sénéchaux, vicomtes, viguiers,
prévôts, lieutenants des baillis, etc., ils nommaient les électeurs.
Ceux-ci se réunissaient au chef-lieu du bailliage, et procédaient à
l'élection des députés aux états généraux. Ils rédigeaient aussi des
cahiers de doléances pour exprimer leurs besoins et leurs vœux.

Le nombre des députés qui devaient être élus dans chaque bailliage
n'était pas déterminé\,; cette question avait alors très peu
d'importance, puisque, dans l'assemblée des états généraux, on votait
par ordre et non par tête.

Tout en cherchant à résumer et à ramener à des règles uniformes la
nomination des députés aux états généraux, il faut reconnaître que les
usages variaient souvent de province à province. Les paysans ne
prenaient pas toujours part aux élections. En Auvergne, par exemple, le
clergé, la noblesse et la bourgeoisie nommaient seuls les députés aux
états généraux. Dans plusieurs circonstances, des corps, comme la
commune de Paris en 1356, l'université en 1413, le parlement en 1557,
obtinrent une représentation spéciale.

\hypertarget{note-vi.-ruxe9cit-officiel-de-larrestation-de-fouquet-ruxe9diguxe9-par-ordre-de-colbert.}{%
\chapter{NOTE VI. RÉCIT OFFICIEL DE L'ARRESTATION DE FOUQUET, RÉDIGÉ PAR
ORDRE DE
COLBERT.}\label{note-vi.-ruxe9cit-officiel-de-larrestation-de-fouquet-ruxe9diguxe9-par-ordre-de-colbert.}}

Aussitôt après l'arrestation de Fouquet (5 septembre 1661), Colbert, qui
avait dirigé toute cette affaire, fit instituer une chambre de justice
pour juger le surintendant et ses complices. Joseph Foucault, père de
l'intendant dont on cite plus loin le journal, fut nommé greffier de ce
tribunal. Il nous reste de lui un recueil des procès-verbaux sous le
titre de Registres de la chambre de justice\footnote{Bibl. Imp., ms.
  nos. 235-245 des 500 de Colbert.} . Il a fait précéder ces
procès-verbaux d'un récit de l'arrestation de Fouquet rédigé d'après les
documents officiels. C'est une pièce d'une authenticité incontestable où
l'on trouve des détails importants\,:

«\,Le bruit d'un voyage que le roi devait faire en Bretagne ayant
longtemps couru sans que personne en pût conjecturer la cause, quoiqu'on
en parlât fort diversement, Sa Majesté, partie de Fontainebleau le
premier jour de septembre 1661, suivie de M. le Prince, de M. le duc de
Beaufort, de MM. de Charost, de Villequier, de Saint-Aignan, de Villeroy
et de peu d'autres seigneurs, prit la poste à Blois et se rendit trois
jours après à Nantes.

«\,M. Fouquet\footnote{Nicolas Fouquet, né en 1615, maître des requêtes
  en 1636, à vingt et un ans, procureur général au parlement de Paris en
  1650, surintendant des finances en 1653, mort en 1680.}, lors
surintendant des finances, et qui peu de jours auparavant avait disposé
de sa charge de procureur général au parlement de Paris en faveur de M.
de Harlay, maître des requêtes, partit un jour avant le roi en relais de
carrosse qui avaient été disposés en divers lieux de sa marche.
M\textsuperscript{me} sa femme et M. de Lyonne l'accompagnèrent jusques
à Nantes, où il se rendit en même temps que Sa Majesté, bien qu'il fût
travaillé d'une fièvre double tierce\footnote{D'après les Mémoires de
  Louis-Henri de Loménie, comte de Brienne, Fouquet fit une partie du
  voyage sur la Loire. Voici le passage (édit de M. F. Barrière, t. II,
  p.~187)\,: «\,M. Fouquet, accompagné de M. de Lyonne son ami intime,
  passa dans une fort grande cabane, à plusieurs reprises et je les
  saluai. Un moment après passa une autre cabane, où était M. Le Tellier
  avec M. Colbert\,; et je les saluai encore, et Ariste dit, sans que je
  fusse préparé à cela\,: «\,Ces deux cabanes, que nous voyons encore
  l'une et l'autre, se suivent avec autant d'émulation que si les
  rameurs disputaient un prix sur la Loire. L'une des deux, ajouta-t-il,
  doit faire naufrage à Nantes. »}.

«\,M. Le Tellier, secrétaire d'État, et M. Colbert, intendant des
finances, firent ce voyage en même carrosse.

«\,La cour étant à Nantes, le roi assista aux états de Bretagne qui
avaient été convoqués et dont M. Boucherat, maître des requêtes, était
commissaire de la part de Sa Majesté. Toute la province était en suspens
et l'on voulait faire appréhender au peuple quelque chose
d'extraordinaire. Mais enfin on connut qu'il n'y avait rien à craindre
que pour M. Fouquet que le roi fit arrêter, et comme cette résolution
était importante et que Sa Majesté n'en voulait commettre l'exécution
qu'à une personne de confiance, elle fit choix du sieur d'Artagnan,
sous-lieutenant de la compagnie des mousquetaires, qu'elle manda le
jeudi premier jour du mois de septembre pour lui prescrire les ordres
qu'il avait à suivre. On le trouva dans le lit avec une grosse fièvre,
nonobstant laquelle le roi lui fit commander de se rendre près de Sa
Majesté en quelque état qu'il fût. Le sieur d'Artagnan ne put obéir à
cet ordre qu'en se faisant porter dans la chambre du roi qui, le voyant
en si mauvais état, ne lui dit autre chose, sinon que prenant une active
confiance en sa fidélité, il avait jeté les yeux sur lui pour
l'exécution d'une résolution qu'il lui aurait communiquée, s'il avait
été en meilleur état\,; mais qu'il fallait remettre la partie à deux ou
trois jours, pendant lesquels il lui recommanda d'avoir soin de sa
santé.

«\,Le vendredi et le samedi, le sieur d'Artagnan fut visité de la part
du roi sous divers prétextes, et le dimanche s'étant rendu chez le roi
sur le midi, Sa Majesté lui demanda tout haut des nouvelles de sa
compagnie et témoigna qu'elle en voulait voir le rôle qu'il lui remit
entré les mains. Le roi entra en lisant dans le cabinet\,; il en ferma
lui-même la porte, dès que le sieur d'Artagnan y fut entré, et, après
quelques paroles qui témoignaient une obligeante confiance, Sa Majesté
lui déclara qu'étant mal satisfaite de M. Fouquet, elle avait résolu de
le faire arrêter. Elle lui recommanda d'exécuter cet ordre avec prudence
et avec adresse, et lui mit en main un paquet dans lequel étaient les
ordres qu'il avait à suivre, lui recommandant d'en aller faire
l'ouverture chez M. Le Tellier. Comme le sieur d'Artagnan se voulait
retirer, le roi lui dit qu'il fallait payer de quelque défaite ceux qui
étaient à la porte, et qui l'avaient vu demeurer si longtemps dans le
cabinet. Ce qui l'obligea de dire à ceux qu'il rencontra, en sortant
qu'il venait de demander au roi un don que Sa Majesté lui avait accordé
de la meilleure grâce du monde, et de ce pas s'étant rendu chez M. Le
Tellier qu'il trouva environné de beaucoup de gens, il lui dit tout haut
que le roi lui avait accordé une grâce, dont il lui avait commandé de
venir demander promptement les expéditions. Ce qui donna occasion à M.
Le Tellier de l'emmener dans son cabinet, où le sieur d'Artagnan se
trouva si faible qu'il fut obligé de demander du vin pour prévenir une
défaillance. S'étant remis il ouvrit le paquet, où il vit une lettre de
cachet pour arrêter M. Fouquet, une autre lettre contenant la route
qu'il fallait tenir, et tout ce qu'il avait à faire pour le conduire
jusques au lieu de sa prison, une autre lettre pour envoyer un brigadier
et dix mousquetaires en la ville d'Ancenis pour exécuter l'ordre qui
leur serait envoyé le lendemain de leur arrivée, qui fut d'arrêter tous
autres courriers que ceux de Sa Majesté, afin d'empêcher que la nouvelle
de cet emprisonnement ne vint à Paris par d'autres voies. Il y avait
encore dans le paquet diverses lettres adressées aux gouverneurs des
places, et toutes ces lettres étaient écrites de la main de M. Le
Tellier.

«\,Le lundi 5 septembre, le roi, pour mieux couvrir ce dessein, avait
fait une partie de chasse\footnote{Le jeune Brienne parle aussi de cette
  partie de chasse (t. II, p.~204, des \emph{Mémoires de H. L. de
  Loménie}, publies par M. F. Barrière).}, pour laquelle il fit
commander les mousquetaires et les chevau-légers qui se trouvèrent tous
à cheval, lorsqu'il sortit du conseil. Il parla encore assez longtemps à
M. Fouquet, tandis que M. Le Tellier alla joindre M. Boucherat, qui
s'était rendu à la porte du conseil par un ordre exprès, et lui donna
une lettre de cachet qu'il avait toute écrite de sa main comme les
autres, par laquelle le roi, faisant part de la résolution qu'il avait
prise à M. Boucherat, lui enjoignait d'aller, aussitôt que M. Fouquet
serait arrêté, saisir les papiers qui se trouveraient en sa maison, et
en celle du sieur Pellisson son commis.

«\,Le roi voyant que toutes choses étaient bien disposées quitta M.
Fouquet, lequel en descendant l'escalier parla à tous ceux qui avaient
quelque chose à lui dire. Il rentra dans sa chaise sur les onze
heures\footnote{On trouvera des différences notables entre ce récit
  officiel, et celui qu'a laisse le jeune Brienne. Je n'ai pas besoin
  d'ajouter qu'entre les détails un peu romanesques donnes par Brienne,
  et le caractère sérieux et positif de notre récit, on ne peut hésiter.},
et comme il sortait du château, dont il avait passé la dernière
sentinelle, le sieur d'Artagnan fit arrêter sa chaise en lui disant
qu'il avait à lui parler. M. Fouquet lui demanda s'il fallait que ce fût
sur-le-champ ou s'il pouvait attendre que ce fût en sa maison. Mais le
sieur d'Artagnan lui ayant fait entendre que ce qu'il avait à lui dire
ne se pouvait remettre, M. Fouquet sortit de sa chaise en ôtant son
chapeau à demi. En cet état, le sieur d'Artagnan lui dit qu'il avait
ordre du roi de l'arrêter prisonnier. À quoi M. Fouquet ne répondit
autre chose, après avoir demandé à voir cet ordre et l'avoir lu, sinon
qu'il avait cru être dans l'esprit du roi mieux que personne du
royaume\footnote{Ces détails sont d'accord avec les discours que le
  jeune Brienne prête à Fouquet. Voy. \emph{Mémoires de H. L. de
  Loménie}, t. II. p.~200.}, et en même temps il acheva de se découvrir,
et l'on observa qu'il changea plusieurs fois de visage en priant le
sieur d'Artagnan que cela ne fît point d'éclat. Ce qui donna occasion au
sieur d'Artagnan de lui dire qu'il entrât dans la maison prochaine qui
se trouva celle du grand archidiacre, dont M. Fouquet avait épousé la
nièce en premières noces.

«\,En y entrant, il aperçut le sieur Codur, une de ses créatures, à qui
il dit en passant ces mots\,: \emph{À M\textsuperscript{me} du Plessis,
à Saint-Mandé}.

«\,Le sieur d'Artagnan envoya incontinent le sieur Desclavaux donner
avis au roi de ce qui s'était passé et dépêcha un mousquetaire en la
ville d'Ancenis, pour donner ordre au brigadier qu'on y avait envoyé
avec dix mousquetaires le jour précédent d'arrêter tous autres courriers
que ceux de Sa Majesté.

«\,Ensuite le sieur d'Artagnan demanda à M. Fouquet les papiers qu'il
avait sur lui, et les ayant mis en un paquet cacheté, il chargea le
sieur de Saint-Mars, maréchal des logis de la compagnie des
mousquetaires, de les porter au roi avec un billet écrit de sa main, par
lequel il faisait savoir à Sa Majesté qu'aussitôt qu'il aurait fait
prendre à M. Fouquet un bouillon qu'il avait envoyé quérir à la bouche,
et que le sieur Saint-Mars serait de retour auprès de lui, il partirait
pour suivre ses ordres.

«\,En effet, dès que le sieur de Saint-Mars fut de retour et que M.
Fouquet eut pris un bouillon, le sieur d'Artagnan le fit monter dans un
des carrosses du roi, dans lequel entrèrent les sieurs de Bertaud,
gouverneur de Briançon, de Maupertuis et Desclavaux, gentilshommes
servants de Sa Majesté.

«\,La première couchée fut à Houdan, où le sieur d'Artagnan demanda de
la part du roi à M. Fouquet un ordre de sa main au commandant de
Belle-Isle de remettre la place entre les mains de celui que Sa Majesté
y enverrait. Ce que M. Fouquet fit incontinent par un billet qui fut
aussitôt porté au roi par le sieur de Maupertuis.

«\,Le mardi 6 septembre, le sieur d'Artagnan partit de Houdan et alla
coucher à Ingrande, où le roi passa à deux heures après minuit.

«\,Le mercredi 7 septembre, M. Fouquet arriva au château d'Angers, et
fut loger dans le château que le commandant avait remis suivant l'ordre
du roi, entre les mains du sieur d'Artagnan qui retint, pour le garder,
soixante mousquetaires avec les sieurs Saint-Mars et de Saint-Léger,
maréchaux des logis de la compagnie, et renvoya le reste au roi.

«\,Cependant M. Boucherat\footnote{Louis Boucherat, ne le 26 août 1616,
  chancelier de France en 1685, mort le 2 septembre 1699.}, qui, dès le
moment que M. Fouquet avait été arrêté, s'était transporté en la maison
où était M\textsuperscript{me} Fouquet, l'a trouvée gardée par six
mousquetaires. Il entra dans la chambre et lui fit entendre avec
civilité l'ordre que le roi lui avait donne de visiter les papiers de M.
son mari. Elle demanda où il était et s'il ne lui serait pas permis de
l'accompagner. Mais M. Boucherat, qui n'avait rien à lui répondre sur
cela, ne songea qu'à exécuter sa commission. Il fit ouvrir les cassettes
qui étaient dans sa chambre, dans lesquelles il ne rencontra aucuns
papiers. Il entra ensuite dans le cabinet de M. Fouquet, d'où il fit
transporter tout ce qu'il y trouva de papiers. On observa que, dans
cette occasion, M\textsuperscript{me} Fouquet fit paraître beaucoup de
courage, qu'elle ne fit rien d'indécent, qu'elle ne dit rien qui
témoignât de la faiblesse, et même qu'elle ne pleura pas.

«\,Le sieur Pellisson\footnote{Paul Pellisson-Fontanier, ne 1624, mort
  en 1693\,; il a laissé plusieurs ouvrages, et entre autres des
  mémoires composes pour la défense de Fouquet.}, commis de M. Fouquet,
et celui en qui il avait le plus de confiance, fut aussi arrêté dans sa
maison par des mousquetaires que le sieur d'Artagnan y avait envoyés, et
M. Boucherat, s'y étant transporté, se fit représenter ses papiers, qui
furent enfermés dans une malle et portés à M. Le Tellier.

«\,Le sieur Pecquet, médecin de M. Fouquet, et Lavallée, son plus ancien
valet de chambre, s'étant présentés pour le servir, furent enfermés avec
lui sans aucune communication avec les gens du dehors.

«\,M. Fouquet, prisonnier, parut inquiet et abattu pendant les premiers
jours de sa détention. Il mit toutes choses en usage pour gagner ses
gardes et pour avoir des nouvelles\,; mais tout cela fut inutile par les
soins et l'application extraordinaire du sieur d'Artagnan, qui faisait
d'ailleurs à son prisonnier tous les bons traitements dont il se pouvait
aviser. Ce qui n'empêcha pas M. Fouquet de tomber dans une maladie qui
le mit à l'extrémité.

«\,Le 22 novembre, M. d'Artagnan reçut ordre du roi d'envoyer quérir le
sieur Pellisson, qui était prisonnier dans le château de Nantes, et de
le faire conduire dans celui d'Angers par tel nombre de mousquetaires
qu'il aviserait. Ce qui fut exécuté par vingt mousquetaires commandés
par le sieur de Saint-Mars, entre les mains duquel M. le maréchal de La
Meilleraye remit le prisonnier, qui arriva au château d'Angers le 25
dudit mois de novembre.

«\,Le 1er décembre, par un nouvel ordre du roi, le sieur d'Artagnan
conduisit les deux prisonniers au faubourg de Saumur, du côté du pont\,;
le second, il les conduisit à la Chapelle-Blanche\,; le troisième, au
faubourg de Tours, et le quatrième au château d'Amboise\footnote{La
  Fontaine, qui a montré tant de dévouement à Fouquet, parle dans ses
  lettres à sa femme (édit. Lahure, t. II. p.~554) de son voyage au
  château d'Amboise où il visita la chambre qu'avait habitée Fouquet.
  «\,Je demandai, dit-il, à voir cette chambre\,; triste plaisir, je
  vous le confesse\,; mais enfin je le demandai. Le soldat qui nous
  conduisait n'avait pas la clef\,; au défaut je fus longtemps à
  considérer la porte et me fis conter la manière dont le prisonnier
  était gardé. Je vous en ferais volontiers la description\,; mais ce
  souvenir est trop affligeant. Qu'est-il besoin que je retrace / Une
  garde au soin nonpareil\,? / Chambre murée, étroite place, / Quelque
  peu d'air pour toute grâce\,; / Jours sans soleil,Nuits sans
  sommeil\,: / Trois portes en six pieds d'espace\,! / Vous peindre un
  tel appartement, / Ce seroit attirer vos larmes. / Je l'ai fait
  insensiblement\,: / Cette plainte a pour moi des charmes. Sans la nuit
  on n'eût jamais pu m'arracher de cet endroit.\,»}, où il mit M.
Fouquet, son médecin et son valet de chambre à la garde du sieur Talois,
enseigne des gardes du corps, suivant le commandement exprès de Sa
Majesté, et partit, le sixième jour de décembre, d'Amboise pour mener le
sieur Pellisson à la Bastille, où il le mit à la garde du sieur de
Bessemaux, le 12 du même mois.

«\,Peu de temps après, le roi donna ordre aux sieurs de Talois et de
Carrat, préposés à la garde de M. Fouquet, de le mener à Paris dans un
carrosse de louage qui leur fut envoyé à cet effet\,; le sieur de Talois
lui fit entendre ce nouvel ordre dont il parut surpris. Il témoigna le
lendemain qu'on lui avait fait plaisir de le préparer à ce voyage, et
que ce changement lui faisait de la peine. Il demanda même à diverses
reprises au sieur de Talois à quoi ce voyage qui le rapprochait du roi
devait aboutir, et si c'était pour quelque chose de mieux ou de pis. Sur
quoi, le sieur de Talois lui dit quelques bonnes paroles pour le
remettre.

«\,Le jour de Noël, le sieur de Talois le fit monter dans un carrosse où
entrèrent le sieur Pecquet, Lavallée, le sieur de Talois, le sieur
Batine, maréchal de la compagnie des mousquetaires, et les sieurs Bonin
et Blondeau, qui avaient amené le carrosse à Amboise, d'où le prisonnier
fut conduit en la ville de Blois par vingt-six mousquetaires. Il coucha
à l'hôtellerie de \emph{la Galère}. Le second jour, il coucha à
Saint-Laurent-des-Eaux, aux \emph{Trois-Rois\,;} le troisième à Orléans,
au faubourg de Paris, à \emph{la Salamandre\,;} le quatrième à Toury, au
\emph{Grand-Cerf\,;} le cinquième à Étampes\,; le sixième à Corbeil, aux
\emph{Carnaux}, d'où il fut conduit au château de Vincennes le dernier
décembre.

«\,Il dit en passant près de sa maison de Saint-Mandé qu'il y aurait
plus de plaisir à prendre à gauche qu'à droite, mais que puisqu'il avait
été si malheureux que de déplaire au roi il fallait prendre patience.

«\,Il fut accueilli avec beaucoup d'injures dans tous les lieux où il
passa\footnote{Ce fait est confirmé par le journal inédit d'Olivier
  d'Ormesson, IIe partie. fol. 27 r°. D'Artagnan raconta à d'Ormesson
  qu'à Angers, «\,les habitants dirent mille injures à M. Fouquet
  lorsqu'il passa par les rues, et voyant le soin que M. d'Artagnan
  prenait de le garder, ils lui disaient\,: «\,Ne craignez pas qu'il
  sorte\,; car si nous l'avions en nos mains, nous le pendrions
  nous-mêmes.\,» La même haine parut à Tours, et il (d'Artagnan) fut
  obligé d'emmener M. Fouquet dès trois heures du matin pour éviter les
  injures du peuple.\,»}, et quelques soins que les gardes pussent
prendre pour écarter la populace, il fut impossible d'empêcher qu'il
n'entendît les imprécations que l'on faisait partout contre lui. Ce
qu'il supporta avec beaucoup de courage et de résolution.

«\,On le mit dans la première chambre du donjon du château que l'on
meubla avec tous les cabinets qui l'accompagnent de meubles qu'on avait
tirés de sa maison de Saint-Mandé. L'on enferma avec lui les sieurs
Pecquet et Lavallée. Le sieur Talois, avec vingt-quatre mousquetaires,
fut chargé de garder le dedans, et le sieur de Marsac, lieutenant au
gouvernement de Vincennes, et capitaine-lieutenant de la compagnie des
petits mousquetaires, fut chargé de la garde des dehors. Mais ne s'étant
pu accommoder sur l'exécution de leurs ordres, le roi prit résolution de
remettre la garde du prisonnier au sieur d'Artagnan, et lui en donna les
ordres le 3 janvier 1662.

«\,Le lendemain, le sieur d'Artagnan s'étant rendu au donjon du château
de Vincennes, sur les quatre heures du matin, avec cinquante
mousquetaires de sa compagnie, deux maréchaux des logis et le sieur
Bertaud, le sieur de Marsac lui remit la place entre les mains\,; M.
Talois remit aussi la personne de M. Fouquet, son médecin et son valet
de chambre. Depuis ce temps, jusques au jugement du procès de M.
Fouquet, le sieur d'Artagnan a continué cette garde avec tant
d'exactitude que lui seul entrait dans la chambre de M. Fouquet. Il lui
portait toutes les choses nécessaires, sans souffrir que tout autre que
lui eût communication avec M. Fouquet, son médecin et son valet de
chambre. Et cela fit que M. Fouquet, qui avait témoigné beaucoup
d'inquiétude et de curiosité pendant les premiers jours de sa détention,
se voyant si bien renfermé et si soigneusement gardé, perdit l'espérance
de recevoir des nouvelles de ce qui se passait au dehors, et ne pensant
plus qu'à soi-même, on ne l'entendit plus parler que du mépris des
vanités du monde et du bon usage qu'il ferait de son affliction, s'il
plaisait au roi de le reléguer en quelque lieu aux extrémités du
royaume.\,»

\hypertarget{note-vii.-le-tellier-et-son-fils-louvois.}{%
\chapter{NOTE VII. LE TELLIER ET SON FILS
LOUVOIS.}\label{note-vii.-le-tellier-et-son-fils-louvois.}}

Le maréchal de camp Saint-Hilaire, fils du général qui eut le bras
emporté par le boulet qui tua Turenne, a laissé sur le règne de Louis
XIV des Mémoires trop peu consultés. Ces Mémoires, qui ont été imprimés
en quatre volumes, diffèrent du manuscrit conservé à la Bibliothèque
impériale du Louvre. C'est d'après le manuscrit que je cite les deux
portraits de Le Tellier et de son fils Louvois.

«\,M. Le Tellier, qui est mort chancelier de France, avait un bon
esprit, beaucoup de jugement et une grande expérience des affaires,
ayant passé par tous les degrés. D'ailleurs, il allait à ses fins avec
beaucoup d'adresse et excellait en patelinage par-dessus tous les
autres. Il était doucereux comme le miel, et dans le fond, aussi
malfaisant, dangereux et rancunier qu'un Italien. Jamais il ne se
haussait, ni ne se baissait\,; toujours le même visage, et le même air,
aussi affable dans un temps que dans un autre. Ce n'est pas qu'il ne fût
prompt et colère\,; mais il savait prendre son temps. Du reste, il
paraissait fort réglé dans ses mœurs et sa dépense, et la conduite qu'il
a tenue lui a si bien réussi qu'il a fait une grosse maison, et s'est
acquis des richesses immenses, que bien des gens ont attribuées à sa
seule économie, qui tenait beaucoup de l'avarice.

«\,Le caractère de M. de Louvois différait en bien des choses de celui
de son père. Son humeur, qui dominait toujours en lui, était fière,
brusque et hautaine, et sa férocité naturelle était toujours peinte sur
son visage\footnote{Ce sont presque les termes dont s'est servi
  Saint-Simon en parlant de Louvois\,: «\, c'était un homme altier,
  brutal, grossier dans toutes ses manières, comme sa figure le montrait
  bien, etc.\,»Voy. p.~408, note.}, et effrayait ceux qui avaient
affaire à lui. Il était sans ménagement pour qui que ce pût être, et
traitait toute la terre haut la main, et même les princes\,; d'ailleurs
avide, jaloux, rancunier et capable de tout sacrifier pour soutenir son
autorité et ses intérêts. Il avait peu d'étude et de connaissance des
sciences et des arts\,; dans le commencement de sa vie, il fut assez
dissipé par les plaisirs ordinaires à la jeunesse vicieuse, et son
esprit parut lourd et pesant. On a dit, à propos de cela, que M. Le
Tellier, qui connaissait parfaitement l'esprit du roi, eut l'adresse de
l'engager à corriger la conduite de son fils, et à le former à ses
manières, afin qu'il s'y attachât davantage et le retardât comme sa
créature. Ses peines ne furent pas inutiles\,; car, après les premières
façons, l'esprit de ce jeune ministre s'ouvrit et parut excellent, et il
devint si assidu, actif et laborieux, qu'il n'y eut jamais rien de tel.
Le roi en fut si content qu'il eut tout crédit près de lui, et que rien
ne s'y faisait que par son moyen. À quoi j'ajouterai que le roi s'est
piqué depuis, sur cet échantillon, de former ses autres ministres.\,»

\hypertarget{note-viii.-jalousie-de-louvois-contre-seignelay.}{%
\chapter{NOTE VIII. JALOUSIE DE LOUVOIS CONTRE
SEIGNELAY.}\label{note-viii.-jalousie-de-louvois-contre-seignelay.}}

Les documents contemporains confirment pleinement ce que dit Saint-Simon
de la jalousie de Louvois contre Seignelay, jalousie \emph{qui écrasa la
marine}. On retranchait des fonds à la marine pour les prodiguer dans
des fêtes que dirigeait Louvois. La révocation de l'édit de Nantes
enleva un grand nombre de soldats à la flotte, « et des
meilleurs\footnote{« Nos matelots n'étaient pas en grand nombre; la
  religion en avait fait évader une infinité des meilleurs. »
  (\emph{Mémoires de Mme de La Fayette}, année 1689; édit. Petitot,
  p.~90.)}. » Louvois s'opposait aux expéditions qui pouvaient élever la
gloire de son rival\footnote{« M. de Louvois, ministre de la guerre qui
  par son opposition à M. de Seignelay, ministre de la marine, était
  contraire en tout au roi d'Angleterre, s'opposa si fortement à ce
  projet (d'une invasion en Angleterre) que le roi très chrétien
  persuadé par ses raisons n'y voulut pas consentir. » (\emph{Mémoires
  de Berwick;} édit. Petilot, p.~353.)}; il alla jusqu'à faire raser des
places et citadelles situées sur les côtes de l'Océan:

« En 1688, dit Foucault\footnote{Foucault était alors intendant de Caen.
  Son journal inédit est conservé à la Bibliothèque Impériale.
  Saint-Simon parle plusieurs fois de ce Nicolas Foucault et en fait
  l'éloge.}, le roi avait fait travailler à la citadelle de Cherbourg
par M. de Vauban; elle était fort avancée, lorsque M. de Louvois,
\emph{pour donner du chagrin à M. de Seignelay plutôt que pour le bien
de la place}, obtint du roi un ordre pour la faire démolir, aussi bien
que le châtelet de Valognes, sous prétexte que le prince d'Orange, ayant
formé le dessein de faire une descente en Normandie, se saisirait de
cette place. Il était mal informé; car le prince d'Orange pensait à
détrôner son beau-frère et à descendre en Angleterre. La démolition de
Cherbourg était achevée, lorsque je suis venu en basse Normandie, et il
ne m'a resté qu'à régler les comptes des entrepreneurs de la démolition.
»

Foucault ajoute un peu plus loin: « J'ai été retenu à Cherbourg, où je
n'ai trouvé qu'un chaos de débris de tours, de bastions et de murailles
renversées. Il y avait autrefois un château; M. de Vauban, ayant cru le
poste important, le fit enceindre de fortifications régulières, et la
dépense fut considérable; mais à peine furent-elles au cordon que M. de
Louvois, ennemi juré de M. de Seignelay, secrétaire d'État de la marine,
fit comprendre au roi que cette place était commandée par des hauteurs;
que si les Anglais faisaient une descente à la Hougue, ils se rendraient
aisément maîtres de cette place; que le prince d'Orange en avait formé
le dessein et devait être incessamment sur cette côte, en sorte qu'il
obtint du roi que les fortifications seraient entièrement démolies. On
envoya M. d'Artagnan, major des gardes, avec une compagnie de
mousquetaires et d'autres troupes, pour s'opposer à la descente du
prince d'Orange, qui ne songeait pas à nous visiter, mais à passer en
Angleterre, où il était appelé, et où il fut déclaré roi. »

Foucault insiste sur le projet de creuser un port à la Hogue ou la
Hougue (département de la Manche), et, comme Saint-Simon, accuse Louvois
de l'avoir fait échouer:

« Au mois d'octobre 1690, on a proposé au roi de faire un port à la
Hougue, qui est l'endroit le plus propre des côtes de Normandie pour y
tenir un grand nombre de vaisseaux commodément et en sûreté. M. de
Combes, ingénieur, a été commis pour examiner la commodité ou
incommodité, et il a trouvé que c'était l'ouvrage le plus facile et le
plus nécessaire que le roi pût faire pour le salut de ses vaisseaux de
la Manche; mais l'avis n'a pas été agréable à M. de Louvois. »

\hypertarget{note-ix.-mort-de-louvois.}{%
\chapter{NOTE IX. MORT DE LOUVOIS.}\label{note-ix.-mort-de-louvois.}}

Saint-Simon dit qu'on sut par l'ouverture du corps de Louvois qu'il
avait été empoisonné, et il ajoute l'histoire du médecin de ce ministre
qui mourut en désespéré, se déclarant coupable de la mort de son maître.
Ces témoignages, qui paraissent bien positifs, ont été sérieusement
discutés par M. Leroy, bibliothécaire de la ville de Versailles, qui a
inséré dans l'\emph{Union de Seine-et-Oise} ( 9 et 12 juillet 1856) des
notes extraites d'une dissertation du chirurgien Dionis \emph{sur la
mort subite} (Paris, 1710): « Dionis, dit M. Leroy, était chirurgien de
Louvois. Il publia plusieurs ouvrages encore recherchés aujourd'hui par
les observations curieuses qu'elles renferment. Dans un de ces ouvrages,
intitulé: \emph{Dissertation sur la mort subite}, voici comment il
raconte la mort de Louvois:

« Le 16 juillet 1691, M. le marquis de Louvois, après avoir dîné chez
lui et en bonne compagnie, alla au conseil. En lisant une lettre au roi,
il fut obligé d'en cesser la lecture, parce qu'il \emph{se sentait fort
oppressé;} il voulut en reprendre la lecture, mais ne pouvant pas la
continuer, il sortit du cabinet du roi, et s'appuyant sur le bras d'un
gentilhomme à lui, il prit le chemin de la surintendance, où il était
logé.

« En passant par la galerie qui conduit de chez le roi à son
appartement, il dit à un de ses gens de me venir chercher au plus tôt.
J'arrivais dans sa chambre comme on le déshabillait j il me dit: «
Saignez-moi vite, car j'étouffe. » Je lui demandai s'il sentait de la
douleur plus dans un des côtés de la poitrine que dans l'autre; il me
montra la région du cœur, me disant: « Voilà où est mon mal. » Je lui
fis une grande saignée en présence de M. Séron, son médecin. Un moment
après, il me dit: « Saignez-moi encore, car je ne suis point soulagé. »
M. Daquin et M. Fagon arrivèrent, qui examinèrent l'état fâcheux où il
était, le voyant souffrir avec des angoisses épouvantables; il sentit un
mouvement dans le ventre comme s'il voulait s'ouvrir; il demanda la
chaise, et peu de temps après s'y être mis, il dit: « Je me sens
évanouir. » Il se jeta en arrière, appuyé sur le bras, d'un côté de M.
Séron, et de l'autre d'un de ses valets de chambre. Il eut des râlements
qui durèrent quelques minutes, et il mourut.

« On voulut que je lui appliquasse des ventouses avec scarifications, ce
que je fis; on lui apporta et on lui envoya de l'eau apoplectique, des
gouttes d'Angleterre, des eaux divines et générales; on lui fit avaler
de tous ces remèdes qui furent inutiles, puisqu'il était mort, et en peu
de temps; car il ne se passa pas une demi-heure depuis le moment qu'il
fut attaqué de son mal jusqu'à sa mort.

« Le lendemain, M. Séron vint chez moi me dire que la famille souhaitait
que ce fût moi qui en fis l'ouverture. Je le fis en présence de MM.
Daquin, Fagon, Duchesne et Séron.

« En faisant prendre le corps pour le porter dans l'antichambre, je vis
son matelas tout baigné de sang; il y en avait plus d une pinte qui
avait distillé pendant vingt-quatre heures par les scarifications que je
lui avais faites aux épaules; et ce qui est de particulier, c'est
qu'étant sur la table, je voulus lui ôter la bande qui était encore à
son bras de la saignée du jour précédent, et que je fus obligé de la
remettre, parce que le sang coulait; ce qui gâtait le drap sur lequel il
était.

« Le cerveau était dans son état naturel et très bien disposé;
\emph{l'estomac était plein de tout ce qu'il avait mangé à son dîner;}
il y avait plusieurs petites pierres dans la vésicule du fiel; \emph{les
poumons étaient gonflés et pleins de sang;} le cœur était gros, flétri,
mollasse et semblable à du linge mouillé, n'ayant pas une goutte de sang
dans ses ventricules.

« On fit une relation de tout ce qu'on avait trouvé, qui fut portée au
roi après avoir été signée par les quatre médecins que je viens de
nommer, et par quatre chirurgiens, qui étaient MM. Félix, Gervais,
Dutertre et moi.

« * Le jugement certain qu'on peut faire de la cause de cette mort est
l'interception de la circulation du sang; les poumons en étaient pleins,
parce qu'il y était retenu, et il n'y en avait point dans le cœur, parce
qu'il n'y en pouvait point entrer; il fallait donc que ses mouvements
cessassent, ne recevant point de sang pour les continuer; c'est ce qui
s'est fait aussi, et ce qui a causé une mort si subite*. »

\hypertarget{note-x.-conduite-de-louis-xiv-envers-barbezieux.}{%
\chapter{NOTE X. CONDUITE DE LOUIS XIV ENVERS
BARBEZIEUX.}\label{note-x.-conduite-de-louis-xiv-envers-barbezieux.}}

M. Barbier, auteur du \emph{Dictionnaire des Anonymes}, a publié le
mémoire suivant, trouvé dans les papiers du procureur général Joly de
Fleury. Il était adressé par Louis XIV à l'archevêque de Reims,
Charles-Maurice Le Tellier, sur la conduite de son neveu Barbezieux.
Cette pièce avait été connue de Voltaire, qui l'a appréciée en ces
termes: « Quoique écrite d'un style extrêmement néglige, elle fait plus
d'honneur au caractère de Louis XIV que les pensées les plus ingénieuses
n'en auraient fait à son esprit. »

Voici ce mémoire:

« A L'ARCHEVÊQUE DE REIMS.

« Que la vie que son neveu a faite à Fontainebleau n'est pas soutenable;
que le public en a été scandalisé;

« Qu'il a passé tous les jours à la chasse et la nuit en débauche;

« Que les commis se relâchent à son exemple;

« Que les officiers ne sauraient trouver le temps de lui parler; qu'ils
se ruinent pour attendre;

« Qu'il est menteur, toujours amoureux, rôdant partout; peu chez lui;
que le monde croit qu'il ne saurait travailler le voyant partout
ailleurs;

« Le retardement des lettres de Catalogne;

« Qu'il se lève tard, passant la nuit à souper en compagnie souvent avec
les princes;

« Qu'il parle et écrit rudement;

« Que, s'il ne change du blanc au noir, il n'est pas possible qu'il
puisse demeurer dans sa charge;

« Qu'il doit bien examiner ce qu'il doit lui conseiller, après avoir su
de lui ses sentiments;

« Que je serais très fâché de faire quelques changements, mais que je ne
le pourrais éviter;

« Qu'il n'est pas possible que les affaires marchent avec une telle
inapplication;

« Que je souhaite qu'il y remédie, sans que je sois obligé d'y mettre la
main;

« Qu'il est impossible qu'on ne soit trompé en beaucoup de choses,
s'appliquant aussi peu; que cela me doit coûter beaucoup;

« Qu'enfin on ne peut pas plus mal faire qu'il fait, et que cela n'est
pas soutenable;

« Que l'on me reprocherait de souffrir ce qu'il fait, dans un temps
comme celui-ci, où les plus grandes affaires et les plus importantes
roulent sur lui;

« Que je ne pourrais me dispenser de prendre un parti pour le bien de
l'État, et même pour me disculper;

« Que je l'en avertis, peut-être trop tard, afin qu'il agisse de la
manière qui conviendra le plus à sa famille;

« Que je les plains tous et lui en particulier, par l'amitié et l'estime
que j'ai pour lui, archevêque de Reims;

« Qu'il donne toute son application à faire voir à son neveu l'abîme où
il se jette et qu'il l'oblige à faire ce qui conviendra le plus à tout
le monde; que je ne veux point perdre son neveu; que j'ai de l'amitié
pour lui; mais que le bien de l'État marche chez moi devant toutes
choses;

« Qu'il ne m'estimerait pas, si je n'avais pas ces sentiments;

« Qu'il faut finir de façon ou d'autre; que je souhaite que ce soit en
faisant bien son devoir et en s'y appliquant tout à fait; mais qu'il ne
le peut faire qu'il ne quitte tous les amusements qui l'en détournent,
pour ne plus faire que sa charge, qui doit être capable seule de
l'amuser;

« Que cette vie est pénible à un homme de son âge; mais qu'il faut
prendre un parti; et se résoudre à ne manquer à rien de ses devoirs et à
ne rien faire qu'il puisse se reprocher à lui-même;

« Qu'il faut qu'il ferme la bouche à tout le monde par sa conduite, et
qu'il me fasse voir qu'il ne manque en rien dans son emploi, qui est
présentement le plus considérable du royaume.

« Louis. »

\emph{Observations de l'archevêque de Reims sur ce mémoire}.

« Le roi a écrit ce mémoire de sa main à Fontainebleau, où je n'avais
pas l'honneur d'être à la suite de Sa Majesté; j'étais à Reims.

« Le roi revint de Fontainebleau à Versailles, le vendredi 28 octobre
1695. Je m'y rendis samedi 29, à midi. Sa Majesté m'appela dans son
cabinet en sortant de table; elle m'y donna ce mémoire, dont j'ai fait
l'usage qui convenait. J'en ai rendu l'original au roi à Marly, vendredi
11 novembre; j'en ai fait cette copie avec la permission de Sa Majesté,
et je la garderai toute ma vie, comme un monument du salut de ma
famille, si mon neveu profite, comme je l'espère, de cet avertissement,
ou du moins comme une marque de la bonté du roi pour moi, qui m'a
pénétré d'une reconnaissance si vive, qu'elle durera, quoi qu'il arrive,
autant que je vivrai. »

L'archevêque avait écrit en tête du mémoire: « J'ordonne, mon cher
neveu, que ce mémoire vous soit remis après ma mort; je vous conjure de
le garder pendant toute votre vie. \emph{Signé:} Archevêque-duc de
Reims.»

\hypertarget{note-xi.-muxe9moire-de-marinier-commis-des-buxe2timents-du-roi-sous-colbert-louvois-et-mansart.}{%
\chapter{NOTE XI. MÉMOIRE DE MARINIER, COMMIS DES BÂTIMENTS DU ROI, SOUS
COLBERT, LOUVOIS ET
MANSART.}\label{note-xi.-muxe9moire-de-marinier-commis-des-buxe2timents-du-roi-sous-colbert-louvois-et-mansart.}}

Je dois à l'obligeance du savant bibliothécaire de Versailles, M. Leroy,
la copie de ce mémoire, qui donne les chiffres exacts des dépenses de
Louis XIV à Versailles et à Marly, de 1664 à 1690. Saint-Simon ne parle
que d'une manière générale de ce palais \emph{si immensément cher} ; et
quant à Marly, il se borne à dire « que Versailles n'a pas coûté Marly.
» On verra par la suite du mémoire la totalité des dépenses de Louis XIV
en bâtiments jusqu'en 1690.

\emph{À Monseigneur, Monseigneur Hardouin Mansart, chevalier de l'ordre
de Saint-Michel, conseiller du roi en ses conseils, surintendant et
ordonnateur général des bâtiments, jardins, tapisseries, arts et
manufactures de Sa Majesté}.

« Monseigneur,

« Le manuscrit que je prends la liberté de vous offrir n'a point encore
vu le jour. Il attendait son légitime protecteur pour paraître; le rang
que vous tenez aujourd'hui, Monseigneur, n'est pas tant l'effet de la
libéralité du prince, que de sa justice et de son discernement; les
superbes édifices dont vous êtes le surintendant et ordonnateur général
tiennent tout leur éclat et toute leur magnificence de la grandeur et de
la beauté d'un génie inconnu jusqu'à vous; mais il n'en fallait pas
moins pour remplir les grandes idées du plus grand prince du monde. Je
n'entreprendrai ici, Monseigneur, ni l'éloge du roi que vous servez, ni
le votre, l'un et l'autre sont fort au-dessus de moi; j'ose seulement
vous supplier très humblement, Monseigneur, de vouloir agréer un travail
qui est le fruit d'un autre infiniment plus étendu que feu mon père a
fait sous les ordres de feu Mgr Colbert. Vous y verrez, Monseigneur, de
grandes choses en peu d'espace, et en peu de temps; j'ai tout pris sur
le mien pour ménager le vôtre! heureux si, par cet essai, je puis vous
persuader du profond respect avec lequel je suis,

« Monseigneur,

« Votre très humble, très obéissant et très soumis serviteur,

« G. M. »

\emph{Mémoires curieux tirés des comptes des bâtiments du roi depuis et
compris l'année 1664, que feu M. Colbert fut surintendant des bâtiments,
et que les dépenses commencèrent à devenir considérables, jusques et
compris l'année 1690, que Sa Majesté les a retranchés à cause de la
guerre}.

Le plan qu'on s'est proposé dans cet ouvrage a été de supputer la
dépense qui a été faite pour chaque maison royale en chacune année, et
composer un total de ce que chaque maison a coûté au roi pendant les
vingt-sept années de ces mémoires. Et à l'égard de Versailles seulement,
on a encore distingué ce qui a été dépensé pour chaque nature d'ouvrage.

Ensuite de ces chapitres particuliers, on a composé un chapitre général
qui contient le total des dépenses que le roi a faites dans ses
bâtiments depuis l'année 1664 jusqu'en 1690 inclusivement.

On aurait pu embellir cet ouvrage, très sommaire dans sa disposition, de
plusieurs traits d'histoire qui l'auraient sans doute rendu agréable;
mais persuadé que M. Félibien n'omettra rien dans son Histoire des
Maisons royales, on n'a pas voulu le prévenir.

On a cru néanmoins qu'il était indispensable de donner une idée générale
de chaque maison royale, avant d'exposer la dépense qui y a été faite,
et cela pour satisfaire en quelque sorte la curiosité des personnes
moins instruites, entre les mains de qui cet ouvrage pourrait tomber
dans la suite.

CATALOGUE DE TOUTES LES MAISONS ROYALES ET ÉDIFICES APPARTENANT À SA
MAJESTÉ.

~

{\textsc{Le château de Versailles et ses dépendances, qui sont:
Trianon.}} {\textsc{-Clagny.}} {\textsc{- Saint-Cyr.}} {\textsc{- Les
églises de Versailles.}} {\textsc{- La machine de la Seine.}} {\textsc{-
L'aqueduc de la rivière d'Eure.}} {\textsc{- Noisy.}} {\textsc{-
Moulineaux.}} {\textsc{- Le château de Saint-Germain en Laye et le
Val.}} {\textsc{- Le château de Marly.}} {\textsc{- Le château de
Fontainebleau.}} {\textsc{- Le château de Chambord.}} {\textsc{- Le
Louvre et les Tuileries.}} {\textsc{- L'Arc de triomphe à Paris.}}
{\textsc{- Le bâtiment et l'église des Invalides.}} {\textsc{- La place
royale de l'hôtel de Vendôme, et couvent des Capucines.}} {\textsc{- Le
Val-de-Grâce à Paris.}} {\textsc{- Le couvent de l'Annonciade de
Meulan.}} {\textsc{- Le canal des communications des mers.}} {\textsc{-
La manufacture des Gobelins et de la Savonnerie.}} {\textsc{- Les
manufactures établies en plusieurs villes de France.}} {\textsc{- Les
Académies de Paris et celle de Rome.}} {\textsc{- Le Palais-Royal (Sa
Majesté l'a donné en propre à Mgr le duc de Chartres, pour partie de la
dot de Mme la duchesse de Chartres).}} {\textsc{- La Bastille.}}
{\textsc{- L'Arsenal.}} {\textsc{-L'Enclos du palais.}} {\textsc{- Le
Châtelet.}} {\textsc{- La Monnaie.}} {\textsc{- La Bibliothèque.}}
{\textsc{-Le Jardin-Royal.}} {\textsc{- Le Collège de France.}}
{\textsc{- L'hôtel des Ambassadeurs.}} {\textsc{-La Pompe du pont Neuf}}
{\textsc{-La Tournelle.}} {\textsc{- L'aqueduc d'Arcueil.}}
{\textsc{-L'Hôpital général.}} {\textsc{- La Pépinière du Roule.}}
{\textsc{- Le château de Madrid.}} {\textsc{-La {[}Muette{]} de
Boulogne.}} {\textsc{- Le château de Vincennes.}} {\textsc{- Le château
de Saint-Léger.}} {\textsc{- Le château de Limours.}} {\textsc{- Le
château de Monceaux.}} {\textsc{- Le château de Compiègne.}} {\textsc{-
Le château d'Amboise.}} {\textsc{- Le château de Marimont.}} {\textsc{-
Le Jardin du Roi, à Toulon.}} {\textsc{- Le château et domaine de
Villers-Cotterêts a été donné à S. A. R. Monsieur, en augmentation
d'apanage.}} {\textsc{- Château-Thierry, engagé à M. le duc de
Bouillon.}} {\textsc{- Le palais du Luxembourg, que le roi a acquis
depuis la mort de Mademoiselle.}} {\textsc{- Le château de Meudon et ses
dépendances, qui appartient à Monseigneur, au moyen de l'échange qu'il
en a fait avec le château de Choisy, qui lui a été légué par
Mademoiselle.}}

~

CHAPITRE PREMIER.

Château de Versailles et ses dépendances.

Le château de Versailles, et ses dépendances, surpasse toutes les idées
que l'on en peut donner; aucun prince de l'Europe n'a porté la dépense
aussi loin que le roi, pour se faire une demeure digne de la majesté
royale, et le succès ne pouvait achever plus parfaitement de couronner
la grandeur de l'entreprise. Ce château est situé sur une élévation qui
commande à tous les environs. Ses aspects sont d'un côté sur Paris, de
l'autre sur les jardins. Aux deux côtés du château sont les deux ailes
eu arrière-corps qui s'étendent du côté du nord et du midi, dont les
vues sont sur les jardins. De quelque côté qu'on envisage cet édifice,
tout y est surprenant, tout y est admirable; on y trouve plus qu'on ne
peut souhaiter; des appartements superbes et commodes, des logements
infinis, des jardins, des fontaines dont les beautés toutes différentes
tiennent plutôt de l'enchantement que de la nature qui n'a jamais rien
produit de si extraordinaire.

Aux deux extrémités d'un canal qui se partage en deux-bras, sont la
Ménagerie et Trianon. La Ménagerie est remplie de ce qu'il y a de plus
rares animaux dans le monde, recherchés avec un soin et une dépense
extraordinaires. Trianon est un palais où le marbre est plus commun que
la pierre, où tout est brillant et splendide; c'est un séjour de repos
et de plaisir où le roi va se promener avec très peu de monde.

Au bout de la grande aile droite du château, en entrant par l'avenue de
Paris, est un grand réservoir, appelé le Château d'Eau, où se rendent
les eaux élevées par la machine de Marly, duquel réservoir elles se
communiquent dans toute la fontaine du petit parc.

Au bout et au-dessous de l'aile gauche est l'Orangerie, dont la
structure est si noble et si magnifique, qu'on est toujours surpris
lorsqu'après l'avoir considérée par dehors, on en examine le dedans.
Jamais entreprise ne fut plus hardie et mieux exécutée que celle de ce
bâtiment.

En sortant des jardins par le milieu du château, vous voyez en face la
principale avenue, et des deux côtés la grande et la petite écurie du
roi; deux édifices pareils en tout dont la beauté attire la curiosité de
tous ceux qui ont du goût pour l'architecture.

Plus loin est le Chenil, et plusieurs autres bâtiments dépendant du
château.

À côté droit du château, dans le même aspect, sont encore plusieurs
grands édifices, savoir:

Derrière le Grand-Commun est le couvent des Récollets que Sa Majesté a
fait bâtir à neuf.

Dans le même alignement du Grand-Commun, en descendant du côté du parc,
est la surintendance des bâtiments, maison très belle et très commode,
destinée pour le logement de M. le surintendant des bâtiments.

Plus loin, du même côté, est le potager du roi, jardin séparé de tous
les autres, dont la culture et la fertilité surpassent tout ce que l'on
en pourrait dire.

De l'autre côté de la ville est la Paroisse que Sa Majesté a fait
construire de fond en comble, aussi bien que le logement des Pères de la
Mission, par qui elle est desservie aux dépens du roi, avec toute la
décence et l'exactitude possibles. C'est un des plus considérables
édifices de la dépendance du château.

Plus loin, du même côté, est le château de Clagny, maison de plaisance
très belle et très agréable, soit par la régularité de l'architecture,
soit par la distribution des appartements et la disposition des jardins.
Elle coûte au roi plus de deux millions.

Au bord de la Seine, sur le chemin de Saint-Germain en Laye, est la
machine de Marly qui élève les eaux de la rivière jusqu'au sommet d'une
tour bâtie sur une montagne. De cette tour, les eaux sont conduites par
des aqueducs et des conduites de fer de fonte aux jardins de Versailles
et de Marly. Cette seule machine demanderait une description
particulière, si c'était le dessein de cet ouvrage; mais on peut juger
de sa beauté et de son succès par l'abondance des eaux qu'elle fournit à
Versailles. On verra ci-après qu'elle coûte au roi trois millions sept
cent mille livres, sans y comprendre les remboursements des héritages
acquis pour le passage des eaux, et aussi sans les conduites de fer de
fonte qui sont confondues avec celles de Versailles.

Quoique le roi ait dépensé près de neuf millions pour la construction
des aqueducs qui devaient conduire les eaux de la rivière d'Eure, de
Maintenon à Versailles, comme ces aqueducs ne sont pas dans leur
perfection, ils ne demandent pas une plus ample description.

La royale maison de Saint-Cyr, dont les dépenses sont confondues avec
celles de Versailles, comme en étant une dépendance, mérite une plus
particulière attention, la piété, la charité, la religion ont été les
bases de cette fondation royale qui procure tous les jours un asile
honorable à un grand nombre de jeunes demoiselles, qui, pourvues des
avantages de la naissance, se trouvent dénuées de ceux de la fortune; il
faut faire preuve\footnote{Preuve de noblesse. Voy. l'\emph{Histoire de
  Saint-Cyr} par M. Théoph. Lavallée.} pour y entrer.

Je n'ai rien dit de la chapelle du château de Versailles, parce qu'elle
n'est point encore bâtie. On y travaille actuellement. Sans doute la
piété du roi n'omettra rien pour la rendre cligne, autant qu'elle le
peut être de la majesté du Dieu qu'elle adore avec tant de sincérité et
de zèle.

Les dépenses qui ont été faites aux châteaux de Noisy et Moulineaux sont
confondues avec celles de Versailles, et ne méritent pas d'attention.

Voilà je crois l'idée la plus sommaire qu'on puisse donner du château de
Versailles, et de ses principales dépendances. Un volume entier ne
suffirait pas pour faire la description exacte des dedans et de chaque
lieu en particulier, quand on n'entreprendrait que de rendre aux arts la
gloire qu'ils s'y sont acquise, sans oser parler des actions héroïques
de notre auguste monarque qui y sont représentées de toutes parts; leur
nombre et leur suite glorieuse ont épuisé nos plus célèbres génies;
l'histoire, toute féconde qu'elle est, aura peine à les rendre sensibles
à la postérité; ce n'est point mon intention d'essayer de la prévenir.

\emph{Dépenses du château de Versailles par année}.

Année.

livres.

sols.

den.

1664

834 037

2

6

1665

783 673

4

»

1666

526 954

7

»

1667

214 300

18

»

1668

618 006

5

7

1669

1 238 375

7

»

1670

1 996 452

12

4

1671

3 396 595

12

6

1672

2 802 718

1

5

1673

847 004

3

10

1674

1 384 269

10

3

1675

1 933 755

8

1

1676

1 348 222

10

10

1677

1 628 638

11

4

1678

2 622 655

3

10

1679

5 667 331

17

»

1680

5 839 761

19

8

1681

3 854 382

2

»

1682

4 235 123

8

7

1683

3 714 572

5

11

1684

5 762 092

2

8

1685

11 314 281

10

10

1686

6 558 210

7

9

1687

5 400 245

18

»

1688

4 551 596

18

2

1689

1 710 055

10

»

1690

368 101

10

1

Somme totale des dépenses du château de Versailles et dépendances.

81 151 414

9

2

Quatre-vingt-un millions, cent cinquante-un mille quatre cent quatorze
livres, neuf sols, deux deniers.

Dans ce total de dépenses de Versailles et dépendances, j'ai compris les
achats de plomb et de marbre en entier, quoiqu'on ait pu en prendre
quelques parties pour d'autres maisons royales; mais j'ai compensé cela
avec plusieurs autres dépenses pour Versailles employées dans d'autres
chapitres des comptes, sous des titres généraux dont il était malaisé de
les distraire, et je crois que la compensation peut être juste.

Après avoir vu en gros le total des dépenses de Versailles et ses
dépendances, il m'a semblé qu'il serait assez curieux de voir séparément
ce qui a été dépensé pour chaque nature d'ouvrage et de dépense, et le
montant de chacune pour les vingt-sept années de ces mémoires.

On verra aussi les dépenses de Clagny, la machine de Marly et la rivière
d'Eure qui seront distinguées des autres dépenses chacune en un article,
quoique comprise dans le total.

\emph{Dépenses de Versailles par chapitres}.

Maçonnerie de Versailles et ses dépendances, compris Trianon, Saint-Cyr,
et les églises de Versailles pendant lesdites 27 années.

livres

21 186 012

sol.

4

den.

1

Charpenterie et bois

2 553 638

1

5

Couvertures

718 679

16

9

Plomberies et achats de plomb

4 558 077

2

6

Menuiserie et marqueterie

2 666 422

2

»

Serrurerie et taillanderie

2 289 062

3

9

Vitrerie

300 878

10

7

Glaces et miroirs

221 631

1

6

Peintures et dorures, sans les achats de tableaux

1 676 286

11

8

Sculptures sans les achats des antiques

2 696 070

6

9

Marbreries et achats de marbres

3 043 502

5

8

Bronze, fonte et cuivre

1 876 504

6

3

Tuyaux de fer, de fonte, compris ceux de la machine

2 265 114

15

8

Pavé, carreau et ciment

1 267 464

13

»

Jardinages, fontaines et rocailles

2 338 715

15

8

Fouilles de terres et convoi

6 038 035

1

10

Journées d'ouvriers

1 381 701

16

8

Diverses et extraordinaires dépenses

1 799 061

12

10

Château de Clagny et Glatigny dépendants de Versailles, sans les
acquisitions de terre\footnote{Il y a eu environ 300 000 liv. dépensées
  pour Clagny, qui sont confondues avec les dépenses de Versailles
  depuis l'année 1682.}

2 074 592

9

5

Machine de Marly, sans les conduites et acquisitions

3 674 864

8

8

Travaux de la rivière d'Eure et de Maintenon, sans les acquisitions

8 612 995

1

»

Remboursements de terres et héritages pris pour le château et
dépendances de Versailles susmentionnées

5 912 104

1

10

Semblable au total précédent par années

81 151 414

9

2

\emph{Autres dépenses pour Versailles}.

Outre toutes les grandes dépenses qui viennent d'être expliquées, il en
a été fait beaucoup d'autres très considérables pendant lesdites 27
années, pour l'embellissement de Versailles et de Trianon.

Voici les plus considérables:

Pour les achats de tableaux anciens et figures antiques de tous les
grands maîtres

livre

509 073

sol.

8

den.

»

Pour les étoffes d'or et d'argent payées sur le fonds des bâtiments

1 075 673

2

6

Pour les grands ouvrages d'argenterie, outre ceux payés par le trésorier
de l'argenterie\footnote{Tous ces grands ouvrages d'argenterie ont été
  portés à la Monnaie pendant la dernière guerre. (Note de l'auteur.)}

3 245 759

10

6

Pour le cabinet des médailles, cristaux, agates, et autres raretés, dont
le roi a acheté les six dernières années de ces mémoires pour

556 069

»

»

Pour les appointements des inspecteurs et préposés auxdits bâtiments et
travaux de Versailles et ses dépendances; gratifications aux contrôleurs
et autres, a été payé pendant lesdites 27 années environ

1 000 000

»

»

Total de ces dernières dépenses

6 386 574

15

2

Et le total précédent

81 151 414

9

2

Total général des dépenses de Versailles

87 537 989

4

4

Quatre-vingt-sept millions, cinq cent trente-sept mille, neuf cent
quatre-vingt-neuf livres, quatre sols, quatre deniers.

En sorte que si l'on joignait à ce total les autres dépenses qui ont été
faites pour les meubles, les grands cabinets, les grands ouvrages
d'argenterie et autres qui n'ont point été payés sur les fonds des
bâtiments, on trouverait que Versailles et ses dépendances coûtent au
roi plus de cent millions, sans les entretènements, dont ceux qui sont
réglés montent à environ deux cent mille livres, et qui ne le sont pas à
plus de trois cent mille livres.

Voici quels sont les entretènements réglés de Versailles et de ses
dépendances.

Les couvertures

livre

7 500

sol.

»

den.

»

Les jardins de Versailles et Trianon, compris les marbres

33 416

»

»

Le potager de Versailles

18 000

»

»

Les fontaines, rocailles et cuivre

19 780

»

»

Les tuyaux de fer de fonte

10 000

»

»

Les figures et sculptures de marbre

1 695

»

»

Menus entretènements au dehors

2 286

»

»

Gages des officiers et matelots du canal

35 970

»

»

Les jardins de Clagny

10 200

»

»

Les entretiens de la machine de Marly

60 000

»

»

TOTAL

198 847

»

»

Nota. Les entretiens ci-dessus peuvent avoir été augmentés de quelque
chose depuis que ces calculs ont été faits; mais cela n'est pas assez
considérable pour être réformé.

CHAPITRE II.

Château de Saint-Germain en Laye et le Val.

Cette maison, illustrée par la naissance du roi, est très ancienne; elle
consiste en deux châteaux, l'un vieil, l'autre neuf. Le vieil château
est beaucoup plus beau et mieux bâti que le neuf. Ils ne sont séparés
l'un de l'autre que d'une grande basse-cour, qui pourrait servir de
manège.

Le vieil château est entièrement isolé, d'une forme assez irrégulière.
Cinq gros pavillons en font le principal ornement. Un balcon de fer
règne dans toute la circonférence du château, à la hauteur des
principaux appartements qui sont très vastes. Ce château a pour
principal aspect les jardins et la forêt; et le château neuf a sa
principale vue sur la rivière de Seine. Le roi, qui y a séjourné très
longtemps, y a fait faire des augmentations considérables. C'est une
demeure toute royale, et quoique la cour n'y habite pas actuellement, ce
ne laisse pas d'être un des plus beaux lieux des environs de Paris, pour
sa situation naturelle.

Le Val est un jardin dépendant de Saint-Germain que Sa Majesté fait
entretenir avec soin, et qui produit une infinité de beaux fruits dans
toutes les saisons, surtout des précoces.

Je ne dis rien des autres dépendances de Saint-Germain, crainte
d'ennuyer.

\emph{Dépenses des châteaux de Saint-Germain en Laye et dépendances par
années}.

Année.

livres.

sols.

den.

1664

193 767

13

6

1665

179 478

14

9

1666

59 124

11

6

1667

56 235

8

4

1668

120 271

18

3

1669

515 214

19

»

1670

597 429

1

4

1671

361 020

11

11

1672

208 516

13

»

1673

97 379

4

3

1674

112 168

19

11

1675

130 306

18

2

1676

176 118

14

10

1677

194 303

14

4

1678

196 770

5

1679

447 401

14

2

1680

607 619

9

2

1681

279 509

9

2

1682

662 826

13

4

1683

460 995

9

8

1684

380 218

19

»

1685

189 598

»

7

1686

47 618

4

5

1687

50 450

18

10

1688

152 950

2

1

1689

33 176

13

6

1690

25 388

15

3

Somme totale

6 455 561

18

»

Six millions, quatre cent cinquante-cinq mille, cinq cent soixante-une
livres, dix-huit sols.

CHAPITRE III.

Château de pavillons de Marly commencés en 1670.

Le château de Marly est situé dans un vallon à un quart de lieue de
Saint-Germain en Laye. Il est composé: 1° d'un gros pavillon carré, qui
est la demeure du roi. Le pavillon est isolé, situé sur le lieu le plus
éminent, et l'on y monte par plusieurs degrés, en sorte qu'il commande à
huit autres pavillons. Ces huit pavillons, aussi isolés, forment une
espèce d'avenue spacieuse au Pavillon-Royal dans les jardins, et n'ont
de communication, les uns avec les autres, que par des berceaux de fer
sur lesquels on a fait plier des arbres qui les couvrent.

Les quatre faces de tous ces pavillons sont peintes à fresque,
d'ornements d'architecture, couverts en terrasses, avec des vases sur
les angles et au-dessus des pilastres.

Le Pavillon-Royal consiste au dedans en quatre vestibules au
rez-de-chaussée, où l'on entre par les quatre faces dudit pavillon. Ces
quatre vestibules conduisent à un grand salon de toute la hauteur du
pavillon, et qui en fait le centre, et dans les quatre angles sont
quatre appartements, qui ont leurs entrées et sorties sur ces
vestibules. Au-dessus de ces quatre appartements, il y en a encore
d'autres plus petits dégagés par un corridor qui tourne autour du dôme
du grand salon.

Dans ce château tous les agréments et les commodités de la vie sont
rassemblés avec tant de soin, d'art et de propreté, qu'il n'y reste rien
à désirer.

Les autres pavillons sont occupés chacun par une des personnes de la
cour, à qui le roi fait l'honneur de les nommer pour être de ses
parties.

La chapelle et le corps de garde sont détachés du château et forment
deux pavillons aux deux côtés de la principale entrée.

Les jardins sont très agréables, surtout dans la saison des fleurs, par
la diversité et l'abondance qui s'y en trouvent.

Les fontaines et les cascades y sont en très grand nombre et très
belles, et depuis peu Sa Majesté a fait encore tomber une cascade en
forme de rivière du haut de l'allée du derrière du château, d'où elle se
décharge dans toutes les autres fontaines des jardins. Je n'ai point
supputé la dépense de cette nouvelle rivière, pour ne point innover aux
calculs de ces mémoires On estime qu'elle passe cent mille écus.

Le roi embellira tous les jours cette maison de plaisance qu'il aime
beaucoup, et qui passerait dans un autre pays pour un chef-d'oeuvre de
l'art et de la nature en l'état qu'elle est. On prétend que c'est Sa
Majesté qui en a donné les principales idées; ce qui est de vrai, c'est
qu'elle est très singulière, et qu'elle ne ressemble à aucune autre
maison royale.

\emph{Dépenses du château et pavillons de Marly}.

Année.

livres.

sols.

den.

1679

470 764

»

11

1680

489 002

17

1

1681

304 881

14

3

1682

305 628

9

11

1683

450 708

2

»

1684

478 872

4

11

1685

676 046

18

»

1686

443 153

6

»

1687

249235

2

5

1688

293 062

4

2

1689

231 807

»

10

1690

108 117

11

9

Somme totale

4 501 279

12

3

Quatre millions, cinq cent un mille, deux cent soixante-dix-neuf livres,
douze sols, trois deniers.

CHAPITRE IV.

Château de Fontainebleau.

Le château est très ancien et très digne d'avoir si souvent fait la
demeure de nos rois. Rien n'est plus agréable que la situation, voisin
d'une forêt et au milieu des plus belles eaux du monde; d'où ce château,
comme on sait, a pris son nom de la \emph{Fontaine-Belle-Eau}, dont la
maison se conserve encore actuellement.

Rien n'est plus charmant que la diversité des vues de ce château. De
nouveaux jardins et de nouveaux canaux offrent de tous côtés des
perspectives toutes différentes. La chapelle y est magnifique, et
desservie par les révérends pères de la Très-Sainte Trinité. Les
plaisirs de la chasse y sont les plus ordinaires et les plus agréables.
Quoique ce château soit très illustre dans sou origine, il l'est devenu
davantage encore par les augmentations et les embellissements que Sa
Majesté y a fait faire, dont on pourra juger par la dépense qui suit.

\emph{Dépenses du château de Fontainebleau}.

Année.

livres.

sols.

den.

1664

339 251

16

»

1665

107 159

18

»

1666

37 200

8

8

1667

27 820

15

6

1668

19 827

»

5

1669

39 396

»

»

1670

23 106

15

3

1671

58 504

6

1

1672

56 560

12

10

1673

24 425

11

1

1674

66 145

17

»

1675

61 670

17

1

1676

36 052

19

»

1677

33 029

18

6

1678

394 509

15

1

1679

264 417

15

1

1680

204 463

»

8

1681

188 886

19

3

1682

80 019

5

6

1683

98 881

11

8

1684

65 967

1

»

1685

220 216

8

7

1686

92 246

5

3

1687

113 014

9

2

1688

87 988

7

2

1689

31 109

5

4

1690

21 853

14

3

Somme totale

2 773 746

13

5

Deux millions, sept cent soixante-treize mille, sept cent quarante-six
livres, treize sols, cinq deniers.

Ne sont compris en ce total les gages d'officiers, et les entretènements
réglés suivant les états.

CHAPITRE V.

Château de Chambord.

Le château est très ancien, bien bâti, bien situé, et dans un très bon
pays de chasse. Son éloignement est cause que le roi il y va pas
souvent. Sa Majesté n'a pas laissé d'y faire de temps en temps des
augmentations et des dépenses assez considérables, comme il suit:

Année.

livres.

sols.

den.

1664

26 936

5

»

1665

6 000

»

»

1666

11 021

2

»

1667

3 496

3

6

1668

12 164

15

6

1669

57 739

12

»

1670

79 367

5

»

1671

16 000

»

»

1672

532

»

»

1673

3 000

1674

6 000

1675

3 000

1676

3 000

1677

3 000

1678

3 795

10

»

1679

4 500

1680

72 200

1681

127 870

9

7

1682

11 667

16

6

1683

196 350

15

»

1684

38 766

1

»

1685

445 770

9

5

1686

14 980

13

»

1687

54 558

15

5

1688

8 197

4

4

1689

8 036

2

9

1690

7 750

16

5

Somme totale

1 225 701

16

5

Douze cent vingt-cinq mille sept cent une livres, seize sols, cinq
deniers.

CHAPITRE VI.

Louvre et les Tuileries.

Le Louvre n'étant point bâti, on n'a fait mention, dans ces mémoires,
des dépenses qui y ont été faites que pour ne rien omettre. Il serait
assez difficile de faire une description agréable de ce qui est
commencé. Le dessin n'en paraît pas encore dans tout son jour; on croit
même que si les vœux de la capitale du royaume étaient écoutés, et qu'il
plut à Sa Majesté de s'y faire bâtir un palais, on prendrait de nouveaux
alignements et de nouveaux dessins. Tout ce que l'on peut dire, c'est
que rien ne paraît plus engageant que la situation de l'emplacement du
Louvre, dans le plus bel endroit de la ville, ayant la rivière de Seine
pour canal, et une étendue infinie pour les jardins et parcs du côté de
la campagne.

La galerie du Louvre est occupée par ce qu'il y a de plus habiles gens
dans les arts, que le roi loge gratis. C'est une marque de distinction
pour eux.

Le palais des Tuileries n'est point habité, quoique très logeable. Sa
façade est très agréable sur le jardin des Tuileries, dont on ne peut
rien dire qui ne soit connu de tout le monde. Ce jardin passe dans toute
l'Europe pour un des mieux entendus, et la plus agréable promenade que
l'on pût souhaiter. C'est un des principaux ornements de la ville de
Paris, aussi coûte-t-il au roi plus de vingt mille livres par an à
entretenir.

\emph{Dépenses du Louvre et des Tuileries à commencer en l'année 1664,
suivant l'ordre de ces mémoires, n'ayant point eu de connaissance de
celles faites les années précédentes, qui ne peuvent pas être bien
considérables}.

Année.

livres.

sols.

den.

1664

1 059 422

15

»

1665

1 110 685

10

2

1666

1 107 973

7

8

1667

1 536 683

8

2

1668

1 096 977

3

11

1669

1 203 781

3

9

1670

1 627 393

19

11

1671

946 409

3

4

1672

213 653

3

1

1673

58 135

18

»

1674

159 485

8

11

1675

63 160

6

6

1676

42 082

14

6

1677

99 67

19

10

1678

119 875

12

8

1679

163 581

9

»

Somme totale

10 608 969

4

5

Dix millions, six cent huit mille, neuf cent soixante-neuf livres,
quatre sols, cinq deniers.

Depuis l'année 1679, il n'a point été fait aucunes dépenses
considérables au Louvre et Tuileries; c'est pourquoi je n'en fais point
de mention.

CHAPITRE VII.

Arc de Triomphe à Paris, commencé en 1699.

Le dessin de cet édifice est superbe, et tient beaucoup de la grandeur
romaine. On en a vu le modèle en plâtre, et on en a jeté les fondements
en pierre, dont les piles sont élevées jusqu'à la hauteur des socles qui
doivent porter les piédestaux des colonnes. Il serait à souhaiter que
cet arc de triomphe fût conduit à sa perfection; il serait d'un grand
ornement à la ville, surtout dans les entrées publiques.

Année.

livres.

sols.

den.

1669

46 278

2

»

1670

99 334

6

»

1671

102 244

3

6

1672

Néant

1673

Néant

1674

14 225

»

»

1675

14 690

12

»

1676

8 900

»

»

1677

41 863

16

6

1678

76 651

11

8

1679

80 676

4

5

1680

12 601

10

9

1681

16 288

11

3

Somme totale

513 755

18

1

Depuis l'année 1681, il n'a été fait aucune dépense audit arc de
triomphe, si ce n'est 1696, pour le parfait payement du modèle et des
fondations de pierre en l'état qu'elles sont. On peut juger, par cette
dépense, de ce que cet édifice pourrait coûter s'il était élevé avec ses
ornements.

CHAPITRE VIII.

Observatoire à Paris, commencé en 1667.

Cet édifice, construit en forme de tour pour observer les astres, est
bâti sur le terrain le plus éminent de Paris, au dehors du faubourg
Saint-Jacques, et commande à toute la ville. Là sont loges ce qu'il y a
de plus célèbres astronomes et mathématiciens, à qui Sa Majesté fournit
toutes sortes de lunettes d'approche et d'instruments de mathématiques
nécessaires pour l'exercice de leur science. Le dessus de l'édifice est
une terrasse pavée de cailloux; l'on y dresse des lunettes selon le
besoin.

Comme ce terrain est au milieu des carrières, on a fait des descentes
qui conduisent dans des voûtes naturelles si profondes et si étendues,
qu'on aurait peine à ne s'y pas égarer sans guide; les lumières mêmes ne
peuvent pas résister à l'humide fraîcheur qui y domine; ou n'y peut
aller qu'avec des flambeaux.

Cet édifice renferme encore beaucoup d'autres singularités qui
demanderaient un trop long détail.

Outre l'édifice de pierre, on a encore fait apporter et dresser à côté
la tour de bois qui était à la machine de Marly, avant qu'elle fût
construite en pierre. Cette tour de bois est encore plus élevée que
l'Observatoire, et par conséquent très utile pour l'observation des
astres.

\emph{Dépenses de l'Observatoire}.

Année.

livres.

sols.

den.

1667

57 758

4

»

1668

99 744

3

»

1669

135 293

6

»

1670

138 694

9

»

1671

118 657

19

6

1672

50 305

14

8

1673

21 803

16

2

1674

14 766

9

»

1675

14 393

13

»

1676

13 225

13

»

1677

27 894

7

»

1678

2 999

18

»

1679

5 195

9

»

1680

5 902

11

6

1681

2 047

10

»

1682

3 407

4

»

1683

2 197

10

6

Depuis, pour transporter la tour de bois de Marly et la mettre en place

10 886

7

4

Somme totale

725 174

4

8

Sept cent vingt-cinq mille, cent soixante-quatorze livres, quatre sols,
huit deniers.

Depuis 1683 jusqu'en 1690, il n'a été fait que très peu de dépenses à
l'Observatoire, hors pour le transport et emplacement de ladite tour de
bois.

CHAPITRE IX.

Hôtel royal et église des Invalides, commencé en 1679.

Cette maison, destinée pour la retraite des soldats devenus invalides au
service de Sa Majesté, est d'une étendue extraordinaire et d'une
régularité parfaite. Sa situation est très belle dans une plaine, en
face du cours la Reine, la rivière entre-deux, de manière que ces objets
différents se prêtent l'un à l'autre un ornement réciproque.

Les dedans de la maison sont très vastes, et en même temps très
logeables. La discipline y est la même que dans une place de guerre.
Elle est gouvernée par un nombre suffisant d'officiers, en sorte que la
paix et le silence y règnent à peu près comme dans un cloître.

L'église est desservie par les pères de la Mission, qui ont leur
logement séparé à côté de l'église, séparée des autres logements. Cette
église est d'un dessin très magnifique. Le grand autel, isolé sous un
dôme entre deux nefs très spacieuses, dont l'une qui a son entrée du
côté de la maison est destinée pour ceux qui y habitent, et l'autre qui
a son entrée par un portail magnifique du côté de la campagne est
destinée pour le public. Rien n'est épargné pour rendre cet édifice
admirable en toutes ses parties, comme il est un des plus glorieux à la
piété du roi.

Les fonds pour la subsistance de cette maison sont levés par les
trésoriers de l'extraordinaire des guerres sur le payement des troupes,
à raison de trois deniers pour livre, et le trésorier des Invalides en
fait l'emploi, suivant qu'il lui est ordonné par le commissaire
ordonnateur.

\emph{Dépenses de l'hôtel royal et église des Invalides}.

Année.

livres.

sols.

den.

1679

56 000

»

»

1680

80 667

11

6

1681

72 000

»

»

1682

87 000

»

»

1683

81 647

5

6

1684

103 332

»

»

1685

147 573

5

9

1686

176 505

15

»

1687

169 460

9

7

1688

186 282

19

»

1689

172 706

4

9

1690

143 432

10

10

1691

233 724

2

7

Somme totale

1 710 332

4

6

Dix-sept cent dix mille, trois cent trente-deux livres, quatre sols, six
deniers.

On a excédé dans ce chapitre les bornes qu'on s'était prescrites, à
cause de la dépense considérable qui a été faite aux Invalides l'année
1691. Il en a été fait d'autres depuis, et l'on peut compter que cet
édifice reviendra à deux millions.

CHAPITRE X.

Place royale de hôtel Vendôme et nouveau couvent des capucines, commencé
en 1685.

Les dépenses de ces deux édifices sont confondues, parce qu'ils ont été
élevés sur le même terrain et en même temps.

La place n'a point encore d'autre nom que celui de l'hôtel dont Sa
Majesté a acquis le fonds pour la construire. Elle n'est point encore
achevée, mais la statue équestre du roi qui doit y être placée est jetée
en bronze et entièrement réparée sur les dessins et par les soins des
sieurs Girardon, premier sculpteur du roi, et Keller, qui en a fait la
fonte.

Le couvent des Capucines est entièrement achevé, et tous ceux qui en ont
vu les dedans conviennent que c'est un des plus beaux couvents de filles
qui soient à Paris. L'église est bâtie dans le goût de simplicité et de
propreté dont ces religieuses font profession. Elle s'enrichit tous les
jours par les monuments superbes des personnes de qualité qui y
choisissent leur sépulture.

\emph{Dépenses de la place royale de l'hôtel de Vendôme, fonte de la
statue équestre du roi, et couvent des Capucines}.

Premièrement. --- L'acquisition de l'hôtel de Vendôme, du prix de six
cent mille livres, les intérêts de moitié du prix, soixante-six mille
livres que le roi a données a M. le duc de Vendôme, au delà dudit prix,
et vingt-cinq mille livres pour les lods et ventes, et les frais du
décret, le tout montant à la somme de 131 208l. 15s. »d.

Année.

livres.

sols.

den.

Acquisition

131 208

15

»

1685

Ouvrages

21 708

3

7

1686

320 969

7

8

1687

467 063

8

3

1688

275 835

14

5

1689

71 215

5

7

1690

174 698

14

10

Somme totale

2 062 699

9

4

CHAPITRE XI.

Le Val-de-Grâce, à Paris.

Cet édifice, que la feue reine mère a fait bâtir, est superbe et
magnifique en toutes ses parties. Il revient à trois millions; mais il
n'en a été pris sur les fonds des bâtiments que trois cent soixante-dix
mille livres dans les années ci-après nommées, savoir:

Année.

livres.

sols.

den.

1666

314 600

7

2

1667

30 571

11

9

1670

6 000

»

»

1681

10 400

»

»

1682

8 711

13

10

Somme totale

370 283

12

9

Il a encore été fait quelques dépenses depuis peu d'années pour revêtir
de marbre le caveau des reines, destiné pour recevoir leurs cœurs et
leurs entrailles.

CHAPITRE XII.

Couvent de l'Annonciade de Meulan, commencé en 1682.

Comme il y a peu de personnes qui sachent ce qui a engagé le roi a faire
bâtir ce couvent, et que j'en suis parfaitement instruit, j'en dirai un
mot.

Il y a eu longues années dans ce couvent une supérieure d'une vertu
extraordinaire, que la feue reine honorait de son estime et de son
amitié, et même quelquefois de ses visites. Le roi y alla aussi dans ses
jeunes années, et y posa la première pierre dans le dessein d'y faire
bâtir. Ce dessein a été différé pendant plusieurs années. Feu mon père,
qui était allié à cette supérieure, la visitait souvent et négocia
auprès de la reine mère l'accomplissement de son projet. En effet la
reine lui ayant renouvelé ses promesses, et le mal dont elle mourut
s'augmentant, elle eut la bonté d'en parler au roi, qui depuis a fait
bâtir ce couvent, qui coûta près de trente mille écus, et de plus Sa
Majesté fait une pension à la communauté, qui n'est pas riche.

\emph{Dépenses dudit couvent}.

Année.

livres.

sols.

den.

1682

20 000

»

»

1683

29 400

»

»

1684

6 659

5

1

1685

11 551

1

»

1686

6 544

»

»

1687

7 270

11

6

1688

6 987

11

6

1689

Néant

1690

Néant

Somme totale

88 412 10

10

1

Quatre-vingt-huit mille, quatre cent douze livres, dix sols, un denier.

CHAPITRE XIII.

Canal de communication des mers en Languedoc, commencé en 1670.

Comme ce canal n'est point encore achevé, je ne dirai rien de
particulier quant à présent, ni de ses dimensions, ni de son usage. On
sait qu'il porte de petits bâtiments. On peut voir sa situation sur la
carte. On verra ici seulement les dépenses qui ont été faites sur les
fonds des bâtiments du roi, qui montent à près de huit millions sans ce
qui a été fourni par les états de Languedoc pour contribuer à une
entreprise si grande et si utile pour le commerce de la province.

\emph{Dépenses}.

Année.

livres.

sols.

den.

1670

125 000

»

»

1671

525 000

»

»

1672

Néant

1673

1 575 452

13

4

1674

1 235 242

14

»

1675

64 105

»

»

1676

768 541

13

4

1677

561 944

8

8

1678

748 716

9

5

1679

1 194 503

18

11

1680

Néant

1681

460 000

»

»

1682

449 057

»

»

1683

28 992

1

8

Somme totale

7 736 555

19

4

Sept millions, sept cent trente-six mille, cinq cent cinquante-cinq
livres, dix-neuf sols, quatre deniers.

Depuis l'année 1683, il n'y a eu aucunes dépenses dans les comptes des
bâtiments pour ledit canal de Languedoc.

CHAPITRE XIV.

Manufactures des Gobelins et de la Savonnerie.

Les dépenses de ces deux manufactures sont jointes ensemble parce que
les tapisseries sont leur principal objet.

Dans celle de la Savonnerie, qui est à Chaillot, près Paris, l'on ne
fait que des ouvrages façon de Turquie et du Levant. Ces ouvrages sont
une espèce de \emph{velours ras}, entièrement de laine et servant à
faire des meubles, comme des portières, des tapis, des formes et des
tabourets.

La manufacture des Gobelins est établie au bout du faubourg
Saint-Marcel, et est bien plus spacieuse. Elle renferme un très grand
nombre d'ouvriers célèbres dans leurs arts, premièrement pour les
tapisseries. On y travaille pour Sa Majesté, en haute et basse lisse,
sur les dessins des plus habiles peintres, soit anciens soit modernes.

Les tapisseries qui s'y font représentent les unes des sujets
d'histoire, d'autres les conquêtes du roi, les maisons royales, les
assemblées et fêtes publiques, et toutes sortes de sujets indifférents.
On sait assez le mérite de tous ces ouvrages où l'art du dessin surpasse
infiniment la richesse et la finesse des étoffes.

Dans la même manufacture sont logés des peintres, des sculpteurs, des
ébénistes et fondeurs, des orfèvres, des lapidaires qui travaillent aux
pierres fines de rapport, \emph{manière de Florence}. Dans le temps de
paix, ces artistes sont occupés uniquement à faire des ouvrages pour le
service de Sa Majesté, et n'ont pas le temps de travailler pour le
public.

Cette maison est pourvue de toutes choses agréables, commodes et
nécessaires; le service divin s'y célèbre; les ouvriers y sont
instruits, et les enfants catéchisés aux dépens de Sa Majesté, ce qui
marque dans quel détail sa piété le fait descendre.

Année.

livres.

sols.

den.

1664

95 885

»

»

1665

95 594

11

»

1666

101 377

11

1

1667

290 744

13

4

1668

214 020

19

2

1669

133 209

13

»

1670

141 486

15

3

1671

176 502

11

»

1672

130 573

5

5

1673

139 747

11

4

1674

122 910

15

4

1675

107 958

13

»

1676

98 004

19

4

1677

110 795

8

6

1678

107 456

15

1

1679

126 933

12

4

1680

117 698

1

6

1681

116 127

5

7

1682

126 358

7

1

1683

146 694

7

2

1684

95 570

9

»

1685

224 321

18

7

1686

123 289

4

9

1687

127 217

1

8

1688

132 961

12

2

1689

146 724

6

3

1690

95 777

17

9

Somme totale

3 645 943

5

1

Trois millions, six cent quarante-cinq mille, neuf cent quarante-trois
livres, cinq sols, un denier.

Pendant la guerre que les ouvrages ont cessé, Sa Majesté a fait des
pensions aux principaux ouvriers de ladite manufacture des Gobelins.

CHAPITRE XV.

Manufactures établies en plusieurs villes de France.

Outre les manufactures des Gobelins et de la Savonnerie, Sa Majesté en a
fait établir encore plusieurs autres en divers endroits du royaume; mais
comme ces dernières ne sont plus du ressort des bâtiments du roi, mais
du contrôle général des finances, je n'entrerai point dans le détail de
ces différents établissements dont les dépenses, aussi glorieuses à Sa
Majesté qu'utiles à l'État, montent pendant les 27 années de ces
mémoires à près de deux millions, comme il suit:

Année.

livres.

sols.

den.

1664

66 121

5

8

1665

254 019

14

»

1666

2 077

3

6

1667

248 14

13

»

1668

179 767

15

»

1669

535 705

16

»

1670

131 030

10

»

1671

110 625

15

2

1672

99 558

5

10

1673

49 046

»

»

1674

8 000

»

»

1675

18 000

»

»

1676

8 000

»

»

1677

8 000

»

»

1678

8 000

»

»

1679

18 298

»

10

1680

19 120

»

»

1681

20 539

15

»

1682

8 000

»

»

1683

15 520

»

»

1684

16 000

»

»

1685

8 000

»

»

1686

8 000

»

»

1687

42 283

13

»

1688

50 690

»

»

1689

22 940

11

»

1690

23 970

10

»

Somme totale

1 979 990

9

»

Dix-neuf cent soixante-dix-neuf mille, neuf cent quatre-vingt-dix
livres, neuf sols.

CHAPITRE XVI.

Pensions des gens de lettres.

L'estime singulière que Sa Majesté a toujours fait des belles-lettres et
des personnes qui par une longue étude et un travail assidu se sont
rendues célèbres dans les sciences a porté Sa Majesté à animer ceux qui
se trouvent nés avec d'heureuses dispositions par l'espérance des
pensions attachées au seul mérite. Ces pensions ne se payent plus sur le
fonds des bâtiments depuis l'année 1690:

Année.

livres.

sols.

den.

1664

80 870

»

»

1665

83 400

»

»

1666

95 507

»

»

1667

92 380

»

»

1668

89 400

»

»

1669

110 550

»

»

1670

107 550

»

»

1671

100 075

»

»

1672

86 800

»

»

1673

84 200

»

»

1674

62 250

»

»

1675

57 550

»

»

1676

49 200

»

»

1677

65 100

»

»

1678

52 400

»

»

1679

54 000

»

»

1680

53 600

»

»

1681

53 500

»

»

1682

52 800

»

»

1683

1 600

»

»

1684

42 100

»

»

1685

46 400

»

»

1686

41 400

»

»

1687

46 900

»

»

1688

44 900

»

»

1689

39 400

»

»

1690

11 966

13

4

Somme totale

1 707 148

13

4

Dix-sept cent sept mille, cent quarante-huit livres, treize sols, quatre
deniers.

CHAPITRE XVII.

Académie de Paris et de Rome.

\emph{Académie française}.

Cette académie est composée tant de la plupart des personnes qui ont les
pensions dont il a été parlé au chapitre précédent, que d'autres
personnes savantes. Elle ne coûte au roi, outre ces pensions, qu'environ
sept mille livres par an. Savoir: environ 6400 livres en jetons
d'argent; 300 livres pour une messe qui y est chantée en musique le jour
de Saint-Louis, et 300 livres qui sont entre les mains du trésorier de
ladite académie pour la fourniture de bois et bougies et transcriptions
de cahiers.

\emph{Académies des sciences et des inscriptions}.

Les dépenses de ces deux académies ne sont pas assez considérables pour
en faire mention, et elles ne se prennent plus sur le fonds des
bâtiments.

\emph{Académie d'architecture de Paris}.

Cette académie ne coûte au roi qu'environ trois mille cinq cents livres
par an, tant pour les appointements d'un professeur qui y tient les
conférences publiques, que pour les assistances des architectes qui s'y
assemblent en particulier et pour les menues nécessités.

\emph{Académie de peinture et sculpture de Paris}.

Cette académie coûte en premier lieu au roi, six mille livres qui se
mettent tous les ans entre les mains de son trésorier;

Plus, deux mille six cent quarante livres par an, pour la subsistance de
dix élèves de peinture et de sculpture à chacun desquels le trésorier
des bâtiments paye 264 livres par an, et de plus Sa Majesté fait
distribuer des prix aux élèves, qui sont des médailles qui se payent sur
le fonds des bâtiments au directeur du balancier du Louvre, où elles
sont frappées. Cette dépense n'est pas fixe.

\emph{Académie de peinture, sculpture et architecture de Rome}.

Sa Majesté a établi et entretient l'Académie de Rome, comme dans un lieu
d'où sont sortis ce que nous avons eu de plus excellents maîtres, et qui
est aussi la source des plus parfaites productions des arts. On y envoie
les élèves pour s'y perfectionner. On peut compter sur une dépense
d'environ soixante mille livres par an, pour l'entretien de cette
académie; et ces fonds sont remis au directeur, qui en doit compte.

Voilà toutes les maisons royales dont j'ai cru devoir exposer les
dépenses en détail, celles qui ont été faites aux autres maisons royales
insérées au catalogue qui est en tête de cet ouvrage, n'étant pas assez
considérables. Ces dépenses seront confondues dans l'état général des
dépenses qui ont été faites dans les bâtiments du roi pendant les
vingt-sept années de ces mémoires. Cet état terminera un travail plus
vaste dans ses opérations qu'il n'est resserré dans sa perfection,
toutes les sommes qui y sont assises étant le fruit d'un choix très
circonspect et des calculs les plus exacts, à cause de la contusion des
comptes.

CHAPITRE XVIII.

État général des dépenses des bâtiments du roi pendant les vingt-sept
années de ces mémoires, suivant les états finaux et arrêtés des comptes
et états au vrai.

Année.

livres.

sols.

den.

1664

3 221 731

2

2

1665

3 269 723

19

3

1666

2 826 770

3

5

1667

3 516 160

3

10

1668

3 616 486

»

2

1669

5 192 957

8

6

1670

6 834 037

16

»

1671

7 865 243

1

2

1672

4 168 354

12

6

1673

3 550 410

3

8

1674

3 898 466

5

10

1675

3 091 587

10

2

1676

3 195 381

7

2

1677

3 265 220

17

9

1678

4 977 253

10

6

1679

9 373 614

10

8

1680

8 615 287

18

9

1681

6 465 309

16

»

1682

6 985 568

13

5

1683

5 995 996

2

10

1684

7 996 163

1

»

1685

15 408 443

19

7

1686

9 064 446

15

6

1687

8 279 526

11

10

1688

7 347 966

6

9

1689

3 644 587

13

4

1690

1 616 134

18

8

Somme totale

153 282 827

10

5

Cent cinquante-trois millions, deux cent quatre-vingt-deux mille, huit
cent vingt-sept livres, dix sols, cinq deniers.

Le roi a tellement augmenté sa maison royale pendant ces vingt-sept
années, que quand Sa Majesté ne ferait point élever de nouveaux
édifices, les seuls entretiens coûteront par nécessité quinze à seize
cent mille livres par an, compris les gages des officiers et autres
employés.

Mais si la grandeur et la magnificence du roi paraît dans la somptuosité
de ses superbes édifices, et si par une dépense si considérable, il
s'élève au-dessus de tous les princes de l'Europe, il ne paraît pas
moins de grandeur dans les motifs qui l'ont porté à exécuter de si
vastes desseins. Élever des palais, bâtir des temples au Seigneur, faire
fleurir les sciences et les arts, c'est immortaliser sa grandeur, sa
piété et son mérite; faire subsister une infinité de personnes, qui par
ce moyen ont trouvé dans le sein de leur patrie de quoi élever leurs
familles; récompenser les gens de mérite et célèbres dans les arts;
encourager les élèves, et leur procurer les moyens d'arriver à la
perfection des plus excellents maîtres, c'est l'effet d'une bonté toute
paternelle, qui mérite au roi, avec autant de justice qu'à l'empereur
Auguste, le glorieux nom de Père de la patrie.

Le roi n'a pu rien faire de plus glorieux, surtout dans les temps de
paix, qui pour un prince moins attentif à sa gloire et au bonheur de ses
peuples auraient été des espèces d'interrègnes, et auraient laissé des
vides à remplir dans son histoire. Mais notre prince compte tous ses
moments, et il croirait avoir perdu un jour, s'il l'avait passé sans
donner quelques marques de sa grandeur, de sa justice ou de sa bonté;
s'il n'était pas aussi grand dans ces temps heureux de repos et de
silence que dans ceux où ses armées portent l'effroi dans les terres
ennemies, nous n'aurions pas vu tous les princes conspirer contre un si
glorieux repos. La religion a paru le motif de leur dernière
considération; mais elle n'en a été que le prétexte, et le roi a soutenu
pendant dix campagnes tant d'efforts redoublés, seul contre tous. Il a
pris leurs villes, gagné des batailles, dissipé leurs armées, déconcerté
leurs projets. La Flandre, la Savoie et l'Allemagne, la Catalogne, les
mers ont été en même temps le théâtre de la guerre; disons mieux des
conquêtes du roi. Que n'a-t-il point fait, ce pieux monarque, pour
épargner le sang de tant d'ennemis, et pour finir une guerre si longue
par une paix aussi glorieuse que solide? L'histoire développera un jour
tous ces secrets de son grand cœur. Mon dessein n'est pas d'entrer dans
une si vaste carrière. Ces faibles caractères échappés à l'ardeur de mon
zèle, partent d'un cœur pénétré de la part qu'il prend à la
reconnaissance publique. Eh! sous un règne si grand, faut-il s'étonner
que le roi soit chéri de ses plus petits sujets, comme de ceux qui,
ayant l'honneur d'approcher de sa royale personne, ont aussi le bonheur
de voir de plus près cette étendue de grandeur, de majesté et de mérite,
qu'on ressent mieux qu'on ne peut l'exprimer, et qui remplit les cœurs
autant d'amour que de respect ?

De l'amour des sujets dépend en quelque sorte la fortune d'un prince;
aussi voyons-nous de quels succès les entreprises du roi sont toujours
suivies. Sa sagesse, qui fait revivre celle du plus sage prince du
monde, anime ses ministres et son conseil. Son héroïque valeur imprimée
dans le cœur et sur le front des généraux qui comptent pour rien le sang
qu'ils versent pour leur prince, passe jusque dans l'âme des soldats, et
l'expérience nous a appris que combattre pour le roi, et vaincre, ç'a
toujours été la même chose.

Une si longue suite de prospérités est le pur ouvrage du Dieu des
armées, qui disposant des volontés de tous les hommes, selon ses
desseins éternels, tient en sa main d'une manière spéciale les coeurs
des rois. Aussi Sa Majesté Très-Chrétienne qui, comptant pour peu ses
propres forces, rapporte à la protection divine tout le bonheur de ses
armes, a cru ne pouvoir mieux lui en marquer sa reconnaissance, qu'en
abolissant dans son royaume tout culte impur, et en nous montrant tous
les jours par la sincérité de son zèle, et par l'assiduité de ses
exemples, que le vrai Dieu du ciel et de la terre doit et veut être
adoré en esprit et en vérité dans l'unité de la religion catholique.
Veuille ce même Dieu conserver longues années la personne sacrée de Sa
Majesté. Ce sont les vœux de tous ses sujets qui ne sauraient trop
souhaiter la durée d'un règne si rempli de piété, de justice et de
gloire.

\end{document}
